\chapter{Semi\naive\ Evaluation}
\label{chapter-seminaive}

\fixme{jeremy}{I should cite and discuss my POPL 2020 paper. But where and how? Check Jeremy's feedback.}

\noindent
In \cref{chapter-datafun} we presented Datafun's syntax and semantics. These semantics are straightforward to implement directly; implementing them \emph{efficiently} is more difficult. Datalog has decades of well-studied implementation and optimization techniques\todolater{citations}.
%
To explore whether these techniques can be transferred to Datafun, in this
chapter we'll examine just one classic Datalog optimization,
\emph{semi\naive\ evaluation}, which makes practical Datalog and Datafun's
defining feature: iterative fixed points.

In \cref{section-seminaive-incremental} we'll see how the direct approach to finding fixed points wastes time by recomputing already-known facts at each iteration, and how semi\naive\ evaluation fixes this by computing the differences between iterations. Our key insight (thought not original to us\todo{citation}) is to see semi\naive\ evaluation as an application of incremental computation, that is, efficiently responding to changes.
%
To apply this insight, in \cref{section-change-structures} we adapt prior work on the incremental lambda calculus~\citep{incremental} to construct a category of incrementalizable monotone maps capable of interpreting Datafun's semantics.
%
Using this construction as a guide, our central contribution~(\cref{section-phi-delta}) is a pair of static Datafun-to-Datafun translations which enable the semi\naive\ fixed-point-finding strategy. Finally, we prove these transformations correct using a logical relation  (\cref{section-seminaive-logical-relation}).

\section{From semi\naive{} evaluation to the incremental \boldfn-calculus}
\label{section-seminaive-incremental}


\subsection{Semi\naive\ evaluation as incremental computation}

Let's return to our example Datalog program~\todo{cross reference Datalog
  transitive closure example}, modified to consider graphs rather than ancestry:

\begin{datalog}
  \atom{path}{X,Z} &\gets \atom{edge}{X,Z}
  \\
  \atom{path}{X,Z} &\gets \atom{edge}{X,Y} \wedge \atom{path}{Y,Z}
\end{datalog}

\noindent
Suppose \name{edge} denotes a linear graph, $\{(1, 2),\, (2, 3),\, \dots,\,
({n-1}, n)\}$. Then \name{path} should denote its reachability relation,
$\setfor{(i, j)}{1 \le i < j \le n}$. How can we compute this? The simplest
approach is to begin with nothing in the \name{path} relation and repeatedly
apply its rules until nothing more is deducible. We can make this strategy
explicit by time-indexing the \name{path} relation:

\begin{datalog}
  \name{path}_{i+1}(X,Z) &\gets \atom{edge}{X,Z}
  \\
  \name{path}_{i+1}(X,Z) &\gets \atom{edge}{X,Y} \wedge \name{path}_i(Y,Z)
\end{datalog}

\noindent
By omission $\name{path}_0 = \emptyset$.
%
From this inductively $\name{path}_i \subseteq \name{path}_{i+1}$, because at step $i+1$ we re-deduce every fact known at step $i$.
%
For example, suppose $\name{path}_i(j, k)$ holds. Then at step $i+1$ the second
rule deduces $\name{path}_{i+1}({j-1}, k)$ from $\atom{edge}{{j-1}, j} \wedge
\name{path}_i(j,k)$.
%
But since $\name{path}_{i+1}(j, k)$ holds, we perform the same deduction at time
$i+2$, and again at $i+3$, $i+4$, etc.

Because we append edges one at a time, $\name{path}_i$ contains all paths of
$i$ or fewer edges.
%
Therefore it takes $n$ steps until we reach our fixed point $\name{path}_{n-1} =
\name{path}_n$.
%
Since step $i$ involves $|\name{path}_i| \in \Theta(i^2)$ deductions, we make
$\Theta(n^3)$ deductions in total.
%
There being only $\Theta(n^2)$ paths in the final result, this is terribly
wasteful; hence we term this \emph{\naive\ evaluation}.

\label{section-seminaive-tc-in-datafun}

Now let's move from Datalog to Datafun.\footnote{In this section we do not
  bother distinguishing monotone variables $\mvar x$ or discrete expressions
  $\eiso e$, as it muddies our examples to no benefit.} The transitive closure
of \name{edge} is the fixed point of the monotone function \name{step} defined
by:

\nopagebreak[2]
\begin{code}
\name{step} \<\name{path} = \name{edge} \cup
\setfor{(x,z)}{(x,y) \in \name{edge},\, (y,z) \in \name{path}}
\end{code}

\noindent
The \naive\ way to compute this is to iterate \name{step}: start from
\(\name{path}_0 = \emptyset\) and compute successive \(\name{path}_{i+1} =
\name{step}\<\name{path}_i\) until \(\name{path}_i = \name{path}_{i+1}\).
%
But as before, $\name{path}_i \subseteq \name{step}\<\name{path}_i$; each iteration re-computes the paths found by its predecessor.
%
Following Datalog, we'd prefer to compute only the \emph{change} between
iterations.
%
So consider $\name{step}'$ defined by:

\nopagebreak[2]
\begin{code}
\name{step}' \<\name{dpath} =
\setfor{(x,z)}{(x,y) \in \name{edge},\, (y,z) \in \name{dpath}}
\end{code}

\newcommand\colorA{\color{ColorA}}
\newcommand\colorB{\color{ColorB}}

\noindent
Observe that

\nopagebreak[4]
\begin{datalog}
  &\mathrel{\hphantom{=}} \name{step} \<(\name{path} \cup \name{dpath})
  \\
  &= \name{edge} \cup \setfor{(x,z)}{(x,y) \in \name{edge},\, (y,z) \in \name{path} \cup \name{dpath}}
  \\
  &= {\colorA \name{edge} \cup \setfor{(x,z)}{(x,y) \in \name{edge},\, (y,z) \in \name{path}}} \cup {\colorB \setfor{(x,z)}{(x,y) \in \name{edge},\, (y,z) \in \name{dpath}}}
  \\
  &= {\colorA\name{step}\<\name{path}} \cup {\colorB\name{step}'\<\name{dpath}}
\end{datalog}

\noindent
In other words, $\name{step}'$ tells us how \name{step} changes as its input
grows.
%
This lets us directly compute the changes $\name{dpath}_i$ between our
iterations $\name{path}_i$:

\begin{datalog}
  \name{dpath}_0
  &= \name{step}\<\emptyset
  = \name{edge}
  \\
  \name{dpath}_{i+1}
  &= \name{step}'\<\name{dpath}_i
  = \setfor{(x,z)}{(x,y) \in \name{edge},\, (y,z) \in \name{dpath}_i}
  \\
  \name{path}_{i+1}
  &= \name{path}_i \cup \name{dpath}_i
\end{datalog}

\noindent These exactly mirror the derivative and accumulator rules for
\(\name{path}_i\) and \(\name{dpath}_i\) we gave earlier.

The problem of semi\naive\ evaluation for Datafun, then, reduces to the problem
of finding functions, like $\name{step}'$, which compute the change in a
function's output given a change to its input.
%
This is a problem of \emph{incremental computation}, and since Datafun is a
functional language, we turn to the \emph{incremental
  \fn-calculus}~\citep{incremental,DBLP:conf/esop/GiarrussoRS19}.


\subsection{Change structures}
\label{section-change-structures}

To make precise the notion of change, an incremental \fn-calculus associates
every type $A$ with a \emph{change structure}, consisting of:%
%
\footnote{Our notion of change structure differs significantly from that of
  \citet{incremental}, although it is similar to the logical relation given in
  \citet{DBLP:conf/esop/GiarrussoRS19}; \todo{we discuss this in
    \cref{section-differences-from-incremental}}. Although we do not use change
  structures \emph{per se} in the proof of correctness sketched in
  \cref{section-seminaive-logical-relation}, they are an important source of
  intuition.}

\begin{enumerate}
\item A type $\D A$ of possible changes to values of type $A$.
\item A relation $\changesat{A}{\dx}{x}{y}$ for $\dx : \D A$ and $x,y : A$,
  read as ``$\dx$ changes $x$ into $y$''.
\end{enumerate}

\noindent
Since the iterations of a fixed point grow monotonically, in Datafun we only
need \emph{increasing} changes.
%
For example, sets change by gaining new elements:

\begin{align*}
  \D\tseteq{A} &= \tseteq{A}
  &
  \changesat{\tseteq{A}}{\dx}{x}{x \cup \dx}
\end{align*}

\noindent
Set changes may be the most significant for fixed point purposes, but to handle
all of Datafun we need a change structure for every type. For products and sums,
for example, the change structure is pointwise:

\begin{center}
  \setlength\tabcolsep{10pt}
  \begin{tabular}{@{}ccc@{}}
    $\D\tunit = \tunit$
    &
    \(\D(A \x B) = \D A \x \D B\)
    &
    \(\D(A + B) = \D A + \D B\)
    \\[\betweenfunctionskip]    % TODO: is this the right distance?
    \(\changesat{\tunit}{\tuple{}}{\tuple{}}{\tuple{}}\)
    &
    \(\infer{
      \changesat{A}{\da}{a}{a'}
      \\
      \changesat{B}{\db}{b}{b'}
    }{\changesat{A \x B}
      {\tuple{\da,\db}}
      {\tuple{a,b}}
      {\tuple{a',b'}}
    }\)
    &
    \(\infer{
      \changesat{A_i}{\dx_i}{x}{x'}
    }{
      \changesat{A_1 + A_2}{\inj i \dx}{\inj i x}{\inj i x'}
    }\)
  \end{tabular}
\end{center}

%% \begin{align*}
%%   \D\tunit &= \tunit
%%   &
%%   \D(A \x B) &= \D A \x \D B
%%   &
%%   \D(A + B) &= \D A + \D B
%% \end{align*}
%%
%% \begin{align*}
%%   \changesat{\tunit}{\tuple{}}{\tuple{}}{\tuple{}}
%%   &&
%%   %% \infer{
%%   %%   \fa{i} \changesat{A_i}{\dx_i}{x_i}{y_i}
%%   %% }{\changesat{A_1 \x A_2}
%%   %%   {\tuple{\vec\dx}}
%%   %%   {\tuple{\vec x}}
%%   %%   {\tuple{\vec y}}
%%   %% }
%%   %
%%   %% \infer{
%%   %%   \fa{i} \changesat{A_i}{\dx_i}{x_i}{y_i}
%%   %% }{\changesat{A_1 \x A_2}
%%   %%   {\tuple{\dx_1,\dx_2}}
%%   %%   {\tuple{x_1,x_2}}
%%   %%   {\tuple{y_1,y_2}}
%%   %% }
%%   %
%%   \infer{
%%     \changesat{A}{\da}{a}{a'}
%%     \\
%%     \changesat{B}{\db}{b}{b'}
%%   }{\changesat{A \x B}
%%     {\tuple{\da,\db}}
%%     {\tuple{a,b}}
%%     {\tuple{a',b'}}
%%   }
%%   &&
%%   \infer{
%%     \changesat{A_i}{\dx}{x}{y}
%%   }{
%%     \changesat{A_1 + A_2}{\inj i \dx}{\inj i x}{\inj i y}
%%   }
%% \end{align*}

\noindent
Since we only consider increasing changes, and $\iso A$ is ordered discretely,
the only ``change'' permitted is to stay the same. Consequently, no information
is necessary to indicate what changed:

\begin{align*}
  \D(\iso A) &= \tunit
  &&
  \changesat{\iso A}{\tuple{}}{x}{x}
\end{align*}

\noindent
Finally we come to the most interesting case: functions.

\begin{align*}
  \D(A \to B) &= \iso A \to \D A \to \D B
  &
  \infer[fn~change]{
    \fa{\changesat A \dx x y}
    \changesat B {\df\<x\<\dx} {f\<x} {g\<y}
  }{
    \changesat{A \to B}{\df}{f}{g}
  }
\end{align*}

\noindent
Observe that a function change $\df$ takes two arguments: a base point $x : \iso A$ and a change $\dx : \D A$.
%
To understand why we need both, consider incrementalizing function application:
we wish to know how $f\<x$ changes as both $f$ and $x$ change.
%
Supposing $\changes{\df}{f}{g}$ and $\changes{\dx}{x}{y}$, how do we find a
change $f\<x \changesto g\<y$ that updates both function and argument?

If changes were given pointwise, taking only a base point, we might take
$\changes{\df}{f} g$ to mean that $\fa{x} \changes{\df\<x}{f\<x}{g\<x}$. But
this only gets us to $g\<x$, not $g\<y$: we've accounted for the change in the
function, but not the argument.
%
We can account for both by giving $\df$ an additional parameter: not just the
base point $x$, but also the change to it $\dx$.
%
Then by inverting \rn{fn~change} we have $\changes{\df\<x\<\dx}{f\<x}{g\<y}$ as
desired.

%% This makes it easy to incrementalize function application, $f\<x$; given
%% changes $\changes \df f g$ and $\changes \dx x y$ to the function and its
%% argument, we want to compute the change that takes us to the updated
%% application $g\<y$. By inverting \textsc{FnChange} we know that
%% $\changes{\df\<x\<\dx}{f\<x}{g\<y}$, so $\df\<x\<\dx$ gives us the desired
%% change.

%% If instead changes were given pointwise, letting $\D(A \to B)= \iso A \to \D B$,
%% then it'd be natural to let $\changes{\df}{f}{g} \iff \fa{x}
%% \changes{\df\<x}{f\<x}{g\<x}$.

Note also the mixture of monotonicity and non-monotonicity in the type $\iso A
\to \D A \to \D B$. Since our functions are monotone (increasing inputs yield
increasing outputs), function changes are monotone with respect to input changes
$\D A$: a larger increase in the input yields a larger increase in the output.
However, there's no reason to expect the change in the output to grow as the
base point increases -- hence the base point argument is discrete, $\iso A$.


\subsection{Zero changes, derivatives, and faster fixed points}
\label{section-derivatives}

If $\changesat A \dx x x$, we call $\dx$ a \emph{zero change} to $x$. Usually
zero changes are boring -- for example, a zero change to a set $x :
\tseteq{A}$ is any $\dx \subseteq x$, and so $\emptyset$ is always a zero
change.
%
However, there is one very interesting exception: function zero changes. Suppose
$\changesat{A \to B}{\df}{f}{f}$. Then inverting \rn{fn change} implies that

\begin{equation*}
  \changesat A \dx x y \implies \changesat B{\df\<x\<\dx}{f\<x}{f\<y}
\end{equation*}

\noindent
In other words, $\df$ yields the change in the output of $f$ given a change to
its input.
%
This is exactly the property of $\name{step}'$ that made it useful for
semi\naive\ evaluation -- indeed, $\name{step}'$ is a zero change to
\name{step}, modulo not taking the base point $x$ as an argument:

\begin{align*}
  \changesat{\tseteq A} \dx x y
  &\implies
  \changesat{\tseteq A}{\name{step}'\<\dx}{\name{step}\<x}{\name{step}\<y}
  \\
  % should make a box that is as wide as \implies here.
  &\textit{that is,}
%  &\omit\hphantom{wut}\clap{\text{\emph{equivalently}}}
  \\[2.5pt]
  x \cup \dx = y
  &\implies
  \name{step}\<x \cup \name{step}'\<\dx = \name{step}\<y
\end{align*}

\noindent
Function zero changes are so important we give them a special name:
\emph{derivatives}. We now have enough machinery to prove correct a
general \emph{semi\naive\ fixed point strategy}. First, observe that:

\begin{lemma}\label{lemma-DeltaL}
  At every semilattice type $L$, we have $\D L = L$ and
  $\changesat{L}{\dx}{x}{y} \iff (x \binvee \dx) = y$.
\end{lemma}

\noindent
This holds by a simple induction on semilattice types $L$. Now, given
a monotone map $f : L \to L$ and its derivative $f' : \iso L \to L \to
L$, we can find $f$'s fixed-point as the limit of the sequence $x_i$
defined:

\begin{align*}
  x_0 &= \bot & x_{i+1} &= x_i \vee \dx_i\\
  \dx_0 &= f\<\bot & \dx_{i+1} &= f'\<x_i\<\dx_i
\end{align*}

\noindent
Observe that the function $f$ itself is only used once, to calculate $\dx_0 = f
\<\bot$.
%
Given this initial ``kickoff'' change, the remaining $x_i$ are calculated
entirely using the derivative $f'$.
%
Let $\fastfix\<(\dx_0,\, f') = \bigvee_i x_i$ be the limit of this sequence.
%
By induction and the derivative property, we have $\changes{\dx_i}{x_i}{f\<x_i}$
and so $x_i = f^i\<x$, and therefore $\fastfix\<(f\<\bot,\, f')$ is the least
fixed point of $f$.
%
And if $L$ has no infinite ascending chains, we will reach this fixed point in
some finite number of iterations $i$ such that $x_i = x_{i+1}$.

\label{section-seminaive-strategy}

This leads directly to our strategy for semi\naive\ Datafun.
%
\Citet{incremental} defines a static transformation $\Deriv e$ which computes
the change in $e$ given the change in its free variables; it
\emph{incrementalizes} $e$.
%
Our goal is not to incrementalize Datafun \emph{per se}, but to find fixed
points faster.
%
Consequently, we define two mutually recursive transformations: $\phi e$, which
computes $e$ faster by replacing fixed points with calls to \fastfix; and
$\delta e$, which incrementalizes $\phi e$ just enough that we can compute
the derivative of fixed point functions.


\section{Change structures for Datafun}
\label{section-change-structures}

To solve the problem of computing how a function's output changes in response to its input, we must first make precise the notion of \emph{change} for each type in our language.
%
To do this, incremental \fn-calculi associate every type $A$ with a \emph{change structure}.
%
In our case, noting that Datafun types denote posets, we define change structures as follows:

\begin{definition}\label{definition-change-structures}
  A \emph{change structure} $A$ consists of a poset $V A$, a poset $\D A$, and a
  relation $R_A \subseteq \D A \x V A \x V A$. For $\dx : \D A$ and $x, y : V
  A$, we will write $(\dx, x, y) \in R_A$ interchangeably as $\changesat A {\dx}
  x y$. This relation must satisfy three
  properties:

  \newlength{\wubwubwub}
  \setlength\wubwubwub{.25\baselineskip}
  \vspace{2\wubwubwub}
  \def\arraystretch{1}
  \begin{tabular}{rp{30em}}
    \emph{Functionality} & If $\changesat A {\dx} x y$ and $\changesat A {\dx} x z$ then $y = z$.
    %% \\ & So $x$ and $\dx$ determine $y$ uniquely.
    \\[\wubwubwub]
    \emph{Soundness} & If $\changesat A {\dx} x y$ then $x \le_A y$.
    %% \\ & In other words, all changes are increasing.
    \\[\wubwubwub]
    \emph{Zero changes} & If $x : V A$ there is some $\dx : \D A$ such that $\changesat A {\dx} x x$.
    %% There is a function $\zero_A : A \to \D A$ such that $\changesat A {\zero_A\<a} a a$.
    %% (We do not require $\zero_A$ to be monotone.)
  \end{tabular}
\end{definition}

\noindent
Some useful terminology and notation: We can think of elements $\dx \in \D A$ as
\emph{changes} or ``diffs'' to values $x \in VA$. The relation $R_A$ tell us how
changes affect values: we gloss $\changesat{A}{\dx} x y$ as ``$\dx$ changes $x$
into $y$''. We say that $\dx$ is a \emph{valid} change to $x$ if there is some $y$ such that $\changesat{A}{\dx} x y$.
%
When $\changes{\dx} x x$ we call $\dx$ a \emph{zero change} to $x$; when we need to pick such a change we write $\zero_x$. By the axiom of choice, the \emph{zero changes} property is equivalent to the existence of such a $\zero$ function.

%% Elements of $\D A$ represent ``deltas'' or ``diffs'', while $V_A$ says how these deltas affect elements of $A$.
%% %
%% Given a delta $\dx \in \D A$ and two values $x,y \in A$, one may read $(\dx, x, y) \in V_A$ as ``$\dx$ changes $x$ into $y$''; hence the suggestive notation $\changes {\dx} x y$.
%
%(Visual similarity with the nonsensical type ascription $\dx : x \to y$ is an unfortunate coincidence.)
%
Although we use multi-letter variable names prefixed with ``d'' for elements $\dx,\dy : \D A$ of delta posets, this is merely a naming convention; we could instead use single-letter variables like $p,q : \D A$ with the same meaning.

%% \XXX
%% As for the properties, \emph{functionality} says that the validity relation $V_A
%% \subseteq \D A \x A \x A$ is a partial function $\D A \x A \partialto A$. The
%% intuition here is that a delta $\dx$ applied to a base value $x$ can only
%% produce at most one updated value $y$ -- \emph{at most} because we do not
%% require that every delta be applicable to every value. Second, since the
%% iterations toward a fixed point grow monotonically, in Datafun we only need
%% consider increasing changes; \emph{soundness} says that all changes are
%% increasing changes.
%% %
%% Finally, \emph{zero changes} requires that every value have some way to remain the same.

To motivate our three properties, it will help to consider an example of a
change structure corresponding to an important Datafun type: finite sets
$\tseteq A$.
%
Recall that our goal is to speed up fixed point computation. Since
iterations toward a fixed point grow monotonically, in Datafun we only need
\emph{increasing} changes. Therefore, changes to sets are themselves sets, to be unioned in:

\label{example-finite-set-change-structure}
\begin{align*}
  V \tseteq A &= \tseteq A
  &
  \D\tseteq{A} &= \tseteq{A}
  &
  \infer{
    x \cup \dx = y
  }{
    \changesat{\tseteq{A}}{\dx}{x}{y}
  }
\end{align*}

\noindent
\emph{Functionality} says that $\changesat{\tseteq A}{\dx} x y$ must be a
partial function from $(\dx,x)$ to $y$. In this case, it's a total function: set
union. \emph{Soundness} requires that all changes are increasing, which is true since $x \subseteq x \cup \dx$.
%
Finally, \emph{zero changes} holds since $x \cup \emptyset = x$; one can leave a set unchanged by adding nothing.%
%
\footnote{\label{footnote-completeness}%
Indeed, sets have not only zero changes but all increasing changes: for any $x \le y$ there is a $\dx$ such that $\changesat{\tseteq A}{\dx}{x}{y}$; for instance one may let $\dx = y \setminus x$, or indeed just $y$. We call this property \emph{completeness,} as it is the converse of soundness.
%
However, while our change structure for sets is complete, we will later observe that completeness is troublesome at function types, so we do not insist on it in general.}

We'll see more examples of change structures later, including ones where the validity relation is a partial rather than a total function, but first, let's revisit our transitive closure example from \cref{section-seminaive-incremental}.
%
Using change structures we can generalize the relation between \name{step} and $\name{step}'$. We call $\name{step}'$ a \emph{derivative}, because it tells us how \name{step}'s output changes in respond to its input changing:

\begin{definition}\label{definition-derivative}
  A \emph{derivative} of a monotone map $f : A \to B$ between change structures $A$, $B$ is a monotone map $f' : \iso V A \to \D A \to \D B$ satisfying the law (for all $x,y,\dx$):

  %% \[
  %% \infer{\changesat A {\dx} x y}{\changesat B {f'\<x \<\dx} {f\<x} {f\<y}}
  %% \]

  \[
  {\changesat A {\dx} x y} \implies {\changesat B {f'\<x \<\dx} {f\<x} {f\<y}}
  \]
\end{definition}

\noindent
We say \emph{a} derivative, not \emph{the} derivative, because derivatives are not necessarily unique. This is because changes are not necessarily unique: for fixed $x,y$ there may be many $\dx$ such that $\changes{\dx} x y$.

\label{example-step-prime-is-a-derivative}
Applying this definition to our change structure for finite sets, we recover the relationship we needed between \name{step} and $\name{step}'$ in \cref{section-seminaive-incremental} for semi\naive\ evaluation:

\begin{align*}
  \changesat{\tseteq A} \ds s t
  &\implies
  \changesat{\tseteq A}{\name{step}'\<s\<\ds}{\name{step}\<s}{\name{step}\<t}
  \\
  % should make a box that is as wide as \implies here.
  &\parbox[t]{\widthof{${}\implies{}$}}{\centering\emph{iff}}
  \\[2.5pt]
  s \cup \ds = t
  &\implies
  \name{step}\<s \cup \name{step}'\<s\<\ds = \name{step}\<t
  \\
  % should make a box that is as wide as \implies here.
  &\parbox[t]{\widthof{${}\implies{}$}}{\centering\emph{iff}}
  \\[2.5pt]
  \name{step}\<(s \cup \ds)
  %% &\parbox[t]{\widthof{${}\implies{}$}}{\centering$=$}
  &=
  \name{step}\<s \cup \name{step}'\<s\<\ds
\end{align*}

%% \begin{align*}
%%   & \text{$\name{step}'$ is a derivative of $\name{step} : \tseteq A \to \tseteq A$}\\
%%   \iff& \fa{\changesat{\tseteq A}{\ds} s t}\, \changesat{\tseteq A}{\name{step}' \<s\<\ds}{\name{step}\<s}{\name{step}\<t}\\
%%   \iff& \fa{s \cup \ds = t}\, \name{step}\<s \cup \name{step}'\<s\<\ds = \name{step}\<t\\
%%   \iff& \fa{s,\ds}\, \name{step}\<(s \cup \ds) = \name{step}\<s \cup \name{step}'\<s\<\ds
%% \end{align*}

%% \begin{align*}
%%   & \text{$f'$ is a derivative of $f : \tseteq A \to \tseteq A$}\\
%%   \iff& \fa{\changesat{\tseteq A}{\dx} x y}\, \changesat{\tseteq A}{f' \<x\<\dx}{f\<x}{f\<y}\\
%%   \iff& \fa{x \cup \dx = y}\, f\<x \cup f'\<x\<\dx = f\<y\\
%%   \iff& \fa{x,\dx}\, f\<x \cup f'\<x\<\dx = f\<(x \cup \dx)
%% \end{align*}

\noindent
This generalization is useful because differentiable maps (that is, maps possessing a derivative in the above sense) \emph{compose;} in fact, they form a category:

\newcommand\ChangePoset{\textbf{$\boldsymbol\Delta$Poset}}

\begin{definition}
  The category \ChangePoset\ has as objects change structures $A,B$ and as morphisms differentiable monotone maps $f : V A \to V B$, that is, maps having at least one derivative $f' : \iso V A \to \Delta A \to \Delta B$ (also monotone). Morphism composition and the identity morphism are both as in \Poset.
\end{definition}

\begin{proof}
  Of course, for this to be a category we need to show that: (1) the identity map is differentiable; (2) the composition of two differentiable maps is differentiable. We also need associativity and identity of composition, but these follow from the same in \Poset. Derivatives for identity and composition can be found as follows:

  \begin{align*}
    \id'\<x\<\dx &= \dx
    &
    (g \compose f)' \<x\<\dx &= g' \<(f\<x) \<(f' \<x\<\dx)
  \end{align*}

  \noindent
  That $\id'$ is a derivative of \id\ is trivial, while for composition we need to pick maps $f',g'$ which are derivatives of $f,g$ respectively; then, applying the definition of derivatives:

  \begin{align*}
    &\changes{\dx} x y
    \\
    \implies&
    \changes{f'\<x\<\dx}{f\<x}{f\<y}
    \\
    \implies&
    \changes{g'\<(f\<x)\<(f'\<x\<\dx)}{g\<(f\<x)}{g\<(f\<y)}
  \end{align*}

  %% \[
  %% \infer*{\changes{\dx} x y}{
  %%   \infer*{\changes{f'\<x\<\dx}{f\<x}{f\<y}}{
  %%     \changes{g'\<(f\<x)\<(f'\<x\<\dx)}{g\<(f\<x)}{g\<(f\<y)}
  %%   }
  %% }
  %% \]

  \noindent
  We also need $\id'\<x\<\dx$ and $(g \compose f)' \<x \<\dx$ to be monotone in $\dx$, which they are, for straightforward compositional reasons; in general, for the remainder of \cref{section-change-structures,section-changeposet}, we omit showing that functions are monotone unless the argument is non-obvious.
\end{proof}

\noindent
In the next section we will sketch the most important structures in \ChangePoset\ needed to support Datafun's semantics, providing a recipe for incrementalizing Datafun.  Applied correctly, this will let us automatically find derivatives for functions used by \prim{fix} expressions, allowing us to employ the semi\naive\ evaluation strategy for finding fixed points faster. \todo{is this still accurate?}


\section{The structure of \ChangePoset}
\label{section-changeposet}

We should note up front that ours is only one among many reasonable notions of change structure.
%
For instance, \citet{DBLP:phd/dnb/Giarrusso20} defines both \emph{basic change structures} (definition 12.1.1), consisting only of a delta-set and a relation, and the more elaborate \emph{change structures} (definition 13.1.1) that have an update operator $\oplus : A \x \D A \to A$, a difference operator $\ominus : A \x A \to \D A$, and composition of changes $\circledcirc : \D A \x \D A \to \D A$; while \citet{mario-thesis} uses a definition based on monoid actions.
%
We will compare these with our approach in more detail in \cref{section-related-work-incremental-computation}, but the ``big picture'' difference is that we are pervasively concerned with \emph{monotone functions,} \emph{increasing changes,} and \emph{higher-order computation;} most of our choices flow from one more more of these considerations.

%% Our eventual destination is a static transformation on Datafun source code which implements semi\naive\ evaluation (\cref{section-phi-delta}).
%% %
%% This was originally presented in \cite{seminaive-datafun}; it predates the construction of \ChangePoset\ presented here and is independent of it.
%% %
%% However, this transformation and the logical relation used to prove it correct (\cref{section-seminaive-logical-relation}) are fairly intricate.
%% %
%% Our aim in presenting \ChangePoset\ is to break the core concepts of this transformation down into small pieces, to show how this complexity arises and suggest potential alternatives for future investigation.

Our eventual destination is a static transformation on Datafun source code which implements semi\naive\ evaluation (\cref{section-phi-delta}).
%
This transformation, originally presented in \cite{seminaive-datafun}, predates the construction of \ChangePoset\ presented here and is independent of it.
%
The transformation itself is quite intricate; our aim in presenting \ChangePoset\ is to break its core concepts down into small pieces, showing how this complexity arises and suggesting potential alternatives for future investigation.
%
In service of this goal, we have chosen what seems the simplest definition of change structure that both supports the features of Datafun and provides useful intuition.
%
Ultimately, however, we deploy a logical relations argument to prove the translation correct (\cref{section-seminaive-logical-relation}).
%
A reader who does not care for a categorical view and is prepared to jump ``in the deep end,'' therefore, may skim this section or jump straight to the definition of the semi\naive\ transformation itself in \cref{section-phi-delta}.

%% %% OLD INTRO
%% there are reasonable alternatives to the definition of \ChangePoset\ and the structures we are about to construct in it to interpret Datafun's semantics. Our eventual destination (\cref{section-phi-delta}) is a static transformation on Datafun source code.
%% %
%% This transformation was originally presented in \cite{seminaive-datafun}; it predates the construction of \ChangePoset\ presented here and is independent of it.
%% %
%% However, the transformation and the logical relation used to prove it correct (\cref{section-seminaive-logical-relation}) are fairly intricate.
%% %
%% Our aim in presenting \ChangePoset\ is to break the core concepts involved in these transformations down into small pieces, to show how this complexity arises and suggest potential alternatives for future investigation.
 
%% %% OLD FOOTNOTE
%% This is far from the only reasonable notion of change structure one might consider. For instance, \citet{DBLP:phd/dnb/Giarrusso20} defines both \emph{basic change structures} (definition 12.1.1), consisting only of a delta-set and a relation, and the more elaborate \emph{change structures} (definition 13.1.1) that have an update operator $\oplus : A \x \D A \to A$, a difference operator $\ominus : A \x A \to \D A$, and composition of changes $\circledcirc : \D A \x \D A \to \D A$; while \citet{mario-thesis} uses a definition based on monoid actions.
%% %
%% We will compare these with our approach in more detail in \cref{section-related-work-incremental-computation}.

%% However, change structures alone will not suffice to achieve our goal of speeding up fixed point computations, as we will discover in \cref{why-is-fix-discrete}.
%% %
%% Our eventual solution involves a pair of static transformations proved correct by a complex logical relation (\cref{section-phi-delta,section-seminaive-logical-relation}).
%% %
%% Explaining these transformations and logical relation without first developing familiarity with the simpler change structures which inspired them would, however, be difficult; hence this section.
%% %
%% But since these simpler change structures are not used in the logical relation or proof of correctness, our choice of properties to impose is somewhat arbitrary; we have chosen what seems the simplest definition which both supports the types of Datafun and provides appropriate intuition.%
%% %
%% %% For instance, if we drop our three properties and require only a poset $\D A$ and the relation $V_A$, this coincides with the \emph{basic change structures} defined by \citet[definition 12.1.1]{DBLP:phd/dnb/Giarrusso20}, save that we use posets where Giarrusso uses sets.
%% %% %
%% %% This definition is simple, but does not capture our intuition that all changes should be increasing (\emph{soundness}) or that a base point $x$ and a change $\dx$ uniquely determine the updated value $y$ (\emph{functionality}).
%% %
%% %% In the other direction, we could require more and insist on \emph{completeness}: if $x \le y : A$ then there exists some $\dx : \D A$ such that $\changesat A {\dx} x y$. Proving completeness of function spaces is problematic, however; we conjecture it can be done if we add even more structure, insisting that (1) completeness is \emph{monotone}, that is, there is a map $\name{diff} : \setfor{(x,y)}{x \le y : A} \to \D A$ such that $x_2 \le x_1 \wedge y_1 \le y_2 \implies \name{diff}(x_1, y_1) \le \name{diff}(x_2,y_2)$; and (2) there is a monotone change composition operator $\name{follow} : \D A \x \D A \to \D A$ such that if $\changesat A {\dx} x y$ and $\changesat A {\dy} y z$ then $\changesat A {\name{follow}(dx, dy)} x z$.
%% %% %
%% %% This begins to look like \todo{Giarrusso's ???}.

\todolater{address some alternative directions here: For instance, observe that derivatives are not necessarily unique. So then why differentiable rather than equipped with derivatives?

  1. Why differentiable rather than equipped with derivatives? A: we don't get proper sums with the latter, because we can't prove equality of the derivatives. In the end we will provide an explicit program transformation which constructs derivatives, which will follow the proof that the maps are differentiable closely.

  2. Are derivatives unique?

  2. We need derivatives to compute seminaive fixed points, but why do we need derivatives for all of Datafun? 3. Is finding derivatives enough?}

Recall that the structures we needed to interpret Datafun into the category \Poset\ in \cref{section-poset-structures,section-semantics} were: products, sums, exponentials, a discreteness comonad to interpret \iso, sets and semilattice objects, equality-test morphisms, and fixed points.
%
In \ChangePoset\ we will cover products, sums, exponentials, the discreteness comonad, and fixed points, as they are the most significant for understanding the broad structure of our approach. \fixme{now}{Is this still accurate?} \todomaybe{cover other structures briefly}{}


\subsection{Products}

%% In many cases the value-structure of \ChangePoset\ is ``inherited'' from \Poset. For instance, the value-poset of the product of two change structures, $V(A \x B)$, is the product of their value posets, $VA \x VB$; and the projection morphism $\pi_i : A_1 \x A_2 \to A_i : \ChangePoset$ is the \Poset-projection $\pi_i : V A_1 \x V A_2 \to V A_i : \Poset$. When this inheritance applies, we omit the definitions of $V A$ and morphisms of our structures, and only give the delta poset $\D A$, the update relation $R_A$, and show the inherited morphisms differentiable.

%% For instance, finite products and terminal objects inherit from \Poset, and their deltas are given component-wise:

%% \begin{center}
%%   \setlength\tabcolsep{10pt}
%%   %% \def\arraystretch{1.333}
%%   \begin{tabular}{c@{\qquad}c}
%%     $\D\tunit = \tunit$
%%     &
%%     \(\changesat{\tunit}{\tuple{}}{\tuple{}}{\tuple{}}\)
%%     \\[1.25ex]
%%     \(\D(A \x B) = \D A \x \D B\)
%%     &
%%     \(\infer[product change]{
%%       \changesat{A}{\da}{a}{a'}
%%       \\
%%       \changesat{B}{\db}{b}{b'}
%%     }{\changesat{A \x B}
%%       {\tuple{\da,\db}}
%%       {\tuple{a,b}}
%%       {\tuple{a',b'}}
%%     }\)
%%   \end{tabular}
%% \end{center}

Products and the terminal object in \ChangePoset\ mirror those in \Poset:

\begin{center}
  \def\arraystretch{1.333}
  \begin{tabular}{c@{\qquad\qquad}c}
    $V\terminalobject = \terminalobject$
    &
    \(V(A \x B) = VA \x VB\)
    \\
    $\D\terminalobject = \terminalobject$
    &
    \(\D(A \x B) = \D A \x \D B\)
    \\[1.25ex]
    \(\changesat{\terminalobject}{\tuple{}}{\tuple{}}{\tuple{}}\)
    &
    \(\infer[product change]{
      \changesat{A}{\da}{a}{a'}
      \\
      \changesat{B}{\db}{b}{b'}
    }{\changesat{A \x B}
      {\tuple{\da,\db}}
      {\tuple{a,b}}
      {\tuple{a',b'}}
    }\)
  \end{tabular}
\end{center}

\noindent
These satisfy functionality, soundness, and zero changes by invoking the corresponding properties at $A$ and $B$.
%
For instance, picking zero changes $\zero_a$, $\zero_b$ for $a,b$ respectively, \rn{product~change} tells us $(\zero_a, \zero_b)$ is a zero change to $(a,b)$.
%
%% For instance, $(a,b)$ has a zero change because $a$ must have a zero change $\changes{\da} a a$, likewise for $b, \db$, and \rn{product change} then tells us that $\changes{(\da,\db)}{(a,b)}{(a,b)}$.

Finally, the terminal map $\fork{}$, projection $\pi_i$, and tupling $\fork{f,g}$ (given $f : A \to B$ and $g : A \to C$) are all the same as in \Poset\ (thus inheriting the necessary universal properties), with derivatives given by:
%
\todo{give types of derivatives not just of morphisms?}

\begin{align*}
  \fork{} &: A \to \terminalobject
  &
  \fork{}' \<a \<\da &= \tuple{}
  \\
  \pi_i &: A_1 \x A_2 \to A_i
  &
  \pi_i' \<(x_1,x_2) \<(\dx_1,\dx_2) &= \dx_i
  \\
  \fork{f,g} &: A \to B \x C
  &
  \fork{f,g}' \<a \<\da &= \tuple{f'\<a\<\da, g'\<a\<\da}
\end{align*}

\noindent
The correctness of $\fork{}'$ is trivial; correctness of $\pi_i'$ follows by inversion of \rn{product change}; and $\fork{f,g}'$ is correct by \rn{product change} and correctness of $f',g'$.\footnotemark

\footnotetext{Note that had we chosen to let $\Delta(A \x B) = \Delta A + \Delta B$, representing a change to a tuple by a change to only one of its components, this would not allow us to differentiate tupling $\fork{f,g}$, since a change to the input may cause both components of the output to change simultaneously.}


\subsection{Sums}

Sums and the initial object also mirror those in \Poset:

\begin{center}
  \def\arraystretch{1.333}  
  \begin{tabular}{c@{\qquad\qquad}c}
    $V \initialobject = \initialobject$
    &
    $V (A + B) = V A + V B$
    \\
    $\Delta\initialobject = \initialobject$
    &
    $\Delta(A + B) = \Delta A + \Delta B$
    \\[1ex]
    $R_\initialobject = \emptyset$
    &
    \(
    \infer[sum change]{
      \changesat{A_i}{\dx}{x}{y}
    }{
      \changesat{A_1+A_2}{\inj i \dx}{\inj i x}{\inj i y}
    }
    \)
  \end{tabular}
\end{center}

\noindent
These satisfy functionality, soundness, and zero changes pretty straightforwardly using the corresponding properties at $A$ and $B$. For instance, $\inj i \zero_x$ is a zero change to $\inj i x$.

The initial map $\krof{}$, injection $\injc_i$, and case-analysis $\krof{f_1,f_2}$ (given $f_1 : A_1 \to C$, $f_2 : A_2 \to C$) are the same as in \Poset\ (inheriting its universal properties), with derivatives as follows:

\begin{align*}
  \krof{} &: \initialobject \to A
  &
  \krof{}' &= \krof{} \quad\text{(the domain is empty)}
  \\
  \injc_i &: A_i \to A_1 + A_2
  &
  \injc_i' \<x \<\dx &= \inj i \dx
  \\
  \krof{f,g} &: A_1 + A_2 \to C
  &
  \krof{f_1,f_2}' \<(\inj i x) \<(\inj j \dx) &=
  \begin{cases}
    f_i' \<x \<\dx & \text{if}~i = j\\
    %% f_i' \<x \<\text{(some zero change to $x$)} & \text{if}~ i \ne j
    \textsf{\itshape anything of type $\D C$} & \text{if}~ i \ne j
  \end{cases}
\end{align*}

\noindent
Correctness of $\krof{}'$ is vacuous; correctness of $\injc_i'$ follows directly from \rn{sum change}; but the definition of $\krof{f_1,f_2}'$ requires explanation.
%
If we take the proposition that $\krof{f_1,f_2}'$ is a derivative of $\krof{f_1,f_2}$ and apply the definition of $R_{A_1+A_2}$ (namely \rn{sum change}), we find that it simplifies to:

\[
\changesat{A_i}{\dx}{x}{y}
\implies
\changesat{C}{\krof{f_1,f_2}' \<(\inj i x) \<(\inj i \dx)}
          {f_i\<x}{f_i\<y}
\]

%% \[
%% \infer{
%%   \changesat{A_i}{\dx}{x}{y}
%% }{
%%   \changesat{C}{\krof{f_1,f_2}' \<(\inj i x) \<(\inj i \dx)}
%%             {f_i\<x}{f_i\<y}
%% }
%% \]

\noindent
This only constrains the behavior of $\krof{f_1,f_2}' \<(\inj i x) \<(\inj j \dx)$ when $i = j$; and in this case, we have $\changesat{C}{f_i'\<x\<\dx}{f_i\<x}{f_i\<y}$ as desired. Since the $i \ne j$ case is unconstrained, any value of type $\D C$ will suffice; all we need for differentiability is to show one exists, i.e. that $\D C$ is inhabited. Fortunately, in this case we have an $x : V A_i$ and a differentiable function $f_i : A_i \to C$. Applying \emph{zero changes} at $A_i$ we can pick a zero-change $\zero_x$ (although it being a zero-change is unnecessary; all we need is an element of $\D A_i$) and take $f_i' \<x \<\zero_x : \D C$. Or, we could use zero-changes at $C$ instead and take $\zero_{(f_i\<x)} : \D C$.

%% apply \emph{zero changes} at $A_i$ and pick some $\dy : \D A_i$ such that $\changesat{A_i}{\dy}{x}{x}$ (although this property is unnecessary; all we need is a value $\D A_i$), and then we have $f_i' \<x\<\dy : \D C$.

This $i \ne j$ case is related to the \emph{partiality} of the validity relation: $\inj 1 \dx$ is never a valid change to $\inj 2 x$.
%
This is hard to avoid given our definition of change structures: to differentiate $\injc_i$ and $\krof{f_1,f_2}$ we need $\D(A_1 + A_2)$ to include both $\D A_1$ and $\D A_2$ somehow; and a change $\dx \in \D A_1$ has no natural meaning applied to a value $x \in V A_2$.
%
Furthermore, the fact that the $i \ne j$ case is unconstrained -- essentially ``dead code'' -- means that if we had defined \ChangePoset\ morphisms as maps \emph{equipped with a particular derivative} (rather than merely differentiable) we would be unable to prove the uniqueness of $\krof{f,g}'$ required by the universal property for sums.%
%
\footnote{We could avoid partiality by defining $\changesat{A_1+A_2}{\inj i \dx}{\inj j x}{\inj j x}$ for $i \ne j$; that is, treating currently ``invalid'' changes as zero-changes. This unfortunately doesn't extend to the function case, which as we'll see shortly also needs a partial validity relation. Moreover, it doesn't ensure uniqueness of $\krof{f_1,f_2}'$: although it requires $\krof{f_1,f_2}'\<(\inj i x)\<(\inj j \dx)$ to be a zero change to $f_i\<x$ when $i \ne j$, there may be multiple such zero changes.}

This could, for instance, be addressed by changing the definition of \ChangePoset\ to only require derivatives to be defined for valid changes. We don't do this because from a type-theoretic perspective, this requires a dependent or refinement type, while we want the types of our derivatives to be simple so our category corresponds closely with a static transformation on Datafun, a simply-typed language.


\subsection{Exponentials}

The values of the exponentials in \ChangePoset\ capture differentiable, monotone maps:

\begin{align*}
  V(A \expto B) &= \text{differentiable monotone maps $VA \to VB$, ordered pointwise}
  \\
  &= (\ChangePoset(A, B),\, \{(f,g) : \fa{x} f\<x \le g\<x\})
\end{align*}

%% \begin{align*}
%%   V(A \expto B) &= \text{differentiable monotone maps $VA \to VB$, ordered pointwise}
%%   \\
%%   &= (\ChangePoset(A, B),\, \{(f,g) : \fa{x} f\<x \le g\<x\})
%%   \\[1ex]
%%   \Delta(A \expto B)
%%   &= \iso VA \expto (\D A \expto \D B)
%%   \\[1ex]
%%   \changesat{A \expto B}{\df}{f}{g}
%%   &\iff
%%   \fa{\changesat{A}{\dx}{x}{y}}
%%   \changesat{B}{\df\<x\<\dx}{f\<x}{g\<y}
%% \end{align*}

\newcommand\curried[1]{\lambda{#1}}
%\renewcommand\curried[1]{\lambda_{#1}}

\noindent
We might expect changes $\D(A \expto B)$ to be given pointwise, as (not necessarily monotone) functions $VA \to \D B$ mapping each input to the change in the corresponding output:

%% \begin{align*}
%%   \D(A \expto B) &= \iso VA \expto (\D A \expto \D B)
%%   &
%%   \infer[fn~change]{
%%     \fa{\changesat A \dx x y}
%%     \changesat B {\df\<x\<\dx} {f\<x} {g\<y}
%%   }{
%%     \changesat{A \to B}{\df}{f}{g}
%%   }
%% \end{align*}

\colorlet{wrong}{rgb,255:red,214;green,92;blue,92}
\colorlet{wrong}{Red}
\definecolor{wrong}{cmyk}{0, 0.8, 0.8, 0.25}
{\color{wrong}
\begin{align*}
  \D(A \expto B) &= \iso V A \expto \D B
  &
  \infer{
    \fa{x} \changesat{B}{\df\<x}{f\<x}{g\<x}
  }{
    \changesat{A \expto B}{\df}{f}{g}
  }
  &&\text{\scshape\ding{55}\: not an exponential}
\end{align*}}

\noindent
However, this choice makes it difficult to differentiate function application.
%
The function application map $\eval : (A \expto B) \x A \to B$ is, of course, given by $\eval\<(f,x) = f\<x$.
%
%% A derivative $\eval'$ would be a function such that $\eval' \<(f,x) \<(\df,\dx)$ tells us how $f\<x$ changes as both $f$ and $x$ change.
To differentiate this is to ask for some $\eval'\<(f,x) \<(\df,\dx)$ that captures how $f\<x$ changes as both $f$ and $x$ change simultaneously:
%
supposing $\changes{\df}{f}{g}$ and $\changes{\dx}{x}{y}$, how do we find a change $f\<x \changesto g\<y$?

Using a pointwise change $\df : VA \to \D B$, we can find $\changes{\df\<x}{f\<x}{g\<x}$; and applying differentiability of $f$ we can find some $\changes{f'\<x\<\dx}{f\<x}{f\<y}$.
%
The former handles a change to the function, the latter a change to the argument.
%
These form two sides of a ``square of changes'':

% https://q.uiver.app/?q=WzAsNCxbMCwwLCJmXFw7eCJdLFszLDAsImZcXDt5Il0sWzAsMywiZ1xcO3giXSxbMywzLCJnXFw7eSJdLFswLDIsImRmXFw7eCIsMSx7InN0eWxlIjp7InRhaWwiOnsibmFtZSI6Imhvb2siLCJzaWRlIjoiYm90dG9tIn19fV0sWzAsMSwiZidcXDt4XFw7ZHgiLDEseyJzdHlsZSI6eyJ0YWlsIjp7Im5hbWUiOiJob29rIiwic2lkZSI6ImJvdHRvbSJ9fX1dLFsyLDMsImcnXFw7eFxcO2R4IiwxLHsic3R5bGUiOnsidGFpbCI6eyJuYW1lIjoiaG9vayIsInNpZGUiOiJib3R0b20ifX19XSxbMSwzLCJkZlxcO3kiLDEseyJzdHlsZSI6eyJ0YWlsIjp7Im5hbWUiOiJob29rIiwic2lkZSI6ImJvdHRvbSJ9fX1dLFswLDMsIj8iLDEseyJjb2xvdXIiOlswLDYwLDYwXSwic3R5bGUiOnsiYm9keSI6eyJuYW1lIjoiZG90dGVkIn19fSxbMCw2MCw2MCwxXV1d
\[\begin{tikzcd}[every label/.append style={font=\footnotesize}]
	{f\<x} &&& {f\<y} \\
    \\
	\\
	{g\<x} &&& {g\<y}
	\arrow["{\color{wrong}\df\<x}"{description}, hook, from=1-1, to=4-1]
	\arrow["{f'\<x\<\dx}"{description}, hook, from=1-1, to=1-4]
	\arrow["{g'\<x\<\dx}"{description}, hook, from=4-1, to=4-4]
	\arrow["{\color{wrong}\df\<y}"{description}, hook, from=1-4, to=4-4]
	\arrow["{?}"{description},
      %color={wrong}, color=Blue,
      dotted, from=1-1, to=4-4]
\end{tikzcd}\]

\noindent
We need the \emph{diagonal} of this square.
%
One approach would be to use pointwise function changes but augment our definition of change structures to allow \emph{composing} changes, and find the diagonal by composing sides.
%
Unfortunately, this is more difficult than it appears: $\eval'$ is applied to $f,x,\df,\dx$, but to compute $\df\<y$ or $g'\<x\<\dx$ we need either $y$ or $g$.
%
This seems to require equipping change structures with an operator $\oplus_A : \iso VA \x \D A \to VA$ that extends the validity relation $R_A$ from a partial to a \emph{total} function (since $\eval'$ is defined for all inputs, not merely valid ones); then we can recover $y = x \oplus \dx$ or $g = f \oplus \df$.
%
But $\oplus_{A \to B}$ is difficult to construct, because we must guarantee that $f \oplus \df$ is a \emph{monotone} function, no matter the value of $\df : \iso V A \to \D B$; it is easy to come up with a $\df$ such that $\fnof{x} f\<x \oplus \df\<x$ (the natural definition of $f\oplus\df$) is non-monotone.\footnote{We further discuss the issue of monotonicity and pointwise changes in \cref{pointwise-changes-monotonicity}.}

Perhaps there is some way through these difficulties; fortunately, there is a simple approach that side-steps them entirely: following the original incremental \fn-calculus~\citep{incremental} we require function changes to produce the diagonal \emph{directly}.
%
Since this diagonal depends on the change $\dx$ to the argument, function changes $\df$ become two-argument functions:

\begin{align*}
  \D(A \expto B) &= \iso VA \expto (\D A \expto \D B)
  &
  \infer[fn~change]{
    \fa{\changesat A \dx x y}
    \changesat B {\df\<x\<\dx} {f\<x} {g\<y}
  }{
    \changesat{A \to B}{\df}{f}{g}
  }
\end{align*}

\noindent
With this definition, function changes are exactly what is needed to incrementalize function application $f\<x$. A change to a function $\changes{\df}{f}{g}$ accepts a change in its argument $\changes{\dx}{x}{y}$ and produces the the change in its output, $\changes{\df\<x\<\dx}{f\<x}{g\<y}$. If we return to our square of changes, we find it now has a zig-zag shape, with the diagonal filled in but missing the vertical sides:

\[\begin{tikzcd}[every label/.append style={font=\footnotesize}]
	{f\<x} &&& {f\<y} \\
    \\
	\\
	{g\<x} &&& {g\<y}
	\arrow["{f'\<x\<\dx}"{description}, hook, from=1-1, to=1-4]
	\arrow["{g'\<x\<\dx}"{description}, hook, from=4-1, to=4-4]
	\arrow["{\df\<x\<\dx}"{description},
      %color={wrong}, color=Blue,
      hook, from=1-1, to=4-4]
    %% \arrow["{\df\<x\<\zero_x}"{description}, hook,from=1-1,to=4-1]
    %% \arrow["{\df\<y\<\zero_y}"{description}, hook,from=1-4,to=4-4]
\end{tikzcd}\]

\noindent
To recover the missing sides, we can apply $\df$ to zero-changes $\zero_x$ and $\zero_y$ instead of $\dx$:

\[\begin{tikzcd}[every label/.append style={font=\footnotesize}]
	{f\<x} &&& {f\<y} \\
    \\
	\\
	{g\<x} &&& {g\<y}
	\arrow["{f'\<x\<\dx}"{description}, hook, from=1-1, to=1-4]
	\arrow["{g'\<x\<\dx}"{description}, hook, from=4-1, to=4-4]
	\arrow["{\df\<x\<\dx}"{description},
      %color={wrong}, color=Blue,
      hook, from=1-1, to=4-4]
    \arrow["{\df\<x\<\zero_x}"{description}, hook,from=1-1,to=4-1]
    \arrow["{\df\<y\<\zero_y}"{description}, hook,from=1-4,to=4-4]
\end{tikzcd}\]

\noindent
Zero changes thus let us recover a pointwise change $\fnof{x} \df\<x\<\zero_x$ from any $\changesat{A \expto B}{\df} f g$.

Note also the mixture of monotonicity and non-monotonicity in $\iso V A \expto
\D A \expto \D B$.
%
Since our functions are monotone (increasing inputs yield increasing outputs),
we expect function changes to be monotone with respect to input changes $\D A$:
a larger increase in the input yields a larger increase in the output.
%
However, there's no reason to expect the change in the output to grow as the
base point increases -- hence the first argument is discrete, $\iso V A$.

Although this nicely solves the problem of differentiating $\eval$, it is not immediately obvious that $R_{A \expto B}$ is functional, sound, and possesses zero-changes.\footnotemark\
%
The first two are quite similar, so we'll tackle them together:
%
Suppose $\changesat{A \expto B}{\df}{f}{g}$ and likewise $\changes{\df} f h$ and fix some $x \in VA$. For functionality, we wish to show $g\<x = h\<x$; for soundness, we wish to show $f\<x \le g\<x$.
%
By zero changes at $A$ we can pick some $\changes{\zero_x} x x$.
%
Inverting \rn{fn~change} we have $\changesat{B}{\df\<x\<\zero_x}{f\<x}{g\<x}$ and likewise $\changes{\df\<x\<\zero_x}{f\<x}{h\<x}$.
%
Then by functionality at $B$ we have $g\<x = h\<x$; and by soundness at $B$ we have $f\<x \le g\<x$.

Showing \emph{zero changes} is simple but illuminating.
%
By definition, every $f : V(A \expto B)$ is differentiable, and a derivative $f'$ of $f$ is exactly a zero change $\changesat{A \expto B}{f'}{f}{f}$:

\begin{align*}
  &\changesat{A \to B}{\df}{f}{f}\\
  \iff&
  \fa{\changesat A \dx x y}
  \changesat B {\df\<x\<\dx} {f\<x} {f\<y}
  &&\text{\rn{fn~change}}
  \\
  \iff& \df\text{ is a derivative of }f
  &&\text{\cref{definition-derivative}}
\end{align*}

\noindent
This happens because we've defined function changes $\changesat{A \expto B}{\df} f g$ to tell us how function application responds to changes in both the function and its argument. If the function \emph{doesn't} change, this reduces to how the function's output changes as its argument changes: exactly what a derivative does.

\footnotetext{We also promised in a footnote on \cpageref{footnote-completeness} to show that completeness was problematic at function types / exponentials. In other words, supposing $f \le g : A \expto B$, why can't we find some $\changes{\df} f g$? Well, we would need $\changesat{B}{\df\<x\<\dx}{f\<x}{g\<y}$ whenever $\changes{\dx} x y$. If we inductively suppose completeness at $B$, we could pick such a change, since by monotonicity we can show $f\<x \le g\<y$. Of course, we need $\df$ to be defined over \emph{all} $x,\dx$, not merely valid ones, but this is nothing the axiom of choice can't handle. More problematic is that we need $\df\<x\<\dx$ to be \emph{monotone} with respect to $\dx$. So merely picking changes is not enough; we have to pick them in a way that preserves monotonicity.

  We conjecture this can be done by strengthening change structures to include (1) monotone completeness and (2) monotone change composition. \emph{Monotone completeness} strengthens completeness by requiring an operator $y \ominus_A x$ defined for $y \ge x : VA$ such that $\changesat{A}{y \ominus_A x} x y$ and which is monotone in $y$ and anti-monotone in $x$, that is, $x' \le x \wedge y \le y' \implies y \ominus x \le y' \ominus x'$. \emph{Monotone change composition} requires a monotone operator $\ocircle_A : \D A \x \D A \to \D A$ such that if $\changes{\dx}{x}{y}$ and $\changes{\dy} y z$ then $\changes{\dy \ocircle \dx}{x}{z}$. Then we can define $g \ominus_{A \expto B} f = \fnof x \fnof{\dx} g'\<x\<\dx \ocircle_B (g\<x \ominus_B f\<x)$ and $\dg \ocircle_{A \expto B} \df = \fnof x \fnof{\dx} \dg \<x\<\dx \ocircle \df\<x\<\zero_x$.

  We have not done this because it considerably complicates the definition of change structures and does not help explain any features of the translation given in \cref{section-phi-delta}, but it might be an interesting direction for future work.
}

Finally, to make $A \expto B$ an exponential we need function application $\eval_{A,B} : (A \expto B) \x A \to B$ (which we have already discussed), and for any $f : C \x A \to B$, its currying $\curried{f} : C \to A \expto B$.
%
%% Finally, to be an exponential object we need function application $\eval_{A,B} : (A \expto B) \x A \to B$, which we have already discussed, and its counterpart, currying: given $f : C \x A \to B$ we must define $\curried{f} : C \to A \expto B$.
%
These are defined as in \Poset, which ensures their universal property holds; but since $V(A \expto B)$ contains only \emph{differentiable} maps, besides $\eval$ and $\curried{f}$ we also require $(\curried{f}\<c)$ to be differentiable:

%% \begin{align*}
%%   \eval\<(f,x) &= f\<x\\
%%   \eval'\<(f,x) \<(\df,\dx) &= \df\<x\<\dx
%%   \\[1ex]
%%   \curried{f} \<\gamma \<x &= f\<(\gamma,x)
%%   \\
%%   (\curried{f} \<\gamma)' \<x \<\dx &=
%%   f' \<(\gamma,x) \<(\textsf{\itshape pick a zero change to }\gamma, \dx)
%%   \\
%%   (\curried f)' \<\gamma \<\dgamma \<x \<\dx &= f' \<(\gamma,x) \<(\dgamma,\dx)
%% \end{align*}

\begin{align*}
  \eval\<(f,x) &= f\<x
  &
  \curried{f} \<c \<x &= f\<(c,x)
  \\
  \eval'\<(f,a) \<(\df,\da) &= \df\<a\<\da
  &
  (\curried f)' \<c \<\dc \<a \<\da &= f' \<(c,a) \<(\dc,\da)
  \\
  &&
  (\curried{f} \<c)' \<a \<\da &=
  f' \<(c,a) \<(\zero_c, \da)
\end{align*}

\noindent
We've already seen how $\eval'$s correctness follows from \rn{fn~change}. Applying $R_{A \expto B}$ and $R_{C \x A}$, we find that $(\curried{f})'$ is a derivative for $\curried{f}$ when $f'$ is a derivative for $f$:

\begin{align*}
  & (\curried f)'\text{ is a derivative of }\curried{f}
  \\
  \iff&
  \fa{\changesat{C}{\dc}{c}{c'}}
  \changesat{A \expto B}{(\curried f)'\<c\<\dc}{\curried{f}\<c}{\curried{f}\<c'}
  \\
  \iff&
  \fa{\changesat{C}{\dc}{c}{c'}}
  \fa{\changesat{A}{\da}{a}{a'}}
  \changesat{B}{(\curried f)'\<c\<\dc\<a\<\da}
            {\curried{f}\<c\<a}
            {\curried{f}\<c'\<a'}
  \\
  \iff&
  \fa{\,\changesat{C \x A}{(\dc,\da)}{(c,a)}{(c',a')}}
  \changesat{B}{f'\<(c,a)\<(\dc,\da)}{f\<(c,a)}{f\<(c',a')}
  \\
  \iff&
  f'\text{ is a derivative of }f
\end{align*}

\noindent
Finally, the correctness of $(\curried f \<c)'$ follows from that of $f'$ by applying $\changesat{C}{\zero_c}{c}{c}$:

\begin{align*}
  &\text{$f'$ is a derivative of $f$ and $\zero_c$ is a zero change to $c$}
  \\
  \implies&
  \fa{\changesat{A}{\da} a {a'}}
  \changesat{B}{f'\<(c,a)\<(\zero_c,\da)}{f\<(c,a)}{f\<(c,a')}
  \\
  \iff&
  \fa{\changesat{A}{\da}{a}{a'}}
  \changesat{B}{(\curried{f} \<c)' \<a\<\da}
            {(\curried{f} \<c) \<a}
            {(\curried{f} \<c) \<a'}
  \\
  \iff&
  (\curried{f} \<c)'\text{ is a derivative of }(\curried{f} \<c)
\end{align*}


%% To explain these definitions, it may help to begin with why the more ``obvious'' definitions do not work. Following the product and sum examples, we might expect the value-poset of the exponential object $V(A \expto B)$ to be the exponential in \Poset, $VA \expto VB$. Moreover we might expect changes $\D(A \expto B)$ to be given pointwise, as functions $V A \to \D B$ mapping each input to the change in the corresponding output. These choices do not give us exponential objects, however, for interconnected reasons, beginning with differentiability of function application -- that is, the map $\eval : (A \expto B) \x A \to B$.

%% \[ \eval \<(f, x) = f(x) \]

%% However, we must restrict this poset to only include \emph{differentiable} maps. Although this makes sense considering \ChangePoset-morphisms are differentiable, it actually arises for complex reasons stemming from the definition of $\D(A \expto B)$.


\subsection{Semilattice change structures and semi\naive\ fixed points}

We've already been introduced to the finite powerset change structure, as our introductory example in \cref{example-finite-set-change-structure}.
%
But to define it properly as a functor $\pfin : \ChangePoset \to \ChangePoset$, inheriting from the corresponding $\pfin$ on \Poset:

\begin{align*}
  V \pfinof A &= \pfinof V A
  &
  \Delta \pfinof A &= \pfinof VA
  &
  \changesat{\pfinof A}{\dx} x y &\iff x \cup \dx = y
\end{align*}

\noindent
%% This plainly satisfies functionality ($\cup$ is a function), soundness (for any $x \le y$, we have $x \cup y = y$, for instance), and zero-changes (for any $x$ we have $x \cup \emptyset = x$).
%
%% We also need derivatives for $\morph{singleton}$ and \morph{isEmpty}, but the types of these morphisms involve the discreteness comonad $\iso$; we'll return to them after we define what $\iso$ means in \ChangePoset.
%
This finite powerset change structure forms the prototype for our change structures for semilattices in general, which we need to support various language features, most importantly fixed points.
%
We saw in \cref{section-seminaive-incremental,section-change-structures,example-step-prime-is-a-derivative} that given a function $f : \pfinof A \to \pfinof A$ and a derivative for it $f' : \iso \pfinof A \to \pfinof A \to \pfinof A$ we can compute its fixed point semi\naive{}ly as follows:

\begin{align*}
  x_0 &= \emptyset & x_{i+1} &= x_i \cup \dx_i\\
  \dx_0 &= f\<\emptyset & \dx_{i+1} &= f'\<x_i\<\dx_i
\end{align*}

\newcommand\semichange[1]{\name{Semi}\,#1}

\noindent
This takes advantage of the fact that the change to a set is another set, and we apply a change using set union/semilattice join. Following this pattern, we can endow any semilattice $L : \Poset$ with a similar change structure:

\begin{align*}
  V L &= L
  &
  \D L &= L
  &
  \changesat{L}{\dx} x y &\iff x \vee \dx = y
\end{align*}

\noindent
This satisfies functionality ($\vee$ is a function), soundness ($x \le x \vee y$), and zero-changes ($x \vee \bot = x$).
%
Let's call this the \emph{semilattice change structure} on $L$.
%
By construction, the finite powerset change structure $PA$ is the semilattice change structure on $PVA$;
%
and our semi\naive\ fixed point strategy generalizes to any semilattice change structure:

\begin{definition}[\semifix]\label{definition-semifix}
  Given a semilattice $L$ with no infinite ascending chains and monotone maps $f : L \to L$ and $f' : \iso L \to L \to L$, let $\semifix_L\<(f,f') = \bigvee_i x_i$ be the limit of the ascending chain defined by:
  
  \begin{align*}
    x_0 &= \bot & x_{i+1} &= x_i \vee \dx_i\\
    \dx_0 &= f\<\bot & \dx_{i+1} &= f'\<x_i\<\dx_i
  \end{align*}
\end{definition}

\begin{theorem} \label{theorem-seminaive-fixed-points}
  $\semifix_L\<(f,f')$ is the least fixed point of $f$ if $f'$ is a derivative of $f$.
\end{theorem}

\begin{proof}
  It suffices to show inductively that $x_{i+1} = f\<x_i$; from this it follows that $x_i = f^i \<\bot$, as in the \naive\ approach to computing a fixed point. We prove this with essentially the same argument used in \cref{proof-seminaive-step-works} (\cpageref{proof-seminaive-step-works}). The base case is $x_1 = x_0 \vee \dx_0 = \bot \vee f\<\bot = f\<\bot = f\<x_0$, and the inductive case is:

  \begin{align*}
    x_{i+2} &= x_{i+1} \vee \dx_{i+1}
    && \text{definition of }x_{i+2}
    \\
    &= f\<x_i \vee f' \<x_i \<\dx_i
    && \text{inductive hypothesis, definition of }\dx_{i+1}
    \\
    &= f\<(x_i \vee \dx_i)
    && f'~\text{is a derivative of}~f\\
    &= f\<x_{i+1}
    && \text{definition of }x_{i+1}
  \end{align*}
\end{proof}


\subsection{Fixed points and discreteness comonads}

\Cref{theorem-seminaive-fixed-points} shows we can speed up fixed points by exploiting the power of derivatives.
%
It may seem as though this justifies a morphism $\morph{fix} : (L \expto L) \to L : \ChangePoset$ for any semilattice change structure $L$ satisfying ACC.
%
However, morphisms in \ChangePoset\ must be differentiable: does $\morph{fix}$ have a derivative? Prior work~\citep{delta-fix,DBLP:conf/esop/Alvarez-Picallo19} has answered this affirmatively. One solution is to find the \emph{fixed point of the function change:}

\[
\morph{fix}' \<f \<\df = \morph{fix} \<(\df \<(\morph{fix} \<f))
\]

\noindent
How and why this works is non-obvious; we refer the reader to \cite{delta-fix} for a full explanation.
%
So there is indeed a morphism $\morph{fix} : (L \expto L) \to L : \ChangePoset$. However, from the perspective of our original goal of speeding up fixed point computations, the derivative of this morphism presents two issues.
%
First, it isn't actually incremental: computing $\morph{fix}'\<f\<\df$ using this derivative requires re-computing $\morph{fix}\<f$ as an argument to $\df$!\footnotemark\
%
Second, we would naturally like to compute $\morph{fix}\<(\df\<(\morph{fix} \<f))$ \emph{semi\naive{}ly,} but we have no guarantee that $(\df \<(\morph{fix}\<f))$ is differentiable!
%
This requires a higher-order derivative; a coherent theory of higher-order derivatives and higher-order change structures would be enormously interesting, but we leave it to future work.

\footnotetext{The need to recompute $\morph{fix}\<f$ could likely be solved by caching intermediate values, which we discuss further in \cref{section-caching}. Somewhat unusually, in this case we want to cache the previous output of an operation rather than its previous input.}

Instead, we deliberately limit the scope of our approach to avoid the need to incrementally maintain fixed points. As we've already seen\todo{bwd ref}, in Datafun \prim{fix} is not treated as a monotone operator; correspondingly the morphisms we require to interpret it are not $\morph{fix} : (L \expto L) \to L$ but rather $\morph{fix} : {\color{IsoRed} \iso}(L \expto L) \to L$. The idea here is that, just as $\iso$ in \Poset\ captures \emph{non-monotonicity} in an otherwise monotone world, in \ChangePoset\ we can use it to capture \emph{non-differentiability} or \emph{non-incrementalizability} in an otherwise differentiable world.

\newcommand\isotriv{\ensuremath{\iso_{\textsf{triv}}}}

More concretely, since we only consider increasing changes and $\iso A$ is ordered discretely, $x \le y : \iso A \iff x = y$, the only possible ``change'' is to stay the same. We can thus extend the discreteness comonad $\iso$ on $\Poset$ to a comonad $\iso$ on $\ChangePoset$ by letting the space of changes be trivial:

\begin{align*}
  V \iso A &= \iso V A
  &
  \Delta\iso A &= \terminalobject
  &
  %% \fa{a : VA} 
  \changesat{\iso A}{()} a a
\end{align*}

\noindent
This straightforwardly satisfies functionality, soundness, and zero changes. Moreover, it inherits the monoidal comonad structure of $\iso$ from \Poset. Fixing some map $f : A \to B$, the derivatives of functorial action, extraction, duplication, and distribution are mostly trivial:

\begin{align*}
  \iso(f) &\isa \iso A \to \iso B
  &
  \iso(f)' \<x \<\tuple{} &= \tuple{}
  \\
  \varepsilon_A &\isa \iso A \to A
  &
  \varepsilon_A' \<x \<\tuple{} &= \zero_x
  \\
  \delta_A &\isa \iso A \to \iso\iso A
  &
  \delta_A' \<x \<\tuple{} &= \tuple{}
  \\
  \isox &\isa \textstyle\prod_i \iso A_i \to \iso\prod_i A_i
  &
  {\isox}' \<x \<\dx &= \tuple{}
  \\
  \isosum &\isa \textstyle\iso\sum_i A_i \to \sum_i \iso A_i
  &
  {\isosum}' \<(\inj i x) \<\tuple{} &= \inj i ()
\end{align*}

\noindent
Finally, observe that any monotone map $f : V\iso A \to VB$ is trivially differentiable by letting $f' \<x \<() = \zero_{(f\<x)}$, confirming our intuition that differentiable maps $\iso A \to B$ should coincide with not-necessarily-differentiable maps $A \to B$.

\todo{exposition linking to the next section; maybe say this doesn't quite match our translation (because we need to concretely find derivative of functions?). or, adjust exposition in the intro to the next section to talk about ChangePoset.}


%% Perhaps surprisingly, $\isotriv$ is not the only monoidal comonad in \ChangePoset\ capable of playing the role of $\iso$ in Datafun's semantics.
%% %
%% Nor is it the one which most closely resembles the static transformation on Datafun code which we will introduce in \cref{section-phi-delta}.
%% %
%% This is primarily an accident of history: the static transformation was discovered first and first presented in \cref{seminaive-datafun}; this categorical account came later, and a revised transformation based on $\isotriv$ might be a fruitful avenue for future work.

%% \newcommand\isodelta{\ensuremath{\iso_{\Delta}}}

%% However, there is at least one reason to consider an alternate interpretation of $\iso$.
%% %
%% The semi\naive\ approach to computing a functions fixed point requires a derivative of it; using this to define the morphism $\morph{fix} : \isotriv(L \expto L) \to L$ relies on the fact that $V(L \expto L)$ contains only differentiable maps.
%% %
%% Exploiting this differentiability constructively, as in a code transformation, would require us to annotate \emph{every} function with a derivative, even if it is not used in a fixed point.\footnotemark\
%% %
%% To avoid this, we can instead exploit the fact that the argument to \morph{fix} is boxed, and creatively reinterpret $\iso A$ to consist of values equipped with zero changes.
%% %
%% Then $\iso(L \expto L)$ will consist of functions equipped with zero changes; and as we've already seen, a zero change to a function is a derivative of it.
%% %
%% So instead of $\isotriv$, consider the comonad $\isodelta$ defined by:

%% \footnotetext{It must be admitted that this reason is rather weak, because a perfectly standard dead code elimination pass would suffice to remove these unnecessary derivatives. Another way of looking at our definition of $\isodelta$ is as a type-directed ``unused derivative elimination pass''.}

%% \begin{align*}
%%   V\isodelta A
%%   &= \text{pairs $(x,\dx)$ such that $\changesat{A}{\dx} x x$, ordered discretely}
%%   \\
%%   \Delta\isodelta A &= \terminalobject
%%   \\
%%   R_{\isodelta A} &= \{((), (x,\dx), (x,\dx))
%%   \mathrel{\text{such that}}
%%   (x,\dx) \in V\isodelta A\}
%%   %% \changesat{\isodelta A}{()}{(x,\dx)}{(x,\dx)}
%% \end{align*}

%% \noindent
%% Functionality, soundness, and zero changes hold straightforwardly. Fixing some $f : A \to B : \ChangePoset$, the comonad structure is as follows:

%% \todo{we can't use $f'$ in the definition of $\isodelta(f)$ because it's not unique. and we need functoriality to preserve id \& associativity. AAAAAAAAAARGH.}

%% \begin{align*}
%%   \isodelta(f) &\isa \isodelta A \to \isodelta B
%%   &
%%   \isodelta(f) \<(x, \dx) &= \color{Red}(f\<x, f'\<x\<\dx)
%%   \XXX
%%   &
%%   \isodelta(f)' \<(x, \dx) \<() &= ()
%%   \\
%%   \varepsilon_A &\isa \isodelta A \to A
%%   &
%%   \varepsilon_A \<(x, \dx) &= x
%%   &
%%   \varepsilon_A' \<(x, \dx) \<() &= \dx
%%   \\
%%   \delta_A &\isa \isodelta A \to \isodelta\isodelta A
%%   &
%%   \delta_A \<(x, \dx) &= ((x, \dx), ())
%%   &
%%   \delta_A' \<(x, \dx) \<() &= ()
%% \end{align*}

%% \noindent
%% The correctness of these derivatives is mostly trivial, except that $\varepsilon_A' \<(x,\dx)$ uses the fact that $\dx$ is a zero change to $x$.
%% %
%% The comonad laws require that $\varepsilon \compose \delta = \isodelta(\varepsilon) \compose \delta = \id$ and that $\delta \compose \delta = \isodelta(\delta) \compose \delta$; these are easily verified by direct computation.
%% %
%% Distribution over products and sums is as follows:

%% \begin{align*}
%%   \isox &\isa \prod_i\isodelta A_i \to \isodelta\prod_i A_i
%%   &
%%   \isosum &\isa \isodelta\sum_i A_i \to \sum_i\isodelta A_i
%%   \\
%%   \isox \<((x_i,\dx_i))_i &= ((x_i)_i, (\dx_i)_i)
%%   &
%%   \isosum \<(\inj i x, \inj i \dx) &= \inj i (x,\dx)
%%   \\
%%   {\isox}' \<\pwild \<\pwild &= ()
%%   &
%%   {\isosum}' \<(\inj i x, \pwild) \<() &= \inj i ()
%% \end{align*}

%% \noindent
%% Finally, we can put all this machinery to work in the definition of $\morph{fix}_L : \isodelta(L \expto L) \rightarrow L$ for any semilattice change structure $L$ via semi\naive\ iteration:

%% \begin{align*}
%%   \morph{fix}_L\<(f,f') &= \semifix\<(f,f')
%%   &
%%   \morph{fix}_L'\<(f,f')\<() &= \bot
%% \end{align*}

%% \noindent
%% Since the input is not allowed to change, the derivative $\morph{fix}'$ needs to compute a zero change; fortunately, $\bot$ is a universal zero change for any semilattice change structure.
%% %
%% Observe that this definition does not use the fact that the exponential object $V(L \expto L)$ contains only differentiable functions; instead it uses the explicitly-supplied zero change $f'$.

%% There is, however, a problem with $\isodelta$. Our end-goal is a source-to-source translation on Datafun code, but the poset $V\isodelta A$ is not definable in Datafun, because Datafun is simply typed: assuming we have types corresponding to $VA$ and $\D A$, Datafun can express $\iso(VA \times \D A)$, but cannot ``filter this down'' to only pairs $(x,\dx)$ such that $\changes{\dx} x x$.
%% %
%% Our solution is to approximate: our translation produces a type corresponding to $\iso(VA \times \D A)$, and we deploy a logical relations argument to fix up the difference.\footnotemark

%% \footnotetext{Unfortunately, this approximation cannot be made to work in \ChangePoset\ itself; if we let $V\isodelta A = VA \x \D A$ directly, then $\varepsilon_A' \<(x, \dx) = $}

%% We will dive into the definition of this translation in the next section, but first, there are a few remaining structures necessary to interpret Datafun into \ChangePoset.

%% \todo{explain (5) that this isn't a simply-typed thing anymore, alas, but (6) we can't let $V\isodelta A = VA \x \D A$ because extract must use zero and then we disobey comonad laws}


%% \subsection{Remaining structures}

%% \XXX

%% This leaves us with join, singleton, isEmpty, collect, and eq, which both use the discreteness comonad $\iso$


%% %% ---------- EVEN OLDER STUFF ----------
%% \subsection{The box comonad and fixed points}

%% Almost all of the remaining structures we need to interpret Datafun's semantics into \ChangePoset\ depend crucially on how we choose to interpret one particular type: the discreteness comonad \iso. For instance, to interpret set types $\tseteq A$ we need a finite powerset $\pfin$ equipped with morphisms $\morph{singleton}_A : \iso A \to \pfinof A$ and $\morph{isEmpty}_A : \iso P A \to \terminalobject + \terminalobject$; to interpret equality we need morphisms $\morph{eq}_A : \iso A \x \iso A \to \pfinof \terminalobject$; and to interpret $\prim{fix}$ we need morphisms $\morph{fix}_L : \iso(L \expto L) \to L$ for semilattice objects $L$.
%% %
%% Perhaps surprisingly, there are several distinct monoidal comonads in \ChangePoset\ which can support all these structures.
%% %
%% We will first consider the simplest of these before explaining why, ultimately, we choose a different interpretation, for reasons to do with fixed points.

%% Since we only wish to capture increasing changes and the type $\iso A$ is ordered discretely, $x \le y : \iso A \iff x = y$, the only possible ``change'' is to stay the same. The comonad $\isotriv$ that most straightforwardly represents this simply lets the space of changes be trivial:

%% \begin{align*}
%%   V \isotriv A &= \iso V A
%%   &
%%   \Delta\isotriv A &= \terminalobject
%%   &
%%   %% \fa{a : VA} 
%%   \changesat{\isotriv A}{()} a a
%% \end{align*}

%% \noindent
%% This straightforwardly satisfies functionality, soundness, and zero changes. Moreover, it inherits the monoidal comonad structure of $\iso$ from \Poset. Fixing some map $f : A \to B$, the derivatives of functorial action, extraction, duplication, and distribution are mostly trivial:

%% \begin{align*}
%%   \isotriv(f) &\isa \isotriv A \to \isotriv B
%%   &
%%   \isotriv(f)' \<x \<\tuple{} &= \tuple{}
%%   \\
%%   \varepsilon_A &\isa \isotriv A \to A
%%   &
%%   \varepsilon_A' \<x \<\tuple{} &= \zero_x
%%   \\
%%   \delta_A &\isa \isotriv A \to \isotriv\isotriv A
%%   &
%%   \delta_A' \<x \<\tuple{} &= \tuple{}
%%   \\
%%   \isox &\isa \textstyle\prod_i \isotriv A_i \to \isotriv\prod_i A_i
%%   &
%%   {\isox}' \<x \<\dx &= \tuple{}
%%   \\
%%   \isosum &\isa \textstyle\isotriv\sum_i A_i \to \sum_i \isotriv A_i
%%   &
%%   {\isosum}' \<(\inj i x) \<\tuple{} &= \inj i ()
%% \end{align*}

%% \pagebreak
%% Set changes are important since most of our motivating examples of fixed points
%% are taken at set type, but to handle all of Datafun we will need change
%% structures for every type. For products, for instance, we use a pointwise change
%% structure:

%% \begin{center}
%%   \setlength\tabcolsep{10pt}
%%   \begin{tabular}{c@{\qquad}c}
%%     $\D\tunit = \tunit$
%%     &
%%     \(\changesat{\tunit}{\tuple{}}{\tuple{}}{\tuple{}}\)
%%     \\[\betweenfunctionskip]    % TODO: is this the right distance?
%%     \(\D(A \x B) = \D A \x \D B\)
%%     &
%%     \(\infer{
%%       \changesat{A}{\da}{a}{a'}
%%       \\
%%       \changesat{B}{\db}{b}{b'}
%%     }{\changesat{A \x B}
%%       {\tuple{\da,\db}}
%%       {\tuple{a,b}}
%%       {\tuple{a',b'}}
%%     }\)
%%   \end{tabular}
%% \end{center}

%% \noindent
%% Here \emph{functionality}, \emph{soundness}, and \emph{zero changes} are trivial for the unit type $\tunit$, and for $A \x B$ they follow directly from the corresponding properties at $A$ and $B$ and the pointwise ordering on products: for instance, a zero change to $(a,b)$ is $(\da,\db)$ where $\da$ is a zero change to $a$ and $\db$ a zero change to $b$.\footnotemark

%% \footnotetext{%
%% One may ask why we let $\D(A \x B) = \D A \x \D B$ rather than $\D A + \D B$, letting $\changesat{A\x B}{\inj 1 \da}{(a,b)}{(a',b)}$ when $\changesat A{\da} a {a'}$ and symmetrically for $B$. This would satisfy our three properties, but recall that our goal is incrementalizing \XXX
%% %
%% }

%% \XXX \todo{sums, functions, box (and thus any discrete type)}

%% \fixme{achim}{Achim is very unhappy with this section, check his notes.}

%% %% To make precise the notion of change, an incremental \fn-calculus associates
%% %% every type $A$ with a \emph{change structure}, consisting of:%
%% %% %
%% %% \footnote{Our notion of change structure differs significantly from that of
%% %%   \citet{incremental}, although it is similar to the logical relation given in
%% %%   \citet{DBLP:conf/esop/GiarrussoRS19}; we discuss this in
%% %%     \cref{section-incremental-lambda-calculus}. Although we do not use change
%% %%   structures \emph{per se} in the proof of correctness sketched in
%% %%   \cref{section-seminaive-logical-relation}, they are an important source of
%% %%   intuition.}

%% %% \begin{enumerate}
%% %% \item A type $\D A$ of possible changes to values of type $A$.
%% %% \item A ternary relation $R_A \subseteq \D A \x A \x A$. As a suggestive
%% %%   notation, we will generally write $(\dx, x, y) \in R_A$ as
%% %%   $\changesat{A}{\dx}{x}{y}$, read as ``$\dx$ changes $x$ into $y$''.
%% %% \end{enumerate}

%% \noindent
%% Since the iterations towards a fixed point grow monotonically, in Datafun we only
%% need \emph{increasing} changes.
%% %
%% For example, sets change by gaining new elements:

%% \begin{align*}
%%   \D\tseteq{A} &= \tseteq{A}
%%   &
%%   \changesat{\tseteq{A}}{\dx}{x}{x \cup \dx}
%% \end{align*}

%% \noindent
%% Set changes may be the most significant for fixed point purposes, but to handle
%% all of Datafun we need a change structure for every type. For products and sums,
%% for example, the change structure is pointwise:
%% %
%% \fixme{achim}{Why not $\Delta(A \x B) = \Delta A + \Delta B$?}
%% %
%% \nopagebreak

%% \begin{center}
%%   \setlength\tabcolsep{10pt}
%%   \begin{tabular}{@{}ccc@{}}
%%     $\D\tunit = \tunit$
%%     &
%%     \(\D(A \x B) = \D A \x \D B\)
%%     &
%%     \(\D(A + B) = \D A + \D B\)
%%     \\[\betweenfunctionskip]    % TODO: is this the right distance?
%%     \(\changesat{\tunit}{\tuple{}}{\tuple{}}{\tuple{}}\)
%%     &
%%     \(\infer{
%%       \changesat{A}{\da}{a}{a'}
%%       \\
%%       \changesat{B}{\db}{b}{b'}
%%     }{\changesat{A \x B}
%%       {\tuple{\da,\db}}
%%       {\tuple{a,b}}
%%       {\tuple{a',b'}}
%%     }\)
%%     &
%%     \(\infer{
%%       \changesat{A_i}{\dx_i}{x}{x'}
%%     }{
%%       \changesat{A_1 + A_2}{\inj i \dx}{\inj i x}{\inj i x'}
%%     }\)
%%   \end{tabular}
%% \end{center}

%% %% \begin{align*}
%% %%   \D\tunit &= \tunit
%% %%   &
%% %%   \D(A \x B) &= \D A \x \D B
%% %%   &
%% %%   \D(A + B) &= \D A + \D B
%% %% \end{align*}
%% %%
%% %% \begin{align*}
%% %%   \changesat{\tunit}{\tuple{}}{\tuple{}}{\tuple{}}
%% %%   &&
%% %%   %% \infer{
%% %%   %%   \fa{i} \changesat{A_i}{\dx_i}{x_i}{y_i}
%% %%   %% }{\changesat{A_1 \x A_2}
%% %%   %%   {\tuple{\vec\dx}}
%% %%   %%   {\tuple{\vec x}}
%% %%   %%   {\tuple{\vec y}}
%% %%   %% }
%% %%   %
%% %%   %% \infer{
%% %%   %%   \fa{i} \changesat{A_i}{\dx_i}{x_i}{y_i}
%% %%   %% }{\changesat{A_1 \x A_2}
%% %%   %%   {\tuple{\dx_1,\dx_2}}
%% %%   %%   {\tuple{x_1,x_2}}
%% %%   %%   {\tuple{y_1,y_2}}
%% %%   %% }
%% %%   %
%% %%   \infer{
%% %%     \changesat{A}{\da}{a}{a'}
%% %%     \\
%% %%     \changesat{B}{\db}{b}{b'}
%% %%   }{\changesat{A \x B}
%% %%     {\tuple{\da,\db}}
%% %%     {\tuple{a,b}}
%% %%     {\tuple{a',b'}}
%% %%   }
%% %%   &&
%% %%   \infer{
%% %%     \changesat{A_i}{\dx}{x}{y}
%% %%   }{
%% %%     \changesat{A_1 + A_2}{\inj i \dx}{\inj i x}{\inj i y}
%% %%   }
%% %% \end{align*}

%% \noindent
%% Since we only consider increasing changes, and $\iso A$ is ordered discretely,
%% the only ``change'' permitted is to stay the same. Consequently, no information
%% is necessary to indicate what changed:

%% \begin{align*}
%%   \D(\iso A) &= \tunit
%%   &&
%%   \changesat{\iso A}{\tuple{}}{x}{x}
%% \end{align*}

%% \noindent
%% Finally we come to the most interesting case: functions.

%% \begin{align*}
%%   \D(A \to B) &= \iso A \to \D A \to \D B
%%   &
%%   \infer[fn~change]{
%%     \fa{\changesat A \dx x y}
%%     \changesat B {\df\<x\<\dx} {f\<x} {g\<y}
%%   }{
%%     \changesat{A \to B}{\df}{f}{g}
%%   }
%% \end{align*}

%% \noindent
%% Observe that a function change $\df$ takes two arguments: a base point $x : \iso A$ and a change $\dx : \D A$.
%% %
%% To understand why we need both, consider incrementalizing function application:
%% we wish to know how $f\<x$ changes as both $f$ and $x$ change.
%% %
%% Supposing $\changes{\df}{f}{g}$ and $\changes{\dx}{x}{y}$, how do we find a
%% change $f\<x \changesto g\<y$ that updates both function and argument?

%% If changes were given pointwise, taking only a base point, we might take
%% $\changes{\df}{f} g$ to mean that $\fa{x} \changes{\df\<x}{f\<x}{g\<x}$. But
%% this only gets us to $g\<x$, not $g\<y$: we've accounted for the change in the
%% function, but not the argument.
%% %
%% We can account for both by giving $\df$ an additional parameter: not just the
%% base point $x$, but also the change to it $\dx$.
%% %
%% Then by inverting \rn{fn~change} we have $\changes{\df\<x\<\dx}{f\<x}{g\<y}$ as
%% desired.

%% %% This makes it easy to incrementalize function application, $f\<x$; given
%% %% changes $\changes \df f g$ and $\changes \dx x y$ to the function and its
%% %% argument, we want to compute the change that takes us to the updated
%% %% application $g\<y$. By inverting \textsc{FnChange} we know that
%% %% $\changes{\df\<x\<\dx}{f\<x}{g\<y}$, so $\df\<x\<\dx$ gives us the desired
%% %% change.

%% %% If instead changes were given pointwise, letting $\D(A \to B)= \iso A \to \D B$,
%% %% then it'd be natural to let $\changes{\df}{f}{g} \iff \fa{x}
%% %% \changes{\df\<x}{f\<x}{g\<x}$.

%% Note also the mixture of monotonicity and non-monotonicity in the type $\iso A
%% \to \D A \to \D B$. Since our functions are monotone (increasing inputs yield
%% increasing outputs), function changes are monotone with respect to input changes
%% $\D A$: a larger increase in the input yields a larger increase in the output.
%% However, there's no reason to expect the change in the output to grow as the
%% base point increases -- hence the base point argument is discrete, $\iso A$.

%% \subsection{Zero changes, derivatives, and faster fixed points}
%% \label{section-derivatives}

%% \todo{Needs to be incorporated into the above categorical discussion.}

%% If $\changesat A \dx x x$, we call $\dx$ a \emph{zero change} to $x$. Usually
%% zero changes are boring -- for example, a zero change to a set $x :
%% \tseteq{A}$ is any $\dx \subseteq x$, and so $\emptyset$ is always a zero
%% change.
%% %
%% However, there is one very interesting exception: function zero changes. Suppose
%% $\changesat{A \to B}{\df}{f}{f}$. Then inverting \rn{fn change} implies that

%% \begin{equation*}
%%   \changesat A \dx x y \implies \changesat B{\df\<x\<\dx}{f\<x}{f\<y}
%% \end{equation*}

%% \noindent
%% In other words, $\df$ yields the change in the output of $f$ given a change to
%% its input.
%% %
%% This is exactly the property of $\name{step}'$ that made it useful for
%% semi\naive\ evaluation -- indeed, $\name{step}'$ is a zero change to
%% \name{step}, modulo not taking the base point $x$ as an argument:

%% \begin{align*}
%%   \changesat{\tseteq A} \dx x y
%%   &\implies
%%   \changesat{\tseteq A}{\name{step}'\<\dx}{\name{step}\<x}{\name{step}\<y}
%%   \\
%%   % should make a box that is as wide as \implies here.
%%   &\parbox[t]{\widthof{${}\implies{}$}}{\centering\emph{i.e.}}
%%   \\[2.5pt]
%%   x \cup \dx = y
%%   &\implies
%%   \name{step}\<x \cup \name{step}'\<\dx = \name{step}\<y
%% \end{align*}

%% \noindent
%% Function zero changes are so important we give them a special name:
%% \emph{derivatives}. We now have enough machinery to prove correct a
%% general \emph{semi\naive\ fixed point strategy}. First, observe that:

%% \begin{restatable}{lemma}{DeltaLattice}\label{lemma-delta-lattice}
%%   At each semilattice type $L$ \fixme{achim}{Where are these defined? Refer to that, please.}, we have $\D L = L$ and
%%   $\changesat{L}{\dx}{x}{y} \iff x \binvee \dx = y$.
%% \end{restatable}

%% \begin{restatable}{proof}{DeltaLatticeProof}
%%   Induct on semilattice types $L$. \todolater{doesn't this need to be more detailed?}
%% \end{restatable}

%% \noindent
%% Now, given a monotone map $f : L \to L$ and a derivative of it $f' : \iso L \to
%% L \to L$, i.e. an $f'$ such that $\changesat{L \to L}{f'} f f$, we can find $f$'s
%% fixed-point as the limit of the sequence $x_i$ defined:

%% \begin{align*}
%%   x_0 &= \bot & x_{i+1} &= x_i \vee \dx_i\\
%%   \dx_0 &= f\<\bot & \dx_{i+1} &= f'\<x_i\<\dx_i
%% \end{align*}

%% \noindent
%% Observe that the function $f$ itself is only used once, to calculate $\dx_0 = f
%% \<\bot$.
%% %
%% Given this initial ``kickoff'' change, the remaining $x_i$ are calculated
%% entirely using the derivative $f'$.
%% %
%% Let $\semifix\<(f,\, f') = \bigvee_i x_i$ be the limit of this sequence.
%% %
%% By induction and the derivative property, we have $\changes{\dx_i}{x_i}{f\<x_i}$
%% and so $x_i = f^i\<x$, and therefore $\semifix\<(f,\, f')$ is the least
%% fixed point of $f$.
%% %
%% And if $L$ has no infinite ascending chains, we will reach this fixed point in
%% some finite number of iterations $i$ such that $x_i = x_{i+1}$.

%% \label{section-seminaive-strategy}

%% This leads directly to our strategy for semi\naive\ Datafun.
%% %
%% The original incremental
%% \fn-calculus~\citep{incremental}
%% defines a static transformation $\Derive e$ which computes the change in $e$
%% given the change in its free variables; it \emph{incrementalizes} $e$.
%% %
%% Our goal is not to incrementalize Datafun \emph{per se}, but to find fixed
%% points faster.
%% %
%% Consequently, we define two mutually recursive transformations: $\phi e$, which
%% computes $e$ faster by replacing fixed points with calls to \semifix; and
%% $\delta e$, which incrementalizes $\phi e$ just enough that we can compute
%% the derivative of fixed point functions.


\section{The \boldphi\ and \bolddelta\ transforms}
\label{section-phi-delta}
\label{why-is-fix-discrete}

\splittodo{explain syntax sugar in phi and delta}{, forward referencing the appendix}

We use two static transformations, $\phi$ and $\delta$, defined in
\cref{figure-phi,figure-delta}. Rather than dive into the gory details
immediately, we first build some intuition.

The speed-up transform $\phi$ computes fixed points semi\naive{}ly by
replacing $\efix f$ by $\semifix\<({f,\, f'})$.
%
To find this derivative $f'$ we'll need a second transform, $\delta$.
%
Since a derivative is a zero change, can $\delta e$ simply find a zero change to
$e$?
%
Unfortunately, this is not strong enough.
%
For example, the zero change to (the derivative of) $\efn x e$ depends not on the
\emph{zero} change to $e$, but on how $e$ changes in response to changes in $x$.
%
To compute derivatives, we need to solve the general problem of computing and
propagating \emph{changes}.
%
So, modelled on the incremental \fn-calculus' $\Derive$ \citep{incremental},
$\delta e$ will compute how $\phi e$ changes as its free variables change.

In generalizing from zero changes to changes, however, we seem to have
jettisoned our original goal: \semifix\ doesn't need the change to $f$, it needs its derivative.
%
Since derivatives are zero changes, function changes and derivatives coincide if
\emph{the function cannot change}.
%
This is why the typing rule for $\efix f$ requires $f : \isofixLtoL$ to be boxed: not because \prim{fix} is non-monotone, but to prevent its argument from changing!
%
So the key strategy of our speed-up transformation $\phi$ is to
{\bfseries\boldmath decorate expressions of type ${\iso A}$ with their zero
  changes.}
%
This makes derivatives available exactly where we need them: at \prim{fix}
expressions.


\subsection{Typing \boldphi\ and \bolddelta}

\begin{figure}\centering
  \begin{align*}
    \Phi\tunit &= \tunit
    &
    \D\tunit &= \tunit
    \\
    \Phi\tseteq A &= \tset{\Phi{\eqt A}}
    \quad\textsf{\small(see \cref{lemma-phi-eqt})}
    &
    \D\tseteq A &= \tseteq A
    \\
    \Phi(\iso A) &= \iso{(\Phi A \x \DP A)}
    &
    \D(\iso A) &= \tunit
    \\
    \Phi(A \x B) &= \Phi A \x \Phi B
    &
    \D(A \x B) &= \D A \x \D B
    \\
    \Phi(A + B) &= \Phi A + \Phi B
    &
    \D(A + B) &= \D A + \D B
    \\
    \Phi(A \to B) &= \Phi A \to \Phi B
    &
    \D(A \to B) &= \iso A \to \D A \to \D B
  \end{align*}

  \todo{roll $\Phi$ into $\Delta$ by letting $\Delta(A \to B) = \iso\Phi A \to \Delta A \to \Delta B$ and $\Delta\tseteq A = \tset{\Phi \eqt A}$. Check for uses of isolated $\Delta$ first.}

  \caption{$\D$ and $\Phi$ type transformations}
  \label{figure-DeltaPhi}
\end{figure}

%% ---- "Go faster" term translation, phi ----
\begin{figure}
  \begin{align*}
    \phi \mvar x &= \mvar x & \phi \dvar x &= \dvar x\\
    \phi(\efn x e) &= \efn x \phi e & \phi(e\<f) &= \phi e\<\phi f\\
    \phi\etuple{e_i}_i &= \etuple{\phi e_i}_i &
    \phi(\pi_i\<e) &= \pi_i\<\phi e\\
    \phi(\inj i e) &= \inj i \phi e
    &
    \phi(\emcase e (\inj i x \caseto f_i)_i)
    &= \emcase{\phi e} (\inj i x \caseto \phi f_i)_i
    \\
    \phi\bot &= \bot &
    \phi(e \vee f) &= \phi e \vee \phi f\\
    \phi(\esetsub{e_i}{i}) &= \esetsub{\phi e_i}{i}
    &
    %% replaced substitution by let-binding
    \phi(\eforvar x e f) &= \eforvar x {\phi e}
        %{\substd{\phi f}{\dvar\dx \substo \zero\<\dvar x}}
        {\eletbox{\dx}{\ebox{\zero\<\dvar x}} \phi f}
    \\
    \phi\ebox{e} &= \eboxtuple{\phi e, \delta e}
    &
    \phi(\eletbox x e f)
    &= \elet{\pboxtuple{\dvar x,\dvar\dx} = \phi e} \phi f
    \\
    \phi(\eeq e f) &= (\eeq {\phi e} {\phi f})
    &
    \phi(\eisempty e) &= \eisempty {\phi e}
    %% \\
    %% \phi(\efixis x e) &\omit\rlap{${}=
    %% \fastfix\<(\subone{(\phi e)}{\mvar x}{\bot},\;
    %% \delta(\efn{x}{e})
    %% )$}
    %% \\
    %% %% split
    %% \phi(\esplit e) &= \emcase{\phi e}
    %% \\
    %% &\omit\rlap{\(\phantom{{}={}}\
    %% \left(\pboxtuple{\inj i \dvar x, \inj i \dvar \dx}
    %% \caseto \inj i {\eboxtuple{\dvar x,\dvar\dx}}\right)_{i}
    %% \)}
    %% \\
    %% &\omit\rlap{\(\phantom{{}={}}\
    %% \left(\pboxtuple{\inj i \dvar x, \inj j \pwild}
    %% \caseto \inj i {\eboxtuple{\dvar x, \dummy\<\dvar x}} \right)_{i\ne j}
    %% \)}
  \end{align*}

  \begin{align*}
    \phi(\efixis x e) &=
    \fastfix\<(\subone{\phi e}{\mvar x}{\bot},\; \delta(\efn{x}{e}))
    \\
    %% split
    \phi(\esplit e) &= \emcase{\phi e}
    \\
    &\phantom{{}={}}\
    \left(\pboxtuple{\inj i \dvar x, \inj i \dvar \dx}
    \caseto \inj i {\eboxtuple{\dvar x,\dvar\dx}}\right)_{i}
    \\
    &\phantom{{}={}}\
    \left(\pboxtuple{\inj i \dvar x, \inj j \pwild}
    \caseto \inj i {\eboxtuple{\dvar x, \dummy\<\dvar x}} \right)_{i\ne j}
  \end{align*}

  \caption{Semi\naive{} speed-up translation, $\phi$}
  \label{figure-phi}
\end{figure}


%% ---- "Derivative" term translation, delta
\begin{figure}\centering
  \[ \delta\bot = \delta\esetsub{e_i}{i} = \delta(\eeq e f)
  = \delta(\efixis x e) = \bot \]
  %
  \begin{align*}
    \delta \mvar x &= \mvar\dx &
    \delta \dvar x &= \dvar\dx\\
    \delta(\efn{x} e) &= \fnof{\pboxvar x} \efn\dx \delta e
    & \delta(e\<f) &= \delta e \<\ebox{\phi e} \<\delta f\\
    \delta\etuple{e_i}_i &= \etuple{\delta e_i}_i
    & \delta(\pi_i\<e) &= \pi_i\<\delta e\\
    \delta(\inj i e) &= \inj i {\delta e} &
    \delta(e \vee f) &= \delta e \vee \delta f\\
    \delta\ebox{e} &= \etuple{} &
    \delta(\eletbox x e f)
    &= \elet{\pboxtuple{\dvar x, \dvar\dx} = \phi e} \delta f
    \\
    \delta(\eisempty e) &= \eisempty {\phi e}
    &
    \delta(\esplit e) &= \emcase{\phi e}
    (\pboxtuple{\inj i \pwild, \pwild}
    \caseto \inj i {\etuple{}} )_i
  \end{align*}
  %
  \begin{align*}
    \delta(\emcase e (\inj i \mvar x \caseto f_i)_i)
    &= \emcase{\esplit{\ebox{\phi e}},\, \delta e}\\
    &\qquad ({\inj i {\pboxvar x},\, \inj i \mvar\dx} \caseto \delta f_i)_{i}\\
    &\qquad ({\inj i {\pboxvar x},\, \inj j \pwild}
    %\caseto \subst{\delta f_i}{\dx \substo \dummy\<\dvar x})_{i\ne j}
    \caseto \elet{\mvar\dx = \dummy\<\dvar x} \delta f_i)_{i\ne j}
    \\
    \delta(\eforvar x e f)
    &= (\eforvar x {\delta e}
    %\substd{\phi f}{\dvar\dx \substo \zero\<\dvar x}) \\
    \eletbox \dx {\zero\<\dvar x} \phi f) \\
    &\vee (\eforvar x {\phi e \vee \delta e}
    %\substd{\delta f}{\dvar\dx \substo \zero\<\dvar x})
    \eletbox{\dx}{\zero\<\dvar x} \delta f)
  \end{align*}

  \caption{Semi\naive{} derivative translation, $\delta$}
  \label{figure-delta}
\end{figure}


In order to decorate expressions with extra information, $\phi$ also needs to
decorate their types. In \cref{figure-DeltaPhi} we give a type translation $\Phi A$
capturing this.
%
In particular, if $e : \iso A$ then $\phi e$ will have type $\Phi(\iso A) =
\iso(\Phi A \x \DP A)$.
%
The idea is that evaluating $\phi e$ will produce a pair
$\eboxraw{\etuple{x,\dx}}$ where $x : \Phi A$ is the sped-up result and $\dx :
\DP A$ is a zero change to $x$.
%
For example, if $e : \iso(A \to B)$, then $\phi e$ will compute $\eboxraw{\etuple{f,f'}}$, where $f'$ is the derivative of $f$.

On types other than $\iso A$, there is no information we need to add, so $\Phi$
simply distributes.
%
In particular, source programs and sped-up programs agree on the shape of
first-order data:

\begin{restatable}{lemma}{PhiEqualityType}\label{lemma-phi-eqt}
  $\Phi\eqt A = \eqt A$ for all equality types $\eqt A$.
\end{restatable}

\begin{restatable}{proof}{PhiEqualityTypeProof}
 Induct on $\eqt A$.
\end{restatable}

\noindent
As we'll see in \cref{section-var-fn-app,section-phi-delta-box}, $\phi$ and $\delta$ are
mutually recursive. To make this work, $\delta e$ must find the change to $\phi
e$ rather than $e$.
%
So if $e : A$ then $\phi e : \Phi A$ and $\delta e : \DP A$.
%
However, so far we have neglected to say what $\phi$ and $\delta$ do to typing
contexts.
%
To understand this, it's helpful to look at what $\Phi$ and $\DP$ do to
functions and to $\iso$.
%
This is because expressions denote functions of their free variables.
%
Moreover, in Datafun free variables come in two flavors, monotone and discrete, and discrete variables are semantically $\iso$-ed.

Viewed as functions of their free variables, $\delta e$ denotes the
\emph{derivative} of $\phi e$.
%
And just as the derivative of a unary function $f\<x$ has \emph{two} arguments,
$\df\<x\<\dx$, the derivative of an expression $e$ with $n$ variables $x_1,
\dots, x_n$ will have $2n$ variables: the original $x_1, \dots, x_n$ and their
changes $\dx_1, \dots, \dx_n$.%
%
\footnote{For notational convenience we assume
  that source programs contain no variables starting with the letter \emph{d}.}
%
However, this says nothing yet about monotonicity or discreteness.
%
To make this precise, we'll use three context transformations, named according
to the analogous type operators $\iso$, $\Phi$, and $\Delta$:
%% %
%% We define these pointwise by their action on singleton contexts (they all
%% preserve empty contexts and distribute across context union):

\todo{explain mono-to-disc renaming confusion}

\begin{align*}
  \iso{(\hm x A)} &= \hd x A & \iso{(\hd x A)} &= \hd x A
  \\
  \Phi(\hm x A) &= \hm x {\Phi A} & \Phi(\hd x A) &= \hd x {\Phi A}, \hd \dx {\DP A}
  \\
  \D(\hm x A) &= \hm \dx {\D A}
  & \D(\hd x A) &= \emptycx\quad\textsf{\small(the empty context)}
\end{align*}

\noindent
Otherwise all three operators distribute; e.g.\ $\iso\emptycx = \emptycx$ and
$\iso(\G_1,\G_2) = \iso\G_1, \iso\G_2$.
%
Intuitively, $\iso\G$, $\Phi\G$, and $\D\G$ mirror the effect of
$\iso$, $\Phi$, and $\D$ on the semantics of $\G$:

\begin{align*}
  \den{\iso\G} &\cong \iso\den\G
  &
  \begin{aligned}
    \den{\Phi(\hm x A)} &\cong \den{\Phi A}
    \\
    \den{\Phi(\hd x A)} &\cong \den{\Phi \iso A}
  \end{aligned}
  &&
  \begin{aligned}
    \den{\D(\hm x A)} &\cong \den{\D A}
    \\
    \den{\D(\hd x A)} &\cong \den{\D \iso A}
  \end{aligned}
\end{align*}

%% \begin{align*}
%%   \multirow{2}{*}{\den{\iso \G} = \iso\den\G}
%%   &&
%%   \den{\D(\hm x A)} &\cong \den{\D A}
%%   &
%%   \den{\Phi(\hm x A)} &\cong \den{\Phi A}
%%   \\
%%   &&
%%   \den{\D(\hd x A)} &\cong \den{\D \iso A}
%%   &
%%   \den{\Phi(\hd x A)} &\cong \den{\Phi \iso A}
%% \end{align*}

\noindent
These defined, we can state the types of $\phi e$ and $\delta e$:

\begin{restatable}[Well-typedness of $\phi$, $\delta$]{theorem}{PhiDeltaWellTyped}
  \label{theorem-phi-delta-well-typed}
  If\/ $\J e \G A$, then $\phi e$ and $\delta e$ have the following types:
  \begin{align*}
    \Jalign {\phi e} {\Phi\G} {\Phi A}\\
    \Jalign {\delta e} {\iso{\Phi\G}, \DP\G} {\DP A}
  \end{align*}
\end{restatable}

\begin{restatable}{proof}{PhiDeltaWellTypedProof}
  By induction on the derivation of $\J e \G A$, although as we'll see shortly
  we will need weakening (\cref{theorem-weakening}) in some places.
  %%; see \cref{proof-phi-delta-well-typed}.
\end{restatable}

\noindent As expected, if we view expressions as functions of their free
variables, and pretend $\G$ is a type, these correspond to $\Phi(\G \to A)$
and $\DP(\G \to A)$ respectively:

\begin{align*}
  \Phi(\G \to A) &= \Phi\G \to \Phi A
  & \DP(\G \to A) &= \iso\Phi\G \to \DP\G \to \DP A
\end{align*}

%% TODO: can I maybe cut this bit?
\noindent
To get the hang of these context and type transformations, suppose $\J
e {\hd x A, \hm y B} C$. Then \cref{theorem-phi-delta-well-typed} tells us:

\begin{align*}
  \Jalign{\phi e} {\hd x{\Phi A},\, \hd \dx {\DP A},\, \hm y {\Phi B}} {\Phi C}
  \\
  \Jalign{\delta e} {\hd x{\Phi A},\, \hd\dx{\DP A},\, \hd y{\Phi B}, \hm\dy{\DP B}} {\DP C}
\end{align*}

%% %% TODO: should we include this? I think the point is pretty well-made already.
%% \noindent
%% These correspond respectively to $\Phi$ and $\DP$ applied to $\iso A \to B \to
%% C$:
%%
%% \begin{align*}
%%   \Phi(\iso A \to B \to C)
%%   &= \iso (\Phi A \x \DP A) \to \Phi B \to \Phi C
%%   \\
%%   \DP(\iso A \to B \to C)
%%   &= \D(\iso (\Phi A \x \DP A) \to \Phi B \to \Phi C)\\
%%   &= \iso (\iso (\Phi A \x \DP A))
%%   \to \D(\iso (\Phi A \x \DP A))
%%   \to \D(\Phi B \to \Phi C)\\
%%   &\cong \iso (\Phi A \x \DP A)
%%   \to \tunit
%%   \to \iso\Phi B \to \DP B \to \DP C\\
%%   &\cong \iso \Phi A \to \iso \DP A \to \iso\Phi B \to \DP B \to \DP C
%% \end{align*}

\noindent
Along with the original program's variables, $\phi e$ requires zero change
variables $\dvar\dx$ for every discrete source variable $\dvar x$. Meanwhile,
$\delta e$ requires changes for \emph{every} source program variable (for
discrete variables these will be zero changes), and moreover is \emph{discrete}
with respect to the source program variables (the ``base points'').

We now have enough information to tackle the definitions of $\phi$ and $\delta$
given in \cref{figure-phi,figure-delta}. In the remainder of this section, we'll
examine the most interesting and important parts of these definitions in detail.


\subsection{Fixed points}

The whole purpose of $\phi$ and $\delta$ is to speed up fixed points, so let's
start there.
%
In a fixed point expression $\efix e$, we know $e : \isofixLtoL$. Consequently the type of $\phi e$ is

\begin{align*}
  \Phi(\isofixLtoL)
  &= \iso(\Phi(\kernfixtL \to \fixtLkern) \x \DP(\kernfixtL \to \fixtLkern))\\
  &= \iso((\Phi\fixt L \to \Phi\fixtLkern)
  \x (\iso\Phi\fixt L \to \DP \fixt L \to \DP \fixtLkern))
  \\
  &= \iso((\kernfixtL \to \fixtLkern) \x
  (\iso\fixt L \to \D \fixt L \to \D\fixtLkern)
  & \text{by \cref{lemma-phi-eqt}, }\Phi\fixt L = \fixt L
  \\
  &= \iso((\kernfixtL \to \fixtLkern) \x (\iso\fixt L \to \fixt L \to \fixtLkern)
  & \text{by \cref{lemma-delta-lattice}, }\Delta \fixt L = \fixt L
\end{align*}

\noindent
The behavior of $\phi e$ is to compute a boxed pair $\eboxtupleraw{f,f'}$, where
$f : \fixt L \to \fixt L$ is a sped-up function and $f' : \iso\fixt L \to \fixt
L \to \fixt L$ is its derivative. This is exactly what we need in order to call
\semifix. Therefore $\phi(\efix e) = \semifix\<\phi e$.
%
However, if we're going to use \semifix\ in the output of $\phi$, we ought to
give it a typing rule and semantics:

\begin{align*}
  \infer{
    \J{e}{\G}{\iso((\kernfixtL \to \fixtLkern) \x (\iso\fixt L \to \fixt L \to \fixtLkern)}
  }{\J{\semifix\<e}{\G}{\fixt L}}
  &&
  \begin{aligned}
    \den{\semifix\<e}\<\g &= \semifix \<(f, f')
    \\
    \text{where}~ & (f,f') = \den{e}\<\g
  \end{aligned}
\end{align*}

\noindent
As for $\delta(\efix e)$, since $e$ can't change (having $\iso$ type), neither
can $\efix e$ (or $\semifix\<\phi e$). All we need is a zero change at type
$\fixtLkern$; by \cref{lemma-delta-lattice}, $\bot$ suffices.


\subsection{Variables, \boldfn-abstraction, and application}
\label{section-var-fn-app}

At the core of a functional language are variables, \fn-abstraction, and
application. The $\phi$ translation leaves these alone, simply distributing over
subexpressions. On variables, $\delta$ yields the corresponding change
variables. On functions and application, $\delta$ is more interesting:

%% \begin{align*}
%%   \begin{aligned}
%%     \delta(\efn x e) &= \fnof{\pboxvar x} \efn\dx \delta e
%%     \\
%%     \delta(e\<f) &= \delta e \<\ebox{\phi f} \<\delta f
%%   \end{aligned}
%%   &&
%%   \begin{aligned}
%% %    \clap{\quad\quad\textit{recalling that}}\\[-.25ex]
%% %    \clap{\quad\;\;\textit{N.B.}}\\[-.25ex]
%% %    \textit{N.B.}\:\:\:
%%     \DP(A \to B) &= \iso\Phi A \to \DP A \to \DP B
%%   \end{aligned}
%% \end{align*}

%% \begin{align*}
%%   \DP(A \to B) &= \iso\Phi A \to \DP A \to \DP B
%%   &
%%   \begin{aligned}
%%     \delta(\efn x e) &= \fnof{\pboxvar x} \efn\dx \delta e
%%     \\
%%     \delta(e\<f) &= \delta e \<\ebox{\phi f} \<\delta f
%%   \end{aligned}
%% \end{align*}

\begin{align*}
  \DP(A \to B) &= \iso\Phi A \to \DP A \to \DP B
  \\
  \delta(\efn x e) &= \fnof{\pboxvar x} \efn\dx \delta e
  \\
  \delta(e\<f) &= \delta e \<\ebox{\phi f} \<\delta f
\end{align*}

%% \begin{align*}
%%   \DP(A \to B) &= \iso\Phi A \to \DP A \to \DP B
%%   &
%%   \delta(\efn x e) &= \fnof{\pboxvar x} \efn\dx \delta e
%%   &
%%   \delta(e\<f) &= \delta e \<\ebox{\phi f} \<\delta f
%% \end{align*}

\noindent
The intuition behind $\delta(\efn x e) = \fnof{\pboxvar x} \efn\dx \delta e$ is
that a function change takes two arguments, a base point $\dvar x$ and a change
$\mvar\dx$, and yields the change in the result of the function, $\delta e$.
However, we are given an argument of type $\iso \Phi A$, but consulting
\cref{theorem-phi-delta-well-typed} for the type of $\delta e$, we need a discrete variable
$\hd x {\Phi A}$, so we use pattern-matching to unbox our argument.
%
The intuition behind $\delta(e\<f) = \delta e \<\ebox{\phi f} \<\delta f$ is
much the same: $\delta e$ needs two arguments, the original input $\phi f$ and
its change $\delta f$, to return the change in the function's output. Moreover,
it's discrete in its first argument, so we need to box it, $\ebox{\phi f}$.

One might ask why this type-checks, since $\phi e$ and $\delta e$ don't use the
same typing context.
%
We're even boxing $\phi f$, hiding all monotone variables; consequently, it gets
the context $\stripcx{\iso\Phi\G, \DP\G}$.
%
However, $\iso$ makes every variable discrete, and $\stripcx{-}$ leaves discrete
variables alone, so this provides \emph{at least} $\iso\Phi\G$, while the
context $\phi f$ needs $\Phi\G$.
%
Thus really this is a question about the interaction of weakening and discreteness: can a discrete variable always substitute for a monotone one?

Indeed it may: making a variable discrete only increases the number of places it can be used, because while some typing rules discard monotone variables, they never discard discrete ones.
%
We formalize this using a weakening relation $\Gamma \weaker \Delta$
(\cref{figure-weakening}; note that $H$ for ``hypothesis'' ranges over all
variable typings, monotone or discrete), which 
%
is standard except for the rule
\rn{disc}, which says that a  discrete hypothesis is weaker than a monotone one.
%
We can then show that typing respects weakening:

\begin{figure}
  \begin{mathpar}
    \infer[empty]{~}{\emptycx \weaker \emptycx}

    \infer[cons]{\Gamma \weaker \Delta}{\Gamma, H \weaker \Delta, H}

    \infer[drop]{\Gamma \weaker \Delta}{\Gamma \weaker \Delta, H}

    \infer[disc]{\Gamma \weaker \Delta}
          {\Gamma,\,\hm x A \weaker \Delta,\,\hd x A}
  \end{mathpar}

  \caption{Weakening relation}
  \label{figure-weakening}  
\end{figure}

%% \label{section-weakening}

\begin{restatable}[Weakening]{theorem}{Weakening}
  \label{theorem-weakening}
  If $\Delta \stronger \Gamma$ and $\J{e}{\Gamma} A$ then $\J{e}{\Delta} A$.
\end{restatable}
\begin{proof}
  By induction on the derivation of $\J{e}{\Gamma} A$; see \cref{proof-weakening}.
\end{proof}

%% The same weakening argument applies (all the more easily) when $\phi e$ is used in a
%% monotone rather than a discrete position.

\noindent
We use this without further note throughout the $\phi$ and $\delta$ transformations.


\subsection{The discreteness comonad, \iso}
\label{section-phi-delta-box}

Our strategy hinges on decorating expressions of type $\iso A$ with their
zero changes, so the translations of $\ebox e$ and $(\eletbox x e f)$ are of
particular interest.
%
The most trivial of these is $\delta\ebox{e} = \etuple{}$; this follows from
$\DP\iso A = \tunit$, since boxed values cannot change.

Next, consider \(\phi\ebox e = \ebox{\etuple{\phi e, \delta e}}\).
%
The intuition here is straightforward: $\phi$ needs to decorate $e$ with its
zero change; since $e$ is discrete and cannot change, we use $\delta e$.
%
However! In general, one cannot use $\delta$ inside the $\phi$ translation and
expect the result to be well-typed; $\phi$ and $\delta$ require different typing
contexts. To see this, let's apply \cref{theorem-phi-delta-well-typed} to
singleton contexts:

\begin{center}
  \setlength\tabcolsep{10pt}
  \begin{tabular}{@{}lll@{}}
    $\G$ \ \textsf{\small(context of $e$)}
    & $\Phi\G$ \ \textsf{\small(context of $\phi e$)}
    & $\iso\Phi\G,\DP\G$ \ \textsf{\small(context of $\delta e$)}
    \\
    %% $e$'s context & $\phi e$'s context & $\delta e$'s context
    %% \\
    \midrule
    $\hm x A$ & $\hm x {\Phi A}$ & $\hd x {\Phi A}, \hm\dx{\DP A}$
    \\
    $\hd x A$
    & $\hd x {\Phi A}, \hd\dx{\DP A}$
    & $\hd x {\Phi A}, \hd\dx{\DP A}$
  \end{tabular}
\end{center}

\noindent
Luckily, although $\Phi\G$ and $\iso\Phi\G,\DP\G$ differ on monotone variables,
they agree on discrete ones. And since $e$ is discrete, it \emph{has} no
free monotone variables, justifying the use of $\delta e$ in
$\phi\ebox{e} = \ebox{\etuple{\phi e, \delta e}}$.

Next we come to $(\eletbox x e f)$, whose $\phi$ and $\delta$ translations are
very similar:

\begin{align*}
  \phi(\eletbox x e f)
  &=
  \elet{\pboxtuple{\dvar x,\dvar\dx} = \phi e} \phi f
  \\
  \delta(\eletbox x e f) &=
  \elet{\pboxtuple{\dvar x,\dvar\dx} = \phi e} \delta f
\end{align*}

\noindent
Since $\dvar x$ is a discrete variable, both $\phi f$ and $\delta f$ need access
to its zero change $\dvar\dx$. Luckily, $\phi e : \iso(\Phi A \x \DP A)$
provides it, so we simply unpack it. We don't use $\delta e$ in $\delta f$, but
this is unsurprising when you consider that its type is $\DP\iso A = \tunit$.


\subsection{Case analysis, \prim{split}, and \name{dummy}}

%\newcommand\evalsto\mapsto %% TODO: remove this

The derivative of case-analysis, $\delta(\emcase{e}{(\inj i \mvar x_i \caseto
  f_i)_i})$, is complex.
%
Suppose $\phi e$ evaluates to $\inj i x$ and its change $\delta e$ evaluates to
$\inj j \dx$.
%
Since $\delta e$ is a change to $\phi e$, the change structure on sums tells us
that $i = j$! (This is because sums are ordered disjointly; the value $x$ can
increase, but the tag $\injc_i$ must remain the same.)
%
So the desired change $\delta(\emcase{e}{\dots})$ is given by $\delta f_i$ in a
context supplying a discrete base point $\dvar x$ (the value $x$) and the change
$\mvar\dx$.
%
To bind $\dvar x$ discretely, we need to use $\ebox{\phi e} : \iso(\Phi A + \Phi
B)$; to pattern-match on this, we need \prim{split} to distribute the $\iso$.

This handles the first two cases, $(\inj i {\pboxvar x},\, \inj i \mvar\dx
\caseto \delta f_i)_i$. Since we know the tags on $\phi e$ and $\delta e$ agree,
these are the only possible cases. However, to appease our type-checker \fixme{jeremy}{Why do you need dead code in order to appease your type checker? What is this type checker? Can't it have impossibility tests, like Idris?} we must
handle the \emph{impossible} case that $i \ne j$. This case is dead code: it
needs to typecheck, but is otherwise irrelevant. It suffices to generate a dummy
change $\dx : \DP A_i$ from our base point $\hd{x}{\Phi A_i}$. We do this using
a simple function $\dummy_A : A \to \D A$ (\cref{figure-dummy}).

\begin{figure}
  \begin{align*}
    \dummy_{\tseteq A} \<\pwild &= \esetraw{}
    &
    \dummy_{A \x B} \<\etuple{x,y} &= \etuple{\dummy\<x, \dummy\<y}
    \\
    \dummy_\tunit \<\etuple{} &= \etuple{}
    &
    \dummy_{A + B} \<(\inj i x) &= \inj i (\dummy\<x)
    \\
    \dummy_{\iso A} \<\pboxvar{x} &= \ebox{\dummy\<\dvar x}
    &
    \dummy_{A \to B} \<f &= \fnof{x} \dummy\<(f\<x)
  \end{align*}
  \caption{The function $\dummy_A : A \to \D A$}
  \label{figure-dummy}
\end{figure}


\todolater{explain that $\phi$, $\delta(\esplit e)$ are effectively book-keeping
  the decorations that the $\phi$ translation adds to $\iso$.}

We also need \dummy\ in the definition of $\phi(\esplit e)$. In effect
$\prim{split} : \iso(A + B) \to \iso A + \iso B$. Observe that

\begin{align*}
  \Phi(\iso (A + B)) &= \iso((\Phi A + \Phi B) \x (\DP A + \DP B))
  \\
  \Phi(\iso A + \iso B) &= \iso(\Phi A \x \DP A) + \iso(\Phi B \x \DP B)
\end{align*}

\noindent
So while $\phi e$ yields a boxed pair of tagged values, $\eboxraw{\etuple{\inj i
    x, \inj j \dx}}$, we need $\phi(\esplit e)$ to yield a tagged boxed pair,
$\inj i {\eboxraw{\etuple{x,\dx}}}$. Again we use \dummy\ to handle the
impossible case $i \ne j$.

Finally, observe that $\delta(\esplit e)$ has type
%
\(
  \DP(\iso A + \iso B)
  = \DP\iso A + \DP\iso B
  = \tunit + \tunit
\).
%
\noindent
All it must do is return $(\inj i \etuple{})$ with a tag that matches
$\phi(\esplit e)$ and $\phi e$; \kw{case}-analysing $\phi e$ suffices.


\subsection{Semilattices and comprehensions}
\label{section-semilattice-delta-phi}

The translation $\phi(e \vee f) = \phi e \vee \phi f$ is straightforward, but $\delta(e \vee f) = \delta e \vee \delta f$ is not as simple as it seems.
%
Restricting to sets, suppose that $\dx$ changes $x$ into $x'$ and $\dy$ changes
$y$ to $y'$. In particular, suppose these changes are \emph{precise}: that $\dx
= x' \setminus x$ and $\dy = y' \setminus y$. Then the precise change from $x
\cup y$ into $x' \cup y'$ is:

\[ (x' \cup y') \setminus (x \cup y)
= (x' \setminus x \setminus y) \cup (y' \setminus y \setminus x)
= (\dx \setminus y) \cup (\dy \setminus x)
\]

\noindent
This suggests letting $\delta(e \cup f) = (\delta e \setminus \phi f) \cup
(\delta f \setminus \phi e)$. This is a valid derivative, but it involves
recomputing $\phi e$ and $\phi f$, and our goal is to avoid recomputation. So
instead, we \emph{overapproximate} the derivative: $\delta e \cup \delta f$
might contain some unnecessary elements, but we expect it to be cheaper to
include these than to recompute $\phi e$ and $\phi f$. This overapproximation
agrees with semi\naive\ evaluation in Datalog: Datalog implicitly unions the
results of different rules for the same predicate (e.g. those for \name{path} in
\cref{section-seminaive-incremental}), and the semi\naive\ translations of these
rules do not include negated premises to compute a more precise difference.

Now let's consider $\eforin x e f$.
%
Its $\phi$-translation is straightforward, with one hitch: because $\hd x {\eqt
  A}$ is a discrete variable, the inner loop $\phi f$ needs access to its zero
change $\hd \dx {\D\eqt A}$.
%
Conveniently, at eqtypes (although not in general), the \dummy\ function
computes zero changes:
%% TODO: explain why this can't work at function types, which is one reason we
%% restrict sets to containing first-order data?

\begin{lemma} \label{lemma-dummy-change}
  If $x : \eqt A$ then
  $\changesat{\eqt A}{\dummy\<x}{x}{x}$.
\end{lemma}

\begin{proof}
  By induction on $\eqt A$, unfolding the definition of $(\changesat{\eqt
    A}{\dummy\<\dx}{x}{x})$ from \cref{section-change-structures} at each step.
  For example, when $\eqt A = \tset{\eqt B}$, we need to show that $x \cup
  \dummy\<\dx = x$, which is true because $\dummy_{\tset{\eqt B}}\<x = \esetraw{}$.
\end{proof}

\noindent For clarity, we write \zero\ rather than \dummy\ when we use it to
produce zero changes; we only call it \dummy\ in dead code.

Finally, we come to $\delta(\eforin x e f)$, the computational heart of the
semi\naive\ transformation, as \kw{for} is what enables embedding relational
algebra (the right-hand-sides of Datalog clauses) into Datafun.
%
Here there are two things to consider, corresponding to the two \kw{for}-clauses
generated by $\delta(\eforin x e f)$.
%
First, if the set $\phi e$ we're looping over gains new elements $\dvar x \in
\delta e$, we need to compute $\phi f$ over these new elements. Second, if the
inner loop $\phi f$ changes, we need to add in its changes $\delta f$ for every
element, new or old, in the looped-over set, $\phi e \vee \delta e$. Just as in
the $\phi$-translation, we use \zero/\dummy\ to calculate zero changes to set
elements.


\subsection{Leftovers}

The $\phi$ rules we haven't yet discussed simply distribute $\phi$ over
subexpressions. The remaining $\delta$ rules mostly do the same, with a few
exceptions. In the case of $\delta(\esetsub{e_i}{i}) = \delta(\eeq e f) = \bot$,
the sub-expressions are discrete and cannot change, so we produce a zero change
$\bot$. This is also the case for $\delta(\eisempty e) = \eisempty{\phi e}$, but
as with $\delta(\esplit e)$, the zero change here is at type $\tunit + \tunit$,
so to get the tag right we must analyse $\phi e$.


\section{Proving the semi\naive\ transformation correct}

\todolater{rewrite from here on}

%\label{section-logical-relation}
\label{section-seminaive-logical-relation}

We have given two program transformations: $\phi e$, which optimizes $e$ by
computing fixed points semi\naive{}ly; and $\delta e$, which finds the change in
$\phi e$ under a change in its free variables.
%
To state the correctness of $\phi$ and $\delta$, we need to show that $\phi e$
preserves the meaning of $e$ and that $\delta e$ correctly updates $\phi e$ with
respect to changes in its variable bindings.
%
Since our transformations modify the types of higher-order expressions to
include the extra information needed for semi\naive\ evaluation, we cannot
directly prove that the semantics is preserved.
%
Instead, we formalize the relationship between $e$, $\phi e$, and $\delta e$
using a logical relation, and use this relation to prove an \emph{adequacy
  theorem} saying that the semantics is preserved for closed, first-order
programs.

\begin{figure}
  \begin{align*}
    \weirdat{\tunit}{\tuple{}}{\tuple{}}{\tuple{}}{\tuple{}}{\tuple{}}
    &\iff \top
    \\
    \weirdat{A_1 \x A_2}{\vec{\dx}}{\vec x}{\vec a}{\vec y}{\vec b}
    &\iff \fa{i} \weirdat{A_i}{\dx_i}{x_i}{a_i}{y_i}{b_i}
    \\
    \weirdat{A_1 + A_2}{\inj i \dx}{\inj j x}{\inj k a}{\inj l y}{\inj m b}
    &\iff i = j = k = l = m \wedge \weirdat{A_i}{\dx}{x}{a}{y}{b}
    \\
    \weirdat{A \to B}{\df}{f}{f'}{g}{g'}
    &\iff
    \fa{\weirdat{A}{\dx}{x}{a}{y}{b}} \\
    &\hphantom{{}\iff{}}
    \weirdat{B}{\df\<x\<\dx}{f\<x}{f'\<a}{g\<y}{g'\<b}
    \\
    \weirdat{\tseteq A}{\dx}{x}{a}{y}{b}
    &\iff (x,y,x \cup \dx) = (a,b,y)
    \\
    \weirdat{\iso A}{\tuple{}}{(x,\dx)}{a}{(y,\dy)}{b}
    &\iff (a,x,\dx) = (b,y,\dy) \wedge \weirdat{A}{\dx}{x}{a}{y}{b}
  \end{align*}
  \caption{Definition of the logical relation}
  \label{figure-seminaive-logical-relation}
\end{figure}



So, inductively on types $A$, letting $a,b \in \den{A}$, $x,y \in \den{\Phi A}$,
and $\dx \in \den{\DP A}$, we define a five place relation
$\weirdat{A}{\dx}{x}{a}{y}{b}$, meaning roughly ``$x,y$ speed up $a,b$
respectively, and $\dx$ changes $x$ into $y$''. The full definition is in
\cref{figure-seminaive-logical-relation}.

At product, sum, and function types this is essentially a more elaborate version
of the change structures given in \cref{section-change-structures}.
%
At set types, changes are still a set of values added to the initial value, but
we additionally insist that the ``slow'' $a,b$ and ``speedy'' $x,y$ are equal.
%
This is because we have engineered the definitions of $\Phi$ and $\phi$ to
preserve behavior on equality types.
%
Finally, since $\iso A$ represents values which cannot change, $\dx$ is an
uninformative empty tuple and the original and updated values are identical.
%
However, the ``speedy'' values are now \emph{pairs} of a value and its zero
change.
%
This ensures that at a boxed function type, we will always have a derivative (a
zero change) available.

The logical relation is defined on simple values, and so before we can state the
fundamental theorem, we have to extend it to contexts $\G$ and substitutions,
letting $\rho,\rho' \in \den{\G}$, $\g,\g' \in \den{\Phi\G}$, and $\dgamma \in
\den{\DP\G}$:

\begin{align*}
  \weirdat{\G}{\dgamma}{\g}{\rho}{\g'}{\rho'}
  &\iff \fa{\hm x A \in \G}
  \weirdat{A}{\dgamma_{\mvar\dx}}{\g_{\mvar x}}{\rho_{\mvar x}}{\g'_{\mvar x}}{\rho'_{\mvar x}}
  \\
  &\hphantom{{}\iff{}}
  %\hspace*{-15.9pt} \wedge
  \hspace*{-18.6pt} \,\binwedge\,
  %\hspace*{-14.7pt} \binwedge
  \fa{\hd x A \in \G}
  \weirdat{\iso A}
          {\etuple{}}
          {(\g_{\dvar x}, \g_{\dvar \dx})}
          {\rho_{\dvar x}}
          {(\g'_{\dvar x}, \g'_{\dvar \dx})}
          {\rho'_{\dvar x}}
\end{align*}

\noindent
With that in place, we can state the fundamental theorem, showing that
$\phi$ and $\delta$ generate expressions which satisfy this logical
relation:

\begin{restatable}[Fundamental property]{theorem}{SeminaiveFundamental}
  \label{theorem-seminaive-fundamental}
  If $\J e \G A$ and $\weirdat{\G}{\dgamma}{\g}{\rho}{\g'}{\rho'}$ then
  \[\weirdat{A}{\den{\delta e} \<\tuple{\g,\dgamma}}{\den{\phi
      e}\<\g}{\den{e}\<\rho}{\den{\phi e}\<\g'}{\den{e}\<\rho'}\]
\end{restatable}

\begin{proof}
  See \cref{proof-seminaive-fundamental}.
\end{proof}

\noindent
This theorem follows by a structural induction on typing derivations as usual,
but requires a number of lemmas.
%
By and large, these lemmas generalize or build on results stated earlier
regarding the simpler change structures from \cref{section-change-structures}.
%
For example, we build on \cref{lemma-phi-eqt,lemma-dummy-change} to
characterize the logical relation at equality types $\eqt A$ and the behavior of
\dummy:

%% \begin{lemma}[Equality Changes]
%%   If $\weirdat{\eqt A}{\dx}{x}{a}{y}{b}$ then $x = a$ and $y = b$.
%% \end{lemma}

%% \begin{lemma}[Dummy is zero at eqtypes]
%%   If $x \in \den{\eqt A}$ then $\weirdat{\eqt A}{\dummy\<x}{x}{x}{x}{x}$.
%% \end{lemma}

\begin{restatable}[Equality changes]{lemma}{EqualityChanges}
  \label{lemma-equality-changes}
  If $\weirdat{\eqt A}{\dx}{x}{a}{y}{b}$ then $x = a$ and $y = b$.
\end{restatable}
% TODO: why is restatable adding so much vertical space here?
\begin{restatable}[Dummy is zero at eqtypes]{lemma}{EqualityDummy}
  \label{lemma-equality-dummy}
  If $x \in \den{\eqt A}$ then $\weirdat{\eqt A}{\dummy\<x}{x}{x}{x}{x}$.
\end{restatable}

\begin{proof}
  In each case, induct on $\eqt A$. See
  \cref{proof-equality-changes,proof-equality-dummy}.
\end{proof}

\noindent
\Cref{lemma-equality-changes} tells us that at equality types, the sped-up version
of a value is the value itself. This is used later to prove our adequacy
theorem.
%
\Cref{lemma-equality-dummy} is an analogue of \cref{lemma-dummy-change}, showing
that \dummy\ function computes zero changes at equality types.
%
This is used in the proof of the fundamental theorem for \kw{for}-loops,
in whose $\phi$ and $\delta$ translations $\zero$ is implemented by \dummy.

Next, we generalize \cref{lemma-delta-lattice} to characterize changes at semilattice
type:

%% The lemma involves the lattice types, showing that a change for a lattice type
%% $L$ is something that can be joined on to it:

\begin{lemma}[Semilattice changes]
  \label{lemma-semilattice-changes}
  At any semilattice type $L$, we have $\Delta L = L$, and moreover
  $\weirdat{L}{\dx}{x}{a}{y}{b}$ iff $x = a$ and $y = b = x \vee_L \dx$
\end{lemma}

\begin{proof}
  By induction on semilattice types $L$, applying \cref{lemma-equality-changes}
  (noting that every semilattice type is an equality type).
\end{proof}

\noindent
We require this lemma in the proofs of the fundamental theorem in all the
cases involving semilattice types -- namely $\bot$, ${\vee}$, \kw{for},
and \prim{fix}.

Since typing rules that involve discreteness (such as the $\iso$ rules)
manipulate the context, we need some lemmas regarding these manipulations.
First, we show that all valid changes for a context with only discrete variables
send substitutions to themselves, recalling that $\stripcx{\G}$ contains only the
discrete variables from $\G$.

\begin{restatable}[Discrete contexts don't change]{lemma}{DiscreteContexts}
  \label{lemma-discrete-contexts}
  If $\weirdat{\stripcx{\G}}{\tuple{}}{\gamma}{\rho}{\gamma'}{\rho'}$ then
  $\gamma = \gamma'$ and $\rho = \rho'$.
\end{restatable}

\begin{restatable}{proof}{DiscreteContextsProof}
  All variables in the stripped contexts are discrete, and therefore the logical
  relation for discrete variables in contexts, which invokes the logical
  relation at $\iso$ type, requires their corresponding components be equal.
\end{restatable}

\noindent
We use this lemma in combination with the next, which says that any valid
context change gives rise to a valid change on a stripped context:

\begin{restatable}[Context stripping]{lemma}{ContextStripping}
  \label{lemma-context-stripping}
  If $\weirdat{\G}{\dgamma}{\gamma}{\rho}{\gamma'}{\rho'}$
  then

  \[
  \weirdat
      {\stripcx{\G}}
      {\tuple{}}
      {\strip_{\Phi\G}(\g)}
      {\strip_{\G}(\rho)}
      {\strip_{\Phi\G}(\g')}
      {\strip_{\G}(\rho')}
  \]

  where $\strip_\G = \fork{\pi_{\dvar x}}_{\hd x A \in \G}$ keeps only the
  discrete variables from a substitution.
\end{restatable}

\begin{restatable}{proof}{ContextStrippingProof}
  Immediate from the definitions. \todolater{be more explicit in the proof of context stripping}
\end{restatable}

\noindent
Jointly, these two lemmas ensure that a valid change to any context is an
identity on the discrete part. We use these in all the cases of the fundamental
theorem involving discrete expressions -- equality $\eeq{e_1}{e_2}$, set
literals $\esetsub{e_i}{i}$, emptiness tests $\eisempty e$, and box
introduction $\ebox e$.

Combining all these lemmas to establish the fundamental theorem, adequacy
follows immediately:

\begin{theorem}[Adequacy]
  If $\J e {\emptycx} {\eqt A}$ then $\den{e} = \den{\phi e}$.
\end{theorem}

\begin{proof}
  By \cref{theorem-seminaive-fundamental}, noting the premise is trivial since
  the context is empty, we have %
  $\weirdat{\eqt A}{\den{\delta e}}{\den{\phi e}}{\den{e}}{\den{\phi e}}{\den{e}}$%
  , which by \cref{lemma-equality-changes} implies $\den{\phi e} = \den{e}$.
\end{proof}


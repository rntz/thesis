\section{Syntax sketch}\label{section syntax sketch}

\newlength\bnfskip\setlength\bnfskip{1ex}
\begin{figure}
  \[
  \begin{array}{rccl}
    \text{types} & A,B &\bnfeq&
    \tunit \bnfor A \x B \bnfor A + B \bnfor A \to B
    \bnfor \iso A \bnfor \tset{\eqt A}
    \\[\bnfskip]
    \text{eqtypes} & \eqt A, \eqt B &\bnfeq&
    \tset{\eqt A} \bnfor
    \tunit \bnfor \eqt A \x \eqt B \bnfor \eqt A + \eqt B
    \\[\bnfskip]
    \text{semilattices} & L,M &\bnfeq& \tset{\eqt A} \bnfor \tunit \bnfor L \x M
    \\[\bnfskip]
    \text{finite eqtypes\cref{note-finite-eqtypes}} & \fint A, \fint B &\bnfeq&
    \tset{\fint A} \bnfor \tunit \bnfor \fint A \x \fint B \bnfor \fint A + \fint B
    \\[\bnfskip]
    \text{fixtypes} & \fixtLkern, \fixt M &\bnfeq&
    \tset{\fint A} \bnfor \tunit \bnfor \fixt L \x \fixt M
    \\[\bnfskip]
    \text{terms} & e,f &\bnfeq& \mvar x \bnfor \dvar x \bnfor \efn x e
    \bnfor e\<f \bnfor \etuple{} \bnfor \etuple{e,f} \bnfor \pi_i\<e
    \\&&&
    \inj i e \bnfor \emcase{e} (\inj i \mvar{x}_i \caseto f_i)_{i\in\{1,2\}}
    \\&&&
    \esetsub{e_i}{i} \bnfor \eforvar x e f
    \\&&&
    \ebox e \bnfor \eletbox x e f
    \\&&&
    \eeq e f \bnfor \eisempty e \bnfor \esplit e
    \\&&&
    \bot \bnfor e \vee f \bnfor \efixis x e
    \\[\bnfskip]
    \text{monotone variables} & \mvar x, \mvar y, \mvar z
    & \multicolumn{2}{l}{\text{are abstract names}}
    \\
    \text{discrete variables} & \dvar x, \dvar y, \dvar z
    & \multicolumn{2}{l}{\text{are abstract names}}
  \end{array}
  \]
  \caption{Datafun syntax}
  \label{figure-syntax}
\end{figure}


\splittodo{refer to syntax figure somewhere}{, \cref{figure-syntax}}

The idea behind Datafun is to capture the essence of Datalog in a typed,
higher-order, functional setting.
%
Since the key restriction that makes Datalog's combination of recursion and
negation tractable -- stratification -- requires tracking \emph{monotonicity,}
we locate Datafun's semantics in the category \Poset\ of partial orders and
monotone maps.
%
Since \Poset\ is cartesian closed, it can interpret the simply typed
\fn-calculus, giving us a notation for writing monotone and higher-order
functions.
%
This lets us \emph{abstract} over Datalog rules, something impossible in Datalog itself!
%
In this section we sketch the construction of Datafun hewing closely to this
semantic intuition.

Datafun begins as the simply-typed \fn-calculus with functions ($\efn x e$ and
$e\<f$), sums ($\inj i e$ and $\emcase{e}{\dots}$), and products ($\etuple{e,f}$
and $\pi_i\<e$).
%
To represent relations, we add a type of finite sets $\tset{\eqt
  A}$,\footnote{To implement set types, their elements must support decidable
equality. In our core calculus, we use a subgrammar of ``eqtypes'', and in our
implementation (which compiles to Haskell) we use typeclass constraints to pick
out such types.} introduced with set literals $\eset{e_0, \ldots e_n}$, and
eliminated using Moggi's monadic bind syntax, $\eforvar{x}{e_1}{e_2}$,
%% %
%% which
%% binds $x$ in $e_2$ and signifies the union over each $x \in e_1$ of $e_2$; in
%% other words, $\bigcup_{x \in e_1} e_2$.
%% %
which binds $x$ in $e_2$ successively to each element of $e_1$ and takes the
union; in other words, $\bigcup_{x \in e_1} e_2$.
%
Since we are working in \Poset, each type comes with a partial order on it; sets
are ordered by inclusion, $x \le y : \tseteq{A} \iff x \subseteq y$.

As long as all primitives are monotone, every definable function is also
monotone. This is necessary for defining fixed points, but may seem too
restrictive. There are many useful non-monotone operations, such as equality
tests $\eeq e f$. For example, $\esetraw{} = \esetraw{}$ is true, but if the
first argument increases to $\esetraw{1}$ it becomes false, a \emph{decrease}
(as we'll see later, in Datafun, $\efalse < \etrue$).

How can we express non-monotone operations if all functions are monotone?
%
We cut this Gordian knot using a \emph{discreteness} type constructor, $\iso A$.
%
The elements of $\iso A$ are the same as those of $A$, but the partial order on
$\iso A$ is discrete, $x \le y : \iso A \iff x = y$.
%
Monotonicity of a function $\iso A \to B$ is vacuous: $x = y$ implies $f(x) \le
f(y)$ by reflexivity.
%
In this way we represent ordinary, possibly non-monotone, functions $A \to B$ as
monotone functions $\iso A \to B$.

Semantically, $\iso$ is a monoidal comonad or necessity modality, and so we base
our syntax on \citet{jrml}'s syntax for the necessity fragment of constructive
S4 modal logic.
%
This involves distinguishing two kinds of variable: discrete variables are in lowercase ($\dvar x, \dvar y, \dvarlong{foo}, \dvarlong{bar}$), while monotone variables are capitals ($\mvar X, \mvar Y, \mvar Z$).
%
Discrete variables obey normal scoping rules, but monotone variables are hidden
within non-monotone expressions.
%
For example, in an equality test $\eeq e f$, the terms $e$ and $f$ cannot refer
to monotone variables bound outside the equality expression.
%
We highlight such expressions with a
\adjustbox{bgcolor=isobg}{\isobgname\ background}.
%
Putting this all together, we construct the type $\iso A$ with the non-monotone
introduction form $\ebox{e}$ and eliminate it by pattern-matching, $\eletbox x e
f$, giving access to a discrete variable $\dvar{x}$.

%% TODO: should we note that we really definitely mean □(L → L) and
%% not (□L → L)? Reader may be confused.

Finally, to express Datalog's recursive queries, Datafun includes a fixed point combinator \prim{fix}, which computes the least fixed point of a map $f$.
%
To ensure termination, this map must be monotone and take a \emph{semilattice type with decidable equality and no infinite ascending chains}, $\fixtLkern$.
%
For our purposes, a semilattice is a partial order with a least element $\bot$ and a least upper bound operation $\vee$ (``join'').
%
Finite sets (with the empty set as least element, and union as join) are an example, as are tuples of semilattices.
%
As long as the semilattice has no infinite ascending chains $x_0 < x_1 < x_2 < \dots$, iteration from the bottom element is guaranteed to find the least fixed point; decidable equality ensures we can tell when this fixed point is reached.%
%
\footnote{We are sweeping a few technical details under the rug here. First, for reasons which will not be explained until \cref{why-is-fix-discrete} we treat \prim{fix} as a non-monotone operator.
%
  \label{note-finite-eqtypes}
  Second, observe that the finite set type $\tseteq{A}$ will possess infinite
  ascending chains if $\eqt A$ has infinitely many inhabitants. Thus we need to
  distinguish a class of \emph{finite} eqtypes $\fint A$. Although their
  grammars in \cref{figure-syntax} are identical, their intent is different. For
  example, if we extended Datafun with integers, they would form an eqtype, but
  not a finite one.}

\todo{mention split and empty, with forward reference to their occurrences in
  examples and full explanation in \cref{section-typing-and-semantics}}

%% \item \todo{it might be better ignore this point for now and only discuss it
%%   when it becomes relevant.} The entire expression $\efixis x e$ is regarded as
%%   non-monotone, and cannot refer to externally-bound monotone variables besides
%%   the loop variable $x$. This may seem unnecessary, as the least-fixed-point
%%   operator is semantically monotone: $\fa{x} f(x) \le g(x)$ implies $f$'s least
%%   fixed point is less than or equal to $g$'s. The restriction is motivated by
%%   concerns that will not become clear until \cref{chapter-seminaive}, and we
%%   introduce it here so that we can discuss one consistent language throughout
%%   this thesis. It has the effect of preventing semantically nested fixed points;
%%   we discuss the kinds of programs this disallows in more detail in
%%   \cref{section-nested-fixed-points}. Note that this does not prevent mutual
%%   recursion, which can be expressed by taking a fixed point at product type, nor
%%   stratified fixed points \`a la Datalog.


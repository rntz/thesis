\section{Syntax}\label{syntax}

%% TODO: edit/rewrite this section
%% TODO: this figure
%\input{fig-syntax}

\todo{TODO: this section needs to be edited and/or rewritten. Also, the fixed
  point operator should probably be made a genuine binding operator rather than
  a higher-order function. This should simplify the story about optimization of
  the phi transformation.}

The idea behind Datafun is to capture the essence of Datalog in a typed,
higher-order, functional setting.
%
Since the key restriction that makes Datalog tractable -- stratification --
requires tracking \emph{monotonicity,} we locate Datafun's semantics in the
category \Poset\ of partial orders and monotone maps.
%category \Poset\ of partial orders and monotone maps.
%
Since \Poset\ is cartesian closed, it can interpret the simply typed
\fn-calculus, giving us a notation for writing monotone and higher-order
functions.
%
This lets us \emph{abstract} over Datalog rules, something not
possible in Datalog itself!
%
In this section we sketch the construction of Datafun hewing closely to this
semantic intuition.

Datafun begins as the simply-typed \fn-calculus with functions ($\fnof x e$ and
$e\<f$), sums ($\inj i e$ and $\ecase{e}{\dots}$), and products ($\etuple{e,f}$
and $\pi_i\<e$).
%
To represent relations, we add a type of finite sets $\tset{\eqt
  A}$,\footnote{To implement set types, their elements must support decidable
  equality. In our core calculus, we use a subgrammar of ``eqtypes'', and in our
  implementation (which compiles to Haskell) we use typeclass constraints to
  pick out such types.} introduced with set literals $\eset{e_0, \ldots e_n}$, and
eliminated using Moggi's monadic bind syntax, $\efor{x}{e_1}{e_2}$, signifying
the union over all $\dvar x \in e_1$ of $e_2$.
%
Since we are working in \Poset, each type comes with a partial order on it; sets
are ordered by inclusion, $x \le y : \tseteq{A} \iff x \subseteq y$.

As long as all primitives are monotone, every definable function is also
monotone. This is necessary for defining fixed points, but may seem too
restrictive. There are many useful non-monotone operations, such as equality
tests $\eeq e f$. For example, $\esetraw{} = \esetraw{}$ is true, but if the
first argument increases to $\esetraw{1}$ it becomes false, a \emph{decrease}
(as we'll see later, in Datafun, $\efalse < \etrue$).

How can we express non-monotone operations if all functions are monotone?
%
We square this circle by introducing the \emph{discreteness} type constructor,
$\iso A$.
%
The elements of $\iso A$ are the same as those of $A$, but the partial order on
$\iso A$ is discrete, $x \le y : \iso A \iff x = y$.
%
Monotonicity of a function $\iso A \to B$ is vacuous: $x = y$ implies $f(x) \le
f(y)$ by reflexivity!
%
In this way we represent ordinary, possibly non-monotone, functions $A \to B$ as
monotone functions $\iso A \to B$.

Semantically, $\iso$ is a monoidal comonad or necessity modality, and so we base
our syntax on \citet{jrml}'s syntax for the necessity fragment of constructive
S4 modal logic.
%
This involves distinguishing two kinds of variable: discrete variables $\dvar x$
are in \emph{\isocolor\isocolorname\ italics}, while monotone variables $x$ are
in $upright\ black\ script$.
%
Discrete variables may be used wherever they're in scope, but crucially,
monotone variables are hidden within non-monotone expressions.
%
For example, in an equality test $\eeq e f$, the terms $e$ and $f$ cannot refer
to monotone variables bound outside the equality expression.
%
We highlight such expressions with a
\adjustbox{bgcolor=isobg}{\isobgname\ background}.
%
Putting this all together, we construct the type $\iso A$ with the non-monotone
introduction form $\ebox{e}$ and eliminate it by pattern-matching, $\eletbox x e
f$, giving access to a discrete variable $\dvar{x}$.

%% TODO: should we note that we really definitely mean □(L → L) and
%% not (□L → L)? Reader may be confused.

\newcommand\isofixLtoL{\iso(\kernfixtL \to \fixtLkern)}

Finally, Datafun includes fixed points, $\efix{f}$. The \prim{fix} combinator takes a function $\isofixLtoL$ and returns its least fixed point.
%
Besides monotonicity of the function, we impose two restrictions on the fixed point operator to ensure well-definedness and termination.
%
First, we require that recursion occur at \emph{semilattice types with no infinite ascending chains}, $\fixtLkern$.
%
A join-semilattice is a partial order with a least element $\bot$ and a least upper bound operation $\vee$ (``join'').
%
Finite sets (with the empty set as least element, and union as join) are an example, as are tuples of semilattices.
%
As long as the semilattice has no infinite ascending chains $x_0 < x_1 < x_2 < \cdots$, iteration from the bottom element is guaranteed to find the least fixed point.\footnotemark

\footnotetext{\label{note:finite-eqtypes}As a technical detail, the finite set
  type $\tseteq{A}$ will possess infinite ascending chains if $\eqt A$ has
  infinitely many inhabitants. Thus we need to distinguish a class of
  \emph{finite} eqtypes $\fint A$. Although their grammars in \cref{fig:syntax}
  are identical, their intent is different. For example, if we extended Datafun
  with integers, they would form an eqtype, but not a finite one.}

Second, we require that the recursive function be boxed, $\isofixLtoL$. Since boxed expressions can only refer to discrete values, and fixed point
functions themselves must be monotone, this has the effect of preventing
semantically nested fixed points. We discuss this in more detail in
\cref{sec:nested-fixed-points}. Note that this does not prevent mutual
recursion, which can be expressed by taking a fixed point at product type, nor
stratified fixed points \`a la Datalog.


\section{Operational semantics}

\begin{figure*}
  \centering

  \textsc{additional syntax}
  \begin{displaymath}
    \begin{array}{rccl}
      \text{expressions} & e,f,g &\bnfeq&
      ... \bnfor \eeqraw[\eqt A]{e}{f}
      \bnfor \bot_L \bnfor e \vee_L f \bnfor \efor[L]{x \in e} f
      \bnfor \prim{fix}_{\fixt L} \<e\\
      &&& \eiter[\eqt A]{v}{e}{f}
      \\[\bnfskip]
      \text{values} & v,u &\bnfeq&
      \fnof{x} e \bnfor \etuple{} \bnfor \etuple{v, u} \bnfor \inj i v
      \bnfor \esetraw{{v}_i}_i \bnfor \eboxraw{v}
      \\[\bnfskip]
      \text{evaluation contexts} & E &\bnfeq&
      \emptyhole \bnfor E\<e \bnfor v\<E \bnfor \etuple{E, e}
      \bnfor \etuple{v, E} \bnfor \pi_i\< E\\
      &&& \inj i E \bnfor \emcase{E}{(\inj i x_i \caseto e_i)_i}\\
      &&& \eboxraw{E} \bnfor \eletbox{x}{E}{e}\\
      &&& \esetraw{\vec v,\, E,\, \vec e}
      \bnfor E \vee_L e \bnfor v \vee_L E \bnfor \efor[L]{x \in E} e\\
      &&& \eeqraw[A]{E}{e} \bnfor \eeqraw[A]{v}{E}
      \bnfor \esplit E \bnfor \eisemptyraw E\\
      &&& \prim{fix}_{\fixt L} \<E
      \bnfor \eiter[A]{v}{E}{f} \bnfor \eiter[A]{v}{u}{E}
    \end{array}
  \end{displaymath}
  \vspace{0pt}

  \textsc{value (in)equality}
  \begin{mathpar}
    \infer{}{\etuple{} \le \etuple{} : \tunit}

    \infer{v_1 \le u_1 : \eqt A \\ v_2 \le u_2 : \eqt B}{%
      \etuple{v_1,\, v_2} \le \etuple{u_1,\, u_2} : \eqt A \x \eqt B}

    \infer{v \le u : A_i}{\inj i v \le \inj i u : A_1 + A_2}

    \infer{\faex{i}{j} v_i = u_j : \eqt A}{
      \esetraw{v_i}_i \le \esetraw{u_j}_j : \tset{\eqt A}
    }

    \infer{v \le u : \eqt A\\ u \le v : \eqt A}{v = u : \eqt A}
  \end{mathpar}
  \vspace{2pt}

  \textsc{$\beta$ reductions}
  \begin{align*}
    (\fnof{x}e)\<v &~\stepsto~ \subone{e}{x}{v}
    &
    \eletbox{x}{\eboxraw{v}}{e} &~\stepsto~ \subone{e}{x}{v}
    \\
    \pi_i\<\etuple{v_1, v_2} &~\stepsto~ v_i
    &
    \emcase{\inj i v}{(\inj j x_j \caseto e_j)_j}
    &~\stepsto~ \subone{e_i}{x_i}{v}
    \\
    \efor[L]{x \in \esetraw{}} e &~\stepsto~ \bot_L
    \\
    \efor[L]{x \in \esetraw{v, \vec u}} e
    &\omit\rlap{$~\stepsto~
      (\subone{e}{x}{v}) \vee_L \efor[L]{x \in \esetraw{\vec u}} e
    $}
  \end{align*}
  \vspace{0pt}

  \textsc{other reductions}
  \begin{align*}
    \bot_{\tset{A}} &~\stepsto~ \esetraw{}
    &
    \bot_{L \x M} &~\stepsto~ \etuple{\bot_L,\,\bot_M}
    \\
    \esetraw{\vec v} \vee_{\tset A} \esetraw{\vec u}
    &~\stepsto~ \esetraw{\vec v,\, \vec u}
    &
    \etuple{v_1, v_2} \binvee_{L \x M} \etuple{u_1, u_2}
    &~\stepsto~ \etuple{v_1 \vee_L u_1,\, v_2 \vee_M u_2}
    \\
    \eisemptyraw{\esetraw{}} &~\stepsto~ \inj 1 \etuple{}
    &
    \eisemptyraw{\esetraw{v,\vec u}} &~\stepsto~ \inj 2 \etuple{}
    \\
    \esplit{\eboxraw{\inj i v}} &~\stepsto~ \inj i {\eboxraw v}
    &
    \eeqraw[\eqt A]{v}{u}
    &~\stepsto~
    \begin{cases}
      \etrue & \text{if}~ v = u : \eqt A\\
      \efalse & \text{otherwise}
    \end{cases}
    \\
    \prim{fix}_{\fixt L} \<v
    &\omit\rlap{$~\stepsto~
      \eiter[\fixt L]{v}{\bot_{\fixt L}}{v \<\bot_{\fixt L}}
    $}
    \\
    \eiter[\eqt A]{v}{u_1}{u_2}
    &\omit\rlap{$~\stepsto~
    \begin{cases}
      u_1 & \text{if}~ u_1 = u_2 : \eqt A\\
      \eiter[\eqt A]{v}{u_2}{v\<u_2} & \text{otherwise}
    \end{cases}$}
  \end{align*}

  \caption{Operational semantics}
  \label{figure-operational-semantics}
\end{figure*}


We consider the denotational semantics to be primary in Datafun; as with
Datalog, any implementation technique is valid so long as it lines up with this
semantics.
%
To show such an implementation is possible, we present a simple call-by-value
structural operational semantics in \cref{figure-operational-semantics} and show that all well-typed terms terminate.
%
In our operational semantics we:

\begin{enumerate}
\item Drop the distinction between discrete and monotone variables, writing both
  in lowercase $x,y,z$, and cease using a \isobgname\ background for
  non-monotone expressions.
\item Assume all equality tests and all semilattice operations ($\bot$, $\vee$,
  $\kw{for}$, and $\prim{fix}$) are subscripted with their type.
\item Add $\prim{iter}$ expressions, which occur as intermediate forms in the
  evaluation of $\prim{fix}$.
\end{enumerate}

\noindent
We use a small-step operational semantics with evaluation contexts
$E$~\citep{felleisen-hieb-1992} to enforce a call-by-value evaluation order; an
evaluation context $E$ is an expression with a hole in it, written $\emptyhole$,
such that whatever is in the hole is next in line to be evaluated (if it is not
a value already). To fill the hole in an evaluation context $E$ with the
expression $e$, we write $\fillhole{E}{e}$.

We define a relation $e \stepsto e'$ for expressions $e$ whose outermost
structure is immediately reducible; we extend this relation to all
expressions with the rule:

\[
\infer[eval context]{e\stepsto e'}{\fillhole E e \stepsto \fillhole E{e'}}
\]

\noindent
%
In our rules for $e \stepsto e'$ where $e$ is an \prim{iter} expression we make
use of a decidable ordering test on values, $v \le u : \eqt A$, and a
corresponding equality test $v = u : \eqt A$. We define these using
inference rules, but they are easily seen to be decidable by induction on $\eqt
A$.

%\subsection{Computing fixed points by iteration}

Our implementation strategy for $\efix f$ is
% \todo{suggested by the proof of \cref{lemma-fixed-point}:}
straightforward: starting from $\bot$, iteratively apply $f$ until quiescence.
%
We introduce the form $\eiter{f}{e_1}{e_2}$ to represent these intermediate
iterative steps. The intention is that $e_1,e_2$ shall be successive iterations
of $f$, with $e_2 = f\<e_1$.
%
The fixed point expression $\efix f$, after evaluating $f$, steps to
$\eiter{f}{\bot}{f\<e}$, which kicks off the first two iterations.
%
Once these have reduced to values, $\eiter f {u_1}{u_2}$ tests $u_1 = u_2$ to
determine if a fixed point has been reached. If so, its value $u_1$ is returned;
otherwise we step to $\eiter{f}{u_2}{f\<u_2}$ to evaluate the next iteration,
and so on.

Observe that values don't step and evaluation is deterministic:

\begin{lemma}[Values don't step]\label{lemma-values-don't-step}
  If $v$ is a value, there is no $e$ such that $v \stepsto e$.
\end{lemma}

\begin{proof}
  The left hand side of each reduction rule can never be a value. This is easily verified by inspection for the rules in \cref{figure-operational-semantics}; and for \rn{eval context} we can see by the definition of evaluation contexts $E$ that filling a hole with a non-value always produces a non-value.
\end{proof}

\begin{lemma}[Determinism]
  If $e \stepsto e'_1$ and $e \stepsto e'_2$ then $e'_1 = e'_2$; thus inductively, since values don't step, if $e \evalsto v$ and $e \evalsto u$ then $v = u$.
\end{lemma}

\begin{proof}
  The left-hand sides of all reduction rules $e \stepsto e'$ are disjoint; there is no term to which two distinct reduction rules could apply. This applies inductively to \rn{eval context} because decompositions of a term into an evaluation context and a reducible subterm are unique. \todolater{why?}
\end{proof}


\subsection{A logical relation for termination}

%% \todolater{extend this to handle adequacy}

\newcommand\oplr[3]{{#2} \prec {#3} : {#1}}
\newcommand\opequiv[3]{{#2} \equiv {#3} : {#1}}
\newcommand\oplrvalue[3]{\oplr{V(#1)}{#2}{#3}}
\newcommand\oplrvalueequiv[3]{\opequiv{V(#1)}{#2}{#3}}
\newcommand\oplrclosed[3]{\oplr{#1}{#2}{#3}}
\newcommand\oplrclosedequiv[3]{\opequiv{#1}{#2}{#3}}

%% \renewcommand\oplrvalue[3]{{#2} \prec_{\textsf{V}} {#3} : {#1}}
%% \renewcommand\oplrvalueequiv[3]{{#2} \equiv_{\textsf{V}} {#3} : {#1}}

\renewcommand\oplrvalue[3]{{#2} \prec {#3} : {#1}}
\renewcommand\oplrvalueequiv[3]{{#2} \equiv {#3} : {#1}}
\renewcommand\oplrclosed[3]{{#2} \prec {#3} : {#1}}
\renewcommand\oplrclosedequiv[3]{{#2} \equiv {#3} : {#1}}

To prove that all well-typed terms terminate according to our operational
semantics, we use a logical relations argument.
%
As a matter of notation, we will let $v,u,w$ range over values; $a,b,c$ range over closed terms; and $\gamma,\sigma$ range over closing substitutions.

Our guiding intuition is that since we need an order structure in our
denotational semantics to prove the definedness of fixed points, we likewise
need an order structure on our syntax to prove the termination of fixed points.
%
To this end we interpret each type $A$ as a \emph{partial preorder,} $\oplr A x y$.
%
A partial preorder is a relation which is transitive and \emph{partially reflexive,} that is, $x \prec y \implies x \prec  x \wedge y \prec y$. While reflexivity may be glossed as ``every element is related to itself'', partial reflexivity glosses as ``if an element is related to anything, it is related to itself''; in other words, unlike reflexivity, it permits some elements to be ``outside the relation'' and unrelated to anything, even themselves.
%
%% Another way of seeing this is that a partial preorder over a set $S$ is equivalent to a subset $\setfor{x \in S}{x \prec x}$ equipped with an ordinary preorder.
%
Any partial preorder $x \prec y$ gives rise to a symmetric, transitive relation $x \equiv y \iff x \prec y \wedge y \prec x$.\footnotemark

\footnotetext{Symmetric, transitive relations are also known as partial equivalence relations (PERs). Moreover, letting $[x]_\equiv$ denote the equivalence class of $x$ (defined only when $x \prec x$), the relation $[x]_\equiv \le [y]_\equiv \iff x \prec y$ is a partial order over these equivalence classes; so our approach may also be considered to interpret types as PERs equipped with partial orders on their equivalence classes.}

In fact we define a mutually inductive collection of partial preorders: on values $\oplrvalue A v u$, an extension to closed terms $\oplrclosed A a b$, on closing substitutions $\oplr \G \gamma \sigma$, on open terms $\oplr {\Gamma \vdash A} e f$, and on open terms paired with closing substitutions $\oplr{\Gamma,A}{\gamma_1,e_1}{\gamma_2,e_2}$. The rules for the value-relation are:
%
\begin{mathpar}
  \infer{~}{\oplrvalue{\tunit}{\etuple{}}{\etuple{}}}

  \infer{
    \oplrvalue A {v_1}{u_1} \\ \oplrvalue B {v_2} {u_2}
  }{
    \oplrvalue{A \x B}{\etuple{v_1,v_2}}{\etuple{u_1,u_2}}
  }

  \infer{
    \oplrvalue{A_i} v u
  }{
    \oplrvalue{A_1 + A_2}{\inj i v}{\inj i u}
  }

  \infer[lr~fn]{
    \oplr{(\hm x A \vdash B)} e f
  }{
    \oplrvalue{A \to B}{\efn x e}{\efn x f}
  }

  \infer{\oplrvalueequiv A v u}{\oplrvalue{\iso A}{\eboxraw{v}}{\eboxraw{u}}}

  \infer[lr~set]{
    \faex{i}{j} \oplrvalueequiv{\eqt A}{v_i}{u_j}
    \\
    \fa{j} \oplrvalue{\eqt A}{u_j}{u_j}
  }{
    \oplrvalue{\tseteq A}{\esetraw{v_i}_i}{\esetraw{u_i}_i}
  }
\end{mathpar}

\noindent
Note that \rn{lr~fn} depends on the relation for open terms, making this definition mutually inductive. The second premise of \rn{lr~set} may seem strange but it is necessary to ensure partial reflexivity. We extend this value-relation to closed terms:

\begin{align*}
  \oplrclosed{A} a b
  &\iff
  \ex{v,u} a \evalsto v \wedge b \evalsto u \wedge \oplrvalue A v u
  \label{equation-lr-closed}
\end{align*}

\noindent
Note that if $a,b$ are values, this definition coincides with the relation on values, since values do not step; this justifies using the same notation for the relation on values and closed terms.
%
We extend this relation to closing substitutions pointwise, noting that discrete hypotheses are required to be equivalent:

\begin{align*}
  \oplr{\Gamma} \gamma \sigma
  &\iff
  (\fa{\hm x A \in \G} \oplrclosed A {\gamma_x}{\sigma_x})
  \wedge (\fa{\hd x A \in \G} \oplrclosedequiv A {\gamma_x}{\sigma_x})
\end{align*}

\noindent
Finally, we extend the relation to open terms, which involves an auxiliary relation on pairs of terms and closing substitutions:

\begin{align*}
  \oplr{\Gamma \vdash A}{e_1}{e_2}
  &\iff
  \fa{\oplr \Gamma {\gamma_1}{\gamma_2}}
  \oplr{\Gamma,A}{\gamma_1,e_1}{\gamma_2,e_2}
  \\
  \oplr{\Gamma,A}{\gamma_1,e_1}{\gamma_2,e_2}
  &\iff
  \fa{i = 1,2}
  \oplrclosed A {\gamma_i(e_1)}{\gamma_i(e_2)}
  \wedge \oplrclosed A {\gamma_1(e_i)}{\gamma_2(e_i)}
\end{align*}

\newcommand{\precright}{%
  \arrow[no line]{r}{\rotatebox[origin=c]{0}{\scalebox{1.4}{$\prec$}}}}
\newcommand{\precdown}{%
  \arrow[no line]{d}{\rotatebox[origin=c]{-90}{\scalebox{1.4}{$\prec$}}}}

\noindent
Note that $\oplr{\Gamma,A}{\gamma_1,e_1}{\gamma_2,e_2}$ may be seen as a transitive square:

\begin{center}
  \tikzset{
    no line/.style={draw=none,
      commutative diagrams/every label/.append style={/tikz/auto=false}}}
  {\begin{tikzcd}
      \gamma_1(e_1) \precright \precdown & \gamma_1(e_2) \precdown\\
      \gamma_2(e_1) \precright & \gamma_2(e_2)
    \end{tikzcd}}
\end{center}

\noindent
This ensures partial reflexivity; if we replace $e_1$ with $e_2$ or vice-versa this square collapses to one of its sides. If we had instead only required the diagonal, $\oplrclosed A {\gamma_1(e_1)}{\gamma_2(e_2)}$, we could not derive $\oplrclosed A {\gamma_1(e_1)} {\gamma_2(e_1)}$ (or the same for $e_2$) as required by partial reflexivity.

\begin{restatable}[Fundamental theorem]{theorem}{TerminationFundamental}
  \label{theorem-termination-fundamental}
%\begin{theorem}[Fundamental theorem]
  If $\J e \Gamma A$ then $\oplr{\Gamma \vdash A} e e$.
\end{restatable}

\noindent
Termination of well-typed programs follows as a corollary by unrolling definitions:

\begin{restatable}[Termination]{theorem}{Termination}
  Every closed, well-typed program $\J{a}{\emptycx}{A}$ terminates.
\end{restatable}

\begin{proof}
\begin{align*}
  & \J{a}{\emptycx}{A}
  \\
  \implies& \oplr{\emptycx \vdash A} a a
  && \text{Fundamental theorem}
  \\
  \implies& \fa{\oplr{\emptycx}{\gamma_1}{\gamma_2}}
  \oplr{\Gamma,A}{\gamma_1,a}{\gamma_2,a}
  && \text{expand the definition}
  \\
  \implies& \oplr{\emptycx,A}{\tuple{},a}{\tuple{},a}
  && \text{since }\oplr{\emptycx}{\tuple{}}{\tuple{}}\text{ vacuously}
  \\
  \implies& \oplrclosed{A} a a
  && \text{expand the definition and simplify}
  \\
  \implies& \ex{v,u} a \evalsto v \wedge a \evalsto u \wedge \oplrvalue A v u
  && \text{expand the definition}
  \\
  \implies& \ex{v} a \evalsto v
  && \text{simplify}
\end{align*}
\end{proof}

\noindent
The proof of the fundamental theorem itself proceeds by induction on $\J e \Gamma A$.
%
The key case is the fixed point rule, whose proof is a syntactic version of
the proof of the existence of least fixed points in the denotational semantics.
%
%% We therefore give the fixed point case here, and leave the other cases to the appendix.
%
We give the proof of the fundamental theorem at the end of this section; to
build up to it we must first develop several auxiliary definitions and lemmas.


\subsection{Metatheory of the logical relation}

\setlist[description]{
  topsep=\parsep,
  font={\mdseries\itshape},
}

\newcommand\goodvalue[1]{\ensuremath{\textsf{Ok}_{\textsf{V}}({#1})}}
\newcommand\goodclosed[1]{\ensuremath{\textsf{Ok}_{\textsf{C}}({#1})}}
\newcommand\goodopen[1]{\ensuremath{\textsf{Ok}({#1})}}

First, any partial preorder over a set $S$ gives rise to a subset $\setfor{x \in S}{x \prec x}$ over which $\prec$ is reflexive and thus a true preorder. It will be convenient to apply this point of view to our logical relations:

\begin{definition}[Good terms] We define the following preordered sets of ``good'' terms:
  \begin{align*}
    \goodvalue{A} &= \setfor{v}{\oplrvalue A v v}
    &&\text{the good values}
    \\
    \goodclosed{A} &= \setfor{a}{\oplrclosed A a a}
    && \text{the good closed terms}
    \\
    \goodopen{\Gamma \vdash A} &= \setfor{e}{\oplr{\Gamma \vdash A} e e}
    && \text{the good open terms}
  \end{align*}

  \noindent
  We preorder these by the corresponding logical relation, so $v \le u : \goodvalue A \iff \oplrvalue A v u$, etc.
\end{definition}

%% %% The fundamental theorem says that all well-typed terms are good.
%% %% %
%% We introduce these sets primarily to define certain functions on them. For instance, the evaluation and application maps:

\begin{lemma}[Closed term evaluation map]\label{lemma-closed-term-evaluation}
  There exists a monotone map $\name{value}_A : \goodclosed A \to \goodvalue A$ such that $\name{value}\<a = v$ if and only if $a \evalsto v$.
\end{lemma}

\begin{proof}
  The definition of $\oplrclosed A a a$ shows that every good closed term evaluates to some good value. Determinism shows this value is unique, so we can name it $\name{value}\<a$. And given $\oplrclosed A a b$, applying its definition and this uniqueness shows that $\oplrvalue A {\name{value}\<a} {\name{value}\<b}$, showing monotonicity.
\end{proof}

\begin{lemma}[Closed term application map]\label{lemma-closed-term-application}
  If $\oplrclosed{A \to B}{a_1}{a_2}$ and $\oplrclosed A {b_1}{b_2}$ then $\oplrclosed B {a_1\<b_1} {a_2\<b_2}$.
%
  Equivalently, there exists a monotone map $\name{apply}_{A,B} : \goodclosed{A \to B} \times \goodclosed A \to \goodvalue B$ such that $\name{apply}\<(a,b) = \name{value}\<(a\<b) = v$ if and only if $a\<b \evalsto v$.
\end{lemma}

\begin{proof}
  Suppose $\oplrclosed{A \to B}{a_1}{a_2}$ and $\oplrclosed A {b_1}{b_2}$.
%
  %% It suffices to find some $v_1,v_2$ such that $a_i\<b_i \evalsto v_i$ and $\oplrvalue B {v_1}{v_2}$.
%
  Unrolling these assumptions, we have $e_1,e_2,u_1,u_2$ satisfying:

  \begin{align*}
    a_1 &\evalsto \efn x e_1
    &
    a_2 &\evalsto \efn x e_2
    &
    e_1 &\prec e_2 : (\hm x A) \vdash B
    \\
    b_1 &\evalsto u_1
    &
    b_2 &\evalsto u_2
    &
    u_1 &\prec u_2 : A
  \end{align*}

  \noindent
%
  From $\oplrvalue A {u_1}{u_2}$ we have $\oplr{(\hm x A)}{(\subto{\mvar x}{u_1})}{(\subto{\mvar x}{u_2})}$, and applying this to $\oplr{(\hm x A) \vdash B}{e_1}{e_2}$ we have a transitive square:

  \begin{center}
    \tikzset{
      no line/.style={draw=none,
        commutative diagrams/every label/.append style={/tikz/auto=false}}}
            {\begin{tikzcd}
                \subone{e_1}{\mvar x}{u_1} \precright \precdown
                &
                \subone{e_2}{\mvar x}{u_1} \precdown\\
                \subone{e_1}{\mvar x}{u_2} \precright
                &
                \subone{e_2}{\mvar x}{u_2}
            \end{tikzcd}}
  \end{center}

  \noindent
  Taking the diagonal of this square, we have $\oplrclosed B {\subone{e_1}{\mvar x}{u_1}} {\subone{e_2}{\mvar x}{u_2}}$ and therefore $v_1,v_2$ such that $\subone{e_i}{\mvar x}{u_i} \evalsto v_i$ and $\oplrvalue B {v_1} {v_2}$.
%
  Thus for $i \in \{1,2\}$ we have:

  \begin{align*}
    a_i\<b_i
    \evalsto (\efn x e_i)\<b_i
    \evalsto (\efn x e_i)\<u_i
    \stepsto \subone{e_i}{\mvar x}{u_i}
    \evalsto v_i
  \end{align*}

  \noindent
  and $\oplrvalue B {v_1}{v_2}$ as desired.
\end{proof}

\begin{lemma}[Closed term pairing]
  If $\oplrclosed A {a_1} {a_2}$ and $\oplrclosed B {b_1} {b_2}$ then $\oplrclosed{A \times B}{\etuple{a_1,b_1}}{\etuple{a_2,b_2}}$.
\end{lemma}

\begin{proof}
  Applying our assumptions' definitions we have $v_1,v_2,u_1,u_2$ such that:

  \begin{align*}
    a_1 &\evalsto v_1
    &
    a_2 &\evalsto v_2
    &
    v_1 &\prec v_2 : A
    \\
    b_1 &\evalsto u_1
    &
    b_2 &\evalsto u_2
    &
    u_1 &\prec u_2 : B
  \end{align*}

  \noindent
  From this we have:

  \begin{align*}
    \etuple{a_1, b_1} \evalsto \etuple{v_1, b_1} &\evalsto \etuple{v_1, u_1}
    &
    \etuple{a_2, b_2} \evalsto \etuple{v_2, b_2} &\evalsto \etuple{v_2, u_2}
  \end{align*}

  \noindent
  And $\oplrvalue{A \times B}{\etuple{v_1, u_1}}{\etuple{v_2, u_2}}$ because $\oplrvalue A {v_1} {v_2}$ and $\oplrvalue B {u_1} {u_2}$, which is what we wished to show.
\end{proof}

\begin{lemma}[Closure under stepping]
  $\prec$ is closed under $\evalsto$; that is, if $\oplrclosed A a b$ and $(a' \evalsto a) \vee (a \evalsto a')$ and $(b' \evalsto b) \vee (b \evalsto b')$, then $\oplrclosed A {a'} {b'}$.
\end{lemma}

\begin{proof}
  $a \evalsto v$ and $b \evalsto u$ such that $\oplrvalue A v u$ so by determinism $a' \evalsto v$ and $b' \evalsto u$, thus $\oplrclosed A {a'}{b'}$.
\end{proof}

%% Finally, we need to establish that $\bot_L$ is a least element of the preorder $\goodclosed L$ at every semilattice type $L$:

%% Finally, we need a lemma correlating various orderings on values of equality types. This will let us ``port'' the ascending chain condition from our semantics to our syntax.

%% \begin{lemma}[Value ordering]\label{lemma-value-ordering}
%%   If $\oplrvalue{\eqt A} v u$ then $\J v {\emptycx}{\eqt A}$ and $\J u {\emptycx}{\eqt A}$ and $\den{v} \le \den{u} : \den{\eqt A}$ and  $v \le u : \eqt A$. (To be precise, by $\den v : \den{\eqt A}$ we mean the map $\den{\J v \emptycx {\eqt A}} : \Poset(\den\emptycx,\den{\eqt A})$ applied to the empty environment $() : \den{\emptycx}$.)
%% \end{lemma}

\begin{lemma}[First-order agreement on values]\label{lemma-value-ordering}
  If $v \in \goodvalue{\eqt A}$ then $\J v {\emptycx}{\eqt A}$ and moreover for $v,u \in \goodvalue{\eqt A}$:

  \[
  \oplrvalue{\eqt A} v u \iff \den{v} \le \den{u} : \den{\eqt A} \iff v \le u : \eqt A
  \]

  \noindent
  (To be precise, by $\den v : \den{\eqt A}$ we mean the map $\den{\J v \emptycx {\eqt A}} : \Poset(\den\emptycx,\den{\eqt A})$ applied to the empty environment $() : \den{\emptycx}$.)
\end{lemma}

\begin{proof}
  By induction on $\eqt A$. To show $\J v \emptycx {\eqt A}$, in each case we apply the definition of $\oplrvalue{\eqt A} v v$ (e.g.\ for $\eqt A_1 \times \eqt A_2$ we find that $v = \etuple{u_1,u_2}$ for some $u_1,u_2 \in \goodvalue{\eqt A_1}$) followed by applying our inductive hypotheses and the following typing rules (here specialized to $\Gamma = \emptycx$):
%
  \begin{mathpar}
    \infer[\rn{unit}]{\quad}{\J {\etuple{}} \emptycx \tunit}

    \infer[\rn{pair}]{(\J{e_i}\emptycx{A_i})_i}{\J{\etuple{e_1,e_2}} \emptycx {A_1 \x A_2}}

    \infer[\rn{inj}]{\J e \emptycx A_i}{\J{\inj i e}\emptycx{A_1 + A_2}}

    \infer[\rn{set}]{(\J {e_i} {\emptycx} {\eqt A})_i}{
      \J {\esetsub{e_i}{i}} \emptycx {\tset{\eqt A}}}
  \end{mathpar}

  \noindent
  As for equivalence of the orderings, observe that:

  \begin{align*}
    \den{\etuple{}} &= \tuple{} : \den{\tunit}
    &
    \den{\etuple{v,u}} &= \tuple{\den{v},\den{u}} : \den{\eqt A \times \eqt B}
    \\
    \den{\inj i v} &= \inj i \den{v} : \den{\eqt A + \eqt B}
    &
    \den{\esetraw{v_i}_i} &= \{\den{v_i}\}_i : \den{\tseteq A}
  \end{align*}

  \noindent
  and therefore:

  \begin{align*}
    \den{\etuple{}} \le \den{\etuple{}} : \den{\tunit} &\iff \top
    \\
    \den{\etuple{v_1,u_1}} \le \den{\etuple{v_2,u_2}} : \den{\eqt A \times \eqt B}
    &\iff
    \den{v_1} \le \den{v_2} : \den{\eqt A}
    \wedge
    \den{u_1} \le \den{u_2} : \den{\eqt B}
    \\
    \den{\inj i v} \le \den{\inj j u} : \den{\eqt A_1 + \eqt A_2}
    &\iff
    i = j \wedge \den{v} \le \den u : \den{\eqt A_i}
    \\
    \den{\esetraw{v_i}_i} \le \den{\esetraw{u_j}_j} : \den{\tseteq A}
    &\iff
    \{\den{v_i}\}_i \subseteq \{\den{u_j}\}_j
    \iff
    %% \{\den{v_i}\}_i \le \{\den{u_j}\}_j : \pfinof{\den{\tseteq A}}
    \faex i j \den{v_i} = \den{u_j} : \den{\eqt A}
  \end{align*}

  \noindent
  In each case, these coincide (after applying our inductive hypothesis) with the rules defining $\oplrvalue{\eqt A} v u$ and $v \le u : \eqt A$, except for $\oplrvalue{\tseteq A}{\esetraw{v_i}_i}{\esetraw{u_j}_j}$, which has the additional premise $\fa j \oplrvalue{\eqt A}{u_j}{u_j}$; but this is satisfied by the assumption $\esetraw{u_j}_j \in \goodvalue{\tseteq A}$. Thus inductively all three preorders coincide on good values of equality types.
%
  \todo{discuss base types}

  %% \begin{description}
  %% \item[Case $\tunit$:] Trivial.

  %% \item[Case $\eqt{A} \times \eqt{B}$:]
  %%   Assume $\oplrvalue{\eqt A \times \eqt B}{\etuple{v_1,u_1}}{\etuple{v_2,u_2}}$. By definition we have $\oplrvalue{\eqt A}{v_1}{v_2}$ and $\oplrvalue{\eqt B}{u_1}{u_2}$ and by IH we have $\J{v_i}{\emptycx}{\eqt A}$ and $\J{u_i}{\emptycx}{\eqt B}$, which shows $\J{(v_i,u_i)}{\emptycx}{\eqt A \times \eqt B}$; and $\den{v_1} \le \den{v_2} : \den{\eqt A}$ and $\den{u_1} \le \den{u_2} : \den{\eqt B}$, which shows $\den{\etuple{v_1,u_1}} \le \den{\etuple{v_2,u_2}} : \den{\eqt A \times \eqt B}$; and $v_1 \le v_2 : \eqt A$ and $u_1 \le u_2 : \eqt B$, which show $\etuple{v_1,u_1} \le \etuple{v_2,u_2} : \eqt A \times \eqt B$.

  %% \item[Case ${\eqt A}_1 + \eqt{A}_2$:] Assume $\oplrvalue{\eqt A_1 + \eqt A_2}{\inj i a}{\inj i b}$ for some $i \in \{1,2\}$. By definition we have $\oplrvalue{A_i}{a}{b}$. Applying our IH we have $\J a \emptycx {A_i}$ and $\J b \emptycx {A_i}$ which shows $\J{\inj i a} \emptycx {A_1 + A_2}$ and $\J{\inj i b} \emptycx {A_1 + A_2}$; and $\den a \le \den b : \den{\eqt A_i}$ which shows $\den{\inj i a} \le \den{\inj i b} : \den{\eqt A_1 + \eqt A_2}$; and $a \le b : \eqt A_i$ which shows $\inj i a \le \inj i b : \eqt A_1 + \eqt A_2$.

  %% \item[Case $\tseteq A$:]
  %%   Assume $\oplrvalue{\tseteq A}{\esetraw{v_i}_i}{\esetraw{u_j}_j}$, and thus $\faex i j \oplrvalueequiv{\eqt A}{v_i}{u_j}$ and $\fa j \oplrvalue{\eqt A}{u_j}{u_j}$. Applying our IH we have $\fa i \J{v_i}\emptycx{\eqt A}$ and $\fa j \J{u_j}\emptycx{\eqt A}$, which shows $\J{\esetraw{v_i}_i}\emptycx{\tseteq A}$ and $\J{\esetraw{u_j}_j}\emptycx{\tseteq B}$; and $\faex i j \den{v_i} = \den{u_j} : \den{\eqt A}$ therefore $\den{\esetraw{v_i}_i} = \{\den{v_i}\}_i \le \{\den{u_j}\}_j = \den{\esetraw{u_j}_j} : \den{\tseteq A}$; and $\faex i j v_i = u_j : \eqt A$ which shows $\esetraw{v_i}_i \le \esetraw{u_j}_j : \tseteq A$.

  %% \item[?:] \todo{case for base types}{}
  %% \end{description}
\end{proof}

\begin{restatable}[Bottom is bottom]{lemma}{BottomIsBottom}
  \label{lemma-bottom-is-bottom}
  \(\bot_L \in \goodclosed L\) and 
  $\fa{a \in \goodclosed L} \oplrclosed{L}{\bot_L} a$.
\end{restatable}

\begin{proof}
  By induction on $L$:

  \begin{description}
    \item[Case $\tunit$:] Follows trivially from $\bot_\tunit \stepsto \etuple{}$ and the relation at $\tunit$.

    \item[Case $L_1 \times L_2$:] We have $a \evalsto \etuple{v_1, v_2}$ for good $v_i$ and by IH we have $\oplrclosed{L_i}{\bot_{L_i}}{v_i}$. Thus by closed term pairing $\oplrclosed{L_1 \times L_2}{\etuple{\bot_{L_1}, \bot_{L_2}}}{\etuple{v_1,v_2}}$ and since $\bot_{L_1 \times L_2} \stepsto \etuple{\bot_{L_1},\bot_{L_2}}$ by closure under stepping we have what we desire.

    \item[Case $\tseteq A$:] Since $\bot_{\tseteq A} \stepsto \esetraw{}$ it suffices to show $\oplrvalue{\tseteq A}{\esetraw{}}{v}$ for $v \in \goodvalue{\tseteq A}$. This follows from \rn{lr~set}; the first premise is vacuous and the second follows from goodness of $v$.
  \end{description}
\end{proof}

\begin{restatable}[Join is join]{lemma}{JoinIsJoin}
  \label{lemma-join-is-join}
  \(a \vee_{L} b\) is the least upper bound of \(a,b \in \goodclosed{L}\) with respect to the logical relation. That is, \(\oplrclosed{L} a {a \vee_{L} b}\) and \(\oplrclosed{L} b {a \vee_{L} b}\) and for any \(c\) such that \(\oplrclosed{L} a c\) and \(\oplrclosed{L} b c\) we have \(\oplrclosed{L} {a \vee_{L} b} c\).
\end{restatable}

\begin{proof}
  By closure under stepping it suffices to show the same for only good values.
  For this it suffices to show that \(\den{\name{value}\<(v \vee_L u)} = \den{v} \vee \den{u}\) because by \cref{lemma-value-ordering} the semantic ordering and the logical relation ordering agree, so a least upper bound in one is a least upper bound in the other. We show this by induction on $L$:

  \begin{description}
    \item[Case $\tunit$:] Follows from \(\etuple{} \vee_L \etuple{} \evalsto \etuple{}\) and the trivial order on $\tunit$.

    \item[Case $L \times M$:] Then we have

      \begin{align*}
        &\phantom{{}={}}
        \den{\name{value}\<(\etuple{v_1, v_2} \vee_{L \times M} \etuple{u_1, u_2})}
        \\
        &= \den{\etuple{\name{value}\<(v_1 \vee_L u_1),\, \name{value}\<(v_2 \vee_M u_2)}}
        && \text{calculation}
        \\
        &= (\den{\name{value} \<(v_1 \vee_L u_1)},\, \den{\name{value} \<(v_2 \vee_M u_2)})
        && \text{calculation}
        \\
        &= (\den{v_1} \vee \den{u_1},\, \den{v_2} \vee \den{u_2})
        && \text{inductive hypothesis}
        \\
        &= (\den{v_1}, \den{v_2}) \vee (\den{u_1}, \den{u_2})
        && \text{join in product semilattice}
        \\
        &= \den{\etuple{v_1, v_2}} \vee \den{\etuple{u_1, u_2}}
        && \text{calculation}
      \end{align*}

    \item[Case $\tseteq A$:] We have

      \begin{align*}
        \den{\name{value}\<(\esetraw{\vec v} \vee_{\tseteq A} \esetraw{\vec u})}
        &= \den{\esetraw{\vec v, \vec u}}
        && \text{calculation}
        \\
        &= \{\den{v_i}\}_i \cup \{\den{u_j}\}_j
        && \text{calculation}
        \\
        &= \den{\esetraw{\vec v}} \vee \den{\esetraw{\vec u}}
        && \text{calculation}
      \end{align*}
  \end{description}
\end{proof}

\begin{lemma}[Discrete contexts make terms equivalent]
  \label{lemma-discrete-equivalence}
  If \(\oplr{\stripcx\G \vdash A} e f\) and \(\oplr\G{\gamma_1}{\gamma_2}\) then \(\oplrclosedequiv A {\gamma_1(e)} {\gamma_2(f)}\).
\end{lemma}

\begin{proof}
  From \(\oplr\G{\gamma_1}{\gamma_2}\) we have \(\opequiv{\stripcx\G}{\gamma_1}{\gamma_2}\) because \(\stripcx\G\) restricts to only discrete hypotheses \(\hd x A \in \G\) for which we know \(\oplrclosedequiv A {\gamma_1(\dvar x)}{\gamma_2(\dvar x)}\). Thus applying \(\oplr{\stripcx\G \vdash A} e f\) we have \(\oplrclosedequiv A {\gamma_1(e)} {\gamma_2(f)}\) as desired.
\end{proof}


\subsection{Proof of the fundamental theorem}

We now have the groundwork to prove the fundamental theorem, starting with the crucial case of fixed point expressions $\efix e$.

\TerminationFundamental*

\begin{proof}
  Unrolling the definition of the logical relation, we may assume $\oplr\Gamma{\gamma_1}{\gamma_2}$ and wish to show from this that $\oplrclosed A {\gamma_1(e)}{\gamma_2(e)}$. We do this by induction on $\J e \G A$.

  \begin{description}[ topsep=\baselineskip, itemsep=\baselineskip, ]
  \item[Case\quad $\infer{\J e \Gamma \isofixLtoL}{\J {\efix e} \Gamma {\fixt L}}$.]
    We wish to show that

    \[\oplrclosed{\fixt L}{\efix[\fixt L] \gamma_1(e)}{\efix[\fixt L] \gamma_2(e)}\]

    \noindent
    By our inductive hypothesis we have $\oplrclosed{\isofixLtoL}{\gamma_1(e)}{\gamma_2(e)}$; applying this we have $v_1,v_2$ such that $\gamma_i(e) \evalsto \eboxraw{v_i}$ and $\oplrclosedequiv{\fixt L \to \fixt L}{v_1}{v_2}$. Thus for $i \in \{1,2\}$:

    \begin{align*}
      \efix[\fixt L] \gamma_i(e)
      \evalsto
      \efix[\fixt L] \eboxraw{v_i}
      \stepsto
      \eiter[\fixt L]{v_i}{\bot_{\fixt L}}{v_i\<\bot_{\fixt L}}
    \end{align*}

    \noindent
    Let $f_i(u) = \name{apply}\<(v_i, u)$ for brevity.
    %
    Then
    applying \cref{lemma-closed-term-evaluation,lemma-closed-term-application} we have:

    \begin{align*}
      \eiter[\fixt L]{v_i}{\bot_{\fixt L}}{v_i\<\bot_{\fixt L}}
      \evalsto
      \eiter[\fixt L]{v_i}
            {\name{value}\<\bot_{\fixt L}}
            {f_i(\name{value}\<\bot_{\fixt L})}
    \end{align*}

    \noindent
    By bottom-is-bottom we have $\oplrvalue{\fixt L}{\name{value}\<\bot_{\fixt L}}{f_i( \bot_{\fixt L})}$.
    %
    To understand the way evaluation will proceed from here, consider the generalized situation $\eiter[\fixt L]{v_i}{u}{f_i(u)}$ where $\oplrvalue{\fixt L}{u}{f_i(u)}$. This steps like so:

    \[
    \eiter[\fixt L]{v_i}{u}{f_i(u)}
    \evalsto
    \begin{cases}
      u & \text{if}~ u = f_i(u) : \eqt A\\
      \eiter[\eqt A]{v}{f_i(u)}{f_i(f_i(u))} & \text{otherwise}
    \end{cases}
    \]

    \noindent
    Starting with $u = \bot_{\fixt L}$, this calculates the sequence $u, f_i(u), f_i^2(u), f_i^3(u), \dots$ until the first $k$ such that $f_i^k(u) = f_i^{k+1}(u) : \fixt L$ and returns $f_i^k(u)$.
    %
    Note that $f_i : \goodclosed{\fixt L} \to \goodvalue{\fixt L}$ is monotone by \cref{lemma-closed-term-application} and therefore this sequence ascends in the logical relation: $\oplrvalue{\fixt L}{f_i^j(u)}{f_i^{j+1}(u)}$.
    %
    Also by monotonicity of $\name{apply}$, since $\oplrvalueequiv{\fixt L \to \fixt L}{v_1}{v_2}$ and $\oplrclosedequiv{\fixt L}{u}{u}$ by partial reflexivity, we have inductively that $\oplrvalueequiv{\fixt L}{f_1^j(u)}{f_2^j(u)}$.
    %
    By \cref{lemma-value-ordering} these also hold in the semantic order, so the denotations both ascend $\den{f_i^j(u)} \le \den{f_i^{j+1}(u)} : \den{\fixt L}$ and coincide $\den{f_1^j(u)} = \den{f_2^j(u)} : \den{\fixt L}$.
    %
    By the ascending chain condition on $\den{\fixt L}$ we know there must be some $k$ such that $\den{f_1^k(u)} = \den{f_1^{k+1}(u)}$; and as the sequences for $f_1,f_2$ coincide, $\den{f_1^k(u)} = \den{f_1^{k+1}(u)} = \den{f_2^k(u)} = \den{f_2^{k+1}(u)}$.
    %
    Applying \cref{lemma-value-ordering} again this shows $f_i^k(u) = f_i^{k+1}(u) : \fixt L$, and therefore $f_i^k(u)$ for the least such $k$ is the value we terminate with; since $\oplrvalueequiv{\fixt L}{f_1^k(u)}{f_2^k(u)}$, applying closure under stepping we are done.

  \item[Cases\quad $\infer{\hm x A \in \G}{\J {\mvar x} \G A}$ \quad $\infer{\hd x A \in \G}{\J {\dvar x} \G A}$.]
    Follows directly from \(\oplr\G{\gamma_1}{\gamma_2}\).

  \item[Case\quad $\infer{\J e {\G,\,\hm x A} B}{\J {\efn x e} \G {A \to B}}$.]
    What we wish to show is equivalent to:

    \begin{align*}
      &\phantom{{} \iff {}}
      \oplrclosed{A \to B}{\gamma_1(\efn x e)}{\gamma_2(\efn x e)}
      \\
      &\iff \oplrclosed{A \to B}{\efn x \gamma_1(e)}{\efn x \gamma_2(e)}
      \\
      &\iff \oplr{(\hm x A) \vdash B}{\gamma_1(e)}{\gamma_2(e)}
    \end{align*}

    \noindent
    For the last it suffices to assume \(\oplrclosed A {a_1} {a_2}\) and show the transitive square at the logical relation for \(B\):

  \begin{center}
    \tikzset{
      no line/.style={draw=none,
        commutative diagrams/every label/.append style={/tikz/auto=false}}}
            {\begin{tikzcd}
                \sub{e}{\gamma_1,\,  \subto{\mvar x}{a_1}} \precright \precdown
                &
                \sub{e}{\gamma_2,\, \subto{\mvar x}{a_1}} \precdown\\
                \sub{e}{\gamma_1,\, \subto{\mvar x}{a_2}} \precright
                &
                \sub{e}{\gamma_2,\, \subto{\mvar x}{a_2}}
            \end{tikzcd}}
  \end{center}

  \noindent
  By our IH we have \(\fa{\oplr{\G,\, \hm x A}{\sigma}{\sigma'}}\oplrclosed B {\sigma(e)}{\sigma'(e)}\), so it suffices to show the transitive square at the logical relation for \(\G, \hm x A\):

  \begin{center}
    \tikzset{
      no line/.style={draw=none,
        commutative diagrams/every label/.append style={/tikz/auto=false}}}
            {\begin{tikzcd}
                ({\gamma_1,\,  \subto{\mvar x}{a_1}}) \precright \precdown
                &
                ({\gamma_2,\, \subto{\mvar x}{a_1}}) \precdown\\
                ({\gamma_1,\, \subto{\mvar x}{a_2}}) \precright
                &
                ({\gamma_2,\, \subto{\mvar x}{a_2}})
            \end{tikzcd}}
  \end{center}

  \noindent
  This holds by \(\oplr\G{\gamma_1}{\gamma_2}\) and \(\oplrclosed A {a_1}{a_2}\) and the definition of the logical relation for contexts.

  \item[Case\quad $\infer{\J e \G {A \to B} \\ \J f \G A}{\J {e\<f} \G B}$.]
    We wish to show \(\oplrclosed B {\gamma_1(e) \<\gamma_1(f)} {\gamma_2(e) \<\gamma_2(f)}\). By IH we have \(\oplrclosed{A \to B}{\gamma_1(e)}{\gamma_2(e)}\) and \(\oplrclosed A {\gamma_1(f)}{\gamma_2(f)}\).
%
    What we wish to show then follows from \cref{lemma-closed-term-application}.

  \item[Case\quad $\infer{\quad}{\J {\etuple{}} \G \tunit}$.] Trivial.

  \item[Case\quad
    $\infer{(\J{e_i}\G{A_i})_i}{\J{\etuple{e_1,e_2}} \G {A_1 \x A_2}}$.] Apply our inductive hypotheses to get \(\gamma_i(e_j) \evalsto v_{i,j}\) with \(\oplrvalue{A_j}{v_{1,j}}{v_{2,j}}\); this shows \(\oplrvalue{A_1 \times A_2}{\etuple{v_{1,1}, v_{1,2}}}{\etuple{v_{2,1}, v_{2,2}}}\) and since \(\gamma_i(\etuple{e_1,e_2}) = \etuple{\gamma_i(e_1), \gamma_i(e_2)} \evalsto \etuple{v_{i,1}, \gamma_i(e_2)} \evalsto \etuple{v_{i,1}, v_{i,2}}\) by closure under stepping we are done.

  \item[Case\quad $\infer{\J e \G {A_1 \x A_2}}{\J{\pi_i\<e}\G{A_i}}$.] By IH we have \(\gamma_j(e) \evalsto \etuple{v_{1,j}, v_{2,j}}\) with \(\oplrvalue{A_i}{v_{i,1}}{v_{i,2}}\); thus we have \(\gamma_j(\pi_i\<e) = \pi_i\<\gamma_j(e) \evalsto \pi_i\<\etuple{v_{1,j}, v_{2,j}} \stepsto v_{i,j}\) and we are done.

  \item[Case\quad $\infer{\J e \G A_i}{\J{\inj i e}\G{A_1 + A_2}}$.]
    By IH we have \(\gamma_j(e) \evalsto v_j\) for some \(\oplrvalue{A_i}{v_1}{v_2}\). Applying the definition of the LR we have \(\oplrvalue{A_1 + A_2}{\inj i v_1}{\inj i v_2}\) and since \(\gamma_j(\inj i e) = \inj i \gamma_j(e) \evalsto \inj i v_j\) by closure under stepping we are done.

  \item[Case\quad $\infer{\J e \G {A_1 + A_2} \\
      (\J {f_i} {\G,\, \hm{x_i}{A_i}} {B})_i
    }{
      \J {\emcase{e} (\inj i {\mvar x_i} \caseto f_i)_i} \G B
    }$.]
    By IH for $e$ and the LR for \(A_1 + A_2\) we have \(\gamma_j(e) \evalsto \inj i v_j\) for some $i$ and \(\oplrvalue{A_i}{v_1}{v_2}\). Using this and our IH for $f_i$ we have \(\oplrclosed B {\sub{f_i}{\gamma_1,\, \subto{\mvar x_i}{v_1}}}{\sub{f_i}{\gamma_2,\, \subto{\mvar x_i}{v_2}}}\). Then by calculation:

    \begin{align*}
      \gamma_j(\emcase{e} (\inj i {\mvar x_i} \caseto f_i)_i)
      &\evalsto
      {\emcase {\inj i v_j} (\inj i {\mvar x_i} \caseto \gamma_j(f_i))_i}
      \\
      &\stepsto
      \sub{f_i}{\gamma_j,\, \subto{\mvar x_i}{v_j}}
    \end{align*}

    \noindent
    and by closure under stepping we are done.

  \item[Case\quad $\infer{\J {e} {\stripcx\G} A}{\J{\ebox e} \G {\iso A}}$.]
    By our IH and \cref{lemma-discrete-equivalence} we have for some $v_i$ that \(\gamma_i(e) \evalsto v_i\) with \(\oplrvalueequiv A {v_1}{v_2}\). Thus we have \(\gamma_i(\eboxraw{e}) = \eboxraw{\gamma_i(e)} \evalsto \eboxraw{v_i}\) and by the definition of the LR for \(\iso A\) we are done.

  \item[Case\quad $\infer{\J e \G {\iso A} \\ \J f {\G,\,\hd x A} B}{
    \J {\eletbox x e f} \G B}$.]
    By our IH for $e$ and the logical relation for \(\iso A\) we have \(\gamma_i(e) \evalsto \eboxraw{v_i}\) for some \(\oplrvalueequiv A {v_1}{v_2}\). Then using this and applying our IH for $f$ we have \(\sub{f}{\gamma_i,\, \subto{\dvar x}{v_i}} \evalsto u_i\) for some \(\oplrvalue B {u_1}{u_2}\). Then we have:

    \begin{align*}
      \eletbox x {\gamma_i(e)} \gamma_i(f)
      &\evalsto \eletbox x {\eboxraw{v_i}} \gamma_i(f)\\
      &\stepsto \sub{f}{\gamma_i,\, \subto{\dvar x}{v_i}}\\
      &\stepsto u_i
    \end{align*}

    \noindent
    And by closure under stepping we are done.

  \item[Case\quad $\infer{\quad}{\J\bot\G {L}}$.] By bottom is bottom.

  \item[Case\quad $\infer{(\J{e_i} \G {L})_i}{\J{e_1 \vee e_2}\G {L}}$.] Applying our IH we have \(\oplrclosed L {\gamma_1(e_i)} {\gamma_2(e_i)}\); since \hyperref[lemma-join-is-join]{join is join} we know in particular that \(\vee\) is a monotone operator on good closed terms and therefore \(\oplrclosed L {\gamma_1(e_1) \vee \gamma_1(e_2)} {\gamma_2(e_1) \vee \gamma_2(e_2)}\) as desired.

  \item[Case\quad $\infer{(\J {e_i} {\stripcx\G} {\eqt A})_i}{
    \J {\esetsub{e_i}{i}} \G {\tset{\eqt A}}}$.]

    Applying \cref{lemma-discrete-equivalence} we have by our IH that \(\oplrclosedequiv{\eqt A}{\gamma_1(e_i)}{\gamma_2(e_i)}\) and thus for some \(v_{i,j}\) we have \(\gamma_j(e_i) \evalsto v_{i,j}\) with \(\oplrvalueequiv{\eqt A}{v_{i,1}}{v_{i,2}}\). Then by calculation we have \(\gamma_j(\esetraw{e_i}_i) \evalsto \esetraw{v_{i,j}}_i\) and applying \rn{lr~set} (using partial reflexivity for its second premise) we have \(\oplrvalue{\tseteq A}{\esetraw{v_{i,1}}_j}{\esetraw{v_{i,2}}_i}\) and by closure under stepping we are done.

  \item[Case\quad $\infer{
      \J e \G {\tseteq A} \\
      \J f {\G,\, \hd x {\eqt A}} {L}
    }{\J {\eforvar x e f} \G {L}}
$.]
%
    We wish to show that

    \[
    \oplrclosed{L}
    {\eforvar x {\gamma_1(e)} {\gamma_1(f)}}
    {\eforvar x {\gamma_2(e)} {\gamma_2(f)}}
    \]

    \noindent
    By our IH we have $\oplrclosed{\tseteq A}{\gamma_1(e)}{\gamma_2(e)}$. Applying the definition of this, let $v_{i,j} \in \goodvalue{\eqt A}$ and $a_{i,j}$ be defined by:

    \begin{align*}
      \{v_{i,j}\}_j &= \name{value}\<(\gamma_i(e))
      &
      a_{i,j} &= \sub{f}{\gamma_i, \subto{\dvar x}{v_{i,j}}}
    \end{align*}

    \noindent
    From the logical relation we have $\faex j k \oplrvalue{\eqt A}{v_{1,j}}{v_{2,k}}$. From this, our IH for $f$, and the definition of the logical relation for contexts we have $\faex i j \oplrclosed{\eqt A}{a_{1,i}}{a_{2,j}}$.
%
    Observe that:

    \begin{align*}
      \eforvar x {\gamma_i(e)} {\gamma_i(f)}
      &\evalsto 
      \eforvar x {\{v_{i,j}\}_j} {\gamma_i(f)}
      \\
      &\evalsto
      \bot_{L}
      \vee_{L} \sub f {\gamma_i, \subto{\dvar x}{v_{i,1}}}
      \vee_{L} \sub f {\gamma_i, \subto{\dvar x}{v_{i,2}}}
      \vee_{L} \dots
      \\
      &=
      \bot_{L} \vee_{L} a_{i,1} \vee_{L} a_{i,2} \vee_{L} \dots
    \end{align*}

    \noindent
    Then by closure under stepping we wish to show:

    \[
    \oplrclosed{L}
    {(\bot_{L} \vee_{L} a_{1,1} \vee_{L} a_{1,2} \vee_{L} \dots)}
    {(\bot_{L} \vee_{L} a_{2,1} \vee_{L} a_{2,2} \vee_{L} \dots)}
    \]

    \noindent
    Applying \hyperref[lemma-bottom-is-bottom]{bottom is bottom} and \hyperref[lemma-join-is-join]{join is join}, this follows from $\faex i j \oplrclosed{\eqt A}{a_{1,i}}{a_{2,j}}$, because then transitively

    \[
    a_{1,i}
    \prec
    a_{2,i}
    \prec
    (\bot_{L} \vee_{L} a_{2,1} \vee_{L} a_{2,2} \vee_{L} \dots)
    : L
    \]

    \noindent
    Showing that the latter is an upper bound of each \(a_{1,i}\) and since \((\bot_{L} \vee_{L} a_{1,1} \vee_{L} a_{1,2} \vee_{L} \dots)\) is least among such upper bounds, we have what we desire.


  \item[Case\quad $\infer{(\J {e_i} {\stripcx\G} {\eqt A})_i}
        {\J {\eeq{e_1}{e_2}} \G \tbool}$.]
    By \cref{lemma-discrete-equivalence} for some $v_{i,j}$ we have \(\gamma_j(e_i) \evalsto v_{i,j}\) such that \(\oplrvalueequiv{\eqt A}{v_{i,1}}{v_{i,2}}\).
%
    It suffices to show that \(\oplrclosedequiv\tbool{(\eeqraw{v_{1,1}}{v_{2,1}})}{(\eeqraw{v_{1,2}}{v_{2,2}})}\).
%
Since pretty plainly \(\oplrvalue\tbool\etrue\etrue\) and \(\oplrvalue\tbool\efalse\efalse\) (by the same reasoning as in the case for \(\esetsub{e_i} i\)), and since \(\eeqraw{v_{1,j}}{v_{2,j}}\) steps to either \(\etrue\) or \(\efalse\) depending on whether \(v_{1,j} = v_{2,j} : \eqt A\), it suffices to show that \(v_{1,1} = v_{2,1} : \eqt A \iff v_{1,2} = v_{2,2} : \eqt A\). By \cref{lemma-value-ordering} we know that this decidable ordering test coincides with equivalence in the logical relation, thus this holds because \(\oplrvalueequiv{\eqt A}{v_{i,1}}{v_{i,2}}\).

  \item[Case\quad $\infer{\J {e} {\stripcx\G} {\tset\tunit}}{
    \J {\eisempty e} \G {\tunit + \tunit}}$.]
    Applying our IH and \cref{lemma-discrete-equivalence} we have \(\oplrclosedequiv{\tset \tunit}{\gamma_1(e)}{\gamma_2(e)}\).
%
    This means we have some \(\vec v, \vec u\) such that \(\gamma_1(e) \evalsto \esetraw{\vec v}\) and \(\gamma_2(e) \evalsto \esetraw{\vec v}\) and \(\oplrvalueequiv{\tset\tunit}{\esetraw{\vec v}}{\esetraw{\vec u}}\). This implies that \(\vec v\) is empty if and only if \(\vec u\) is empty. In one case \(\eisempty {\gamma_i(e)} \evalsto \inj 1 \etuple{}\); in the other, \(\inj 2 \etuple{}\). So by closure under stepping it suffices to show \(\inj j \etuple{} \in \goodvalue{\tunit + \tunit}\) for \(j \in \{1,2\}\); which it is by unrolling the definitions involved.

  \item[Case\quad $\infer{\J e \G {\iso{(A_1 + A_2)}}}{\J{\esplit e} \G {\iso A_1 + \iso A_2}}$.]
    By our IH and the LR for \(\iso(A_1 + A_2)\) we have for some \(i \in \{1,2\}\) and \(\oplrvalueequiv{A_i}{v_1}{v_2}\) that \(\gamma_j(e) \evalsto \eboxraw{\inj i v_j}\). Then \(\gamma_j(\esplit e) \evalsto \esplit \eboxraw{\inj i v_j} \stepsto \inj i \eboxraw{v_j}\), and using the definition of the LR we have \(\oplrvalue{\iso A_1 + \iso A_2}{\inj i \eboxraw{v_1}}{\inj i \eboxraw{v_2}}\) and by closure under stepping we are done.

  \end{description}
\end{proof}

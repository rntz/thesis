\section{Operational semantics}

\begin{figure*}
  \centering

  \textbf{Additional syntax}
  \begin{displaymath}
    \begin{array}{rrcl}
      %% expressions
      \text{expressions} & e,f,g &\bnfeq&
      ... \bnfor \bot_L \bnfor e \vee_L e \bnfor \efor[L]{x \in e} f
      \bnfor \efixis[L]{x}{e}\\
      %% &&& \eiter{\eq{A}}{e}{x}{e} \bnfor \eiterstep{\eq{A}}{e}{e}{x}{e}\\
      %% && \iterle{\eq{A}}{e}{e}{x}{e} \bnfor \iterlestep{\eq{A}}{e}{e}{e}{x}{e}
      %% \vspace{0.5em}\\
      %% %% values
      %% v,uw,
      %% &\bnfeq& \fn\bind{x} e \bnfor (v, v) \bnfor \ms{in}_i\; v
      %% \bnfor \ms{true} \bnfor \ms{false} \bnfor \setlit{\vec{v}}\\
      %% \textsf{values}
      %% \vspace{0.5em}\\
      %% %% contexts
      %% E
      %% &\bnfeq& \hole \bnfor E\;e \bnfor v\;E \bnfor (E, e) \bnfor (v, E) \bnfor \ms{in}_i\;E
      %% \bnfor \pi_i \; E\\
      %% \textsf{evaluation}
      %% && E \vee_L e \bnfor v \vee_L E \bnfor \tforin{L}{x \in E} e\\
      %% \textsf{contexts}
      %% && \ifthen{E}{e}{e}\\
      %% && \case{E}{x}{e}{x}{e}\\
      %% && \iter{\eq{A}}{E}{x}{e} \bnfor \iterstep{\eq{A}}{v}{E}{x}{e}\\
      %% && \iterle{\eq{A}}{E}{e}{x}{e} \bnfor \iterle{\eq{A}}{v}{E}{x}{e}\\
      %% && \iterlestep{\eq{A}}{v}{v}{E}{x}{e}
    \end{array}
  \end{displaymath}

  %% ~
  %% %% OPSEM: Rules for (v \ale u : A)
  %% \begin{minipage}{0.45\textwidth}
  %%   \centering
  %%   \textbf{Rules for $v \ale u : \eq{A}$ and $v \aeq u : \eq{A}$}
  %%   \begin{mathpar}
  %%     \infer{\ms{false} \ale \ms{false} : \bool}{}
  %%     \and
  %%     \infer{\ms{false} \ale \ms{true} : \bool}{}
  %%     \and
  %%     \infer{\ms{true} \ale \ms{true} : \bool}{}
  %%     \and
  %%     %% rules for set inequality
  %%     \infer[\rn{\subseteq}]
  %%           { \setlit{\vec{v_i}} \ale \setlit{\vec{u_i}} : \Set{\eq{A}} }
  %%           { \forall{v_i}\,\exists{u_j}\; (v_i \aeq u_j : \eq{A}) }
  %%           \and
  %%           \infer%% [\rn{\ale_{\x}}]
  %%               { (v_1, u_1) \ale (v_2, u_2) : \eq{A} \x \eq{B} }
  %%               { v_1 \ale v_2 : \eq{A} & u_1 \ale u_2 : \eq{B} }
  %%               \and
  %%               \infer%% [\rn{\ale_{+}}]
  %%                   { \ms{in}_i\; v \ale \ms{in}_i\; u : \eq{A}_1 + \eq{A}_2 }
  %%                   { v \ale u : \eq{A}_i }
  %%                   \and
  %%                   \infer[\rn{{\aeq}}]{v \aeq u : \eq{A}}{v \ale u : \eq{A} & u \ale v : \eq{A}}
  %%   \end{mathpar}
  %% \end{minipage}
  %% %% OPSEM: Rules for \step
  %% \vspace{1em}
  %% \begin{displaymath}
  %%   \begin{array}{ccl}
  %%     \multicolumn{3}{c}{\textbf{$\beta$-reductions}}\\
  %%     (\fn\bind{x}e_1) \; e_2 &\step& \sub{e_2/x} e_1\\
  %%     \pi_i \; (v_1, v_2) &\step& v_i\\
  %%     \rawcase{\ms{in}_i\,v}{\widevec{\ms{in_j}\,x_j \cto e_j}}
  %%     &\step& \sub{v/x_i} e_i\\
  %%     \ifthen{\ms{true}}{e_1}{e_2} &\step& e_1\\
  %%     \ifthen{\ms{false}}{e_1}{e_2} &\step& e_2

  %%     %% rules for unit
  %%     \vspace{1em}\\
  %%     \multicolumn{3}{c}{\textbf{Evaluating }\unit}\\
  %%     \unit_2 &\step& \ms{false}\\
  %%     \unit_{\Set{A}} &\step& \{\}\\
  %%     \unit_{L \x M} &\step& (\unit_L, \unit_M)\\
  %%     \unit_{A \to L} &\step& \fn\bind{x} \unit_L\\
  %%     \unit_{A \mto L} &\step& \fn\bind{x} \unit_L

  %%     %% rules for \vee
  %%     \vspace{1em}\\
  %%     \multicolumn{3}{c}{\textbf{Evaluating }\vee}\\
  %%     \ms{false} \vee_2 v &\step& v\\
  %%     \ms{true} \vee_2 v &\step& \ms{true}\\
  %%     %% the rule we've all been waiting for
  %%     \setlit{\vec{v}} \vee_{\Set{A}} \setlit{\vec{u}} &\step& \setlit{\vec{v}, \vec{u}}\\
  %%     (v_1, v_2) \vee_{L \x M} (u_1, u_2) &\step& (v_1 \vee_L u_1, v_2 \vee_M u_2)\\
  %%     v \vee_{A \to L} u &\step& \fn\bind{x} v\;x \vee_L u\;x\\
  %%     v \vee_{A \mto L} u &\step& \fn\bind{x} v\;x \vee_L u\;x

  %%   \end{array}
  %%   \begin{array}{ccl}

  %%     %% rules for \bigvee
  %%     %% \vspace{1em}\\
  %%     \multicolumn{3}{c}{\textbf{Evaluating }\bigvee}\\
  %%     \tforin{L}{x \in \{\}} e &\step& \unit_L\\
  %%     \tforin{L}{x \in \setlit{v, \vec{u}}} e
  %%     &\step& \sub{v/x} e \vee_L \tforin{L}{x \in \setlit{\vec{u}}} e

  %%     %% rules for \ms{fix}
  %%     \vspace{1em}\\
  %%     \multicolumn{3}{c}{\textbf{Evaluating \ms{fix} and \ms{iter}}}\\
  %%     \tfix{\fineq{L}}{x}{e} &\step& \iter{\fineq{L}}{\unit_{\fineq{L}}}{x}{e}\\
  %%     \iter{\eq{A}}{v}{x}{e} &\step& \iterstep{\eq{A}}{v}{\sub{v/x} e}{x}{e}\\
  %%     \iterstep{\eq{A}}{v_1}{v_2}{x}{e}
  %%     &\step& \begin{cases}
  %%       v_1 & \text{if}~{v_1 \aeq v_2 : \eq{A}}\\
  %%       \iter{\eq{A}}{v_2}{x}{e} & \text{otherwise}
  %%     \end{cases}\\
  %%     %% rules for fixle, iterle
  %%     \tfixle{\eq{L}}{x}{e_\top}{e} &\step& \iterle{\eq{L}}{e_\top}{\unit_{\eq{L}}}{x}{e}\\
  %%     \iterle{\eq{A}}{v_\top}{v}{x}{e}
  %%     &\step& \begin{cases}
  %%       \iterlestep{\eq{A}}{v_\top}{v}{\sub{v/x} e}{x}{e} & \text{if}~{v \ale v_\top : \eq{A}}\\
  %%       v_\top & \text{otherwise}
  %%     \end{cases}\\
  %%     \iterlestep{\eq{A}}{v_\top}{v_1}{v_2}{x}{e}
  %%     &\step& \begin{cases}
  %%       v_1 &\text{if}~{v_1 \aeq v_2 : \eq{A}}\\
  %%       \iterle{\eq{A}}{v_\top}{v_2}{x}{e} & \text{otherwise}
  %%     \end{cases}
  %%   \end{array}
  %% \end{displaymath}

  \todo{finish this}

  \caption{Operational semantics}
  \label{figure-operational-semantics}
\end{figure*}

%% \begin{figure}
%%   \todo{operational semantics figure}
  
%%   \caption{Operational semantics}
%%   \label{figure-operational-semantics}
%% \end{figure}


We consider the denotational semantics to be primary in Datafun; as with
Datalog, any implementation technique is valid so long as it lines up with these
semantics.
%
To show such an implementation is possible, we present a simple call-by-value
structural operational semantics in \cref{figure-operational-semantics} and show
that all well-typed terms terminate.
%
In our operational semantics we:

\begin{enumerate}
\item Drop the distinction between discrete and monotone variables, writing both
  in lowercase $x,y,z$, and cease using a \isobgname\ background for
  non-monotone expressions.
\item Assume all equality tests and all semilattice operations ($\bot$, $\vee$,
  $\kw{for}$, and $\kw{fix}$) are subscripted with their type.
\item Add $\prim{iter}$ expressions, which occur as intermediate forms in the
  evaluation of $\kw{fix}$.
\end{enumerate}

\noindent
We use a small-step operational semantics with evaluation contexts
$E$~\citep{felleisen-hieb-1992} to enforce a call-by-value evaluation order; an
evaluation context $E$ is an expression with a hole in it, written $\emptyhole$,
such that whatever is in the hole is next in line to be evaluated (if it is not
a value already). To fill the hole in an evaluation context $E$ with the
expression $e$, we write $\fillhole{E}{e}$.

We define a relation $e \stepsto e'$ for expressions $e$ whose outermost
structure is immediately reducible; we extend this relation to all
expressions with the rule:

\[
\infer{e\stepsto e'}{\fillhole E e \stepsto \fillhole E{e'}}
\]

\noindent
In our rules for $e \stepsto e'$ where $e$ is an \prim{iter} expression we make
use of a decidable ordering test on values, $v \le u : \eqt A$, and a
corresponding equality test $v = u : \eqt A$. We define these using
inference rules, but they are easily seen to be decidable by induction on $\eqt
A$.

%\subsection{Computing fixed points by iteration}

Our implementation strategy for $\efix f$ is
% \todo{suggested by the proof of \cref{lemma-fixed-point}:}
straightforward: starting from $\bot$, iteratively apply $f$ until quiescence.
%
To model this process, we introduce the form
$\eiter{f}{e_1}{e_2}$, which evaluates two successive function iterations
$e_1,e_2$.
%
The fixed point expression$\efix f$, after evaluating $f$, steps to
$\eiter{f}{\bot}{f\<e}$, which kicks off the first two iterations.
%
Once these have reduced to values, $\eiter f {u_1}{u_2}$ tests $u_1 = u_2$ to
determine if a fixed point has been reached. If so, its value $u_1$ is returned;
otherwise we step to $\eiter{f}{u_2}{f\<u_2}$ to evaluate the next iteration,
and so on.

%% \subsection{A logical relation for termination}

%% \todo{logical relation for termination and adequacy}

\section{Operational semantics}

\begin{figure*}
  \centering

  \textsc{additional syntax}
  \begin{displaymath}
    \begin{array}{rccl}
      \text{expressions} & e,f,g &\bnfeq&
      ... \bnfor \eeqraw[\eqt A]{e}{f}
      \bnfor \bot_L \bnfor e \vee_L f \bnfor \efor[L]{x \in e} f
      \bnfor \prim{fix}_{\fixt L} \<e\\
      &&& \eiter[\eqt A]{v}{e}{f}
      \\[\bnfskip]
      \text{values} & v,u &\bnfeq&
      \fnof{x} e \bnfor \etuple{} \bnfor \etuple{v, u} \bnfor \inj i v
      \bnfor \esetraw{{v}_i}_i \bnfor \eboxraw{v}
      \\[\bnfskip]
      \text{evaluation contexts} & E &\bnfeq&
      \emptyhole \bnfor E\<e \bnfor v\<E \bnfor \etuple{E, e}
      \bnfor \etuple{v, E} \bnfor \pi_i\< E\\
      &&& \inj i E \bnfor \emcase{E}{(\inj i x_i \caseto e_i)_i}\\
      &&& \eboxraw{E} \bnfor \eletbox{x}{E}{e}\\
      &&& \esetraw{\vec v,\, E,\, \vec e}
      \bnfor E \vee_L e \bnfor v \vee_L E \bnfor \efor[L]{x \in E} e\\
      &&& \eeqraw[A]{E}{e} \bnfor \eeqraw[A]{v}{E}
      \bnfor \esplit E \bnfor \eisemptyraw E\\
      &&& \prim{fix}_{\fixt L} \<E
      \bnfor \eiter[A]{v}{E}{f} \bnfor \eiter[A]{v}{u}{E}
    \end{array}
  \end{displaymath}
  \vspace{0pt}

  \textsc{value (in)equality}
  \begin{mathpar}
    \infer{}{\etuple{} \le \etuple{} : \tunit}

    \infer{v_1 \le u_1 : \eqt A \\ v_2 \le u_2 : \eqt B}{%
      \etuple{v_1,\, v_2} \le \etuple{u_1,\, u_2} : \eqt A \x \eqt B}

    \infer{v \le u : A_i}{\inj i v \le \inj i u : A_1 + A_2}

    \infer{\faex{i}{j} v_i = u_j : \eqt A}{
      \esetraw{v_i}_i \le \esetraw{u_j}_j : \tset{\eqt A}
    }

    \infer{v \le u : \eqt A\\ u \le v : \eqt A}{v = u : \eqt A}
  \end{mathpar}
  \vspace{2pt}

  \textsc{$\beta$ reductions}
  \begin{align*}
    (\fnof{x}e)\<v &~\stepsto~ \subone{e}{x}{v}
    &
    \eletbox{x}{\eboxraw{v}}{e} &~\stepsto~ \subone{e}{x}{v}
    \\
    \pi_i\<\etuple{v_1, v_2} &~\stepsto~ v_i
    &
    \emcase{\inj i v}{(\inj j x_j \caseto e_j)_j}
    &~\stepsto~ \subone{e_i}{x_i}{v}
    \\
    \efor[L]{x \in \esetraw{}} e &~\stepsto~ \bot_L
    \\
    \efor[L]{x \in \esetraw{v, \vec u}} e
    &\omit\rlap{$~\stepsto~
      (\subone{e}{x}{v}) \vee_L \efor[L]{x \in \esetraw{\vec u}} e
    $}
  \end{align*}
  \vspace{0pt}

  \textsc{other reductions}
  \begin{align*}
    \bot_{\tset{A}} &~\stepsto~ \esetraw{}
    &
    \bot_{L \x M} &~\stepsto~ \etuple{\bot_L,\,\bot_M}
    \\
    \esetraw{\vec v} \vee_{\tset A} \esetraw{\vec u}
    &~\stepsto~ \esetraw{\vec v,\, \vec u}
    &
    \etuple{v_1, v_2} \binvee_{L \x M} \etuple{u_1, u_2}
    &~\stepsto~ \etuple{v_1 \vee_L u_1,\, v_2 \vee_M u_2}
    \\
    \eisemptyraw{\esetraw{}} &~\stepsto~ \inj 1 \etuple{}
    &
    \eisemptyraw{\esetraw{v,\vec u}} &~\stepsto~ \inj 2 \etuple{}
    \\
    \esplit{\eboxraw{\inj i v}} &~\stepsto~ \inj i {\eboxraw v}
    &
    \eeqraw[\eqt A]{v}{u}
    &~\stepsto~
    \begin{cases}
      \etrue & \text{if}~ v = u : \eqt A\\
      \efalse & \text{otherwise}
    \end{cases}
    \\
    \prim{fix}_{\fixt L} \<v
    &\omit\rlap{$~\stepsto~
      \eiter[\fixt L]{v}{\bot_{\fixt L}}{v \<\bot_{\fixt L}}
    $}
    \\
    \eiter[\eqt A]{v}{u_1}{u_2}
    &\omit\rlap{$~\stepsto~
    \begin{cases}
      u_1 & \text{if}~ u_1 = u_2 : \eqt A\\
      \eiter[\eqt A]{v}{u_2}{v\<u_2} & \text{otherwise}
    \end{cases}$}
  \end{align*}

  \caption{Operational semantics}
  \label{figure-operational-semantics}
\end{figure*}


We consider the denotational semantics to be primary in Datafun; as with
Datalog, any implementation technique is valid so long as it lines up with these
semantics.
%
To show such an implementation is possible, we present a simple call-by-value
structural operational semantics in \cref{figure-operational-semantics}\todo{ and show that all well-typed terms terminate}.
%
In our operational semantics we:

\begin{enumerate}
\item Drop the distinction between discrete and monotone variables, writing both
  in lowercase $x,y,z$, and cease using a \isobgname\ background for
  non-monotone expressions.
\item Assume all equality tests and all semilattice operations ($\bot$, $\vee$,
  $\kw{for}$, and $\prim{fix}$) are subscripted with their type.
\item Add $\prim{iter}$ expressions, which occur as intermediate forms in the
  evaluation of $\prim{fix}$.
\end{enumerate}

\noindent
We use a small-step operational semantics with evaluation contexts
$E$~\citep{felleisen-hieb-1992} to enforce a call-by-value evaluation order; an
evaluation context $E$ is an expression with a hole in it, written $\emptyhole$,
such that whatever is in the hole is next in line to be evaluated (if it is not
a value already). To fill the hole in an evaluation context $E$ with the
expression $e$, we write $\fillhole{E}{e}$.

We define a relation $e \stepsto e'$ for expressions $e$ whose outermost
structure is immediately reducible; we extend this relation to all
expressions with the rule:

\[
\infer{e\stepsto e'}{\fillhole E e \stepsto \fillhole E{e'}}
\]

\noindent
In our rules for $e \stepsto e'$ where $e$ is an \prim{iter} expression we make
use of a decidable ordering test on values, $v \le u : \eqt A$, and a
corresponding equality test $v = u : \eqt A$. We define these using
inference rules, but they are easily seen to be decidable by induction on $\eqt
A$.

%\subsection{Computing fixed points by iteration}

Our implementation strategy for $\efix f$ is
% \todo{suggested by the proof of \cref{lemma-fixed-point}:}
straightforward: starting from $\bot$, iteratively apply $f$ until quiescence.
%
To model this process, we introduce the form
$\eiter{f}{e_1}{e_2}$, which evaluates two successive function iterations
$e_1,e_2$.
%
The fixed point expression $\efix f$, after evaluating $f$, steps to
$\eiter{f}{\bot}{f\<e}$, which kicks off the first two iterations.
%
Once these have reduced to values, $\eiter f {u_1}{u_2}$ tests $u_1 = u_2$ to
determine if a fixed point has been reached. If so, its value $u_1$ is returned;
otherwise we step to $\eiter{f}{u_2}{f\<u_2}$ to evaluate the next iteration,
and so on.


%% \subsection{A logical relation for termination}

%% \todo{logical relation for termination}

%% \todolater{extend this to handle adequacy}

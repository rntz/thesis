\section{Applying the semi\naive\ transformation to transitive closure}
\label{section-seminaive-trans}

Let's try applying the semi\naive\ transform to a simple Datafun program: the
transitive closure function \name{trans} from
\cref{section-datafun-transitive-closure}:

\begin{code}
  \name{trans} \< \pbox{\dvarlong{edge}}
  = \efixis{r}{\dvarlong{edge} \cup (\dvarlong{edge} \relcomp \mvar r)}
  \\
  \mvar s \relcomp \mvar t =
  \efor{\etuple{\dvar x, \yone} \in \mvar s}
  \efor{\etuple{\ytwo , \dvar z} \in \mvar t}
  \efor{\eeq{\yone}{\ytwo}}
  \eset{\etuple{\dvar x, \dvar z}}
\end{code}

\noindent
In the process we'll discover that besides $\phi$ itself we need a few simple
optimisations to actually speed up our program: most importantly, we need to
propagate $\bot$ expressions.

In our experience, performing $\phi$ and $\delta$ by hand is easiest when you work inside-out. At the core of transitive closure is a relation composition, $(\dvar e \relcomp p)$, and at the core of relation composition is a ``one-sided conditional'', $\efor{\eeq{\dvar y_1}{\dvar y_2}} \eset{\etuple{\dvar x, \dvar z}}$. Let's take a look at its $\phi$ and $\delta$ translations:

%% Avoid overfull hboxes in this section
%% \setlength\codeoffset{10pt}

\begin{flail}
  &\phantom{{}={}}
  \phi(\efor {\eeq \yone \ytwo} \eset{\etuple{\dvar x, \dvar z}})
  \\
  &= \phi(\efor {\etuple{} \in \eeq \yone \ytwo} \eset{\etuple{\dvar x,
      \dvar z}})
  && \text{desugar}
  \\
  &= \efor{\etuple{} \in \eeq \yone \ytwo}
  \phi{\eset{\etuple{\dvar x, \dvar z}}}
  && \text{apply $\phi$ and omit unused \kw{let}}\\
  &= \efor{\eeq \yone \ytwo} \eset{\etuple{\dvar x, \dvar z}}
  && \text{resugar}
\end{flail}

\noindent
Frequently, as in this case, $\phi$ does nothing interesting. For brevity we'll
skip such no-op translations. Now for the $\delta$ translation:

\begin{flail}
  &\mathrel{\hphantom{=}}
  \delta(\efor {\eeq \yone \ytwo} \eset{\etuple{\dvar x, \dvar z}})
  \\
  &= \delta(\efor{\etuple{} \in \eeq \yone \ytwo} \eset{\etuple{\dvar x, \dvar z}})
  && \text{desugar}
  \\
  &= \hphantom{{}\cup} \efor{\etuple{} \in \delta(\eeq \yone \ytwo)}
  \phi\eset{\etuple{\dvar x, \dvar z}}
  %% this generates two "Overfull \vbox" warnings. :(
  && \multirow{2}{*}{\text{apply $\delta$ and omit unused \kw{let}s}}
  \\
  &\hphantom{={}} \cup
  \efor{\etuple{} \in \phi(\eeq \yone \ytwo) \cup \delta(\eeq \yone \ytwo)}
  \delta\eset{\etuple{\dvar x, \dvar z}}
  \\
  &= \phantom{{}\cup}
  \efor{\etuple{} \in \bot} \eset{\etuple{\dvar x, \dvar z}}
  && \multirow{2}{*}{\text{apply $\phi/\delta$ to $\eeq \yone \ytwo$ and $\eset{\etuple{\dvar x, \dvar z}}$}}
  \\
  &\phantom{={}} \cup \efor{\etuple{} \in \phi(\eeq \yone \ytwo) \cup \bot} \bot
  \\
  &= \bot && \text{propagate }\bot
\end{flail}

\noindent
The core insight here is that neither $\eeq \yone \ytwo$ nor
$\eset{\etuple{\dvar x, \dvar z}}$ can change. Propagating this information --
for example, rewriting $(\efor {...} \bot)$ to $\bot$ -- can simplify
derivatives and eliminate expensive \kw{for}-loops.

Now let's pull out and examine $\efor{\etuple{\ytwo, \dvar z} \in t}
\efor{\eeq \yone \ytwo} \eset{\etuple{\dvar x, \dvar z}}$. The $\phi$
translation is again a no-op, and the $\delta$ translation is:

\begin{flail}
  &\mathrel{\hphantom{=}}
  \delta(\efor{\etuple{\ytwo, \dvar z} \in t}
  \efor{\eeq \yone \ytwo} \eset{\etuple{\dvar x, \dvar z}})
  \\
  &= \hphantom{{}\cup} \efor{\etuple{\ytwo, \dvar z} \in \dt}
  \phi(\efor{\eeq \yone \ytwo} \eset{\etuple{\dvar x, \dvar z}})
  && {\text{apply $\delta$ and omit unused \kw{let}s}}
  \\
  &\hphantom{{}=} \cup \efor{\etuple{\ytwo, \dvar z} \in t \cup \dt}
  \delta(\efor{\eeq \yone \ytwo} \eset{\etuple{\dvar x, \dvar z}})
  \\
  &= \efor{\etuple{\ytwo, \dvar z} \in \dt}
  \efor{\eeq \yone \ytwo} \eset{\etuple{\dvar x, \dvar z}}
  && \text{apply prior work, propagate }\bot
\end{flail}

\noindent Tackling the outermost \kw{for} loop:

\begin{flail}
  &\mathrel{\hphantom{=}}
  \delta(
  \efor{\etuple{\dvar x, \yone} \in s}
  \efor{\etuple{\ytwo, \dvar z} \in t}
  \efor{\eeq \yone \ytwo} \eset{\etuple{\dvar x, \dvar z}})
  \\
  &= \efor{\etuple{\dvar x, \yone} \in \ds}
  \phi(\efor{\etuple{\ytwo, \dvar z} \in t}
  \efor{\eeq \yone \ytwo} \eset{\etuple{\dvar x, \dvar z}})
  && \text{apply $\delta(\kw{for} \dots)$}
  \\
  &\cup \efor{\etuple{\dvar x, \yone} \in s \cup \ds}
  \delta(\efor{\etuple{\ytwo, \dvar z} \in t}
  \efor{\eeq \yone \ytwo} \eset{\etuple{\dvar x, \dvar z}})
  \\
  &= \efor{\etuple{\dvar x, \yone} \in \ds}
  \efor{\etuple{\ytwo, \dvar z} \in t}
  \efor{\eeq \yone \ytwo} \eset{\etuple{\dvar x, \dvar z}}
  && \text{apply prior work}
  \\
  &\cup
  \efor{\etuple{\dvar x, \yone} \in s \cup \ds}
  \efor{\etuple{\ytwo, \dvar z} \in \dt}
  \efor{\eeq \yone \ytwo} \eset{\etuple{\dvar x, \dvar z}}
  \\
  &= (\ds \relcomp t) \cup ((s \cup \ds) \relcomp \dt)
  && \text{rewrite using ${\relcomp}$}
\end{flail}

\noindent
This, then, is the derivative $\delta(s \relcomp t)$ of relation composition.
With a bit of rewriting, this is equivalent to $(\ds \relcomp t) \cup (s \relcomp
\dt) \cup (\ds \relcomp \dt)$, which is perhaps the derivative a human would
give.

Let's use this to figure out $\phi(\name{trans}\<\eboxvar{e})$. Working inside
out, we start with the derivative of the loop body, $\delta(\dvar e \cup (\dvar
e \relcomp p))$:

\begin{flail}
  &\phantom{{}={}}
  \delta({\dvar e \cup (\dvar e \relcomp p)})\\
  &= \delta\dvar e \cup \delta(\dvar e \relcomp p)\\
  &= \delta\dvar e
  \cup (\delta\dvar e \relcomp p)
  \cup ((\dvar e \cup \delta\dvar e) \relcomp \deep)
  \\
  &= \bot \cup (\bot \relcomp p) \cup ((\dvar e \cup \bot) \relcomp \deep)
  && \delta\dvar e ~\text{is a zero change; insert }\bot
  \\
  &= \dvar e \relcomp \deep
  && \text{propagate}~\bot
\end{flail}

\noindent
The penultimate step requires a new optimization.
%
By definition $\delta\dvar e = \dvar{de}$, but since $\dvar e$ is discrete we
know $\dvar{de}$ is a zero change, so we may safely replace it by $\bot$.

Putting everything together, we have:

\begin{flail}
  &\phantom{{}={}}
  \phi(\efixis{p}{\dvar e \cup (\dvar e \relcomp p)}\\
  &= \phi(\efix \ebox{\fnof{p} \dvar e \cup (\dvar e \relcomp p)})
  && \text{desugaring}
  \\
  &= \semifix\< \phi\ebox{\fnof{p} \dvar e \cup (\dvar e \relcomp p)}
  \\
  &= \semifix\<\bigeboxtuple{\phi({\fnof{p} \dvar e \cup (\dvar e \relcomp p)}),\
  \delta({\fnof{p} \dvar e \cup (\dvar e \relcomp p)})}
  \\
  &= \semifix\<\bigeboxtuple{
      ({\fnof{p} \dvar e \cup (\dvar e \relcomp p)}),\
      ({\fnof{\pboxvar p} \fnof{\deep} \dvar e \relcomp \deep})}
  && \text{previous work}
\end{flail}

\noindent
Examining the recurrence produced by this use of \semifix, we recover the
semi\naive\ transitive closure algorithm from
\cref{section-seminaive-tc-in-datafun}:

\begin{align*}
  x_0 &= \bot & x_{i+1} &= x_i \cup \dx_i\\
  \dx_0 &= ({\fnof{p} \dvar e \cup (\dvar e \relcomp p)}) \<\bot
  = \dvar e
  &
  \dx_{i+1} &=
  ({\fnof{\pboxvar p} \fnof{\dx} \dvar e \relcomp \deep})
  \<\ebox{x_i} \<\dx_i
  = \dvar e \relcomp \dx_i
\end{align*}

\setlength\codeoffset{20pt}     %restore.


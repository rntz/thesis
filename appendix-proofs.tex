%% Adjust description lists to be more suitable for proofs by cases.
\setlist[description]{
  topsep=\parsep,
  font={\mdseries\itshape},
}


\chapter{Proofs omitted from main text}

\newcommand\nextlemma{\par\pagebreak[3]\vspace{1\baselineskip}}

We state these lemmas and theorems in dependency order, so that nothing is used
before it has been proven. This is not always the order in which they are stated
in the text. \todolater{But maybe it should be?} \fixmelater{jeremy}{Is dividing the appendix into sections according to your chapters the right thing to do?}


\section{Datafun}

\todo{Should I restate lemmas and theorems from operational semantics metatheory?}

%% \TerminationFundamental*

%% \begin{proof}
%%   By induction on $\J e \Gamma A$.

%%   \begin{description}
%%     \item[Case ...] \XXX
%%   \end{description}
%% \end{proof}



\todomaybe{include the figure defining weakening relation here}{}

\begin{lemma}\label{lemma-weakening-variables}
  If $\hm x A \in \Gamma$ or $\hd x A \in \Gamma$ and $\Gamma \weaker \Delta$
  then $\hm x A \in \Delta$ or $\hd x A \in \Delta$.
\end{lemma}
\begin{proof}
  Recall that although we write them differently we regard $\mvar x$ and $\dvar
  x$ as the same variable.
%
  Let $H$ stand for the hypothesis in our assumption, either $\hm x A$ or $\hd x
  A$ respectively. Then by induction on the derivation of $\Gamma
  \weaker \Delta$:

  \begin{description}[itemsep=1ex]
  \item[Case\, $\emptycx \weaker \emptycx$.] By contradiction, since $H \in
    \emptycx$ is impossible.

  \item[Case\, $\infer{\Gamma' \weaker \Delta'}{\Gamma',H' \weaker
      \Delta',H'}$.] If $H = H'$ we are done. Otherwise, $H \in \Gamma'$, so
    apply the IH.

  \item[Case\, $\infer{\Gamma \weaker \Delta'}{\Gamma \weaker \Delta',H}$.]
    Apply the IH.

  \item[Case\, $\infer{\Gamma' \weaker \Delta'}{\Gamma',\,\hm x A \weaker
      \Delta',\hd x A}$.] If $H = \hm x A$ we are done. Otherwise, $H \in
    \Gamma'$, so apply the IH.
  \end{description}
\end{proof}


\nextlemma
\begin{lemma}\label{lemma-weakening-strip-context}
  If $\Gamma \weaker \Delta$ then $\stripcx{\Gamma} \weaker \stripcx{\Delta}$.
\end{lemma}
\begin{proof}
  By induction on $\Gamma \weaker \Delta$:

  \begin{description}
    \item[Case $\emptycx \weaker \emptycx$.] Immediate.

  \item[Case\, $\infer{\Gamma' \weaker \Delta'}{\Gamma',H \weaker \Delta',H}$.]\

    Either $H$ is discrete $\hd x A$, in which case $\stripcx{\Gamma',H} =
    \stripcx{\Gamma'}, H \weaker \stripcx{\Delta'}, H = \stripcx{\Delta',H}$ by
    \rn{cons} and our inductive hypothesis; or $H$ is monotone $\hm x A$, in
    which case $\stripcx{\Gamma',H} = \stripcx{\Gamma'} \weaker \stripcx{\Delta'} = \stripcx{\Delta',H}$ by our inductive hypothesis alone.

  \item[Case\, $\infer{\Gamma \weaker \Delta'}{\Gamma \weaker \Delta',H}$.]\

    Then $\stripcx{\Gamma} \weaker \stripcx{\Delta'}$ by our inductive
    hypothesis. Depending upon whether $H$ is monotone or discrete, we have
    either $\stripcx{\Delta', H} = \stripcx{\Delta'}$ (in which case our
    inductive hypothesis suffices) or $\stripcx{\Delta', H} = \stripcx{\Delta'},
    H$, in which case by \rn{cons} and transitivity $\stripcx{\Gamma}
    \weaker \stripcx{\Delta'}, H$. \todo{prove transitivity}

  \item[Case\, $\infer{\Gamma' \weaker \Delta'}{\Gamma',\,\hm x A \weaker
      \Delta',\hd x A}$.]\

    Then $\stripcx{\Gamma', \hm x A} = \stripcx{\Gamma'}$ while
    $\stripcx{\Delta', \hd x A} = \stripcx{\Delta'}, \hd x A$. By our IH, we
    have $\stripcx{\Gamma'} \weaker \stripcx{\Delta'}$, and therefore by
    \rn{drop} we have $\stripcx{\Gamma'} \weaker \stripcx{\Delta'}, \hd x A$ as
    desired.
  \end{description}
\end{proof}


\section{Semi\naive\ evaluation}
\label{appendix-seminaive}

\emph{Theorems and lemmas from \cref{chapter-seminaive}.}

%\begin{figure}
  \centering

  \todolater{typing rules for syntax sugar used in phi/delta}
  %% \textsc{typing rules for syntax sugar used in $\phi$/$\delta$}
  %% \begin{mathpar}
  %%   \infer*{
  %%     \J{e}{\Gamma}{\iso(A \x B)}\\
  %%     \J{f}{\Gamma,\,\hd x A,\,\hd y B}{C}
  %%   }{
  %%     \J{\elet{\pboxtuple{\dvar x, \dvar y} = e} f}
  %%       {\Gamma}
  %%       {C}
  %%   }
  %% \end{mathpar}
  %% \vspace{0pt}

  \textsc{additional desugarings}

  \begin{align*}
    % φ(split e)
    \phi(\esplit e) &\stareq \emcase{\phi e}
    \\
    &\qquad
    \left(\pboxtuple{\inj i \dvar x, \inj i \dvar \dx}
    \caseto \inj i {\eboxtuple{\dvar x,\dvar\dx}}\right)_{i}
    \\
    &\qquad
    \left(\pboxtuple{\inj i \dvar x, \inj j \pwild}
    \caseto \inj i {\eboxtuple{\dvar x, \dummy\<\dvar x}} \right)_{i\ne j}
    \\
    &=
    \eletbox{\color{fresh} \dvar z}{\phi e}
    \\
    &\phantom{{}={}}
    \emcase{\esplit{\ebox{\pi_1\<\color{fresh} z}}}
    \\
    &\phantom{{}={}\quad}
    (\inj i {\color{fresh}\mvar y} \caseto \eletbox{x}{\color{fresh}\mvar y}
    \\
    &\phantom{{}={}\quad (\inj i \mvar y \caseto {}}
    \emcase{\esplit \ebox{\pi_2\<\color{fresh}\dvar z}}
    \\
    &\phantom{{}={}\quad (\inj i \mvar y \caseto {} \quad}
    \inj i {\color{fresh}\mvar\dy}
    \caseto
    \eletbox{\dx}{\color{fresh}\mvar\dy}
    \inj i \eboxraw{\etuple{\dvar x, \dvar\dx}}
    \\
    &\phantom{{}={}\quad (\inj i \mvar y \caseto {} \quad}
    \inj{i+1 \bmod 2} \pwild \caseto
    \inj i \ebox{\etuple{\dvar x, \dummy\<\dvar x}}
    )_i
    \\[.5\baselineskip]
    % δ(split e)
    \delta(\esplit e) &\stareq \emcase{\phi e}
    (\pboxtuple{\inj i \pwild, \pwild}
    \caseto \inj i {\etuple{}} )_i
    \\
    &= \eletbox{\color{fresh} y}{\phi e}
    \emcase{\pi_1\<{\color{fresh}\dvar y}}
    (\inj i \pwild \caseto \inj i \ptuple{})_{i \in \{1,2\}}
    \\[.5\baselineskip]
    % δ(case e of ...)
    \delta(\emcase e (\inj i \mvar x \caseto f_i)_i)
    &\stareq
    \emcase{\esplit{\ebox{\phi e}},\, \delta e}\\
    &\qquad ({\inj i {\pboxvar x},\, \inj i \mvar\dx} \caseto \delta f_i)_{i}\\
    &\qquad ({\inj i {\pboxvar x},\, \inj j \pwild}
    %\caseto \subst{\delta f_i}{\dx \substo \dummy\<\dvar x})_{i\ne j}
    \caseto \elet{\mvar\dx = \dummy\<\dvar x} \delta f_i)_{i\ne j}
    \\
    &=
    \emcase{\esplit{\ebox{\phi e}}}\\
    &\qquad
    (
    \inj i {\color{fresh}\mvar y} \caseto
    \eletbox{x}{\color{fresh}\mvar y}
    \\
    &\qquad\phantom{(\inj i {\mvar y} \caseto {}}
    (\efn{\dx} \delta f_i)
    \<(\emcase{\delta e}
    \inj i \mvar\dx \caseto \mvar\dx
    \\
    &\qquad\phantom{(\inj i {\mvar y} \caseto {} (\efn{\dx} \delta f_i) \<(\emcase{\delta e}}
    \inj{i+1 \bmod 2} \pwild \caseto \dummy\<\dvar x
    ))_i
    %% \\
    %% &\qquad\phantom{(\inj i {\mvar y} \caseto {}}
    %% \emcase{\delta e}
    %% \\
    %% &\qquad\phantom{(\inj i {\mvar y} \caseto {}}\quad
    %% \inj i {\mvar\dx} \caseto \delta f_i\\
    %% &\qquad\phantom{(\inj i {\mvar y} \caseto {}}\quad
    %% \inj{i+1 \bmod 2} \pwild \caseto 
    %% )_{i \in \{1,2\}}
  \end{align*}

  \centering\itshape
  Fresh variables introduced by desugaring are colored {\color{fresh}\freshname}.

  \caption{Syntax sugar in $\phi$ and $\delta$ transformations}
  \label{appendix-seminaive-syntax-sugar}
\end{figure}


%% \begin{definition}[Syntax sugar in $\phi$ and $\delta$ transformations]
%%   \label{appendix-seminaive-syntax-sugar}
%% \end{definition}


\todo{find all references to this proof}
%% \DeltaLattice*
%% \DeltaLatticeProof*


\nextlemma
\PhiEqualityType*
\PhiEqualityTypeProof*


\nextlemma
\Weakening*

\begin{proof}\label{proof-weakening}
  By induction on the derivation of $\J{e}{\Gamma} A$.

  \begin{description}[topsep=1em,itemsep=1em]
    \item[Cases\quad $\infer{\hm x A \in \G}{\J {\mvar x} \G A}$%
      \quad
      $\infer{\hd x A \in \G}{\J {\dvar x} \G A}$.]\
%
      By \cref{lemma-weakening-variables}.

    \item[Cases\quad $\infer{\quad}{\J {\etuple{}} \G \tunit}$%
      \quad $\infer{\quad}{\J\bot\G {L}}$.]\
%
      Trivial.

    \item[Cases where the premises have the same context as the conclusion, namely:]

      \begin{mathpar}
        \infer{\J e \G {A \to B} \\ \J f \G A}{\J {e\<f} \G B}

        \infer{(\J{e_i}\G{A_i})_i}{\J{\etuple{e_1,e_2}} \G {A_1 \x A_2}}

        \infer{\J e \G {A_1 \x A_2}}{\J{\pi_i\<e}\G{A_i}}

        \infer{(\J{e_i} \G {L})_i}{\J{e_1 \vee e_2}\G {L}}

        \infer{\J e \G {\iso{(A + B)}}}{\J{\esplit e} \G {\iso A + \iso B}}

        \infer{\J e \Gamma \isofixLtoL}{\J{\efix e} \Gamma {\fixt L}}
      \end{mathpar}

      Apply the same typing rule to our inductive hypotheses.

    \item[Cases where the premises add hypotheses to the context, namely:]

      \begin{mathpar}
        \infer{\J e {\G,\,\hm x A} B}{\J {\efn x e} \G {A \to B}}

        \infer{\J e \G {\iso A} \\ \J f {\G,\,\hd x A} B}{\J {\eletbox x e f} \G B}

        \infer{\J e \G {A_1 + A_2} \\
          (\J {f_i} {\G,\, \hm{x_i}{A_i}} {B})_i
        }{
          \J {\emcase{e} (\inj i {\mvar x_i} \caseto f_i)_i} \G B
        }

        \infer{
          \J e \G {\tset A} \\
          \J f {\G,\, \hd x A} {L}
        }{\J {\eforvar x e f} \G {L}}
      \end{mathpar}

      Apply the inductive hypotheses, using \rn{cons} when necessary to show
      that the modified contexts also satisfy our precondition, for example,
      $\Delta,\hm{x}{A} \stronger \Gamma,\hm{x}{A}$.

    \item[Case where the premises strip the context, namely:]

      \begin{mathpar}
        \infer{\J {e} {\stripcx\G} A}{\J{\ebox e} \G {\iso A}}

        \infer{(\J {e_i} {\stripcx\G} {\eqt A})_i}
              {\J {\esetsub{e_i}{i}} \G {\tset{\eqt A}}}

        \infer{(\J {e_i} {\stripcx\G} {\eqt A})_i}
              {\J {\eeq{e_1}{e_2}} \G \tbool}

              \infer{\J {e} {\stripcx\G} {\tset\tunit}}
                    {\J {\eisempty e} \G {\tunit + \tunit}}
      \end{mathpar}

      Then $\stripcx{\Delta} \stronger \stripcx{\Gamma}$ by
      \cref{lemma-weakening-strip-context}, so we apply the inductive
      hypotheses.

    %% \item[Case\quad $\infer{\J e {\stripcx{\G},\, \hm x {\fixt L}} {\fixt L}}{%
    %%   \J{\efixis x e} \G {\fixt L}}$.]\

    %%   Then by combining \cref{lemma-weakening-strip-context} and \rn{cons} we
    %%   have $\stripcx{\Delta}, \hm x {\fixt L} \stronger \stripcx{\Gamma}, \hm x
    %%   {\fixt L}$ and we apply our inductive hypothesis.

  \end{description}
\end{proof}


\nextlemma
\PhiDeltaWellTyped*
\PhiDeltaWellTypedProof*

%% \begin{proof}\label{proof-phi-delta-well-typed}
%%   By induction on the typing derivation $\J e \Gamma A$. The proof is utterly
%%   mechanical.

%%   \begin{description}[topsep=1em,itemsep=1em]
%%     \item[Case\quad $\infer{\hm x A \in \G}{\J {\mvar x} \G A}$,\quad $\phi\mvar{x} =
%%       \mvar{x}$,\quad $\delta\mvar{x} = \mvar\dx$.]\

%%       Then $\phi \mvar x = \mvar{x}$ and $\delta\mvar x = \mvar\dx$, so we
%%       need to show that $\J{\mvar{x}}{\Phi\Gamma}{\Phi A}$ and
%%       $\J{\mvar\dx}{\iso\Phi\Gamma, \Delta\Phi\Gamma}{\Delta\Phi A}$. We know
%%       that $\Phi$ and $\Delta$ distribute over contexts, so since $\hm x A \in
%%       \Gamma$ we know that $\Phi(\hm x A) = \hm x {\Phi A} \in \Phi\Gamma$ and
%%       $\Delta\Phi(\hm x A) = \hm{\dx}{\Delta\Phi A} \in \Delta\Phi\Gamma$, which
%%       suffices by \rn{var}.

%%     \item[Case\quad $\infer{\hd x A \in \G}{\J {\dvar x} \G A}$,\quad
%%       $\phi\dvar{x} = \dvar{x}$,\quad $\delta\dvar{x} = \dvar\dx$.]\

%%       Then $\phi\dvar{x} = \dvar{x}$ and $\delta\dvar{x} = \dvar\dx$. By the
%%       same argument as the previous case, it suffices to show that $\hd{x}{\Phi
%%         A} \in \Phi\Gamma$ and $\hd{\dx}{\Delta\Phi A} \in \Delta\Phi\Gamma$,
%%       which is true by distributing $\Phi$ and $\Delta$.

%%     \item[Case\quad $\infer{\J e {\G,\,\hm x A} B}{\J {\efn x e} \G {A \to B}}$,\quad
%%       $\phi(\efn x e) = \efn x \phi e$,\quad
%%       $\delta(\efn x e) = \fnof{\pboxvar x}\efn\dx \delta e$.]\

%%       Then by our inductive hypothesis,
%%       \[
%%       \infer{
%%         \J{\phi e}{\Phi\Gamma,\, \hm x {\Phi A}}{\Phi B}
%%       }{\J{\efn x \phi e}{\Phi\Gamma}{\Phi A \to \Phi B}}
%%       \]
%%       which handles the case for $\phi$.

%%       The case for $\delta$ is complicated by the use of syntax sugar,
%%       $(\fnof{\pboxvar x} ...) \desugars (\efn y \eletbox x {\mvar y} ...)$. For
%%       simplicity's sake we first derive a typing rule for this sugared syntax:

%%       \[
%%       \infer{
%%         \J{f}{\Gamma,\, \hd x A}{B}
%%       }{
%%         \J{\fnof{\pboxvar x} f}{\Gamma}{\iso A \to B}
%%       }
%%       \]

%%       \noindent
%%       which is justified by the following expansion and
%%       weakening~(\cref{theorem-weakening}):

%%       \[
%%       \infer*{
%%         \infer*{
%%           \infer*{
%%             \infer*{~}{
%%               {\hm y {\iso A}} \in {\Gamma, \hm y {\iso A}}
%%           }}{\J{\mvar{y}}{\Gamma, \hm y {\iso A}}{\iso A}}
%%           \\
%%           \infer*[right={weakening}]{
%%             \J{f}{\Gamma,\, \hd x A}{B}
%%           }{\J{f}{\Gamma, \hm y {\iso A}, \hd x A}{B}}
%%         }{
%%           \J{\eletbox x {\mvar y} f}{\Gamma, \hm y {\iso A}}{B}
%%       }}{
%%         \J{\efn y \eletbox x {\mvar y} f}{\Gamma}{\iso A \to B}
%%       }
%%       \]

%%       \noindent
%%       Putting this syntactic sugar to work, we have:

%%       \[
%%       \infer*[right=sugar]{
%%         \infer*{
%%           \J{\delta e}
%%             {\iso\Phi\Gamma,\, \Delta\Phi\Gamma,\, \hd x {\iso A},\,
%%               \hm{\dx}{\Delta\Phi A}}
%%             {\Delta\Phi B}
%%         }{
%%           \J{\efn\dx \delta e}
%%             {\iso\Phi\Gamma,\, \Delta\Phi\Gamma,\, \hd x {\iso\Phi A}}
%%             {\Delta\Phi A  \to \Delta\Phi B}
%%         }
%%       }{
%%         \J{\fnof{\pboxvar x} \efn\dx \delta e}
%%           {\iso\Phi\Gamma,\, \Delta\Phi\Gamma}
%%           {\iso \Phi A \to \Delta\Phi A \to \Delta\Phi B}
%%       }
%%       \]

%%       and if we rearrange the typing context of the premise at the top, we see
%%       that it matches our inductive hypothesis:
%% %
%%       \begin{align*}
%%         &\phantom{{}={}}\iso\Phi\Gamma,\, \Delta\Phi\Gamma,\, \hd x {\iso A},\,
%%         \hm{\dx}{\Delta\Phi A}\\
%%         &=
%%         \iso\Phi\Gamma,\, \hd x {\iso A},\:
%%         \Delta\Phi\Gamma,\, \hm{\dx}{\Delta\Phi A}\\
%%         &=
%%         \iso\Phi(\Gamma,\, \hm x A),\, \Delta\Phi(\Gamma,\, \hm x A)
%%       \end{align*}
%% %
%%       which finishes the case for $\delta$.

%%     \item[Case\quad $\infer{\J e \G {A \to B} \\ \J f \G A}{\J {e\<f} \G B}$,\quad
%%       $\phi(e\<f) = \phi e\<\phi f$,\quad
%%       $\delta(e\<f) = \delta e \<\ebox{\phi f} \<\delta f$.]\

%%       The $\phi$ case fairly straightforward: by our inductive hypotheses,

%%       \[
%%       \infer*{
%%         \J{\phi e}{\Phi\Gamma}{\Phi A \to \Phi B}\\
%%         \J{\phi f}{\Phi\Gamma}{\Phi A}
%%       }{
%%         \J{\phi e\<\phi f}{\Phi\Gamma}{\Phi B}
%%       }
%%       \]

%%       The $\delta$ case is not more complex, but the types get larger, so it
%%       does not fit easily in one derivation. Applying \rn{app} it suffices to
%%       show that, in the context ${\iso\Phi\Gamma,\, \Delta\Phi\Gamma}$, we have
%%       $\delta e \isa {\iso\Phi A \to \Delta\Phi A \to \Delta\Phi B}$ and
%%       ${\ebox{\phi f}} \isa {\iso\Phi A}$ and $\delta f \isa {\Delta\Phi B}$.
%%       The first and last of these are supplied directly by our inductive
%%       hypotheses, while the last needs a use of weakening, observing that
%%       $
%%       \stripcx{\iso\Phi\Gamma, \Delta\Phi\Gamma}
%%       \stronger
%%       \stripcx{\iso\Phi\Gamma}
%%       =
%%       \iso\Phi\Gamma
%%       \stronger
%%       \Phi\Gamma
%%       $ (using \cref{lemma-weakening-strip-context}).

%%       \[
%%       \infer*{
%%         \infer*[right=weakening]{
%%           \J{\phi f}{\Phi\Gamma}{\Phi A}
%%         }{
%%           \J{\phi f}
%%             {\stripcx{\iso\Phi\Gamma,\, \Delta\Phi\Gamma}}
%%             {\Phi A}
%%       }}{
%%         \J{\ebox{\phi f}}{\iso\Phi\Gamma,\, \Delta\Phi\Gamma}{\iso\Phi A}
%%       }
%%       \]

%%     \item[Case\quad $\infer{\quad}{\J {\etuple{}} \G \tunit}$,\quad
%%       $\phi() = ()$,\quad $\delta() = ()$.]\

%%       Trivial.

%%     \item[Case\quad
%%       $\infer{(\J{e_i}\G{A_i})_i}{\J{\etuple{e_1,e_2}} \G {A_1 \x A_2}}$,\quad
%%       $\phi(e_1, e_2) = (\phi e_1, \phi e_2)$,\quad
%%       $\delta(e_1, e_2) = (\delta e_1, \delta e_2)$.]\

%%       Immediate.

%%     \item[Case\quad $\infer{\J e \G {A_1 \x A_2}}{\J{\pi_i\<e}\G{A_i}}$,\quad
%%       $\phi(\pi_i\<e) = \pi_i\<\phi e$,\quad
%%       $\delta(\pi_i\<e) = \pi_i\<\delta e$.]\

%%       Immediate.

%%     \item[Case\quad $\infer{\J e \G {A_1 + A_2} \\
%%         (\J {f_i} {\G,\, \hm{x_i}{A_i}} {B})_i
%%       }{
%%         \J {\emcase{e} (\inj i {\mvar x_i} \caseto f_i)_i} \G B
%%       }$.]\

%%       Recall that
%%       \begin{align*}
%%         \phi(\emcase e (\inj i \mvar x \caseto f_i)_i)
%%         &= \emcase{\phi e} (\inj i \mvar x \caseto \phi f_i)_i
%%         \\
%%         \delta({\emcase{e} (\inj i {\mvar x_i} \caseto f_i)_i})
%%         &=
%%         \emcase{\esplit{\ebox{\phi e}},\, \delta e}\\
%%         &\qquad ({\inj i {\pboxvar x},\, \inj i \mvar\dx} \caseto \delta f_i)_{i}\\
%%         &\qquad ({\inj i {\pboxvar x},\, \inj j \pwild}
%%         %\caseto \subst{\delta f_i}{\dx \substo \dummy\<\dvar x})_{i\ne j}
%%         \caseto \elet{\mvar\dx = \dummy\<\dvar x} \delta f_i)_{i\ne j}
%%       \end{align*}

%%       The $\phi$ case is straightforward, observing that $\Phi(A_1 + A_2) = \Phi
%%       A_1 + \Phi A_2$:
%% %
%%       \[
%%       \infer*{
%%         \infer*[right=ih]{~}{\J{\phi e}{\Phi\Gamma}{\Phi A_1 + \Phi A_2}}
%%         \and
%%         \Big(
%%         \infer*[right=ih]{~}
%%                {\J{\phi f_i}{\Phi\Gamma,\, \hm x \Phi A_i}{\Phi B}}
%%         \Big)_i
%%       }{
%%         \J{\emcase{\phi e} (\inj i \dvar x \caseto \phi f_i)_i}{\Phi\Gamma}{\Phi B}
%%       }
%%       \]

%%       The $\delta$ case is more interesting, as it uses syntax sugar. First we
%%       give a rule for this syntax sugar and establish its derivability:

%%       \[
%%       \infer{
%%         \J{e}{\Omega}{\iso (A_1 + A_2) \x (B_1 + B_2)}\\
%%         (\J{f_{i,j}}{\Omega,\, \hd x A_i,\, \hm y B_j}{C})_{i,j}
%%       }{
%%         \J{\emcase{e} (\inj i \pboxvar{x},\, \inj j \mvar{y} \caseto f_{i,j})_{i,j}}
%%           {\Omega}
%%           {C}
%%       }
%%       \]

%%       This is derivable as follows:

%%       \[
%%       \infer*{
%%         \XXX
%%       }{
%%         \J{\emcase{e} (\inj i \pboxvar{x},\, \inj j \mvar{y} \caseto f_{i,j})_{i,j}}
%%           {\Omega}
%%           {C}
%%       }
%%       \]

%%       We can use this to type the $\delta$-translation as follows:

%%       \[
%%       \infer*{
%%         \J{(\esplit{\ebox{\phi e}},\, \delta e)}
%%           {\iso\Phi\Gamma,\, \Delta\Phi\Gamma}
%%           {?}
%%         \and
%%         ...
%%       }{
%%         \J{\emcase{\esplit{\ebox{\phi e}},\, \delta e} \text{...}}
%%           {\iso\Phi\Gamma,\, \Delta\Phi\Gamma}
%%           {\Delta\Phi B}
%%       }
%%       \]

%%       \todolater{prove this case}

%%     \item[Case\quad $\infer{\J {e} {\stripcx\G} A}{\J{\ebox e} \G {\iso A}}$,\quad
%%       $\phi\ebox{e} = \ebox{(\phi e, \delta e)}$,\quad
%%       $\delta\ebox{e} = ()$.]\label{case-well-typed-box}\

%%       Observe that
%%       \begin{align*}
%%         \Phi\iso A &= \iso(\Phi A \x \Delta\Phi A)
%%         & \Delta\Phi\iso A &= \tunit
%%       \end{align*}

%%       Thus the $\delta$ case is trivial, and the $\phi$ case follows by
%%       weakening our inductive hypotheses:

%%       \[
%%       \infer*{
%%         \infer*{
%%           \infer*[right=ih]{~}{
%%             \J{\phi e}{\stripcx{\Phi\Gamma}}{\Phi A}
%%           }
%%           \\
%%           \infer*[right=ih]{~}{
%%             \J{\delta e}{\stripcx{\Phi\Gamma}}{\Delta\Phi A}
%%           }
%%         }{
%%           \J{(\phi e, \delta e)}{\stripcx{\Phi\Gamma}}{\Phi A \x \Delta\Phi A}
%%         }
%%       }{
%%         \J{\ebox{(\phi e, \delta e)}}{\Phi\Gamma}{\iso(\Phi A \x \Delta\Phi A)}
%%       }
%%       \]

%%       To justify the typing of our inductive hypotheses, observe that
%%       $\Phi\stripcx{\Gamma} = \stripcx{\Phi\Gamma}$ because $\Phi$ preserves the
%%       mode (discrete or monotone) of hypotheses. Justifying
%%       $\stripcx{\Phi\Gamma} = \iso\Phi\stripcx{\Gamma},\,
%%       \Delta\Phi\stripcx{\Gamma}$ requires a little more footwork:

%%       \begin{align*}
%%         &\phantom{{}={}}\iso\Phi\stripcx{\Gamma},\, \Delta\Phi\stripcx{\Gamma}\\
%%         &= \iso\stripcx{\Phi\Gamma},\, \Delta\stripcx{\Phi\Gamma}
%%         &&\text{because }\Phi\stripcx{\Gamma} = \stripcx{\Phi\Gamma}\\
%%         &= \iso\stripcx{\Phi\Gamma}
%%         &&\text{$\stripcx{\Phi\Gamma}$ has only discrete hypotheses, which $\Delta$ drops}\\
%%         &= \stripcx{\Phi\Gamma}
%%         &&\text{$\stripcx{\Phi\Gamma}$ has only discrete hypotheses, which $\iso$ leaves alone}
%%       \end{align*}

%%     \item[Case\quad $\infer{\J e \G {\iso A} \\ \J f {\G,\,\hd x A} B}{
%%       \J {\eletbox x e f} \G B}$.]\

%%       Recall that
%%       \begin{align*}
%%         \phi(\eletbox{x} e f) &= \elet{\pboxtuple{\dvar x,\dvar\dx} = \phi e} \phi
%%         \\
%%         \delta(\eletbox{x} e f) &=
%%         \elet{\pboxtuple{\dvar x, \dvar\dx} = \phi e} \delta f
%%       \end{align*}

%%       This involves some syntax sugar, so let us first derive the following
%%       typing rule:

%%       \[
%%       \infer{
%%         \J{e}{\Gamma}{\iso(A \times B)}\\
%%         \J{f}{\Gamma,\, \hd x A,\, \hd y B}{C}
%%       }{
%%         \J{\elet{\pboxtuple{\dvar x, \dvar y} = e} f}{\Gamma}{C}
%%       }
%%       \]

%%       Expanding our syntax sugar, we have

%%       \[
%%       \elet{\pboxtuple{\dvar x, \dvar y} = e} f
%%       \desugars
%%       \elet{y = e} \XXX
%%       \]

%%       \todolater{what does this desugar to?}

%%       Putting this syntax sugar to work, we have:
%%       \[
%%       \infer*{
%%         \infer*[right=ih]{~}{
%%           \J{\phi e}{\Phi\Gamma}{\iso(\Phi A \x \Delta\Phi A)}
%%         }
%%         \\
%%         \infer*[right=ih]{~}{
%%           \J{\phi f}{\Phi\Gamma, \hd x {\Phi A}, \hd\dx{\Delta\Phi A}}{\Phi B}
%%         }
%%       }{
%%         \J{\elet{\pboxtuple{\dvar x,\dvar\dx} = \phi e} \phi f}
%%           {\Phi\Gamma}
%%           {\Phi B}
%%       }
%%       \]

%%       And similarly:
%%       \[
%%       \infer*{
%%         \J{\phi e}
%%           {\iso\Phi\Gamma, \Delta\Phi\Gamma}
%%           {\iso(\Phi A \x \Delta\Phi A)}
%%         \\
%%         \J{\delta f}
%%           {\iso\Phi\Gamma, \Delta\Phi\Gamma, \hd x {\Phi A}, \hd\dx{\Delta\Phi A}}
%%           {\Delta\Phi B}
%%       }{
%%         \J{\elet{\pboxtuple{\dvar x,\dvar\dx} = \phi e} \delta f}
%%           {\iso\Phi\Gamma, \Delta\Phi\Gamma}
%%           {\Phi B}
%%       }
%%       \]
%%       where the two premises come from our induction hypotheses, the latter
%%       directly, and the former by way of weakening: \todolater{finish proving
%%         this case}

%%     \item[Case\quad $\infer{\quad}{\J\bot\G {L}}$,\quad $\phi\bot = \bot$,
%%       \quad $\delta\bot = \bot$.]\

%%       Trivial, noting only that $\Phi{L} = L$ by \cref{lemma-phi-eqt}
%%       and $\Delta\Phi\eqt{L} = \eqt{L}$ by
%%       \cref{lemma-phi-eqt,lemma-delta-lattice} are both semilattice equality
%%       types as required.

%%     \item[Case\quad $\infer{(\J{e_i} \G {L})_i}{\J{e_1 \vee e_2}\G {L}}$,\quad
%%       $\phi(e_1 \vee e_2) = \phi e_1 \vee \phi e_2$,\quad
%%       $\delta(e_1 \vee e_2) = \delta e_1 \vee \delta e_2$.]\

%%       Immediate, noting only that $\Phi\eqt{L} = L$ by \cref{lemma-phi-eqt}
%%       and $\Delta\Phi\eqt{L} = \eqt{L}$ by
%%       \cref{lemma-phi-eqt,lemma-delta-lattice} are both semilattice equality
%%       types as required.

%%     \item[Case\quad $\infer{(\J {e_i} {\stripcx\G} {\eqt A})_i}{
%%       \J {\esetsub{e_i}{i}} \G {\tset{\eqt A}}}$,\quad
%%       $\phi(\esetsub{e_i}{i}) = \esetsub{\phi e_i}{i}$,\quad
%%       $\delta\esetsub{e_i}{i} = \bot$.]\

%%       Immediate.

%%     \item[Case\quad $\infer{
%%         \J e \G {\tset A} \\
%%         \J f {\G,\, \hd x A} {L}
%%       }{\J {\eforvar x e f} \G {L}}$.]\

%%       Recall that
%%       \begin{align*}
%%         \phi(\eforvar x e f)
%%         &= \eforvar x {\phi e}{\eletbox{\dx}{\ebox{\zero\<\dvar x}} \phi f}\\
%%         \delta(\eforvar x e f)
%%         &= (\eforvar x {\delta e}
%%         %\substd{\phi f}{\dvar\dx \substo \zero\<\dvar x}) \\
%%         \eletbox \dx {\zero\<\dvar x} \phi f) \\
%%         &\vee (\eforvar x {\phi e \vee \delta e}
%%         %\substd{\delta f}{\dvar\dx \substo \zero\<\dvar x})
%%         \eletbox{\dx}{\zero\<\dvar x} \delta f)
%%       \end{align*}

%%       \todolater{prove this case}

%%     \item[Case\quad $\infer{(\J {e_i} {\stripcx\G} {\eqt A})_i}
%%           {\J {\eeq{e_1}{e_2}} \G \tbool}$,\quad
%%           $\phi(\eeq e f) = \eeq{\phi e}{\phi f}$,\quad
%%           $\delta(\eeq e f) = \bot$.]\

%%       The $\phi$ case requires only that $\Phi \eqt{A}$ be an equality type,
%%       which it is by \cref{lemma-phi-eqt}; the $\delta$ case type checks because
%%       $\Delta\Phi\tbool = \tbool$.

%%     \item[Case\quad $\infer{\J {e} {\stripcx\G} {\tset\tunit}}{
%%       \J {\eisempty e} \G {\tunit + \tunit}}$,\quad
%%       $\phi(\eisempty e) = \eisempty{\phi e}$,\quad
%%       $\delta(\eisempty e) = \eisempty{\phi e}$.]\

%%       Immediate, observing that $\Phi\tset{\tunit} = \tset{\tunit}$ and
%%       $\Phi(\tunit + \tunit) = \Delta\Phi(\tunit + \tunit) = \tunit + \tunit$.

%%     \item[Case\quad $\infer{\J e \G {\iso{(A + B)}}}{\J{\esplit e} \G {\iso A + \iso
%%           B}}$.]\

%%       Recall that
%%       \begin{align*}
%%         \phi(\esplit e) &= \emcase{\phi e}
%%         \\
%%         &\phantom{{}={}}\
%%         \left(\pboxtuple{\inj i \dvar x, \inj i \dvar \dx}
%%         \caseto \inj i {\eboxtuple{\dvar x,\dvar\dx}}\right)_{i}
%%         \\
%%         &\phantom{{}={}}\
%%         \left(\pboxtuple{\inj i \dvar x, \inj j \pwild}
%%         \caseto \inj i {\eboxtuple{\dvar x, \dummy\<\dvar x}} \right)_{i\ne j}
%%       \end{align*}

%%       \todolater{prove this case}

%%     \item[Case\quad
%%       $\infer{\J e \Gamma \isofixLtoL}{\J{\efix e} \Gamma {\fixt L}}$
%%       %% $\infer{\J e {\stripcx{\G},\, \hm x {\fixt L}} {\fixt L}}{%
%%       %% \J{\efixis x e} \G {\fixt L}}$.
%%     ]\

%%       Recall that
%%       \begin{align*}
%%         \phi(\efix e)
%%         &= \semifix\<(\phi e,\; \delta e)
%%         & \Phi\fixt{L} &= \fixt{L} \quad\textsf{(\cref{lemma-phi-eqt})}
%%         \\
%%         \delta(\efix e) &= \bot
%%         & \Delta\Phi\fixt{L} &= \fixt{L} \quad\textsf{(\cref{lemma-delta-lattice,lemma-phi-eqt})}
%%       \end{align*}

%%       This justifies the case for $\delta$ immediately. As for $\phi$, recall
%%       the typing rule for $\semifix$:

%%       \[
%%       \infer{
%%         \J{e}{\G}{\iso((\kernfixtL \to \fixtLkern) \x (\iso\fixt L \to \fixt L \to \fixtLkern)}
%%       }{\J{\semifix\<e}{\G}{\fixt L}}
%%       \]

%%       \todolater{prove this case}

%%   \end{description}
%% \end{proof}


\nextlemma
\EqualityChanges*
\begin{proof}
  \label{proof-equality-changes}
  By induction on $\eqt A$, applying the definition from
  \cref{figure-seminaive-logical-relation}:

  \begin{description}
    \item[Case $\tunit$.] Trivial.

    \item[Case $\eqt A \x \eqt B$.] Then our assumption is equivalent to
%
      \[\weirdat{\eqt A \x \eqt B}{(\dx_1,\dx_2)}{(x_1,x_2)}{(a_1,a_2)}{(y_1,y_2)}{(b_1,b_2)}\]
%
      and by unfolding this we have
      \(\weirdat{\eqt{A}}{\dx_1}{x_1}{a_1}{y_1}{b_1}\) and
      \(\weirdat{\eqt{B}}{\dx_2}{x_2}{a_2}{y_2}{b_2}\), which by our inductive
      hypotheses show \(x_1 = a_1\), \(y_1 = b_1\) and \(x_2 = a_2\), \(y_2 = b_2\),
      which suffices.

    \item[Case $\eqt A_1 + \eqt A_2$.] Then for some $i \in \{1,2\}$ our
      assumption is equivalent to
%
      \[
      \weirdat{\eqt A_1 + \eqt A_2}{\inj i \dx}{\inj i x}{\inj i a}
              {\inj i y}{\inj i b}
      \]
%
      and by unfolding this we have \(\weirdat{\eqt{A}_i}{\dx}{x}{a}{y}{b}\),
      which by our inductive hypothesis implies \(x=a\) and \(y=b\), which
      suffices.

    \item[Case $\tseteq{A}$.] Then our assumption unfolds to \((x,y,x \cup \dx)
      = (a,b,y)\), which suffices.

  \end{description}
\end{proof}


\nextlemma
\EqualityDummy*
\begin{proof}
  \label{proof-equality-dummy}
  By induction on \(\eqt{A}\), applying the definitions of $\dummy$ and
  $\weirdat{\eqt{A}}{\dummy\<x}{x}{x}{x}{x}$
  (\cref{figure-dummy,figure-seminaive-logical-relation}).

  \begin{description}
  \item[Case $\tunit$.] Trivial.

  \item[Case $\eqt{A} \times \eqt{B}$.]

    Letting \(x = (y,z)\), we have $\dummy\<x = \dummy\<(y,z) =
    (\dummy\<y,\dummy\<z)$. By our inductive hypotheses, we have
    \(\weirdat{\eqt{A}}{\dummy\<y}{y}{y}{y}{y}\) and likewise for $z$. By
    definition this shows that
    \[
    \weirdat{\eqt{A} \times \eqt{B}}{(\dummy\<y, \dummy\<z)}
            {(y,z)}{(y,z)}{(y,z)}{(y,z)}
    \]
    as desired.

  \item[Case $\eqt{A}_1 + \eqt{A}_2$.]

    Without loss of generality we have \(x = \inj i y\) for some $i \in
    \{1,2\}$. Applying the definition of \dummy\ we have \(\dummy\<x = \inj i
    (\dummy\<y)\). By our inductive hypothesis we have
    \(\weirdat{\eqt{A}_i}{\dummy\<y}{y}{y}{y}{y}\), which suffices to show
    \[
    \weirdat{\eqt{A}_i}{\inj i (\dummy\<y)}
            {\inj i y}{\inj i y}{\inj i y}{\inj i y}
    \]
    as desired.

  \item[Case $\tset{\eqt{A}}$.]

    Unfolding our theorem's definition, we need to show that
    \((x, x, x \cup \dummy\<x) = (x, x, x)\), or in other words
    \(x = x \cup \dummy_{\tseteq{A}}\<x\), which is trivial since \(\dummy_{\tseteq{A}}\<x = \esetraw{}\).
  \end{description}
\end{proof}


\nextlemma
\DiscreteContexts*
\DiscreteContextsProof*

%% \begin{proof}
%%   \label{proof-discrete-contexts}

%%   First, let's observe the types of the contexts we are dealing with. We have
%%   \(\rho, \rho' \in \den{\stripcx{\Gamma}}\) and \(\gamma, \gamma' \in
%%   \den{\Phi\stripcx{\Gamma}}\). In particular this means that all four will
%%   contain only discrete variables, namely, for every $\hd x A \in \Gamma$, we
%%   will have $\rho_{\dvar x}, \rho'_{\dvar{x}}$ and $\gamma_{\dvar{x}},
%%   \gamma_{\dvar \dx}, \gamma'_{\dvar{x}}, \gamma'_{\dvar\dx}$.

%%   Next, unfolding the definition of our assumption, we have
%%   \[
%%   \fa{\hm x A \in \stripcx{\Gamma}}
%%   \weirdat{A}{\dgamma_{\mvar\dx}}
%%           {\gamma_{\mvar{x}}}{\rho_{\mvar{x}}}
%%           {\gamma'_{\mvar{x}}}{\rho'_{\mvar{x}}}
%%   \]
%%   which is boring --- there \emph{are} no monotone hypotheses
%%   $\hm x A \in \stripcx{\Gamma}$ --- but also:
%%   \[
%%   \fa{\hd x A \in \stripcx{\Gamma}}
%%   \weirdat{\iso A}{()}
%%           {(\gamma_{\dvar x}, \gamma_{\dvar \dx})}
%%           {\rho_{\dvar x}}
%%           {(\gamma'_{\dvar x}, \gamma'_{\dvar \dx})}
%%           {\rho'_{\dvar x}}
%%   \]

%%   \noindent
%%   Unfolding the logical relation for the box type $\iso$, we have for each
%%   $\hd x A \in \stripcx{\Gamma}$ that
%% %
%%   \[
%%   (\rho_{\dvar x}, \gamma_{\dvar x}, \gamma_{\dvar\dx}) =
%%   (\rho'_{\dvar x}, \gamma'_{\dvar x}, \gamma'_{\dvar\dx})
%%   ~~\wedge~~ \weirdat{A}{\gamma_{\dvar\dx}}
%%          {\gamma_{\dvar x}}{\rho_{\dvar x}}
%%          {\gamma'_{\dvar x}}{\rho'_{\dvar x}}
%%   \]
%% %
%%   of which the first conjunct tells us that each component of $\rho$ equals
%%   its partner in $\rho'$ and the same for $\gamma$ and $\gamma'$, as desired.
%% \end{proof}


\nextlemma
\ContextStripping*
\ContextStrippingProof*

%% \begin{proof}
%%   \label{proof-context-stripping}
%%   To prove this, it suffices to show for each $\hd x A \in \stripcx{\Gamma}$
%%   (since after all there are no monotone $\hm x A \in \stripcx{\Gamma}$) that
%%   \[
%%   \weirdat{\iso A}{()}
%%           {(\strip(\gamma)_{\dvar x},\,
%%             \strip(\gamma)_{\dvar\dx})}
%%           {\strip(\rho)_{\dvar x}}
%%           {(\strip(\gamma')_{\dvar x},\,
%%             \strip(\gamma')_{\dvar\dx})}
%%           {\strip(\rho')_{\dvar x}}
%%   \]
%%   but since $\strip$ merely projects out the discrete variables, this is
%%   equivalent to
%%   \[
%%   \weirdat{\iso A}{()}
%%           {(\gamma_{\dvar x},\, \gamma_{\dvar\dx})}
%%           {\rho_{\dvar x}}
%%           {(\gamma'_{\dvar x},\, \gamma'_{\dvar\dx})}
%%           {\rho'_{\dvar x}}
%%   \]
%%   which is true by our assumption.
%% \end{proof}


\nextlemma
\begin{lemma}[Applying box]
  \label{lemma-applying-box}
  %
  Given $\J e {\stripcx\Gamma} A$ and $\gamma : \den{\Gamma}$,
  \[
  \den{\ebox{e}} \<\gamma
  = \mkbox_\Gamma(\den{e})(\gamma)
  = \den{e} \<(\morphstrip_\Gamma(\gamma))
  \]
\end{lemma}

\begin{proof}
  Recall that the box comonad \iso's functorial action, duplication map $\delta_A : \iso A \to \iso\iso A$, and distribution $\isox : \prod_i\iso A_i \to \iso\prod_i A_i$ are all no-ops. Then:
  
  \begin{align*}
    \den{\ebox{e}} \<\gamma
    &= \mkbox_\Gamma(\den{e})(\gamma)
    && \textsf{definition of}~\den{\ebox{e}}\\
    &= \iso\den{e}(\isox(\delta_A(\gamma_{\dvar x})_{\hd x A \in \Gamma}))
    && \textsf{definition of}~\mkbox\\
    &= \den{e}(\gamma_{\dvar x})_{\hd x A \in \Gamma}
    && \textsf{no-ops}\\
    &= \den{e}(\morphstrip_\Gamma(\gamma))
    && \textsf{definition of \morphstrip}
  \end{align*}
\end{proof}


\nextlemma
\begin{lemma}[Correctness of \semifix]
  \label{lemma-semifix}
  %
  If $\weirdat{\fixt L \to \fixt L}{g'}{g}{f}{g}{f}$, then $\semifix\<(g,g') = \morph{fix}\<f$.
\end{lemma}

\begin{proof}
  First, let's expand our assumption:
  \begin{align*}
    \fa{\weirdat{\fixt L}{dx} x a y b}
    \weirdat{\fixt L}
            {g'\<x\<dx}
            {g\<x}
            {f\<a}
            {g\<y}
            {f\<b}
  \end{align*}

  \noindent
  If we apply \cref{lemma-semilattice-changes} and simplify slightly, this is equivalent to:
%
  \begin{equation}\label{fix-assumption}
    \fa{x,dx : \fixt L}
    g\<x = f\<x ~\textsf{and}~ g\<(x \vee dx) = f\<(x \vee dx) = g\<x \vee g'\<x\<dx
  \end{equation}

  \noindent
  This in particular implies that $f = g$.
  
  Now, recall that $\morph{fix}\<f$ is defined as the limit $\bigvee_i f^i\<\bot$ of the iterations of $f$, while $\semifix\<(g, g')$ is the limit $\bigvee_i x_i$ of the sequence $x_i$ given by:
  %
  \begin{align*}
    x_0 &= \bot & x_{i+1} &= x_i \vee dx_i\\
    dx_0 &= g\<\bot & dx_{i+1} &= g'\<x_i\<dx_i
  \end{align*}

  \noindent
  Thus it suffices to show that $x_i = f^i\<\bot$, which we will show
  inductively, along with $x_i \vee dx_i = f\<x_i$. To establish the base case,
  $x_0 = \bot = f^0\<\bot$ by definition and $x_0 \vee dx_0 = \bot \vee
  g\<\bot = f\<\bot$ because $g = f$. Inductively assuming that $x_i =
  f^i\<\bot$ and $x_i \vee dx_i = f\<x_i$, we have that $x_{i+1} = x_i \vee dx_i
  = f\<x_i = f\<(f^i\<\bot) = f^{i+1}\<\bot$ as desired, and finally:
  %
  \begin{align*}
    x_{i+1} \vee dx_{i+1}
    &= (x_i \vee dx_i) \vee g'\<x_i\<dx_i
    && \text{expanding definitions}\\
    &= f\<x_i \vee g'\<x_i\<dx_i
    && \text{inductive hypothesis}\\
    &= g\<x_i \vee g'\<x_i\<dx_i
    && \text{because}~f = g\\
    &= f\<(x_i \vee dx_i)
    && \text{by \cref{fix-assumption}}\\
    &= f\<x_{i+1}
    && \text{definition of }x_{i+1}
  \end{align*}
\end{proof}


\nextlemma

\SeminaiveFundamental*
\begin{proof}
  \label{proof-seminaive-fundamental}
  By induction on the derivation of $\J e \Gamma A$.
  %
  We will refer to the other premise $\weirdat\Gamma\dgamma\gamma\rho{\gamma'}{\rho'}$ as simply ``the assumption''.

  \begin{description}[topsep=\baselineskip,itemsep=\baselineskip]

    %% CASES: VARIABLES
  \item[Case $\infer{\hm x A \in \G}{\J{\mvar x}\G A}$.]
    We wish to show:

    \begin{align*}
      &\weirdat A
               {\den{\delta \mvar x}\<(\gamma,\dgamma)}
               {\den{\phi \mvar x}\<\gamma}
               {\den{\mvar x}\<\rho}
               {\den{\phi \mvar x}\<\gamma'}
               {\den{\mvar x}\<\rho'}
      \\
      \iff&
      \weirdat A
               {\den{\mvar\dx}\<(\gamma,\dgamma)}
               {\den{\mvar x}\<\gamma}
               {\den{\mvar x}\<\rho}
               {\den{\mvar x}\<\gamma'}
               {\den{\mvar x}\<\rho'}
      \\
      \iff&
      \weirdat A
               {\dgamma_{\mvar\dx}}
               {\gamma_{\mvar x}}
               {\rho_{\mvar x}}
               {\gamma'_{\mvar x}}
               {\rho'_{\mvar x}}
    \end{align*}

    \noindent
    which follows from the definition of our assumption.

  \item[Case $\infer{\hd x A \in \G}{\J{\dvar x}\G A}$.]
    We wish to show:

    \begin{align*}
      &\weirdat A
               {\den{\delta \dvar x}\<(\gamma,\dgamma)}
               {\den{\phi \dvar x}\<\gamma}
               {\den{\dvar x}\<\rho}
               {\den{\phi \dvar x}\<\gamma'}
               {\den{\dvar x}\<\rho'}
      \\
      \iff&
      \weirdat A
               {\den{\dvar\dx}\<(\gamma,\dgamma)}
               {\den{\dvar x}\<\gamma}
               {\den{\dvar x}\<\rho}
               {\den{\dvar x}\<\gamma'}
               {\den{\dvar x}\<\rho'}
      \\
      \iff&
      \weirdat A
               {\gamma_{\dvar\dx}}
               {\gamma_{\dvar x}}
               {\rho_{\dvar x}}
               {\gamma'_{\dvar x}}
               {\rho'_{\dvar x}}
    \end{align*}

    If we apply our assumption we get:

    \begin{align*}
      &
      \weirdat \G \dgamma \gamma \rho {\gamma'} {\rho'}
      \\
      \implies&
      \weirdat {\iso A}
               {\tuple{}}
               {\tuple{\gamma_{\dvar x}, \gamma_{\dvar\dx}}}
               {\rho_{\dvar x}}
               {\tuple{\gamma'_{\dvar x}, \gamma'_{\dvar\dx}}}
               {\rho'_{\dvar x}}
      \\
      \implies&
      \weirdat A {\gamma_{\dvar\dx}}
               {\gamma_{\dvar x}}
               {\rho_{\dvar x}}
               {\gamma'_{\dvar x}}
               {\rho'_{\dvar x}}
    \end{align*}

    \noindent
    as desired.

    %% CASE: LAMBDA
  \item[Case $\infer{\J e {\G,\hm x A} B}{\J{\efn x e} \G {A \to B}}$]
    Recall that $\phi(\efn x e) = \efn x \phi e$ and $\delta(\efn x e) = \fnof{\pboxvar x} \efn\dx \delta e$. We wish to show

    \[
    \weirdat{A \to B}
            {\den{\fnof{\pboxvar x} \efn\dx \delta e}\<(\gamma,\dgamma)}
            {\den{\efn x \phi e}\<\gamma}
            {\den{\efn x e}\<\rho}
            {\den{\efn x \phi e}\<\gamma'}
            {\den{\efn x e}\<\rho'}
    \]

    \noindent
    Applying the definition of the logical relation at $A \to B$, it suffices to assume (a) $\weirdat A \dx x a y b$ and prove

    \[
    \weirdat B
            {\den{\fnof{\pboxvar x} \efn\dx \delta e}\<(\gamma,\dgamma)\<x\<\dx}
            {\den{\efn x \phi e}\<\gamma\<x}
            {\den{\efn x e}\<\rho\<a}
            {\den{\efn x \phi e}\<\gamma'\<y}
            {\den{\efn x e}\<\rho'\<b}
    \]

    \noindent
    which, by calculation, is:

    \[
    \weirdat
    B
    {\den{\delta e}\<\sigma}
    {\den{\phi e} \<(\gamma, \subto{\mvar x} x)}
    {\den{e} \<(\rho, \subto{\mvar x} a)}
    {\den{\phi e} \<(\gamma, \subto{\mvar x} y)}
    {\den{e} \<(\rho, \subto{\mvar x} b)}
    \]

    \noindent
    where $\sigma = (\gamma,\dgamma,\subto{\dvar x}{x}, \subto{\mvar\dx}{\dx})$. This follows from our inductive hypothesis if we can show that:

    \[
    \weirdat
        {\G,\hm x A}
        {(\dgamma, \subto{\mvar\dx}\dx)}
        {(\gamma, \subto{\mvar x} x)}
        {(\rho, \subto{\mvar x} a)}
        {(\gamma', \subto{\mvar x} y)}
        {(\rho', \subto{\mvar x} b)}
    \]

    \noindent
    and this follows from our assumption and (a).

    %% CASE: FUNCTION APPLICATION
  \item[Case $\infer{
      \J e \G {A \to B} \\
      \J f \G A
    }{
      \J {e\<f} \G B
    }$.]
    Recall that $\phi(e\<f) = \phi e \< \phi f$ and $\delta(e\<f) = \delta e \<\ebox{\phi f} \<\delta f$. Thus we wish to show:

    \[
    \weirdat B
    {\den{\delta e \<\ebox{\phi f} \<\delta f} \<(\gamma,\dgamma)}
    {\den{\phi e \<\phi f} \<\gamma}
    {\den{e\<f}\<\rho}
    {\den{\phi e \<\phi f} \<\gamma'}
    {\den{e\<f}\<\rho'}
    \]

    Let:

    \begin{align*}
      \dx &= \den{\delta e}\<(\gamma,\dgamma)
      &&&
      \dy &= \den{\delta f}\<(\gamma,\dgamma)
      \\
      x &= \den{\phi e} \<\gamma
      &
      a &= \den{e}\<\rho
      &
      y &= \den{\phi f}\<\gamma
      &
      b &= \den{f}\<\rho
      \\
      x' &= \den{\phi e}\<\gamma'
      &
      a' &= \den{e}\<\rho'
      &
      y' &= \den{\phi f}\<\gamma'
      &
      b' &= \den{f}\<\rho'
    \end{align*}

    \noindent
    By our IH for $f$ we have $\weirdat A \dy y b {y'}{b'}$. By this and our IH for $e$ we have $\weirdat B {\dx\<y\<\dy}{x\<y}{a\<b}{x'\<y'}{a'\<b'}$. By calculation, this is equal to what we wish to show.
%
    \todo{Not quite! Need a form of weakening.}

    %% \noindent
    %% By calculation, we have for any appropriately typed $\sigma$:

    %% \begin{align*}
    %%   {\den{\delta e \<\ebox{\phi f} \<\delta f} \<\sigma}
    %%   &=
    %%   \den{\delta e}\<\sigma
    %%   %% \<(\den{\ebox{\phi f}} \<\sigma)
    %%   \<(\den{\phi f} \<\morphstrip_{\Phi\G}(\sigma))
    %%   \<(\den{\delta f} \<\sigma)
    %%   \\
    %%   {\den{\phi e \<\phi f} \<\sigma}
    %%   &=
    %%   \den{\phi e} \<\sigma \<(\den{\phi f} \<\sigma)
    %% \end{align*}

  \item[Case $\infer{ }{\J{\etuple{}}\G\tunit}$.]
    We wish to show:

    \begin{align*}
    &\weirdat \tunit
             {\den{\etuple{}}\<(\gamma,\dgamma)}
             {\den{\etuple{}}\<\gamma}
             {\den{\etuple{}}\<\rho}
             {\den{\etuple{}}\<\gamma'}
             {\den{\etuple{}}\<\rho'}
    \\
    \iff
    &\weirdat \tunit {\tuple{}}{\tuple{}}{\tuple{}}{\tuple{}}{\tuple{}}
    \\
    \iff& \top
    \end{align*}

  \item[Case $\infer{(\J{e_i}\G{A_i})_i}{\J{\etuple{e_1,e_2}}\G{A_1 \times A_2}}$:]
    Recall that $\phi(\etuple{e_1,e_2}) = \etuple{\phi e_1, \phi e_2}$ and $\delta(\etuple{e_1, e_2}) = \etuple{\delta e_1, \delta e_2}$. Thus we wish to show:

    \[
    \weirdat {A_1 \times A_2}
             {\den{\etuple{\delta e_1, \delta e_2}}\<(\gamma,\dgamma)}
             {\den{\etuple{\phi e_1, \phi e_2}}\<\gamma}
             {\den{\etuple{e_1, e_2}}\<\rho}
             {\den{\etuple{\phi e_1, \phi e_2}}\<\gamma'}
             {\den{\etuple{e_1, e_2}}\<\rho'}
    \]

    \noindent
    Since in general $\den{(f_1, f_2)} \<\sigma = \tuple{\den{f_1}\<\sigma, \den{f_2}\<\sigma}$, applying the definition of the LR at $A_1 \times A_2$ this is equivalent to:

    \[
    \fa i
    \weirdat{A_i}
    {\den{\delta e_i} \<(\gamma,\dgamma)}
    {\den{\phi e_i}\<\gamma}
    {\den{e_i}\<\rho}
    {\den{\phi e_i}\<\gamma'}
    {\den{e_i}\<\rho'}
    \]

    \noindent
    which holds by our IH.

  \item[Case $\infer{\J e \G {A_1 \times A_2}}{\J{\pi_i\<e}\G{A_i}}$.]
    Recall that $\phi(\pi_i\<e) = \pi_i\<\phi e$ and $\delta(\pi_i\<e) = \pi_i\<\delta e$ and observe that $\den{\pi_i\<f}\<\sigma = \pi_i(\den{f}\<\sigma)$. Applying this, what we wish to show is

    \[
    \weirdat {A_i}
    {\pi_i\<(\den{\delta e}\<(\gamma,\dgamma))}
    {\pi_i \<(\den{\phi e}\<\gamma)}
    {\pi_i \<(\den{e}\<\rho)}
    {\pi_i \<(\den{\phi e}\<\gamma')}
    {\pi_i \<(\den{e}\<\rho')}
    \]

    \noindent
    which is a direct consequence of our IH.

  \item[Case $\infer{\J e \G {A_i}}{\J {\inj i e} \G {A_1 + A_2}}$.]
    Recall that $\phi(\inj i e) = \inj i \phi e$ and $\delta(\inj i e) = \inj i \delta e$. Observe that in general $\den{\inj i f}\<\sigma = \inj i (\den{f}\<\sigma)$ and therefore what we wish to show is equivalent to:

    \[
    \weirdat{A_1 + A_2}
    {\inj i (\den{\delta e}\<(\gamma,\dgamma))}
    {\inj i (\den{\phi e}\<\gamma)}
    {\inj i (\den{e}\<\rho)}
    {\inj i (\den{\phi e}\<\gamma')}
    {\inj i (\den{e}\<\rho')}
    \]
    
    \noindent
    which is by definition equivalent to our inductive hypothesis.

  \item[Case $\infer{ }{\J \bot \G {L}}$.]
    Recall that $\phi \bot = \bot$ and $\delta \bot = \bot$ and $\den{\bot}\<\sigma = \bot$. Thus it STS $\weirdat{L} \bot \bot \bot \bot \bot$, which holds by \cref{lemma-semilattice-changes}.

  \item[Case $\infer{(\J{e_i} \G {L})_i}{\J{e_1 \vee e_2} \G {L}}$.]
    Recall that $\phi(e_1 \vee e_2) = \phi e_1 \vee \phi e_2$. We wish to show:

    \[
    \weirdat{L}
    {\den{\delta e_1 \vee \delta e_2} \<(\gamma,\dgamma)}
    {\den{\phi e_1 \vee \phi e_2}\<\gamma}
    {\den{e_1 \vee e_2}\<\rho}
    {\den{\phi e_1 \vee \phi e_2}\<\gamma'}
    {\den{e_1 \vee e_2}\<\rho'}
    \]

    Let:

    \begin{align*}
      \dx_i &= \den{\delta e_i}\<(\gamma,\dgamma)
      &
      x_i &= \den{\phi e_i}\<\gamma
      &
      x'_i &= \den{\phi e_i}\<\gamma'
      &
      a_i &= \den{e_i}\<\rho
      &
      a'_i &= \den{e_i}\<\rho'
    \end{align*}

    Since $\den{f \vee g}\<\sigma = \den{f}\<\sigma \vee \den{g}\<\sigma$, what we wish to show is equivalent to:

    \[
    \weirdat{L} {\dx_1 \vee \dx_2}
    {x_1 \vee x_2} {a_1 \vee a_2}
    {x'_1 \vee x'_2} {a'_1 \vee a'_2}
    \]

    \noindent
    Applying \cref{lemma-semilattice-changes} this is equivalent to:

    \begin{align*}
      x_1 \vee x_2 &= a_1 \vee a_2
      &
      x'_1 \vee x'_2 &= a'_1 \vee a'_2 = x_1 \vee x_2 \vee \dx_1 \vee \dx_2
    \end{align*}
    
    By our IH and assumption and \cref{lemma-semilattice-changes} we have:

    \begin{align*}
      x_i &= a_i
      &
      x'_i &= a'_i = x_i \vee \dx_i
    \end{align*}

    \noindent
    which suffices by associativity and commutativity of $\vee$.

  \item[Case $\infer{(\J{e_i} \G {\eqt A})_i}{\J{\esetraw{e_i}_i} \G {\tseteq A}}$.]
    Recall that $\phi \esetraw{e_i}_i = \esetraw{\phi e_i}_i$ and $\delta \esetraw{e_i}_i = \bot$. Noting that $\den{\bot}\<(\gamma,\dgamma) = \bot$, by \cref{lemma-semilattice-changes} what we want to show is equivalent to:

    \begin{align*}
      \den{\esetraw{\phi e_i}_i}\<\gamma
      =
      \den{\esetraw{e_i}_i} \<\rho
      =
      \den{\esetraw{\phi e_i}_i} \<\gamma'
      =
      \den{\esetraw{e_i}_i} \<\rho'
    \end{align*}


    \noindent
    Note that $\den{\esetraw{f_i}_i} \<\sigma = \bigvee_i \{\mkbox_\G(\den{f_i})(\sigma)\} = \{\den{f_i}\<(\morphstrip \<\sigma)\}_i$.
%
    Also observe that by \cref{lemma-discrete-contexts,lemma-context-stripping} and our assumption, we have

    \begin{eqnarray}
      \label{hooligan}
      &\weirdat{\stripcx\G}{\tuple{}}{\morphstrip\<\gamma}{\morphstrip\<\rho}
      {\morphstrip\<\gamma'}{\morphstrip\<\rho'}
      \\
      \label{football}
      &\morphstrip\<\gamma = \morphstrip\<\gamma'
      \quad\text{and}\quad
      \morphstrip\<\rho = \morphstrip\<\rho'
    \end{eqnarray}

    \noindent
    So applying \cref{football} it suffices to show that
    $\{\den{\phi e_i} \<(\morphstrip \<\gamma)\}_i = \{\den{e_i} \<(\morphstrip \<\rho)\}_i$, for which it suffices to show $\den{\phi e_i} \<(\morphstrip \<\gamma) = \den{e_i} \<(\morphstrip \<\rho)$. This holds by our IH for $e_i$ and \cref{hooligan} and \cref{lemma-equality-changes}.
        
  \item[Case $\infer{(\J {e_i} {\stripcx{\G}} {\eqt A})_i}{\J{\eeq{e_1}{e_2}} \G \tbool}$.]

    Recall that $\phi(\eeq{e_1}{e_2}) = \eeq{\phi e_1}{\phi e_2}$ and $\delta(\eeq{e_1}{e_2}) = \bot$. Thus what we wish to show is:

    \[
    \weirdat{\tbool}
            {\den{\bot} \<(\gamma,\dgamma)}
            {\den{\eeqraw{\phi e_1}{\phi e_2}} \<\gamma}
            {\den{\eeqraw{e_1}{e_2}} \<\rho}
            {\den{\eeqraw{\phi e_1}{\phi e_2}} \<\gamma'}
            {\den{\eeqraw{e_1}{e_2}} \<\rho'}
    \]
    
    \noindent
    Observing that $\den{\bot}\<(\gamma,\dgamma) = \emptyset$ and applying the definition of the logic relation for $\tbool = \tset{\tunit}$, this is equivalent to:

    \[
    \den{\eeq{\phi e_1}{\phi e_2}}\<\gamma
    =
    \den{\eeq{\phi e_1}{\phi e_2}}\<\gamma'
    =
    \den{\eeq{e_1}{e_2}}\<\rho
    =
    \den{\eeq{e_1}{e_2}}\<\rho'
    \]

    \noindent
    Observe by calculation that

    \[
    \den{\eeq{f_1}{f_2}}\<\sigma
    =
    \begin{cases}
      \{()\} &\text{if}\ \den{f_1} \<(\morphstrip\<\sigma) = \den{f_2}\<(\morphstrip\<\sigma)\\
      \emptyset & \text{otherwise}
    \end{cases}
    \]

    \noindent
    Thus it suffices to show $\den{\phi e_i} \<(\morphstrip\<\gamma) = \den{\phi e_i} \<(\morphstrip\<\gamma') = \den{e_i} \<(\morphstrip\<\rho) = \den{e_i} \<(\morphstrip\<\rho')$. By our assumption and \cref{lemma-discrete-contexts,lemma-context-stripping} we know $\morphstrip\<\gamma = \morphstrip\<\gamma'$ and $\morphstrip\<\rho = \morphstrip\<\rho'$ and $\weirdat{\stripcx\G}{\tuple{}}{\morphstrip\<\gamma}{\morphstrip\<\rho}{\morphstrip\<\gamma'}{\morphstrip\<\rho'}$.
%
    By the first two equalities it now suffices to show $\den{\phi e_i} \<(\morphstrip \<\gamma) = \den{e_i} \<(\morphstrip \<\rho)$.
%
    Applying our IH to the remaining third proposition we have

    \[
    \weirdat{\eqt A}{\den{\delta e_i} \<(\morphstrip\<\gamma)}
    {\den{\phi e_i} \<(\morphstrip \<\gamma)}
    {\den{e_i} \<(\morphstrip \<\rho)}
    {\den{\phi e_i} \<(\morphstrip \<\gamma')}
    {\den{e_i} \<(\morphstrip \<\rho')}
    \]

    \noindent
    which by \cref{lemma-equality-changes} implies $\den{\phi e_i} \<(\morphstrip \<\gamma) = \den{e_i} \<(\morphstrip \<\rho)$ as desired.

  \item[Case $\infer{\J e {\stripcx\G} {\tset \tunit}}{\J {\eisempty e} \Gamma {\tunit + \tunit}}$.]

    Recall that $\phi(\eisempty e) = \delta(\eisempty e) = \eisempty{\phi e}$.
%
    Note that by \cref{lemma-discrete-contexts,lemma-context-stripping} we have $\morphstrip\<\gamma = \morphstrip\<\gamma'$ and $\morphstrip\<\rho = \morphstrip\<\rho'$ and $\weirdat{\stripcx\G}{()}{\morphstrip\<\gamma}{\morphstrip\<\rho}{\morphstrip\<\gamma'}{\morphstrip\<\rho'}$; applying this to our inductive hypothesis and invoking \cref{lemma-equality-changes} on the result, we have

    \begin{align*}
      \den{e}\<(\morphstrip\<\rho)
      = \den{e}\<(\morphstrip\<\rho')
      = \den{\phi e} \<(\morphstrip\<\gamma)
      = \den{\phi e} \<(\morphstrip\<\gamma')
    \end{align*}

    \noindent
    Thus all are equal to the same value. Now, observe by calculation that

    \[
    \den{\eisempty f} \<\sigma = 
    \begin{cases}
      \inj i () & \text{if}~\den{f}\<(\morphstrip\<\sigma) = \emptyset\\
      \inj 2 () & \text{otherwise}
    \end{cases}
    \]

    \noindent
    Thus, there is some $i$ such that $\den{\eisempty e}\<\rho = \den{\eisempty e}\<\rho' = \den{\eisempty{\phi e}}\<\gamma = \den{\eisempty{\phi e}}\<\gamma' = \inj i ()$ and we have $\weirdat{\tunit + \tunit}{\inj i ()}{\inj i ()}{\inj i ()}{\inj i ()}{\inj i ()}$ as desired.

    \todo{weakening! for the delta.}

    %% CASE: CASE ANALYSIS
  \item[Case $\infer{
      \J e \G {A_1 + A_2} \\
      (\J {f_i} {\G,\, \hm{x_i}{A_i}} {B})_i
    }{
      \J{\emcase e (\inj i \mvar x \caseto f_i)_i} \Gamma B
    }$.] Recall that
%
    \begin{align*}
      \phi(\emcase e (\inj i \mvar x \caseto f_i)_i)
      &= \emcase{\phi e} (\inj i \mvar x \caseto \phi f_i)_i
      \\
      \delta(\emcase e (\inj i \mvar x \caseto f_i)_i)
      &=
    \emcase{\esplit{\ebox{\phi e}}}\\
    &\qquad
    (
    \inj i {\color{fresh}\mvar y} \caseto
    \eletbox{x}{\color{fresh}\mvar y}
    \\
    &\qquad\phantom{(\inj i {\mvar y} \caseto {}}
    (\efn{\dx} \delta f_i)
    \\
    &\qquad\phantom{(\inj i {\mvar y} \caseto {}}
    (\emcase{\delta e}
    \inj i \mvar\dx \caseto \mvar\dx
    \\
    &\qquad\phantom{(\inj i {\mvar y} \caseto {} (\emcase{\delta e}}
    \inj{i+1 \bmod 2} \pwild \caseto \dummy\<\dvar x
    ))_i
    \end{align*}

    \noindent
    Let $\rho_1 = \rho$, $\rho_2 = \rho'$, $\gamma_1 = \gamma$, $\gamma_2 = \gamma'$.
%
    By our inductive hypothesis for $e$ there must be some $k \in \{1,2\}$ and some $dx, x_i, a_i$ such that $\weirdat{A_k}{dx}{x_1}{a_1}{x_2}{a_2}$ and:
%
    \begin{align*}
      \den{e}\<\rho_i &= \inj k a_i
      &
      \den{\phi e}\<\gamma_i &= \inj k x_i
      & \den{\delta e}\<(\gamma, \dgamma) &= \inj k dx
    \end{align*}

    \noindent
    Applying this and our inductive hypothesis for $f_k$ it will suffice to show that:
%
    \begin{align}\label{case-a}
      \den{\emcase e (\inj i \mvar x \caseto f_i)_i} \<\rho_i
      &= \den{f_k}\<(\rho_i, \mvar x \mapsto a_i)
      \\\label{case-b}
      \den{\phi(\emcase e (\inj i \mvar x \caseto f_i)_i)} \<\gamma_i
      &= \den{\phi f_k}\<(\gamma_i, \mvar x \mapsto x_i)
      \\\label{case-c}
      \den{\delta(\emcase e (\inj i \mvar x \caseto f_i)_i)}
      \<(\gamma, \dgamma)
      &= \den{\delta f_k}
      \<(\gamma, \dgamma, \dvar x \mapsto x_1, \mvar\dx \mapsto dx)
    \end{align}

    \noindent
    Showing this is a matter of calculation. \Cref{case-a,case-b} are fairly straightforward to calculate, so we only show \cref{case-c} in detail. Before starting, it will be useful to give some abbreviations for subterms of $\delta(\emcase e (\inj i \mvar x \caseto f_i)_i)$:

    \begin{align*}
      h_i &= \emcase{\delta e}
      \inj i \mvar\dx \caseto \mvar\dx;\;
      \inj{i+1 \bmod 2} \pwild \caseto \dummy\<\dvar x
      \\
      g_i &= \eletbox{x}{\color{fresh}\mvar y}
      (\efn{\dx} \delta f_i) \<h_i
    \end{align*}

    \noindent
    It will also be useful to note the type of $\phi e$ as it occurs in the sub-expression $\esplit \ebox{\phi e}$ being immediately analyzed by $\delta(\emcase e (\inj i \mvar x \caseto f_i)_i)$; it has been \emph{weakened} to the type $\J{\phi e}{\iso\Phi\Gamma}{\Phi A_1 + \Phi A_2}$.

    \begin{align*}
      &\phantom{{}={}}
      \den{\delta(\emcase e (\inj i \mvar x \caseto f_i)_i)}
      \<(\gamma, \dgamma)
      \\
      &=
      \krof{\den{g_i}}_i
      \<(\morph{dist}^\x_+
      ((\gamma, \dgamma),\, \den{\esplit \ebox{\phi e}}\<(\gamma, \dgamma)))
      \\
      &=
      \krof{\den{g_i}}_i
      \<(\morph{dist}^\x_+
      ((\gamma, \dgamma),\,
      \morph{dist}^\iso_+ (\den{\ebox{\phi e}}\<(\gamma, \dgamma))))
      \\
      &=
      \krof{\den{g_i}}_i
      \<(\morph{dist}^\x_+
      ((\gamma, \dgamma),\,
      \den{\ebox{\phi e}}\<(\gamma, \dgamma)))
      && \morph{dist}^\iso_+~\text{is the identity}
      \\
      &=
      \krof{\den{g_i}}_i
      \<(\morph{dist}^\x_+
      ((\gamma, \dgamma),\,
      \den{\phi e}\<(\morphstrip_{\iso\Phi\Gamma} \<(\gamma, \dgamma))))
      && \text{\cref{lemma-applying-box}}
      \\
      &=
      \krof{\den{g_i}}_i
      \<(\morph{dist}^\x_+ ((\gamma, \dgamma),\, \den{e} \<\gamma))
      \\
      &=
      \krof{\den{g_i}}_i
      \<(\morph{dist}^\x_+ ((\gamma, \dgamma),\, \inj k x_1))
      \\
      &=
      \krof{\den{g_i}}_i \<(\inj k (\gamma, \dgamma, \mvar y \mapsto x_1))
      \\
      &= \den{g_k} \<(\gamma, \dgamma, \mvar y \mapsto x_1)
      \\
      &= \den{\eletbox{x}{\color{fresh}\mvar y}
        (\efn{\dx} \delta f_k) \<h_k}
      \<(\gamma, \dgamma, \mvar y \mapsto x_1)
      \\
      &= \den{(\efn{\dx} \delta f_k) \<h_k}
      \<(\gamma, \dgamma, \dvar x \mapsto x_1)
      \\
      &= \den{(\efn{\dx} \delta f_k)}\<(\gamma, \dgamma, \dvar x \mapsto x_1)
      \<(\den{h_k} \<(\gamma, \dgamma, \dvar x \mapsto x_1))
      \\
      &= (dx \mapsto \den{\delta f_k}\<(\gamma, \dgamma, \dvar x \mapsto x_1, \mvar\dx \mapsto dx))
      \\
      &\phantom{{}={}} \<(\den{h_k} \<(\gamma, \dgamma, \dvar x \mapsto x_1))
      \\
      &= \den{\delta f_k}\<(\gamma, \dgamma, \dvar x \mapsto x_1, \mvar\dx \mapsto (\den{h_k} \<(\gamma, \dgamma, \dvar x \mapsto x_1)))
    \end{align*}

    \noindent
    And therefore it suffices to show that $\den{h_k} \<(\gamma, \dgamma, \dvar x \mapsto x_1) = dx$. Without loss of generality, assume $k = 1$:

    \begin{align*}
      &\phantom{{}={}}
      \den{h_1} \<(\gamma, \dgamma, \dvar x \mapsto x_1)
      \\
      &= \den{\emcase{\delta e}
      \inj 1 \mvar\dx \caseto \mvar\dx;\;
      \inj 2 \pwild \caseto \dummy\<\dvar x}
      \<(\gamma, \dgamma, \dvar x \mapsto x_1)
      \\
      &= \krof{\den{\mvar\dx}, \den{\dummy\<\dvar x}}
      \<(\morph{dist}^x_+
      ((\gamma, \dgamma, \dvar x \mapsto x_1),\,
      \den{\delta e}\<(\gamma, \dgamma, \dvar x \mapsto x_1)))
      \\
      &= \krof{\den{\mvar\dx}, \den{\dummy\<\dvar x}}
      \<((\gamma, \dgamma, \dvar x \mapsto x_1),\, (\inj 1 dx))
      \\
      &= \krof{\den{\mvar\dx}, \den{\dummy\<\dvar x}}
      \<(\inj 1 (\gamma, \dgamma, \dvar x \mapsto x_1, \mvar\dx \mapsto dx))
      \\
      &= \den{\mvar\dx}
      \<(\gamma, \dgamma, \dvar x \mapsto x_1, \mvar\dx \mapsto dx)
      \\
      &= dx
    \end{align*}

    Which is what we wished to show.

    %% CASE: BOX INTRODUCTION
  \item[Case $\infer{
      \J e {\stripcx{\Gamma}} A
    }{
      \J{\ebox{e}}{\Gamma}{\iso A}
    }$,\, 
    $\phi \ebox e = \ebox{\etuple{\phi e, \delta e}}$,\, 
    $\delta \ebox e = \etuple{}$.
  ]~\\

    \noindent
    For brevity, let
    \begin{align*}
       \gamma_s &= \morphstrip_{\Phi\Gamma}(\gamma) &
       \gamma_s' &= \morphstrip_\Gamma(\gamma') &
       \rho_s &= \morphstrip_{\Phi\Gamma}(\rho) &
       \rho_s' &= \morphstrip_\Gamma(\rho')
    \end{align*}

    By applying \cref{lemma-context-stripping} to our assumption, we have
%
    \begin{equation}\label{equation-strip-valid}
    \weirdat{\Gamma}{()}{\gamma_s}{\rho_s}{\gamma_s'}{\rho_s'}
    \end{equation}

    \noindent
    By applying \cref{lemma-discrete-contexts} we further know

    \begin{equation}\label{equation-strip-equal}
      \gamma_s = \gamma_s' \quad\text{and}\quad \rho_s = \rho_s'
    \end{equation}

    \noindent
    We wish to show:
%
    \[
    \weirdat{\iso A}
            {\den{\etuple{}} \<(\gamma, \dgamma)}
            {\den{\ebox{\etuple{\phi e, \delta e}}} \<\gamma}
            {\den{\ebox e} \<\rho}
            {\den{\ebox{\etuple{\phi e, \delta e}}} \<\gamma'}
            {\den{\ebox e} \<\rho'}
    \]

    Applying \cref{lemma-applying-box} and further simplifying, this is equivalent to:
%
    \begin{equation*}
    \weirdat{\iso A}
            {()}
            {(\den{\phi e} \<\gamma_s,\, \den{\delta e} \<\gamma_s)}
            {\den{e} \<\rho_s}
            {(\den{\phi e} \<\gamma_s,\, \den{\delta e} \<\gamma_s)}
            {\den{e} \<\rho_s}
    \end{equation*}

    \noindent
    Applying the definition of the logical relation at $\iso A$, this requires that $\den{e}\<\rho_s = \den{e}\<\rho_s'$ and $\den{\phi e}\<\gamma_s = \den{\phi e}\<\gamma_s'$ and $\den{\delta e}\<\gamma_s = \den{\delta e}\<\gamma_s'$, which hold by \cref{equation-strip-equal}, and that:
%
    \[
    \weirdat{A}
            {\den{\delta e}\<\gamma_s}
            {\den{\phi e}\<\gamma_s}
            {\den{e}\<\rho_s}
            {\den{\phi e}\<\gamma_s'}
            {\den{e}\<\rho_s'}
    \]
%
    which follows from our inductive hypothesis applied to \cref{equation-strip-valid}.

    %% CASE LET BOX
  \item[Case $\infer{
      \J e \Gamma {\iso A}\\
      \J f {\Gamma, \hd x A} B
    }{
      \J{\eletbox x e f} \Gamma B
    }$.]
    Observe that
    \begin{align*}
    \phi(\eletbox x e f) &= \elet{\pboxtuple{\dvar x, \dvar\dx} = \phi e} \phi f\\
    \delta(\eletbox x e f) &= \elet{\pboxtuple{\dvar x, \dvar\dx} = \phi e} \delta f
    \end{align*}

    Further observe that

    \begin{equation*}
      \den{\elet{\pboxtuple{\dvar x, \dvar\dx} = \phi e} \phi f} \<\gamma
      =
      \den{\phi f} \<(\gamma,\, x,\, dx)
    \end{equation*}

    \noindent
    and likewise for $\delta f$ in place of $\phi f$ and/or $\gamma'$ in place of $\gamma$.
%
    For brevity, let
    \begin{align*}
      x, dx &= \den{\phi e}\<\gamma &
      x', dx' &= \den{\phi e}\<\gamma'\\
      a &= \den{e}\<\rho &
      a' &= \den{e}\<\rho'
    \end{align*}

    Using this observation and these abbreviations, we have
%
    \begin{align*}
      \den{\eletbox x e f} \<\rho &= \den{f}\<(\rho, a)
      \\
      \den{\elet{\pboxtuple{\dvar x, \dvar\dx} = \phi e} \phi f} \<\gamma
      &= \den{\phi f}\<(\gamma, x, dx)
      \\
      \den{\elet{\pboxtuple{\dvar x, \dvar\dx} = \phi e} \delta f} \<\gamma
      &= \den{\delta f}\<(\gamma, x, dx)
    \end{align*}
%
    and likewise for $\gamma', \rho'$. Then what we wish to show is that
%
    \[
    \weirdat B
             {\den{\delta f}\<(\gamma, x, dx)}
             {\den{\phi f}\<(\gamma, x, dx)}
             {\den{f}\<(\rho, a)}
             {\den{\phi f}\<(\gamma', x', dx')}
             {\den{f}\<(\rho', a')}
    \]

    \noindent
    By our inductive hypothesis for $f$, it suffices to show that
%
    \[\weirdat{\Gamma, \hd x A}
              {\dgamma}
              {(\gamma, x, dx)}
              {(\rho, a)}
              {(\gamma', x', dx')}
              {(\rho', a')}
    \]

    \noindent
    Since our assumption tells us that
%
    \(
    \weirdat{\Gamma}\dgamma\gamma\rho{\gamma'}{\rho'}
    \),
%
    it suffices to show that
%
    \(
    \weirdat{\iso A}{()}{(x, dx)}{a}{(x', dx')}{a'}
    \),
%
    which follows directly from our inductive hypothesis for $e$.

    %% CASE: FOR LOOPS
  \item[Case $\infer{
      \J e \Gamma {\tseteq A}\\
      \J f {\Gamma, \hd x {\eqt A}} {L}
    }{
      \J{\eforvar x e f} \Gamma {L}
    }$.] Recall that:
%
    \begin{align*}
      \phi(\eforvar x e f) &=
      \eforvar x {\phi e}{\eletbox{\dx}{\ebox{\zero\<\dvar x}} \phi f}
      \\
      \delta(\eforvar x e f)
      &= (\eforvar x {\delta e}
      \eletbox \dx {\ebox{\zero\<\dvar x}} \phi f) \\
      &\vee (\eforvar x {\phi e \vee \delta e}
      \eletbox{\dx}{\ebox{\zero\<\dvar x}} \delta f)
    \end{align*}

    This case of the proof is quite complex; it will help to have a few abbreviations. First, we will be considering various definitons which differ only in whether they use the primed or un-primed versions of $\gamma,\rho$, so it will help to refer to these by subscript: $\gamma_1 = \gamma$ and $\gamma_2 = \gamma' $ and $\rho_1 = \rho$ and $\rho_2 = \rho'$.

    With this in mind, apply \cref{lemma-semilattice-changes} to our inductive hypothesis for $e$:
%
    \begin{align}
      &\weirdat{\tseteq A}
            {\den{\delta e}\<(\gamma_1,\dgamma)}
            {\den{\phi e}\<\gamma_1}
            {\den{e}\<\rho_1}
            {\den{\phi e}\<\gamma_2}
            {\den{e}\<\rho_2}
      \\\label{for-ih1}
      \iff&
      \den{\phi e}\<\gamma_1 = \den{e}\<\rho_1
      ~\text{and}~
      \den{\phi e}\<\gamma_2 = \den{e}\<\rho_2 = \den{\phi e}\<\gamma_1 \vee \den{\delta e}\<(\gamma_1,\dgamma)
    \end{align}

    Moving to our inductive hypothesis for $f$, for any $x \in \den{\eqt A}$, let:
%
    \begin{align*}
      \rho_i^x &= (\rho_i, \dvar x \mapsto x)
      &
      \gamma_i^x &= (\gamma_i, \dvar x \mapsto x, \dvar\dx \mapsto \dummy\<x)
      &
      \dgamma^x &= (\gamma_1^x, \dgamma)
    \end{align*}
%
    By our assumption and \cref{lemma-equality-dummy} we have
%
    $\weirdat{\Gamma, \hd x {\eqt A}}{\dgamma^x}{\gamma_1^x}{\rho_1^x}{\gamma_2^x}{\rho_2^x}$. From this and our inductive hypothesis for $f$, applying \cref{lemma-semilattice-changes}:
%
    \begin{align}
      &\weirdat{L}
      {\den{\delta f}\<\dgamma^x}
      {\den{\phi f}\<\gamma_1^x}
      {\den{f}\<\rho_1^x}
      {\den{\phi f}\<\gamma_2^x}
      {\den{f}\<\rho_2^x}
      \\\label{for-ih2}
      \iff&
      \den{\phi f}\<\gamma_1^x = \den{f}\<\rho_1^x
      ~\text{and}~
      \den{\phi f}\<\gamma_2^x = \den{f}\<\rho_2^x
      = \den{\phi f}\<\gamma_1^x \vee \den{\delta f}\<\dgamma^x
    \end{align}
%

    We summarize \cref{for-ih1,for-ih2} as follows, introducing variables $s_i, ds, F_i, dF$:
%
    \begin{align}
      \label{for-sF}
      s_i &= \den{e}\<\rho_i = \den{\phi e}\<\gamma_i
      &
      F_i(x) &= \den{f}\<\rho_i^x = \den{\phi f}\<\gamma_i^x
      \\\label{for-dsF}
      ds &= \den{\delta e}\<(\gamma_1,\dgamma)
      &
      dF(x) &= \den{\delta f}\<\dgamma^x
      \\\label{for-crux}
      s_1 \cup ds &= s_2 & F_1(x) \vee dF(x) &= F_2(x)
    \end{align}

    \noindent
    Now let's give abbreviations to the denotations about which we are trying to prove something:
%
    \begin{align*}
      l_i &= \den{\eforvar x e f}\<\rho_i
      \\
      m_i &= \den{\phi(\eforvar x e f)}\<\gamma_i
      \\
      dm &= \den{\delta(\eforvar x e f)}\<(\gamma_1, \dgamma)
    \end{align*}
%
    Then what we wish to show is that
    \(\weirdat{L}{dm}{m_1}{l_1}{m_2}{l_2}\), or equivalently by applying \cref{lemma-semilattice-changes}, that
%
    $m_1 = l_1$ and $m_2 = l_2 = m_1 \vee dm$.
%
    We will do this by showing that
%
    \begin{align}
      \label{for-wts1}
      l_i = m_i &= \bigvee_{x \in s_i} F_i(x)
      \\\label{for-wts2}
      dm &= \Big(\bigvee_{x \in ds} F_1(x)\Big)
      \vee \Big(\bigvee_{x \in s_1 \cup ds} dF(x)\Big)
    \end{align}
%
    from which our result follows because:

    \begin{align*}
      m_2 &= \bigvee_{x \in s_2} F_2(x)\\
      &= \bigvee_{x \in s_1 \cup ds} (F_1(x) \vee dF(x))
      && \text{\cref{for-crux}}\\
      &= \Big(\bigvee_{x \in s_1} F_1(x)\Big)
      \vee \Big(\bigvee_{x \in ds} F_1(x)\Big)
      \vee \Big(\bigvee_{x \in s_1 \cup ds} dF(x)\Big)
      && \text{reassociate}\\
      &= m_1 \vee dm
    \end{align*}

    \noindent
    So it suffices to show \cref{for-wts1,for-wts2}. This is mostly a matter of pushing through denotations. For instance, starting with $l_i$:
%
    \begin{align*}
      l_i &= \den{\eforvar x e f}\<\rho_i\\
      &= \morph{collect}(\den{f})(\rho_i, \den{e}\<\rho_i)
      && \text{definition of }\den{\kw{for} ...}
      \\
      &= \bigvee_{x \in \den{e}\<\rho_i} \den{f}\<(\rho_i, \dvar x \mapsto x)
      && \text{definition of \morph{collect}}
      \\
      &= \bigvee_{x \in s_i} F_i(x)
      && \text{\cref{for-sF}}
    \end{align*}

    And $m_i$:
%
    \begin{align*}
      m_i &=
      \den{\eforvar x {\phi e} \eletbox\dx{\ebox{\zero\<\dvar x}} \phi f}
      \<\gamma_i\\
      &= \bigvee_{x \in \den{\phi e}\<\gamma_i}
      \den{\eletbox\dx{\ebox{\zero\<\dvar x}} \phi f}
      \<(\gamma_i, \dvar x \mapsto x)
      \\
      &= \bigvee_{x \in s_i}
      \den{\phi f}\<(\gamma_i, \dvar x \mapsto x, \dvar\dx \mapsto
      \den{\ebox{\zero\<\dvar x}}\<(\gamma_i, \dvar x \mapsto x))
      && \text{\cref{for-sF}}
      \\
      &= \bigvee_{x \in s_i}
      \den{\phi f}\<(\gamma_i, \dvar x \mapsto x, \dvar\dx \mapsto
      \den{\zero\<\dvar x}\<(\morphstrip \<(\gamma_i, \dvar x \mapsto x)))
      && \text{\cref{lemma-applying-box}}
      \\
      &= \bigvee_{x \in s_i}
      \den{\phi f}\<(\gamma_i, \dvar x \mapsto x, \dvar\dx \mapsto
      \dummy\<x)
      && \text{pushing definitions}
      \\
      &= \bigvee_{x \in s_i} F_i(x)
    \end{align*}

    And finally, $dm$. By expanding in the same manner as directly above we have that:
%
    \begin{align*}
      \den{\eforvar x {\delta e} \eletbox\dx{\ebox{\zero\<\dvar x}} \phi f}
      \<(\gamma_1, \dgamma)
      &= \bigvee_{x \in ds} F_1(x)
      \\
      \den{\eforvar x {\phi e \vee \delta e}
        \eletbox\dx{\ebox{\zero\<\dvar x}} \delta f}
      \<(\gamma_1, \dgamma)
      &= \bigvee_{x \in s_1 \cup ds} dF(x)
    \end{align*}

    And therefore:
    \begin{align*}
      dm &= \den{\delta(\eforvar x e f)}\<(\gamma_1, \dgamma)\\
      &=
      \den{\eforvar x {\delta e} \eletbox\dx{\ebox{\zero\<\dvar x}} \phi f}
      \<(\gamma_1, \dgamma)
      \\
      &\vee \den{\eforvar x {\phi e \vee \delta e}
        \eletbox\dx{\ebox{\zero\<\dvar x}} \delta f}
      \<(\gamma_1, \dgamma)
      \\
      &= \Big(\bigvee_{x \in ds} F_1(x)\Big) \vee
      \Big(\bigvee_{x \in s_1 \cup ds} dF(x)\Big)
    \end{align*}
%
    Which is what we wished to show.

    %% CASE: FIX
  \item[Case $\infer{
      \J e \Gamma \isofixLtoL
    }{
      \J{\efix e}\Gamma {\fixt L}
    }$.]
    Recall that
    %
    \begin{align*}
      \phi(\efix e) &= \semifix\<\phi e
      & \Phi\fixt{L} &= \fixt{L} \quad\textsf{(\cref{lemma-phi-eqt})}
      \\
      \delta(\efix e) &= \bot
      & \Delta\Phi\fixt{L} &= \fixt{L} \quad\textsf{(\cref{lemma-delta-lattice,lemma-phi-eqt})}
    \end{align*}

    \noindent
    For brevity, let
    \begin{align*}
      f &= \den{e}\<\rho & f' &= \den{e}\<\rho'\\
      (g_1, g_2) &= \den{\phi e}\<\gamma
      & (g_1', g_2') &= \den{\phi e}\<\gamma'
    \end{align*}

    What we wish to show is:
%
    \begin{align*}
    &\weirdat{\fixt L}
             {\den{\bot}\<(\gamma,\dgamma)}
             {\den{\semifix\<\phi e}\<\gamma}
             {\den{\efix e}\<\rho}
             {\den{\semifix\<\phi e}\<\gamma'}
             {\den{\efix e}\<\rho'}\\
    \iff&
    \weirdat{\fixt L}
            {\bot}
            {\semifix\<(\den{\phi e}\<\gamma)}
            {\morph{fix}\<(\den{e}\<\rho)}
            {\semifix\<(\den{\phi e}\<\gamma')}
            {\morph{fix}\<(\den{e}\<\rho')}\\
    \iff&
    \weirdat{\fixt L}
            {\bot}
            {\semifix \<(g_1, g_2)}
            {\morph{fix} \<f}
            {\semifix \<(g_1', g_2')}
            {\morph{fix} \<f'}\\
    \iff&
    \semifix\<(g_1, g_2) = \morph{fix}\<f
    = \semifix\<(g_1', g_2') = \morph{fix}\<f'
    \qquad \textsf{(applying \cref{lemma-semilattice-changes})}
    \end{align*}

    By our inductive hypothesis we have
%
    \(
    \weirdat{\isofixLtoL}
               {\den{\delta e}\<(\gamma,\dgamma)}
               {(g_1, g_2)}
               {f}
               {(g_1', g_2')}
               {f'}
    \)
%
    and expanding gives us:
    \begin{gather}
      \label{fix-eqn}
      f = f' ~\text{and}~ g_1 = g_1' ~\text{and}~ g_2 = g_2'
      \\
      \label{fix-gf}
      \weirdat{\fixt L \to \fixt L}{g_2}{g_1}{f}{g_1'}{f'}
    \end{gather}

    Applying \cref{fix-eqn} reduces our goal to showing $\semifix\<(g_1, g_2)
    = \morph{fix}\<f$, which holds by \cref{lemma-semifix} applied to \cref{fix-gf,fix-eqn}.

    %% CASE: SPLIT
  \item[Case $\infer{\J e \G {\iso{(A_1 + A_2)}}}{
      \J{\esplit e} \G {\iso A_1 + \iso A_2}}$.]
%
     Recall that
    %
    \begin{align*}
      \phi(\esplit e)
      &=
      \eletbox{\color{fresh} \dvar z}{\phi e}
      \\
      &\phantom{{}={}}
      \emcase{\esplit{\ebox{\pi_1\<\color{fresh} z}}}
      \\
      &\phantom{{}={}\quad}
      (\inj i {\color{fresh}\mvar y} \caseto \eletbox{x}{\color{fresh}\mvar y}
      \\
      &\phantom{{}={}\quad (\inj i \mvar y \caseto {}}
      \emcase{\esplit \ebox{\pi_2\<\color{fresh}\dvar z}}
      \\
      &\phantom{{}={}\quad (\inj i \mvar y \caseto {} \quad}
      \inj i {\color{fresh}\mvar\dy}
      \caseto
      \eletbox{\dx}{\color{fresh}\mvar\dy}
      \inj i \eboxraw{\etuple{\dvar x, \dvar\dx}}
      \\
      &\phantom{{}={}\quad (\inj i \mvar y \caseto {} \quad}
      \inj{i+1 \bmod 2} \pwild \caseto
      \inj i \ebox{\etuple{\dvar x, \dummy\<\dvar x}}
      )_i
      \\
      \delta(\esplit e)
      &= \eletbox{\color{fresh} y}{\phi e}
      \\
      &\phantom{{}={}}
      \emcase{\pi_1\<{\color{fresh}\dvar y}}
      (\inj i \pwild \caseto \inj i \ptuple{})_{i \in \{1,2\}}
    \end{align*}

    \noindent
    By unpacking our inductive hypothesis there must be some $k \in \{1,2\}$ and some $dx, x, a$ such that $\weirdat{A_k}{dx} x a x a$ and:

    \begin{align*}
      \den{e} \<\rho = \den{e} \<\rho' &= \inj k a
      &
      \den{\phi e}\<\gamma = \den{\phi e}\<\gamma' &= (\inj k x,\, \inj k dx)
      &
      \den{\delta e}\<(\gamma,\dgamma) = ()
    \end{align*}

    \noindent
    Applying the definition of what we wish to show, it therefore suffices to show that:

    \begin{align}
      \label{split-a}
      \den{\esplit e}\<\rho = \den{\esplit e}\<\rho' &= \inj k a\\
      \label{split-b}
      \den{\phi(\esplit e)}\<\gamma = \den{\phi(\esplit e)}\<\gamma'
      &= \inj k (x,\, dx)\\
      \label{split-c}
      \den{\delta(\esplit e)}\<(\gamma,\dgamma) &= \inj k ()
    \end{align}

    \noindent
    Showing this is a matter of calculation. \Cref{split-a} is immediate upon
    noting that $\isosum = \morph{id}$ and therefore $\den{\esplit e}\<\rho =
    \den{e}\<\rho$. The other two are no more difficult but considerably more
    tedious, so we omit the details of the calculations. \todolater{provide
      calculations for at least one; do the delta one, it's simpler}
    
  \end{description}

\end{proof}

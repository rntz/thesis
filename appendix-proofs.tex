\chapter{Proofs}

We state these lemmas and theorems in dependency order, so that nothing is used
before it has been proven. This is not always the order in which they are stated
in the text.

\section{Semi\naive\ evaluation}

\emph{Theorems and lemmas from \cref{chapter-seminaive}.}

\PhiDeltaWellTyped*
\begin{proof}\label{proof-phi-delta-well-typed}
  \todo{prove the phi and delta transforms are well typed}
\end{proof}

\EqualityChanges*
\begin{proof}
  \label{proof-equality-changes}
  By induction on $\eqt A$, applying the definition from
  \cref{figure-seminaive-logical-relation}:

  \begin{description}
    \item[Case $\tunit$.] Trivial.

    \item[Case $\eqt A \x \eqt B$.] Then our assumption is equivalent to
%
      \[\weirdat{\eqt A \x \eqt B}{(\dx_1,\dx_2)}{(x_1,x_2)}{(a_1,a_2)}{(y_1,y_2)}{(b_1,b_2)}\]
%
      and by unfolding this we have
      \(\weirdat{\eqt{A}}{\dx_1}{x_1}{a_1}{y_1}{b_1}\) and
      \(\weirdat{\eqt{B}}{\dx_2}{x_2}{a_2}{y_2}{b_2}\), which by our induction
      hypotheses show \(x_1 = a_1\), \(y_1 = b_1\) and \(x_2 = a_2\), \(y_2 = b_2\),
      which suffices.

    \item[Case $\eqt A_1 + \eqt A_2$.] Then for some $i \in \{1,2\}$ our
      assumption is equivalent to
%
      \[
      \weirdat{\eqt A_1 + \eqt A_2}{\inj i \dx}{\inj i x}{\inj i a}
              {\inj i y}{\inj i b}
      \]
%
      and by unfolding this we have \(\weirdat{\eqt{A}_i}{\dx}{x}{a}{y}{b}\),
      which by our induction hypothesis implies \(x=a\) and \(y=b\), which
      suffices.

    \item[Case $\tseteq{A}$.] Then our assumption unfolds to \((x,y,x \cup \dx)
      = (a,b,y)\), which suffices.
      
  \end{description}
\end{proof}

\EqualityDummy*
\begin{proof}
  \label{proof-equality-dummy}
  \todo{prove}
\end{proof}

\DiscreteContexts*
\begin{proof}
  \label{proof-discrete-contexts}
  \todo{prove}
\end{proof}

\ContextStripping*
\begin{proof}
  \label{proof-context-stripping}
  \todo{prove}
\end{proof}

\SeminaiveFundamental*
\begin{proof}
  \label{proof-seminaive-fundamental}
  \todo{prove fundamental property for phi and delta}
\end{proof}

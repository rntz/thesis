\chapter{Proofs}

\newcommand\nextlemma{\par\pagebreak[3]\vspace{1\baselineskip}}

We state these lemmas and theorems in dependency order, so that nothing is used
before it has been proven. This is not always the order in which they are stated
in the text. \todo{But maybe it should be?}


\section{Datafun}

\begin{lemma}\label{lemma-weakening-variables}
  If $\hm x A \in \Gamma$ or $\hd x A \in \Gamma$ and $\Gamma \weaker \Delta$
  then $\hm x A \in \Delta$ or $\hd x A \in \Delta$.
\end{lemma}
\begin{proof}
  Recall that although we write them differently we regard $\mvar x$ and $\dvar
  x$ as the same variable.
%
  Let $H$ stand for the hypothesis in our assumption, either $\hm x A$ or $\hd x
  A$ respectively. Then by induction on the derivation of $\Gamma
  \weaker \Delta$:

  \begin{description}[itemsep=1ex]
  \item[Case\, $\emptycx \weaker \emptycx$.] By contradiction, since $H \in
    \emptycx$ is impossible.

  \item[Case\, $\infer{\Gamma' \weaker \Delta'}{\Gamma',H' \weaker
      \Delta',H'}$.] If $H = H'$ we are done. Otherwise, $H \in \Gamma'$, so
    apply the IH.

  \item[Case\, $\infer{\Gamma \weaker \Delta'}{\Gamma \weaker \Delta',H}$.]
    Apply the IH.

  \item[Case\, $\infer{\Gamma' \weaker \Delta'}{\Gamma',\,\hm x A \weaker
      \Delta',\hd x A}$.] If $H = \hm x A$ we are done. Otherwise, $H \in
    \Gamma'$, so apply the IH.
  \end{description}
\end{proof}


\nextlemma
\begin{lemma}\label{lemma-weakening-strip-context}
  If $\Gamma \weaker \Delta$ then $\stripcx{\Gamma} \weaker \stripcx{\Delta}$.
\end{lemma}
\begin{proof}
  By induction on $\Gamma \weaker \Delta$:

  \begin{description}
    \item[Case $\emptycx \weaker \emptycx$.] Immediate.

  \item[Case\, $\infer{\Gamma' \weaker \Delta'}{\Gamma',H \weaker \Delta',H}$.]\

    Either $H$ is discrete $\hd x A$, in which case $\stripcx{\Gamma',H} =
    \stripcx{\Gamma'}, H \weaker \stripcx{\Delta'}, H = \stripcx{\Delta',H}$ by
    \rn{cons} and our inductive hypothesis; or $H$ is monotone $\hm x A$, in
    which case $\stripcx{\Gamma',H} = \stripcx{\Gamma'} \weaker \stripcx{\Delta'} = \stripcx{\Delta',H}$ by our inductive hypothesis alone.

  \item[Case\, $\infer{\Gamma \weaker \Delta'}{\Gamma \weaker \Delta',H}$.]\

    Then $\stripcx{\Gamma} \weaker \stripcx{\Delta'}$ by our inductive
    hypothesis. Depending upon whether $H$ is monotone or discrete, we have
    either $\stripcx{\Delta', H} = \stripcx{\Delta'}$ (in which case our
    inductive hypothesis suffices) or $\stripcx{\Delta', H} = \stripcx{\Delta'},
    H$, in which case by \rn{cons} \todo{and transitivity} $\stripcx{\Gamma}
    \weaker \stripcx{\Delta'}, H$.

  \item[Case\, $\infer{\Gamma' \weaker \Delta'}{\Gamma',\,\hm x A \weaker
      \Delta',\hd x A}$.]\

    Then $\stripcx{\Gamma', \hm x A} = \stripcx{\Gamma'}$ while
    $\stripcx{\Delta', \hd x A} = \stripcx{\Delta'}, \hd x A$ \XXX
  \end{description}
\end{proof}


\nextlemma
\Weakening*

\begin{proof}\label{proof-weakening}
  By induction on the derivation of $\J{e}{\Gamma} A$.

  \begin{description}[topsep=1em,itemsep=1em]
    \item[Cases\quad $\infer{\hm x A \in \G}{\J {\mvar x} \G A}$%
      \quad
      $\infer{\hd x A \in \G}{\J {\dvar x} \G A}$.]\
%
      By \cref{lemma-weakening-variables}.

    \item[Cases\quad $\infer{\quad}{\J {\etuple{}} \G \tunit}$%
      \quad $\infer{\quad}{\J\bot\G {\eqt L}}$.]\
%
      Trivial.

    \item[Cases where the premises have the same context as the conclusion, namely:]

      \begin{mathpar}
        \infer{\J e \G {A \to B} \\ \J f \G A}{\J {e\<f} \G B}

        \infer{(\J{e_i}\G{A_i})_i}{\J{\etuple{e_1,e_2}} \G {A_1 \x A_2}}

        \infer{\J e \G {A_1 \x A_2}}{\J{\pi_i\<e}\G{A_i}}

        \infer{(\J{e_i} \G {\eqt L})_i}{\J{e_1 \vee e_2}\G {\eqt L}}

        \infer{\J e \G {\iso{(A + B)}}}{\J{\esplit e} \G {\iso A + \iso B}}
      \end{mathpar}

      Apply the same typing rule to our inductive hypotheses.

    \item[Cases where the premises add hypotheses to the context, namely:]

      \begin{mathpar}
        \infer{\J e {\G,\,\hm x A} B}{\J {\efn x e} \G {A \to B}}

        \infer{\J e \G {\iso A} \\ \J f {\G,\,\hd x A} B}{\J {\eletbox x e f} \G B}

        \infer{\J e \G {A_1 + A_2} \\
          (\J {f_i} {\G,\, \hm{x_i}{A_i}} {B})_i
        }{
          \J {\emcase{e} (\inj i {\mvar x_i} \caseto f_i)_i} \G B
        }

        \infer{
          \J e \G {\tset A} \\
          \J f {\G,\, \hd x A} {\eqt L}
        }{\J {\eforvar x e f} \G {\eqt L}}
      \end{mathpar}

      Apply the inductive hypotheses, using \rn{cons} when necessary to show
      that the modified contexts also satisfy our precondition, for example,
      $\Delta,\hm{x}{A} \stronger \Gamma,\hm{x}{A}$.

    \item[Case where the premises strip the context, namely:]

      \begin{mathpar}
        \infer{\J {e} {\stripcx\G} A}{\J{\ebox e} \G {\iso A}}

        \infer{(\J {e_i} {\stripcx\G} {\eqt A})_i}
              {\J {\esetsub{e_i}{i}} \G {\tset{\eqt A}}}

        \infer{(\J {e_i} {\stripcx\G} {\eqt A})_i}
              {\J {\eeq{e_1}{e_2}} \G \tbool}

              \infer{\J {e} {\stripcx\G} {\tset\tunit}}
                    {\J {\eisempty e} \G {\tunit + \tunit}}
      \end{mathpar}

      Then $\stripcx{\Delta} \stronger \stripcx{\Gamma}$ by
      \cref{lemma-weakening-strip-context}, so we apply the inductive
      hypotheses.

    \item[Case\quad $\infer{\J e {\stripcx{\G},\, \hm x {\fixt L}} {\fixt L}}{%
      \J{\efixis x e} \G {\fixt L}}$.]\

      Then by combining \cref{lemma-weakening-strip-context} and \rn{cons} we
      have $\stripcx{\Delta}, \hm x {\fixt L} \stronger \stripcx{\Gamma}, \hm x
      {\fixt L}$ and we apply our inductive hypothesis.

  \end{description}
\end{proof}

\section{Semi\naive\ evaluation}

\emph{Theorems and lemmas from \cref{chapter-seminaive}.}


\DeltaLattice*
\DeltaLatticeProof*


\nextlemma
\PhiEqualityType*
\PhiEqualityTypeProof*


\nextlemma
\PhiDeltaWellTyped*

\begin{proof}\label{proof-phi-delta-well-typed}
  By induction on the typing derivation $\J e \Gamma A$.

  \begin{description}[topsep=1em,itemsep=1em]
    \item[Case\quad $\infer{\hm x A \in \G}{\J {\mvar x} \G A}$,\quad $\phi\mvar{x} =
      \mvar{x}$,\quad $\delta\mvar{x} = \mvar\dx$.]\

      Then $\phi \mvar x = \mvar{x}$ and $\delta\mvar x = \mvar\dx$, so we
      need to show that $\J{\mvar{x}}{\Phi\Gamma}{\Phi A}$ and
      $\J{\mvar\dx}{\iso\Phi\Gamma, \Delta\Phi\Gamma}{\Delta\Phi A}$. We know
      that $\Phi$ and $\Delta$ distribute over contexts, so since $\hm x A \in
      \Gamma$ we know that $\Phi(\hm x A) = \hm x {\Phi A} \in \Phi\Gamma$ and
      $\Delta\Phi(\hm x A) = \hm{\dx}{\Delta\Phi A} \in \Delta\Phi\Gamma$, which
      suffices by \rn{var}.

    \item[Case\quad $\infer{\hd x A \in \G}{\J {\dvar x} \G A}$,\quad
      $\phi\dvar{x} = \dvar{x}$,\quad $\delta\dvar{x} = \dvar\dx$.]\

      Then $\phi\dvar{x} = \dvar{x}$ and $\delta\dvar{x} = \dvar\dx$. By the
      same argument as the previous case, it suffices to show that $\hd{x}{\Phi
        A} \in \Phi\Gamma$ and $\hd{\dx}{\Delta\Phi A} \in \Delta\Phi\Gamma$,
      which is true by distributing $\Phi$ and $\Delta$.

    \item[Case\quad $\infer{\J e {\G,\,\hm x A} B}{\J {\efn x e} \G {A \to B}}$,\quad
      $\phi(\efn x e) = \efn x \phi e$,\quad
      $\delta(\efn x e) = \fnof{\pboxvar x}\efn\dx \delta e$.]\

      Then by our inductive hypothesis,
      \[
      \infer{
        \J{\phi e}{\Phi\Gamma,\, \hm x {\Phi A}}{\Phi B}
      }{\J{\efn x \phi e}{\Phi\Gamma}{\Phi A \to \Phi B}}
      \]
      which handles the case for $\phi$.

      The case for $\delta$ is complicated by the use of syntax sugar,
      $(\fnof{\pboxvar x} ...) \desugars (\efn y \eletbox x {\mvar y} ...)$. For
      simplicity's sake we first derive a typing rule for this sugared syntax:

      \[
      \infer{
        \J{f}{\Gamma,\, \hd x A}{B}
      }{
        \J{\fnof{\pboxvar x} f}{\Gamma}{\iso A \to B}
      }
      \]

      \noindent
      which is justified by the following expansion and
      weakening~(\cref{theorem-weakening}):

      \[
      \infer*{
        \infer*{
          \infer*{
            \infer*{~}{
              {\hm y {\iso A}} \in {\Gamma, \hm y {\iso A}}
          }}{\J{\mvar{y}}{\Gamma, \hm y {\iso A}}{\iso A}}
          \\
          \infer*[right={weakening}]{
            \J{f}{\Gamma,\, \hd x A}{B}
          }{\J{f}{\Gamma, \hm y {\iso A}, \hd x A}{B}}
        }{
          \J{\eletbox x {\mvar y} f}{\Gamma, \hm y {\iso A}}{B}
      }}{
        \J{\efn y \eletbox x {\mvar y} f}{\Gamma}{\iso A \to B}
      }
      \]

      \noindent
      Putting this syntactic sugar to work, we have:

      \[
      \infer*[right=sugar]{
        \infer*{
          \J{\delta e}
            {\iso\Phi\Gamma,\, \Delta\Phi\Gamma,\, \hd x {\iso A},\,
              \hm{\dx}{\Delta\Phi A}}
            {\Delta\Phi B}
        }{
          \J{\efn\dx \delta e}
            {\iso\Phi\Gamma,\, \Delta\Phi\Gamma,\, \hd x {\iso\Phi A}}
            {\Delta\Phi A  \to \Delta\Phi B}
        }
      }{
        \J{\fnof{\pboxvar x} \efn\dx \delta e}
          {\iso\Phi\Gamma,\, \Delta\Phi\Gamma}
          {\iso \Phi A \to \Delta\Phi A \to \Delta\Phi B}
      }
      \]

      and if we rearrange the typing context of the premise at the top, we see
      that it matches our inductive hypothesis:
%
      \begin{align*}
        &\phantom{{}={}}\iso\Phi\Gamma,\, \Delta\Phi\Gamma,\, \hd x {\iso A},\,
        \hm{\dx}{\Delta\Phi A}\\
        &=
        \iso\Phi\Gamma,\, \hd x {\iso A},\:
        \Delta\Phi\Gamma,\, \hm{\dx}{\Delta\Phi A}\\
        &=
        \iso\Phi(\Gamma,\, \hm x A),\, \Delta\Phi(\Gamma,\, \hm x A)
      \end{align*}
%
      which finishes the case for $\delta$.

    \item[Case\quad $\infer{\J e \G {A \to B} \\ \J f \G A}{\J {e\<f} \G B}$,\quad
      $\phi(e\<f) = \phi e\<\phi f$,\quad
      $\delta(e\<f) = \delta e \<\ebox{\phi f} \<\delta f$.]\

      The $\phi$ case fairly straightforward: by our inductive hypotheses,

      \[
      \infer*{
        \J{\phi e}{\Phi\Gamma}{\Phi A \to \Phi B}\\
        \J{\phi f}{\Phi\Gamma}{\Phi A}
      }{
        \J{\phi e\<\phi f}{\Phi\Gamma}{\Phi B}
      }
      \]

      The $\delta$ case is not more complex, but the types get larger, so it
      does not fit easily in one derivation. Applying \rn{app} it suffices to
      show that, in the context ${\iso\Phi\Gamma,\, \Delta\Phi\Gamma}$, we have
      $\delta e \isa {\iso\Phi A \to \Delta\Phi A \to \Delta\Phi B}$ and
      ${\ebox{\phi f}} \isa {\iso\Phi A}$ and $\delta f \isa {\Delta\Phi B}$.
      The first and last of these are supplied directly by our inductive
      hypotheses, while the last needs a use of weakening, observing that
      $
      \stripcx{\iso\Phi\Gamma, \Delta\Phi\Gamma}
      \stronger
      \stripcx{\iso\Phi\Gamma}
      =
      \iso\Phi\Gamma
      \stronger
      \Phi\Gamma
      $ (using \cref{lemma-weakening-strip-context}).

      \[
      \infer*{
        \infer*[right=weakening]{
          \J{\phi f}{\Phi\Gamma}{\Phi A}
        }{
          \J{\phi f}
            {\stripcx{\iso\Phi\Gamma,\, \Delta\Phi\Gamma}}
            {\Phi A}
      }}{
        \J{\ebox{\phi f}}{\iso\Phi\Gamma,\, \Delta\Phi\Gamma}{\iso\Phi A}
      }
      \]

    \item[Case\quad $\infer{\quad}{\J {\etuple{}} \G \tunit}$,\quad
      $\phi() = ()$,\quad $\delta() = ()$.]\

      Trivial.

    \item[Case\quad
      $\infer{(\J{e_i}\G{A_i})_i}{\J{\etuple{e_1,e_2}} \G {A_1 \x A_2}}$,\quad
      $\phi(e_1, e_2) = (\phi e_1, \phi e_2)$,\quad
      $\delta(e_1, e_2) = (\delta e_1, \delta e_2)$.]\

      Immediate.

    \item[Case\quad $\infer{\J e \G {A_1 \x A_2}}{\J{\pi_i\<e}\G{A_i}}$,\quad
      $\phi(\pi_i\<e) = \pi_i\<\phi e$,\quad
      $\delta(\pi_i\<e) = \pi_i\<\delta e$.]\

      Immediate.

    \item[Case\quad $\infer{\J e \G {A_1 + A_2} \\
        (\J {f_i} {\G,\, \hm{x_i}{A_i}} {B})_i
      }{
        \J {\emcase{e} (\inj i {\mvar x_i} \caseto f_i)_i} \G B
      }$.]\

      Recall that
      \begin{align*}
        \phi(\emcase e (\inj i \mvar x \caseto f_i)_i)
        &= \emcase{\phi e} (\inj i \mvar x \caseto \phi f_i)_i
        \\
        \delta({\emcase{e} (\inj i {\mvar x_i} \caseto f_i)_i})
        &=
        \emcase{\esplit{\ebox{\phi e}},\, \delta e}\\
        &\qquad ({\inj i {\pboxvar x},\, \inj i \mvar\dx} \caseto \delta f_i)_{i}\\
        &\qquad ({\inj i {\pboxvar x},\, \inj j \pwild}
        %\caseto \subst{\delta f_i}{\dx \substo \dummy\<\dvar x})_{i\ne j}
        \caseto \elet{\mvar\dx = \dummy\<\dvar x} \delta f_i)_{i\ne j}
      \end{align*}

      The $\phi$ case is straightforward, observing that $\Phi(A_1 + A_2) = \Phi
      A_1 + \Phi A_2$:
%
      \[
      \infer*{
        \infer*[right=ih]{~}{\J{\phi e}{\Phi\Gamma}{\Phi A_1 + \Phi A_2}}
        \and
        \Big(
        \infer*[right=ih]{~}
               {\J{\phi f_i}{\Phi\Gamma,\, \hm x \Phi A_i}{\Phi B}}
        \Big)_i
      }{
        \J{\emcase{\phi e} (\inj i \dvar x \caseto \phi f_i)_i}{\Phi\Gamma}{\Phi B}
      }
      \]

      The $\delta$ case is more interesting, as it uses syntax sugar. First we
      give a rule for this syntax sugar and establish its deriveability:

      \[
      \infer{
        \J{e}{\Omega}{\iso (A_1 + A_2) \x (B_1 + B_2)}\\
        (\J{f_{i,j}}{\Omega,\, \hd x A_i,\, \hm y B_j}{C})_{i,j}
      }{
        \J{\emcase{e} (\inj i \pboxvar{x},\, \inj j \mvar{y} \caseto f_{i,j})_{i,j}}
          {\Omega}
          {C}
      }
      \]

      This is derivable as follows:

      \[
      \infer*{
        \XXX
      }{
        \J{\emcase{e} (\inj i \pboxvar{x},\, \inj j \mvar{y} \caseto f_{i,j})_{i,j}}
          {\Omega}
          {C}
      }
      \]

      We can use this to type the $\delta$-translation as follows:

      \[
      \infer*{
        \J{(\esplit{\ebox{\phi e}},\, \delta e)}
          {\iso\Phi\Gamma,\, \Delta\Phi\Gamma}
          {?}
        \and
        ...
      }{
        \J{\emcase{\esplit{\ebox{\phi e}},\, \delta e} \text{...}}
          {\iso\Phi\Gamma,\, \Delta\Phi\Gamma}
          {\Delta\Phi B}
      }
      \]

      \todo{prove this case}

    \item[Case\quad $\infer{\J {e} {\stripcx\G} A}{\J{\ebox e} \G {\iso A}}$,\quad
      $\phi\ebox{e} = \ebox{(\phi e, \delta e)}$,\quad
      $\delta\ebox{e} = ()$.]\label{case-well-typed-box}\

      Observe that
      \begin{align*}
        \Phi\iso A &= \iso(\Phi A \x \Delta\Phi A)
        & \Delta\Phi\iso A &= \tunit
      \end{align*}

      Thus the $\delta$ case is trivial, and the $\phi$ case follows by
      weakening our inductive hypotheses:

      \[
      \infer*{
        \infer*{
          \infer*[right=ih]{~}{
            \J{\phi e}{\stripcx{\Phi\Gamma}}{\Phi A}
          }
          \\
          \infer*[right=ih]{~}{
            \J{\delta e}{\stripcx{\Phi\Gamma}}{\Delta\Phi A}
          }
        }{
          \J{(\phi e, \delta e)}{\stripcx{\Phi\Gamma}}{\Phi A \x \Delta\Phi A}
        }
      }{
        \J{\ebox{(\phi e, \delta e)}}{\Phi\Gamma}{\iso(\Phi A \x \Delta\Phi A)}
      }
      \]

      To justify the typing of our inductive hypotheses, observe that
      $\Phi\stripcx{\Gamma} = \stripcx{\Phi\Gamma}$ because $\Phi$ preserves the
      mode (discrete or monotone) of hypotheses. Justifying
      $\stripcx{\Phi\Gamma} = \iso\Phi\stripcx{\Gamma},\,
      \Delta\Phi\stripcx{\Gamma}$ requires a little more footwork:

      \begin{align*}
        &\phantom{{}={}}\iso\Phi\stripcx{\Gamma},\, \Delta\Phi\stripcx{\Gamma}\\
        &= \iso\stripcx{\Phi\Gamma},\, \Delta\stripcx{\Phi\Gamma}
        &&\text{because }\Phi\stripcx{\Gamma} = \stripcx{\Phi\Gamma}\\
        &= \iso\stripcx{\Phi\Gamma}
        &&\text{$\stripcx{\Phi\Gamma}$ has only discrete hypotheses, which $\Delta$ drops}\\
        &= \stripcx{\Phi\Gamma}
        &&\text{$\stripcx{\Phi\Gamma}$ has only discrete hypotheses, which $\iso$ leaves alone}
      \end{align*}

    \item[Case\quad $\infer{\J e \G {\iso A} \\ \J f {\G,\,\hd x A} B}{
      \J {\eletbox x e f} \G B}$.]\

      Recall that
      \begin{align*}
        \phi(\eletbox{x} e f) &= \elet{\pboxtuple{\dvar x,\dvar\dx} = \phi e} \phi
        \\
        \delta(\eletbox{x} e f) &=
        \elet{\pboxtuple{\dvar x, \dvar\dx} = \phi e} \delta f
      \end{align*}

      This involves some syntax sugar, so let us first derive the following
      typing rule:

      \[
      \infer{
        \J{e}{\Gamma}{\iso(A \times B)}\\
        \J{f}{\Gamma,\, \hd x A,\, \hd y B}{C}
      }{
        \J{\elet{\pboxtuple{\dvar x, \dvar y} = e} f}{\Gamma}{C}
      }
      \]

      Expanding our syntax sugar, we have

      \[
      \elet{\pboxtuple{\dvar x, \dvar y} = e} f
      \desugars
      \elet{y = e} \XXX
      \]

      \todo{what does this desugar to?}

      Putting this syntax sugar to work, we have:
      \[
      \infer*{
        \infer*[right=ih]{~}{
          \J{\phi e}{\Phi\Gamma}{\iso(\Phi A \x \Delta\Phi A)}
        }
        \\
        \infer*[right=ih]{~}{
          \J{\phi f}{\Phi\Gamma, \hd x {\Phi A}, \hd\dx{\Delta\Phi A}}{\Phi B}
        }
      }{
        \J{\elet{\pboxtuple{\dvar x,\dvar\dx} = \phi e} \phi f}
          {\Phi\Gamma}
          {\Phi B}
      }
      \]

      And similarly:
      \[
      \infer*{
        \J{\phi e}
          {\iso\Phi\Gamma, \Delta\Phi\Gamma}
          {\iso(\Phi A \x \Delta\Phi A)}
        \\
        \J{\delta f}
          {\iso\Phi\Gamma, \Delta\Phi\Gamma, \hd x {\Phi A}, \hd\dx{\Delta\Phi A}}
          {\Delta\Phi B}
      }{
        \J{\elet{\pboxtuple{\dvar x,\dvar\dx} = \phi e} \delta f}
          {\iso\Phi\Gamma, \Delta\Phi\Gamma}
          {\Phi B}
      }
      \]
      where the two premises come from our induction hypotheses, the
      latter directly, and the former by way of weakening: \todo{finish proving
        this case}

    \item[Case\quad $\infer{\quad}{\J\bot\G {\eqt L}}$,\quad $\phi\bot = \bot$,
      \quad $\delta\bot = \bot$.]\

      Trivial, noting only that $\Phi\eqt{L} = \eqt L$ by \cref{lemma-phi-eqt}
      and $\Delta\Phi\eqt{L} = \eqt{L}$ by
      \cref{lemma-phi-eqt,lemma-delta-lattice} are both semilattice equality
      types as required.

    \item[Case\quad $\infer{(\J{e_i} \G {\eqt L})_i}{\J{e_1 \vee e_2}\G {\eqt L}}$,\quad
      $\phi(e_1 \vee e_2) = \phi e_1 \vee \phi e_2$,\quad
      $\delta(e_1 \vee e_2) = \delta e_1 \vee \delta e_2$.]\

      Immediate, noting only that $\Phi\eqt{L} = \eqt L$ by \cref{lemma-phi-eqt}
      and $\Delta\Phi\eqt{L} = \eqt{L}$ by
      \cref{lemma-phi-eqt,lemma-delta-lattice} are both semilattice equality
      types as required.

    \item[Case\quad $\infer{(\J {e_i} {\stripcx\G} {\eqt A})_i}{
      \J {\esetsub{e_i}{i}} \G {\tset{\eqt A}}}$,\quad
      $\phi(\esetsub{e_i}{i}) = \esetsub{\phi e_i}{i}$,\quad
      $\delta\esetsub{e_i}{i} = \bot$.]\

      Immediate.

    \item[Case\quad $\infer{
        \J e \G {\tset A} \\
        \J f {\G,\, \hd x A} {\eqt L}
      }{\J {\eforvar x e f} \G {\eqt L}}$.]\

      Recall that
      \begin{align*}
        \phi(\eforvar x e f)
        &= \eforvar x {\phi e}{\eletbox{\dx}{\ebox{\zero\<\dvar x}} \phi f}\\
        \delta(\eforvar x e f)
        &= (\eforvar x {\delta e}
        %\substd{\phi f}{\dvar\dx \substo \zero\<\dvar x}) \\
        \eletbox \dx {\zero\<\dvar x} \phi f) \\
        &\vee (\eforvar x {\phi e \vee \delta e}
        %\substd{\delta f}{\dvar\dx \substo \zero\<\dvar x})
        \eletbox{\dx}{\zero\<\dvar x} \delta f)
      \end{align*}

      \todo{prove this case}

    \item[Case\quad $\infer{(\J {e_i} {\stripcx\G} {\eqt A})_i}
          {\J {\eeq{e_1}{e_2}} \G \tbool}$,\quad
          $\phi(\eeq e f) = \eeq{\phi e}{\phi f}$,\quad
          $\delta(\eeq e f) = \bot$.]\

      The $\phi$ case requires only that $\Phi \eqt{A}$ be an equality type,
      which it is by \cref{lemma-phi-eqt}; the $\delta$ case type checks because
      $\Delta\Phi\tbool = \tbool$.

    \item[Case\quad $\infer{\J {e} {\stripcx\G} {\tset\tunit}}{
      \J {\eisempty e} \G {\tunit + \tunit}}$,\quad
      $\phi(\eisempty e) = \eisempty{\phi e}$,\quad
      $\delta(\eisempty e) = \eisempty{\phi e}$.]\

      Immediate, observing that $\Phi\tset{\tunit} = \tset{\tunit}$ and
      $\Phi(\tunit + \tunit) = \Delta\Phi(\tunit + \tunit) = \tunit + \tunit$.

    \item[Case\quad $\infer{\J e \G {\iso{(A + B)}}}{\J{\esplit e} \G {\iso A + \iso
          B}}$.]\

      Recall that
      \begin{align*}
        \phi(\esplit e) &= \emcase{\phi e}
        \\
        &\phantom{{}={}}\
        \left(\pboxtuple{\inj i \dvar x, \inj i \dvar \dx}
        \caseto \inj i {\eboxtuple{\dvar x,\dvar\dx}}\right)_{i}
        \\
        &\phantom{{}={}}\
        \left(\pboxtuple{\inj i \dvar x, \inj j \pwild}
        \caseto \inj i {\eboxtuple{\dvar x, \dummy\<\dvar x}} \right)_{i\ne j}
      \end{align*}

      \todo{prove this case}

    \item[Case\quad $\infer{\J e {\stripcx{\G},\, \hm x {\fixt L}} {\fixt L}}{%
      \J{\efixis x e} \G {\fixt L}}$.]\

      Recall that
      \begin{align*}
        \phi(\efixis x e)
        &= \fastfix\<(\subone{\phi e}{\mvar x}{\bot},\; \delta(\efn{x}{e}))
        \\
        \delta(\efixis x e) &= \bot
        \\
        \Phi\fixt{L} &= \fixt{L} \quad\textsf{(\cref{lemma-phi-eqt})}
        \\
        \Delta\Phi\fixt{L} &= \fixt{L} \quad\textsf{(\cref{lemma-delta-lattice,lemma-phi-eqt})}
      \end{align*}

      This justifies the case for $\delta$ immediately. As for $\phi$, recall
      the typing rule for $\fastfix$: \todo{oops, this is going to be different
        if we keep fix different}

      \[
      \infer{
        \J{e}{\G}{\iso((\kernfixtL \to \fixtLkern) \x (\iso\fixt L \to \fixt L \to \fixtLkern)}
      }{\J{\fastfix\<e}{\G}{\fixt L}}
      \]

      \todo{prove this case}

  \end{description}
\end{proof}


\nextlemma
\EqualityChanges*
\begin{proof}
  \label{proof-equality-changes}
  By induction on $\eqt A$, applying the definition from
  \cref{figure-seminaive-logical-relation}:

  \begin{description}
    \item[Case $\tunit$.] Trivial.

    \item[Case $\eqt A \x \eqt B$.] Then our assumption is equivalent to
%
      \[\weirdat{\eqt A \x \eqt B}{(\dx_1,\dx_2)}{(x_1,x_2)}{(a_1,a_2)}{(y_1,y_2)}{(b_1,b_2)}\]
%
      and by unfolding this we have
      \(\weirdat{\eqt{A}}{\dx_1}{x_1}{a_1}{y_1}{b_1}\) and
      \(\weirdat{\eqt{B}}{\dx_2}{x_2}{a_2}{y_2}{b_2}\), which by our inductive
      hypotheses show \(x_1 = a_1\), \(y_1 = b_1\) and \(x_2 = a_2\), \(y_2 = b_2\),
      which suffices.

    \item[Case $\eqt A_1 + \eqt A_2$.] Then for some $i \in \{1,2\}$ our
      assumption is equivalent to
%
      \[
      \weirdat{\eqt A_1 + \eqt A_2}{\inj i \dx}{\inj i x}{\inj i a}
              {\inj i y}{\inj i b}
      \]
%
      and by unfolding this we have \(\weirdat{\eqt{A}_i}{\dx}{x}{a}{y}{b}\),
      which by our inductive hypothesis implies \(x=a\) and \(y=b\), which
      suffices.

    \item[Case $\tseteq{A}$.] Then our assumption unfolds to \((x,y,x \cup \dx)
      = (a,b,y)\), which suffices.

  \end{description}
\end{proof}


\nextlemma
\EqualityDummy*
\begin{proof}
  \label{proof-equality-dummy}
  By induction on \(\eqt{A}\), applying the definitions of $\dummy$ and
  $\weirdat{\eqt{A}}{\dummy\<x}{x}{x}{x}{x}$
  (\cref{figure-dummy,figure-seminaive-logical-relation}).

  \begin{description}
  \item[Case $\tunit$.] Trivial.

  \item[Case $\eqt{A} \times \eqt{B}$.]

    Letting \(x = (y,z)\), we have $\dummy\<x = \dummy\<(y,z) =
    (\dummy\<y,\dummy\<z)$. By our inductive hypotheses, we have
    \(\weirdat{\eqt{A}}{\dummy\<y}{y}{y}{y}{y}\) and likewise for $z$. By
    definition this shows that
    \[
    \weirdat{\eqt{A} \times \eqt{B}}{(\dummy\<y, \dummy\<z)}
            {(y,z)}{(y,z)}{(y,z)}{(y,z)}
    \]
    as desired.

  \item[Case $\eqt{A}_1 + \eqt{A}_2$.]

    Without loss of generality we have \(x = \inj i y\) for some $i \in
    \{1,2\}$. Applying the definition of \dummy\ we have \(\dummy\<x = \inj i
    (\dummy\<y)\). By our inductive hypothesis we have
    \(\weirdat{\eqt{A}_i}{\dummy\<y}{y}{y}{y}{y}\), which suffices to show
    \[
    \weirdat{\eqt{A}_i}{\inj i (\dummy\<y)}
            {\inj i y}{\inj i y}{\inj i y}{\inj i y}
    \]
    as desired.

  \item[Case $\tset{\eqt{A}}$.]

    Unfolding our theorem's definition, we need to show that
    \((x, x, x \cup \dummy\<x) = (x, x, x)\), or in other words
    \(x = x \cup \dummy_{\tseteq{A}}\<x\), which is trivial since \(\dummy_{\tseteq{A}}\<x = \esetraw{}\).
  \end{description}
\end{proof}


\nextlemma
\DiscreteContexts*
\DiscreteContextsProof*

%% \begin{proof}
%%   \label{proof-discrete-contexts}

%%   First, let's observe the types of the contexts we are dealing with. We have
%%   \(\rho, \rho' \in \den{\stripcx{\Gamma}}\) and \(\gamma, \gamma' \in
%%   \den{\Phi\stripcx{\Gamma}}\). In particular this means that all four will
%%   contain only discrete variables, namely, for every $\hd x A \in \Gamma$, we
%%   will have $\rho_{\dvar x}, \rho'_{\dvar{x}}$ and $\gamma_{\dvar{x}},
%%   \gamma_{\dvar \dx}, \gamma'_{\dvar{x}}, \gamma'_{\dvar\dx}$.

%%   Next, unfolding the definition of our assumption, we have
%%   \[
%%   \fa{\hm x A \in \stripcx{\Gamma}}
%%   \weirdat{A}{\dgamma_{\mvar\dx}}
%%           {\gamma_{\mvar{x}}}{\rho_{\mvar{x}}}
%%           {\gamma'_{\mvar{x}}}{\rho'_{\mvar{x}}}
%%   \]
%%   which is boring --- there \emph{are} no monotone hypotheses
%%   $\hm x A \in \stripcx{\Gamma}$ --- but also:
%%   \[
%%   \fa{\hd x A \in \stripcx{\Gamma}}
%%   \weirdat{\iso A}{()}
%%           {(\gamma_{\dvar x}, \gamma_{\dvar \dx})}
%%           {\rho_{\dvar x}}
%%           {(\gamma'_{\dvar x}, \gamma'_{\dvar \dx})}
%%           {\rho'_{\dvar x}}
%%   \]

%%   \noindent
%%   Unfolding the logical relation for the box type $\iso$, we have for each
%%   $\hd x A \in \stripcx{\Gamma}$ that
%% %
%%   \[
%%   (\rho_{\dvar x}, \gamma_{\dvar x}, \gamma_{\dvar\dx}) =
%%   (\rho'_{\dvar x}, \gamma'_{\dvar x}, \gamma'_{\dvar\dx})
%%   ~~\wedge~~ \weirdat{A}{\gamma_{\dvar\dx}}
%%          {\gamma_{\dvar x}}{\rho_{\dvar x}}
%%          {\gamma'_{\dvar x}}{\rho'_{\dvar x}}
%%   \]
%% %
%%   of which the first conjunct tells us that each component of $\rho$ equals
%%   its partner in $\rho'$ and the same for $\gamma$ and $\gamma'$, as desired.
%% \end{proof}


\nextlemma
\ContextStripping*
\ContextStrippingProof*

%% \begin{proof}
%%   \label{proof-context-stripping}
%%   To prove this, it suffices to show for each $\hd x A \in \stripcx{\Gamma}$
%%   (since after all there are no monotone $\hm x A \in \stripcx{\Gamma}$) that
%%   \[
%%   \weirdat{\iso A}{()}
%%           {(\strip(\gamma)_{\dvar x},\,
%%             \strip(\gamma)_{\dvar\dx})}
%%           {\strip(\rho)_{\dvar x}}
%%           {(\strip(\gamma')_{\dvar x},\,
%%             \strip(\gamma')_{\dvar\dx})}
%%           {\strip(\rho')_{\dvar x}}
%%   \]
%%   but since $\strip$ merely projects out the discrete variables, this is
%%   equivalent to
%%   \[
%%   \weirdat{\iso A}{()}
%%           {(\gamma_{\dvar x},\, \gamma_{\dvar\dx})}
%%           {\rho_{\dvar x}}
%%           {(\gamma'_{\dvar x},\, \gamma'_{\dvar\dx})}
%%           {\rho'_{\dvar x}}
%%   \]
%%   which is true by our assumption.
%% \end{proof}


\nextlemma
\SeminaiveFundamental*
\begin{proof}
  \label{proof-seminaive-fundamental}
  \todo{prove fundamental property for phi and delta}
\end{proof}

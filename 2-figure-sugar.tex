\begin{figure}
  \[\begin{array}{rccl}
    \text{types} & A,B &\bnfeq& ... \bnfor \tbool
    \\[\bnfskip]
    \text{terms} & e,f &\bnfeq& ...
%    \bnfor \ewhen e f
    \bnfor \etrue \bnfor \efalse
    \\&&&
    \efor{C} e \bnfor \esetfor{e}{C}
    \\&&&
    %\edcase{e} (p_i \caseto e_i)_i
%    \edcase{e} p \caseto f_1 \casebar \pwild \caseto f_2
    \ematch{p}{e}{f_1}{f_2}
    \\&&&
    \efixis x e
    \\[\bnfskip]
    \text{discrete patterns} & p &\bnfeq&
    \dvar x \bnfor \pwild \bnfor \peq{e}
    \bnfor \ptuple{p_1, p_2} \bnfor \inj i p
    \\[\bnfskip]
    \text{loop clauses} & C,D &\bnfeq&
    p \in e \bnfor e \bnfor C,D \bnfor %\emptycx
  \end{array}\]

  \begin{align*}
    \begin{aligned}
      \tbool &\desugars \tset \tunit
      \\
      \efalse &\desugars \eset{}
      \\
      \etrue &\desugars \eset{\etuple{}}
      \\
      \esetfor{e}{C} &\desugars \efor{C} \eset{e}
      %% \\
      %% \ewhen e f &\desugars \eforin{\pwild}{e} f
    \end{aligned}
    &&
    \begin{aligned}
      \efor{%\emptycx
} e &\desugars e
      \\
      \efor{C,D} e &\desugars \efor{C} \efor{D} e
      \\
      \efor{e}{f} &\desugars \eforin{\pwild}{e} f
      \\
      \eforin{p}{e} f &\desugars
      \eforvar{\color{fresh} x}{e}
      \ematch{p}{\color{fresh} x}{f}{\bot}
    \end{aligned}
  \end{align*}

  \begin{align*}
    \efixis x e &\desugars \efix\ebox{\efn x e}
    \\
    \fnof{\pboxvar{p}} e
    &\desugars
    \fnof{\color{fresh} \mvar y} \eletbox p {\color{fresh}\mvar y} e
    \\
    \elet{\pboxtuple{x, y} = e} f
    &\desugars
    %% \eletbox{\color{fresh} z}{e}
    %% \sub{f}{
    %%   \subto{\dvar x}{\eiso{\pi_1\<\dvar z}},\,
    %%   \subto{\dvar y}{\eiso{\pi_2\<\dvar z}}
    \eletbox{\color{fresh} z}{e}
    \eletbox x {\ebox{\pi_1\<z}}
    \eletbox y {\ebox{\pi_2\<z}}
    f
    \\
  %% \end{align*}
  %% \begin{align*}
    \ematch{\pwild}{e}{f_1}{f_2} &\desugars f_1
    \\
    \ematch{\dvar x}{e}{f_1}{f_2} &\desugars \eletbox{x}{\ebox{e}} f_1
    \\
    \ematch{\peq{e_1}}{e_2}{f_1}{f_2}
    &\desugars \emcase{\eisempty{(\eeq{e_1}{e_2})}}
    \inj 1 \pwild \caseto f_2
    \casebar \inj 2 \pwild \caseto f_1
    \\[-4pt]
    &\phantom{{}\desugars{}}
    \textit{\small(NB. \(\eisemptyraw e\) yields \(\injc_1\) if \(e\) is empty, i.e.\ \(\efalse\).)}
    \\
    \ematch{\ptuple{p_1,p_2}}{e}{f_1}{f_2}
    &\desugars
    %\ematch{p_1}{\pi_1\<e}{(\ematch{p_2}{\pi_2\<e}{f_1}{f_2})}{f_2}
    \ematchif{p_1}{\pi_1\<e}
    \\[-4pt]
    &\phantom{{}\desugars{}}
    \ematchthen{(\ematch{p_2}{\pi_2\<e}{f_1}{f_2})}
    \\[-4pt]
    &\phantom{{}\desugars{}}
    \ematchelse{f_2}
    \\
    \ematch{\inj i p}{e}{f_1}{f_2}
    &\desugars \emcase{\esplit{\ebox{e}}}
    \\[-4pt]
    &\phantom{{}\desugars\quad}
    \inj i {\color{fresh} \mvar x} \caseto
    \eletbox{\color{fresh} \dvar y}{\color{fresh} \mvar x}
    (\ematch{p}{\color{fresh}\dvar y}{f_1}{f_2})
    \\[-4pt]
    &\phantom{{}\desugars\quad}
    \inj{(3-i)} {\color{fresh} \mvar x} \caseto f_2
  \end{align*}

  \centering\itshape
  Fresh variables introduced by desugaring are colored {\color{fresh}\freshname}.

  \caption{Syntax sugar}
  \label{figure-sugar}
\end{figure}

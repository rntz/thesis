\section{Typing and Semantics}
\label{section-typing-and-semantics}

\begin{figure}
  \begin{mathpar}
    %% \begin{array}{r@{\hskip 1em}ccl}
    %% \text{types} & A,B &\bnfeq& \tunit \bnfor A \x B \bnfor A + B \bnfor A \to B
    %%                             %% \bnfor \iso A \bnfor \tset{\eqt A}
    %% \\[.5ex]
    %% \text{eqtypes} & \eqt A, \eqt B &\bnfeq&
    %% \tset{\eqt A} \bnfor
    %% \tunit \bnfor \eqt A \x \eqt B \bnfor \eqt A + \eqt B
    %% \\[.5ex]
    %% \text{semilattices} & L,M &\bnfeq& \tset{\eqt A} \bnfor \tunit \bnfor L \x M
    %% \\[.5ex]
    %% \text{finite eqtypes} & \fint A, \fint B &\bnfeq&
    %% \tset{\fint A} \bnfor \tunit \bnfor \fint A \x \fint B \bnfor \fint A + \fint B
    %% \\[.5ex]
    %% \text{fixtypes} & \fixt L, \fixt M &\bnfeq&
    %% \tset{\fint A} \bnfor \tunit \bnfor \fixt L \x \fixt M
    %% \end{array}
    %%
    \begin{array}{r@{\hskip 1em}ccl}
      \text{contexts} & \G &\bnfeq& \emptycx \bnfor \G,\, H \\
      \text{hypotheses} & H &\bnfeq& \hm x A \bnfor \hd x A
    \end{array}

    \begin{array}{rcl}
      \stripcxraw{\emptycx} & = & \emptycx\\
      \stripcxraw{\G,\, \hm x A} & = & \stripcxraw\G\\
      \stripcxraw{\G,\, \hd x A} & = & \stripcxraw\G,\, \hd x A
    \end{array}
    \\
    \infer[\rn{var}]{\hm x A \in \G}{\J x \G A}

    \infer[\rn{dvar}]{\hd x A \in \G}{\J {\dvar x} \G A}

    \infer[\rn{lam}]{\J e {\G,\,\hm x A} B}{\J {\efn x e} \G {A \to B}}

    \infer[\rn{app}]{\J e \G {A \to B} \\ \J f \G A}{\J {e\<f} \G B}

    \infer[\rn{unit}]{\quad}{\J {\etuple{}} \G \tunit}

    \infer[\rn{pair}]{(\J{e_i}\G{A_i})_i}{\J{\etuple{e_1,e_2}} \G {A_1 \x A_2}}

    \infer[\rn{prj}]{\J e \G {A_1 \x A_2}}{\J{\pi_i\<e}\G{A_i}}

    \infer[\rn{inj}]{\J e \G A_i}{\J{\inj i e}\G{A_1 + A_2}}

    \infer[\rn{case}]{\J e \G {A_1 + A_2} \\
      (\J {f_i} {\G,\, \hm{x_i}{A_i}} {B})_i
    }{
      \J {\emcase{e} (\inj i {\mvar x_i} \caseto f_i)_i} \G B
    }

    \infer[\rn{box}]{\J {e} {\stripcx\G} A}{\J{\ebox e} \G {\iso A}}

    \infer[\rn{letbox}]{\J e \G {\iso A} \\ \J f {\G,\,\hd x A} B}{
      \J {\eletbox x e f} \G B}

    \infer[\rn{bot}]{\quad}{\J\bot\G {\eqt L}}

    \infer[\rn{join}]{(\J{e_i} \G {\eqt L})_i}{\J{e_1 \vee e_2}\G {\eqt L}}

    \infer[\rn{set}]{(\J {e_i} {\stripcx\G} {\eqt A})_i}{
      \J {\esetsub{e_i}{i}} \G {\tset{\eqt A}}}

    \infer[\rn{for}]{
      \J e \G {\tset A} \\
      \J f {\G,\, \hd x A} {\eqt L}
    }{\J {\eforvar x e f} \G {\eqt L}}

    \infer[\rn{eq}]{(\J {e_i} {\stripcx\G} {\eqt A})_i}
          {\J {\eeq{e_1}{e_2}} \G \tbool}

    \infer[\rn{empty?}]{\J {e} {\stripcx\G} {\tset\tunit}}{
      \J {\eisempty e} \G {\tunit + \tunit}}

    \infer[\rn{split}]{\J e \G {\iso{(A + B)}}}{\J{\esplit e} \G {\iso A + \iso B}}

    \infer[\rn{fix}]{\J e {\stripcx{\G},\, \hm x {\fixt L}} {\fixt L}}{%
      \J{\efixis x e} \G {\fixt L}}
  \end{mathpar}

  \caption{Datafun typing rules}
  \label{figure-typing}
\end{figure}


The syntax of core Datafun is given in \cref{figure-syntax} and its typing rules
in \cref{figure-typing}. Contexts are lists of hypotheses $H$; a
hypothesis gives the type of either a monotone variable $\hm x A$ or a discrete
variable $\hd x A$. The stripping operation $\stripcx\G$ drops all monotone
hypotheses from the context $\G$, leaving only the discrete ones.
%
The typing judgement $\J{e}{\G}{A}$ may be read as ``assuming the variables in
$\G$ have their given types, the term $e$ has type $A$''.

% TODO d\kern0pt var ?
The \rn{var} and \rn{dvar} rules say that both monotone hypotheses $\hm x A$ and
discrete hypotheses $\hd x A$ justify ascribing their variable the type $A$.
%
The \rn{lam} rule is the familiar rule for $\fn$-abstraction. However, note that
we introduce the argument variable $\hm x A$ as a monotone hypothesis, not
a discrete one. (This is the ``right'' choice because in \Poset\ the exponential
object is the poset of monotone functions.)
%
The application rule \rn{app} is standard, as are the rules \rn{unit},
\rn{pair}, \rn{prj}, \rn{inj}. Case analysis \rn{case} is also standard, noting
only that as with \rn{lam}, the variables $\hm {x_i}{A_i}$ bound in each branch
$f_i$ are monotone.

\rn{box} says that $\ebox{e}$ has type $\iso A$ when $e$ has type $A$ in the
stripped context $\stripcx\G$. This restricts $e$ to refer only to discrete
variables, ensuring we don't smuggle any information we must treat monotonically
into a discretely-ordered $\iso$ expression. The elimination rule \rn{letbox}
for $(\eletbox x e f)$ allows us to ``cash in'' a boxed expression $e : \iso A$
by binding its result to a discrete variable $\hd x A$ in the body $f$.

At this point, our typing rules correspond to standard constructive S4 modal
logic~\cite{jrml}. We get to Datafun by adding a handful of domain-specific
types and operations.
%
First, \rn{split} provides an operator $\prim{split} : \iso(A + B) \to \iso A +
\iso B$ to distribute box across sum types.\footnote{An alternative syntax,
  pursued in \citet{datafun}, would be to give two rules for $\kw{case}$,
  depending on whether or not the scrutinee could be typechecked in a stripped
  context. \todo{here we provide this as syntax sugar, in order to simplify and
    clarify our semantics.}}
%
The other direction, $\iso A + \iso B \to \iso (A + B)$, is already derivable,
as is the isomorphism $\iso A \times \iso B \cong \iso (A \times B)$.
%
This is used implicitly by box pattern-matching -- e.g., in the pattern
$\pboxtuple{\inj 1 \dvar x, \inj 2 \dvar y}$, the variables $\dvar x$ and $\dvar
y$ are both discrete, which is information we propagate via these conversions.
\todo{example of box pattern desugaring? move later?}
%
%% TODO: an example of desugaring pattern-matching here?
%
Semantically, all of these operations are the identity, as we
shall see shortly.

This leaves only the rules for manipulating sets and other semilattices.
\rn{bot} and \rn{join} tell us that $\bot$ and $\vee$ are valid at any
semilattice type $L$, that is, at sets and products of semilattice types.
%
The rule for set-elimination, \rn{for}, is almost a monadic bind.
%
However, we generalize it by allowing $\eforvar x e f$ to eliminate into any
semilattice type, not just sets, denoting a ``big semilattice join'' rather than
a ``big union''.
%
\todo{explain why bot/join/for require an equality lattice.}

The set-introduction rule \rn{set} gives $\esetsub{e_i}{i\in I}$ type $\tseteq
A$ when each of the $e_i$ has type $\eqt A$.
%
Just as in \rn{box}, each $e_i$ has to typecheck in a stripped context;
constructing a set is a discrete operation, since $1 \le 2$ but $\esetraw{1}
\not\subseteq \esetraw{2}$.

Likewise discrete is equality comparison $\eeq{e_1}{e_2}$, whose rule \rn{eq} is
otherwise straightforward; and \rn{empty?}, which requires more explanation. The
idea is that $\eisempty e$ determines whether $e : \tset{\tunit}$ is empty,
returning $\inj 1 \etuple{}$ if it is, and $\inj 2 \etuple{}$ if it isn't. This
lets us turn ``booleans'' (sets of units) into values we can \kw{case}-analyse.
This is, however, not monotone, because while booleans are ordered $\efalse <
\etrue$, sum types are ordered disjointly; $\inj 1 ()$ and $\inj 2 ()$ are
incomparable.

\todo{fix this sentence:}
%
Finally, the rule \rn{fix} for fixed points $\efixis x e$ takes a function $e :
\isofixLtoL$ and yields an expression of type $\fixtLkern$.
%
The restriction to ``fixtypes'' ensures $\fixt L$ has no infinite ascending
chains, guaranteeing the recursion will terminate.
%
\todo{but why the context stripping?}


\subsection{Semantics}
\label{section-semantics}

\todo{rewrite from here on; check seminaive.pdf and first paper}

The syntax of core Datafun can be interpreted in $\Poset$, the category of
partially ordered sets and monotone maps. That is, an object of $\Poset$ is a
pair $(A, \leq_A)$ consisting of a set $A$ and a reflexive, transitive,
antisymmetric relation $\leq_A \subseteq A \times A$, while a morphism $f : A
\to B$ is a function such that $x \leq_A y \implies f(x) \leq_B f(y)$.

\subsubsection{Bicartesian Structure}

The bicartesian closed structure of $\Poset$ is largely the same as in $\Set$.
%
The product and sum sets are constructed the same way, and ordered pointwise:

\begin{align*}
  (a,b) \le_{A \x B} (a',b') &\iff a \le_A a' \wedge b \le_B b'\\
  \inj i x \le_{A_1 + A_2} \inj j y &\iff i = j \wedge x \le_{A_i} y
\end{align*}

\noindent Projections $\pi_i$, injections $\injc_i$, tupling $\fork{f,g}$ and
case-analysis $\krof{f,g}$ are all the same as in \Set, pausing only to note
that all these operations preserve monotonicity, as we need.

The exponential $A \expto B$ consists of the monotone maps $f : A \to B$, again
ordered pointwise:

\[ f \le_{A \expto B} g \iff \fa{x \le_A y} f\<x \le_B g\<y \]

\noindent
Currying $\fn$ and evaluation are the same as in \Set. Supposing $f : A \x B \to
C$, then:

\begin{align*}
  \fn(f) &\isa A \to (B \expto C) &
  \eval_{A,B} &\isa (A \expto B) \x A \to B
  \\
  \fn(f) &= x \mapsto y \mapsto f(x,y) &
  \eval_{A,B} &= (g,x) \mapsto g(x)
\end{align*}

\noindent
Monotonicity here follows from the monotonicity of $f$ and $g$ and the pointwise
ordering of $A \expto B$.


\section{Typing and denotational semantics}
\label{section-typing-and-semantics}

\begin{figure}
  \begin{mathpar}
    %% \begin{array}{r@{\hskip 1em}ccl}
    %% \text{types} & A,B &\bnfeq& \tunit \bnfor A \x B \bnfor A + B \bnfor A \to B
    %%                             %% \bnfor \iso A \bnfor \tset{\eqt A}
    %% \\[.5ex]
    %% \text{eqtypes} & \eqt A, \eqt B &\bnfeq&
    %% \tset{\eqt A} \bnfor
    %% \tunit \bnfor \eqt A \x \eqt B \bnfor \eqt A + \eqt B
    %% \\[.5ex]
    %% \text{semilattices} & L,M &\bnfeq& \tset{\eqt A} \bnfor \tunit \bnfor L \x M
    %% \\[.5ex]
    %% \text{finite eqtypes} & \fint A, \fint B &\bnfeq&
    %% \tset{\fint A} \bnfor \tunit \bnfor \fint A \x \fint B \bnfor \fint A + \fint B
    %% \\[.5ex]
    %% \text{fixtypes} & \fixt L, \fixt M &\bnfeq&
    %% \tset{\fint A} \bnfor \tunit \bnfor \fixt L \x \fixt M
    %% \end{array}
    %%
    \begin{array}{r@{\hskip 1em}ccl}
      \text{contexts} & \G &\bnfeq& \emptycx \bnfor \G,\, H \\
      \text{hypotheses} & H &\bnfeq& \hm x A \bnfor \hd x A
    \end{array}

    \begin{array}{rcl}
      \stripcxraw{\emptycx} & = & \emptycx\\
      \stripcxraw{\G,\, \hm x A} & = & \stripcxraw\G\\
      \stripcxraw{\G,\, \hd x A} & = & \stripcxraw\G,\, \hd x A
    \end{array}
    \\
    \infer[\rn{var}]{\hm x A \in \G}{\J x \G A}

    \infer[\rn{dvar}]{\hd x A \in \G}{\J {\dvar x} \G A}

    \infer[\rn{lam}]{\J e {\G,\,\hm x A} B}{\J {\efn x e} \G {A \to B}}

    \infer[\rn{app}]{\J e \G {A \to B} \\ \J f \G A}{\J {e\<f} \G B}

    \infer[\rn{unit}]{\quad}{\J {\etuple{}} \G \tunit}

    \infer[\rn{pair}]{(\J{e_i}\G{A_i})_i}{\J{\etuple{e_1,e_2}} \G {A_1 \x A_2}}

    \infer[\rn{prj}]{\J e \G {A_1 \x A_2}}{\J{\pi_i\<e}\G{A_i}}

    \infer[\rn{inj}]{\J e \G A_i}{\J{\inj i e}\G{A_1 + A_2}}

    \infer[\rn{case}]{\J e \G {A_1 + A_2} \\
      (\J {f_i} {\G,\, \hm{x_i}{A_i}} {B})_i
    }{
      \J {\emcase{e} (\inj i {\mvar x_i} \caseto f_i)_i} \G B
    }

    \infer[\rn{box}]{\J {e} {\stripcx\G} A}{\J{\ebox e} \G {\iso A}}

    \infer[\rn{letbox}]{\J e \G {\iso A} \\ \J f {\G,\,\hd x A} B}{
      \J {\eletbox x e f} \G B}

    \infer[\rn{bot}]{\quad}{\J\bot\G {\eqt L}}

    \infer[\rn{join}]{(\J{e_i} \G {\eqt L})_i}{\J{e_1 \vee e_2}\G {\eqt L}}

    \infer[\rn{set}]{(\J {e_i} {\stripcx\G} {\eqt A})_i}{
      \J {\esetsub{e_i}{i}} \G {\tset{\eqt A}}}

    \infer[\rn{for}]{
      \J e \G {\tset A} \\
      \J f {\G,\, \hd x A} {\eqt L}
    }{\J {\eforvar x e f} \G {\eqt L}}

    \infer[\rn{eq}]{(\J {e_i} {\stripcx\G} {\eqt A})_i}
          {\J {\eeq{e_1}{e_2}} \G \tbool}

    \infer[\rn{empty?}]{\J {e} {\stripcx\G} {\tset\tunit}}{
      \J {\eisempty e} \G {\tunit + \tunit}}

    \infer[\rn{split}]{\J e \G {\iso{(A + B)}}}{\J{\esplit e} \G {\iso A + \iso B}}

    \infer[\rn{fix}]{\J e {\stripcx{\G},\, \hm x {\fixt L}} {\fixt L}}{%
      \J{\efixis x e} \G {\fixt L}}
  \end{mathpar}

  \caption{Datafun typing rules}
  \label{figure-typing}
\end{figure}


Our guiding intuition so far has been that Datafun is a language for writing
monotone, higher-order functions.
%
Here we substantiate that intuition by giving typing rules for core Datafun and
showing how to interpret well-typed Datafun terms into \Poset, the category of
partially ordered sets and monotone maps.

\subsection{Typing rules}

The syntax of core Datafun is given in \cref{figure-syntax} and its typing rules
in \cref{figure-typing}. Contexts are lists of hypotheses $H$; a
hypothesis gives the type of either a monotone variable $\hm x A$ or a discrete
variable $\hd x A$. The stripping operation $\stripcx\G$ drops all monotone
hypotheses from the context $\G$, leaving only the discrete ones.
%
The typing judgement $\J{e}{\G}{A}$ may be read as ``assuming the variables in
$\G$ have their given types, the term $e$ has type $A$''.

% TODO d\kern0pt var ?
The \rn{var} and \rn{dvar} rules say that both monotone hypotheses $\hm x A$ and
discrete hypotheses $\hd x A$ justify ascribing their variable the type $A$.
%
The \rn{lam} rule is the familiar rule for $\fn$-abstraction. However, note that
we introduce the argument variable $\hm x A$ as a monotone hypothesis, not
a discrete one. (This is the ``right'' choice because in \Poset\ the exponential
object is the poset of monotone functions.)
%
The application rule \rn{app} is standard, as are the rules \rn{unit},
\rn{pair}, \rn{prj}, \rn{inj}. Case analysis \rn{case} is also standard, noting
only that as with \rn{lam}, the variables $\hm {x_i}{A_i}$ bound in each branch
$f_i$ are monotone.

\rn{box} says that $\ebox{e}$ has type $\iso A$ when $e$ has type $A$ in the
stripped context $\stripcx\G$. This restricts $e$ to refer only to discrete
variables, ensuring we don't smuggle any information we must treat monotonically
into a discretely-ordered $\iso$ expression. The elimination rule \rn{letbox}
for $(\eletbox x e f)$ allows us to ``cash in'' a boxed expression $e : \iso A$
by binding its result to a discrete variable $\hd x A$ in the body $f$.

At this point, our typing rules correspond to standard constructive S4 modal
logic~\cite{jrml}. We get to Datafun by adding a handful of domain-specific
types and operations.
%
First, \rn{split} provides an operator $\prim{split} : \iso(A + B) \to \iso A +
\iso B$ to distribute box across sum types.\footnote{An alternative syntax,
  pursued in \citet{datafun}, would be to give two rules for $\kw{case}$,
  depending on whether or not the scrutinee could be typechecked in a stripped
  context. \todo{here we provide this as syntax sugar, in order to simplify and
    clarify our semantics.}}
%
The other direction, $\iso A + \iso B \to \iso (A + B)$, is already derivable,
as is the isomorphism $\iso A \times \iso B \cong \iso (A \times B)$.
%
This is used implicitly by box pattern-matching -- e.g., in the pattern
$\pboxtuple{\inj 1 \dvar x, \inj 2 \dvar y}$, the variables $\dvar x$ and $\dvar
y$ are both discrete, which is information we propagate via these conversions.
\todo{example of box pattern desugaring? move later?}
%
%% TODO: an example of desugaring pattern-matching here?
%
Semantically, all of these operations are the identity, as we
shall see shortly.

This leaves only the rules for manipulating sets and other semilattices.
\rn{bot} and \rn{join} tell us that $\bot$ and $\vee$ are valid at any
semilattice type $L$, that is, at sets and products of semilattice types.
%
The rule for set-elimination, \rn{for}, is almost a monadic bind.
%
However, we generalize it by allowing $\eforvar x e f$ to eliminate into any
semilattice type, not just sets, denoting a ``big semilattice join'' rather than
a ``big union''.
%
\todo{explain why bot/join/for require an equality lattice.}

The set-introduction rule \rn{set} gives $\esetsub{e_i}{i\in I}$ type $\tseteq
A$ when each of the $e_i$ has type $\eqt A$.
%
Just as in \rn{box}, each $e_i$ has to typecheck in a stripped context;
constructing a set is a discrete operation, since $1 \le 2$ but $\esetraw{1}
\not\subseteq \esetraw{2}$.

Likewise discrete is equality comparison $\eeq{e_1}{e_2}$, whose rule \rn{eq} is
otherwise straightforward; and \rn{empty?}, which requires more explanation. The
idea is that $\eisempty e$ determines whether $e : \tset{\tunit}$ is empty,
returning $\inj 1 \etuple{}$ if it is, and $\inj 2 \etuple{}$ if it isn't. This
lets us turn ``booleans'' (sets of units) into values we can \kw{case}-analyse.
This is, however, not monotone, because while booleans are ordered $\efalse <
\etrue$, sum types are ordered disjointly; $\inj 1 ()$ and $\inj 2 ()$ are
incomparable.

Finally, the rule \rn{fix} for fixed points $\efixis x e$ permits a recursive expression $e : \fixt{L}$ which refers to its own result as $\mvar x$.
%
The restriction to ``fixtypes'' ensures $\fixt L$ has no infinite ascending
chains, guaranteeing the recursion will terminate.
%
\todo{but why the context stripping?}


\subsection{The category \Poset\ and its structures}
\label{section-poset-structures}

\splittodo{rewrite from here on.}{Talk about the intuition more. check
  seminaive.pdf and first paper}

An object of $\Poset$ is a pair $(A, \le_A)$ consisting of a set $A$ and a
reflexive, transitive, antisymmetric relation $\le_A \subseteq A \times A$. For
convenience, we usually denote these by a single letter $A$, leaving $\le_A$
implicit. Following this convention, a morphism $f : A \to B$ is a function such
that $x \le_A y \implies f(x) \le_B f(y)$.

\subsubsection{Bicartesian structure}

The bicartesian closed structure of $\Poset$ is largely the same as in $\Set$.
%
The product and sum sets are constructed the same way, and ordered pointwise:

\begin{align*}
  (a,b) \le_{A \x B} (a',b') &\iff a \le_A a' \wedge b \le_B b'\\
  \inj i x \le_{A_1 + A_2} \inj j y &\iff i = j \wedge x \le_{A_i} y
\end{align*}

\noindent Projections $\pi_i$, injections $\injc_i$, tupling $\fork{f,g}$ and
case-analysis $\krof{f,g}$ are all the same as in \Set, pausing only to note
that all these operations preserve monotonicity, as we need.

The exponential $A \expto B$ consists of the monotone maps $f : A \to B$, again
ordered pointwise:

\[ f \le_{A \expto B} g \iff \fa{x \le_A y} f\<x \le_B g\<y \]

\noindent
Currying $\fn$ and evaluation are the same as in \Set. Supposing $f : A \x B \to
C$, then:

\begin{align*}
  \fn(f) &\isa A \to (B \expto C) &
  \eval_{A,B} &\isa (A \expto B) \x A \to B
  \\
  \fn(f) &= x \mapsto y \mapsto f(x,y) &
  \eval_{A,B} &= (g,x) \mapsto g(x)
\end{align*}

\noindent
Monotonicity here follows from the monotonicity of $f$ and $g$ and the pointwise
ordering of $A \expto B$.


\subsubsection{The discreteness comonad}

Given a poset $(A, \le_A)$ we define the discreteness comonad $\iso(A, \le_A)$
as $(A, \le_{\iso A})$, where \( a \le_{\iso A} a' \iff a = a' \).
%
That is, the discrete order preserves the underlying elements, but reduces the
partial order to mere equality.
%
This forms a rather boring comonad whose functorial action $\iso(f)$, extraction $\varepsilon_A : \iso A \to A$, and duplication $\delta_A : \iso A \to \iso \iso A$ are all identities on the underlying sets:

\nopagebreak[2]
\begin{align*}
  \iso(f) &= f & \varepsilon_A &= a \mapsto a & \delta_A &= a \mapsto a
\end{align*}

\noindent
This makes the functor and comonad laws trivial. Monotonicity holds in each case because \emph{all} functions are monotone with respect to $\le_{\iso A}$.
%
It is also immediate that $\iso$ is monoidal with respect to \emph{both}
products and coproducts. That is, $\iso (A \times B) \cong \iso A \times \iso B$
and $\iso (A + B) \cong \iso A + \iso B$.
%
In both cases the isomorphism is witnessed by identity on the underlying
elements.
%
These lift to $n$-ary products and sums as well, which we write as $\isox : \prod_i \iso A_i \to \iso\prod_i A_i$ and $\isosum : \iso \sum_i
A_i \to \sum_i \iso A_i$.
%% %
%% We will write $\isox : \prod_i \iso A_i \to \iso\prod_i A_i$ to name the map
%% witnessing distributivity of products over $\iso$, and $\isosum : \iso \sum_i
%% A_i \to \sum_i \iso A_i$ to name the map witnessing distributivity of $\iso$
%% over coproducts.


\subsubsection{Sets and semilattices}

Given a poset $(A, \le_A)$ we define the finite powerset poset $\pfinof(A,
\le_A)$ as $(\powerset_{\mathrm{fin}}\, A, \subseteq)$, that is, the finite
subsets of $A$ ordered by subset inclusion.
%
Note that the subset ordering completely ignores the element ordering $\le_A$.
%
Finite sets admit a pair of useful morphisms:

\begin{align*}
  \morph{singleton} &\isa \iso A \to \pfinof A
  &
  \morph{isEmpty} &\isa \iso \pfinof A \to \terminalobject + \terminalobject
  \\
  \morph{singleton} &= a \mapsto \{a\}
  &
  \morph{isEmpty} &= X \mapsto 
  \begin{cases}
    \inj 1 () &\text{when }X = \emptyset\\
    \inj 2 () &\text{otherwise}
  \end{cases}
\end{align*}

\noindent
The \morph{singleton} function takes a value and makes a singleton set out of
it. The domain must be discrete, as otherwise the map will not be monotone (sets
are ordered by inclusion, and set membership relies on equality, not the partial
order). Similarly, the emptiness test \morph{isEmpty} also takes a discrete
set-valued argument, because otherwise the boolean test would not be monotone.

Sets also form a semilattice, with the least element given by the empty set, and
join given by union.
%
For this and other semilattices $L \in \Poset$, in particular products of
semilattices, we write $\morphjoin{L}{n} : L^n \to L$ to denote the $n$-ary
semilattice join (least upper bound).
%
Moreover, if $f : A \times \iso B \to L$, we define a morphism
$\pcollect{f} : A \times \pfinof{B} \to L$ as follows:

\begin{displaymath}
 \pcollect{f}  = (a, X) \mapsto \bigvee_{b \in X} f(a, b)
\end{displaymath}

\noindent
We will use this to interpret \kw{for}-loops. However, it is worth noting that
the discreteness of \morph{singleton} means finite sets do not quite form a
monad in $\Poset$.


\subsubsection{Equality} Every object $A \in \Poset$ admits an equality-test morphism \morph{eq}:

\begin{align*}
  \morph{eq} &\isa \iso A \x \iso A \to \pfinof{\terminalobject}\\
  \morph{eq} &= (x,y) \mapsto 
  \begin{cases}
    \{()\} &\text{if }x = y\\
    \emptyset &\text{otherwise}
  \end{cases}
\end{align*}

\noindent
The domain must be discrete, since $x = y$ and $y \le z$ certainly doesn't imply $x = z$. \todo{explain we only use equality where it is decidable}


\subsubsection{Fixed points}

Given a semilattice $L \in \Poset$ without infinite ascending chains, we can
define a family of fixed point morphisms $\morph{fix} : \iso (L \expto L) \to L$
as follows:

\begin{displaymath}
  \morph{fix} = f \mapsto \bigvee_{n \in \mathbb{N}} f^n(\bot)
\end{displaymath}

\noindent
A routine inductive argument shows this must yield a least fixed point.
\todo{sketch this fixed point argument} 
\todo{explain why fix is discrete}


\subsection{Interpretation of Datafun in \Poset}
\label{section-semantics}

%% ---- Semantics in a Datafun Model ----
\begin{figure*}
  \centering
  \textsc{type and context denotations}
  \vspace{2pt}

  %% TODO: revert to this more readable version if space allows.
  \begin{align*}
    \den{\tunit} &= \terminalobject & \den{A \to B} &= \den{A} \expto \den{B}
    \\
    \den{\tseteq A} &= \pfinof{\den{\eqt A}}
    & \den{A \x B} &= \den{A} \x \den{B}
    \\
    \den{\iso A} &= \iso{\den{A}} & \den{A + B} &= \den{A} + \den{B}
  \end{align*}

  \begin{align*}
    \den{\G} &= \prod_{H \in \G} \den{H} &
    \den{\hm x A} &= \den{A} & \den{\hd x A} &= \iso{\den{A}} &
    \den{\G \vdash A} &= \Poset(\den\G, \den A)
  \end{align*}
  \vspace{-4pt}                  % yes this matters

  \textsc{term denotations}
  \vspace{2pt}

  \begin{displaymath}
    \def\arraystretch{1.25} % lots of space
    \begin{array}{rcl}
      \den{\J {\mvar x} \G A} &=& \pi_{\mvar x} \qquad\quad\, \text{(for $\hm x A \in \G$)} \\
      \den{\J {\dvar x} \G A} &=& \pi_{\dvar x} \then \varepsilon \qquad \text{(for $\hd x A \in \G$)} \\
      \den{\J {\efn x e} \G {A \to B}} &=&
        \lambda_{\mvar x}\den{\J e {\G,\, \hm x A} B} \\
      \den{\J {e_1\<e_2} \G B} &=& \fork{\den{\J{e_1}\G{A \to B}},\, \den{\J{e_2}\G A}} \then \eval \\
      \den{\J {\etuple{e_1,\, e_2}} \G {A_1 \times A_2}} &=&
           \fork{\den{\J {e_1} \G {A_1}},\, \den{\J {e_2} \G {A_2}}} \\
      \den{\J {\pi_i\<e} \G {A_i}} &=& \den{\J e \G {A_1 \times A_2}} \then \pi_i \\
      \den{\J {\inj i e} \G {A_1 + A_2}} &=& \den{\J e \G {A_i}} \then \injc_i \\
      \den{\J {\emcase{e} (\inj i {\mvar x_i} \caseto f_i)_i} \G B} &=&
        \fork{\id_{\den\G},\, \den{\J e \G {A_1 + A_2}}}
        \then \morph{dist}^\x_+
        \then \bigkrof{\den{\J {f_i} {\G,\, \hm {x_i} {A_i}} B}}_{i\in\{1,2\}} \\
      \den{\J {\ebox e} \G {\iso A}} &=& \mkbox_\G(\den {\J e {\stripcx \G} A}) \\
      \den{\J {\elet{\ebox x = e} f} \G B} &=&  \fork{\id_{\den{\G}}, \den{\J e \G {\iso A}}} \then \den{\J f {\G, \hd x A} B}  \\
      \den{\J \bot \G L} &=& \fork{} \then \morphjoin{\den L}{0} \\
      \den{\J {e \vee f} \G L} &=& \fork{\den{\J e \G L},\, \den{\J f \G L}} \then \morphjoin{\den L}{2} \\
      \den{\J {\eisempty e} \G {1+1}}&=& \mkbox_\Gamma(\den{\J e {\stripcx \G} {\tset \tunit}}) \then \morph{isEmpty} \\
      \den{\J {\esplit e} \G {\iso A + \iso B}} &=& \den{\J e \G {\iso(A + B)}}\then \isosum \\
      \den{\J {\eeq{e_1}{e_2}} \G {\tbool}} &=&
          \fork{\mkbox_\Gamma(\den{\J {e_1} {\stripcx \G} {\eqt A}}),\,
                \mkbox_\Gamma(\den{\J {e_2} {\stripcx \G} {\eqt A}})}
          \then \morph{eq} \\
      \den{\J {\efixis x e} \G {\fixt L \kern 1pt}} &=&
        \mkbox_\G(\lambda_{\mvar x}
          \den{\J e {\stripcx{\G}, \hm x {\fixt L}} {\fixt L \kern1pt}})
        \then \morph{fix}
      \\
      \den{\J {\esetsub{e_i}{i}} \G {\tseteq A}}
      &=& \fork{\mkbox_\Gamma(\den{\J {e_i} {\stripcx \G} {\eqt A}}) \then \morph{singleton}}_i \then \morphjoin{\pfinof{\den{\eqt A}}}
      \\
      \den{\J {\eforvar x e f} \G L} &=& \fork{\id_{\den\G},\den{\J e \G {\tseteq A}}} \then \pcollect{\den{\J f {\G, \hd x {\eqt A}} L}} \\
    \end{array}
  \end{displaymath}
  \vspace{2pt} % yes, this matters

  \textsc{auxilliary operations}
  \vspace{2pt}

  \begin{align*}
    \morph{dist}^\x_+ &\isa A \x (B_1 + B_2) \to (A \x B_1) + (A \x B_2)
    &
    \mkbox_\Gamma &\isa
    %\Poset(\den{\stripcx \G}, A) \to \Poset(\den{\G}, \iso A)
    \den{\stripcx \G \vdash A} \to \den{\G \vdash \iso A}
    \\
    % this could be simpler if it distributed in the opposite direction.
    \morph{dist}^\x_+ &= \fork{\pi_2 \then \krof{\lambda (\fork{\pi_2,\pi_1} \then \injc_i)}_i, \pi_1}
    \then \eval
    &
    \mkbox_\Gamma(f) &= \fork{\pi_{\dvar x} \then \delta}_{\hd x A \in \G} \then \isox \then \iso(f)
  \end{align*}

  \caption{Semantics of Datafun}
  \label{figure-semantics}\label{def:strip}
\end{figure*}


\Cref{figure-semantics} shows how to interpret Datafun into \Poset\ using the
structures developed above.
%
We interpret Datafun types and typing contexts as \Poset-objects $\den{A},
\den{\G}$ and well-typed Datafun terms (or more precisely, their typing
derivations) $\J e \G A$ as \Poset-morphisms $\den{\G} \to \den A$.
%
This follows the usual interpretation for constructive S4~\cite{depaiva-s4},
with the addition of sets, semilattices, fixed points, and the ability to
distribute $\iso$ over sums.
%
We give the interpretation in combinatory style; to increase readability, we
freely use $n$-ary products to represent our typing context, to avoid the
book-keeping of reassociating binary products.

Regarding notation, we write composition in diagrammatic or ``pipeline'' order
with a centered dot, so $f \cdot g : A \to C$ means $f : A \to B$ followed by $g
: B \to C$.
%
If $f_i : A \to B_i$ we write $\fork{f_i}_i : A \to \prod_i B_i$ for the
``tupling map'' such that $\fork{f_i}_i \then \pi_j = f_j$.
%
In particular, $\fork{}$ is the map into the terminal object.
%
Dually, if $g_i : A_i \to B$ we write $\krof{g_i}_i : \sum_i A_i \to B$ for the
``case-analysis map'' such that $\injc_j \then \krof{g_i}_i = g_j$.

\todo{lemma that the interpretation of a semilattice type is a semilattice?}


\chapter{Implementation and Efficiency}
\label{chapter-implementation}

\newcommand\deep{\mathvar{dp}}

The previous chapter was entirely theoretical, formalizing the intuition that
semi\naive\ evaluation works by computing the changes between iterations toward
a fixed point by, first, constructing a theory of changes for Datafun; and
second, applying that theory to construct and prove correct a program
transformation which implements this strategy. However, the purpose of
semi\naive\ evaluation is not to push changes around, but to compute results
faster. We have proven that our transformed program computes the same result,
but not shown that it does so more efficiently.
%
In this chapter we remedy this experimentally, observing that at least two
further optimizations are necessary to achieve asymptotic performance
improvements.

%% In this chapter we remedy this experimentally. First, we apply our program
%% transformation to our running example, transitive closure. In the process we
%% observe a few additional optimizations necessary to reap the efficiency
%% benefits of seminaive evaluation. We discuss our implementation of the
%% semi\naive\ program transformation and these optimizations and measure their
%% effect on the performance of some example program.

First, in \cref{section-seminaive-trans} we apply the semi\naive\ program
transformation by hand to our running example, transitive closure.
%
In the process we uncover some obvious inefficiencies in the transformed code
and demonstrate how to optimize them away.
%
In \cref{section-implementation} we discuss our implementation of a
Datafun-to-Haskell compiler, which we use to demonstrate experimentally that
semi\naive\ evaluation can produce asymptotic performance improvements when
combined with these optimizations.

Second, in \cref{section-change-minimization} we observe that even with these
optimizations, there remain cases where we do asymptotically more work than
necessary, not because of inefficiencies in the transformed program, but because
of the imprecision of our derivatives. This results in overly large changes
which accumulate across fixed point iterations. We implement a simple solution
based on change minimization and test it experimentally.

\section{Applying the semi\naive\ transformation to transitive closure}
\label{section-seminaive-trans}

\todo{rewrite this to avoid ewhen}

Let's try applying the semi\naive\ transform to a simple Datafun program: the
transitive closure function \name{trans} from
\cref{section-datafun-transitive-closure}:

\begin{code}
  \name{trans} \< \pbox{\dvarlong{edge}}
  = \efixis{r}{\dvarlong{edge} \cup (\dvarlong{edge} \relcomp \mvar r)}
  \\
  \mvar s \relcomp \mvar t =
  \efor{\etuple{\dvar x, \yone} \in \mvar s,\,
        \etuple{\ytwo , \dvar z} \in \mvar t}
  \ewhen{\eeq{\yone}{\ytwo}} \eset{\etuple{\dvar x, \dvar z}}
\end{code}

\noindent
In the process we'll discover that besides $\phi$ itself we need a few simple
optimisations to actually speed up our program: most importantly, we need to
propagate $\bot$ expressions.

In our experience, performing $\phi$ and $\delta$ by hand is easiest when you
work inside-out. At the core of transitive closure is a relation composition,
$(\dvar e \relcomp p)$, and at the core of relation composition is a
\kw{when}-expression. Let's take a look at its $\phi$ and $\delta$ translations:

%% Avoid overfull hboxes in this section
\setlength\codeoffset{17.1pt}

\begin{flail}
  \phi(\ewhen {\eeq \yone \ytwo} \eset{\etuple{\dvar x, \dvar z}})
  &= \phi(\efor {\etuple{} \in \eeq \yone \ytwo} \eset{\etuple{\dvar x,
      \dvar z}})
  && \text{desugar \kw{when}}
  \\
  &= \efor{\etuple{} \in \eeq \yone \ytwo}
  \phi{\eset{\etuple{\dvar x, \dvar z}}}
  && \text{apply $\phi$, omitting an unused \kw{let}}\\
  &= \ewhen{\eeq \yone \ytwo} \eset{\etuple{\dvar x, \dvar z}}
  && \text{resugar}
\end{flail}

\noindent
Frequently, as in this case, $\phi$ does nothing interesting. For brevity we'll
skip such no-op translations.

\begin{flail}
  &\mathrel{\hphantom{=}}
  \delta(\ewhen {\eeq \yone \ytwo} \eset{\etuple{\dvar x, \dvar z}})
  \\
  &= \delta(\efor{\etuple{} \in \eeq \yone \ytwo} \eset{\etuple{\dvar x, \dvar z}})
  && \text{desugar \kw{when}}
  \\
  &= \hphantom{{}\cup} \efor{\etuple{} \in \delta(\eeq \yone \ytwo)}
  \phi\eset{\etuple{\dvar x, \dvar z}}
  %% this generates two "Overfull \vbox" warnings. :(
  && \multirow{2}{*}{\text{apply $\phi$, omitting unused \kw{let}s}}
  \\
  &\hphantom{={}} \cup
  \efor{\etuple{} \in \phi(\eeq \yone \ytwo) \cup \delta(\eeq \yone \ytwo)}
  \delta\eset{\etuple{\dvar x, \dvar z}}
  \\
  &= \efor{\etuple{} \in \bot} \eset{\etuple{\dvar x, \dvar z}}
  \cup \efor{\etuple{} \in \phi(\eeq \yone \ytwo) \cup \bot} \bot
  && \text{apply $\phi(\eeq \yone \ytwo)$ and $\delta\eset{\etuple{\dvar x, \dvar z}}$}
  \\
  &= \bot && \text{propagate }\bot
\end{flail}

\noindent
The core insight here is that neither $\eeq \yone \ytwo$ nor
$\eset{\etuple{\dvar x, \dvar z}}$ can change. Propagating this information --
for example, rewriting $(\efor {...} \bot)$ to $\bot$ -- can simplify
derivatives and eliminate expensive \kw{for}-loops.

Now let's pull out and examine $\efor{\etuple{\ytwo, \dvar z} \in t}
\ewhen{\eeq \yone \ytwo} \eset{\etuple{\dvar x, \dvar z}}$. The $\phi$
translation is again a no-op.

\begin{flail}
  &\mathrel{\hphantom{=}}
  \delta(\efor{\etuple{\ytwo, \dvar z} \in t}
  \ewhen{\eeq \yone \ytwo} \eset{\etuple{\dvar x, \dvar z}})
  \\
  &= \hphantom{{}\cup} \efor{\etuple{\ytwo, \dvar z} \in \dt}
  \phi(\ewhen{\eeq \yone \ytwo} \eset{\etuple{\dvar x, \dvar z}})
  && {\text{apply $\delta$, omitting some unused \kw{let}s}}
  \\
  &\hphantom{{}=} \cup \efor{\etuple{\ytwo, \dvar z} \in t \cup \dt}
  \delta(\ewhen{\eeq \yone \ytwo} \eset{\etuple{\dvar x, \dvar z}})
  \\
  &= \efor{\etuple{\ytwo, \dvar z} \in \dt}
  \ewhen{\eeq \yone \ytwo} \eset{\etuple{\dvar x, \dvar z}}
  && \text{applying prior work, propagating }\bot
\end{flail}

\noindent Tackling the outermost \kw{for} loop:

\begin{flail}
  &\mathrel{\hphantom{=}}
  \delta(
  \efor{\etuple{\dvar x, \yone} \in s}
  \efor{\etuple{\ytwo, \dvar z} \in t}
  \ewhen{\eeq \yone \ytwo} \eset{\etuple{\dvar x, \dvar z}})
  \\
  &= \efor{\etuple{\dvar x, \yone} \in \ds}
  \phi(\efor{\etuple{\ytwo, \dvar z} \in t}
  \ewhen{\eeq \yone \ytwo} \eset{\etuple{\dvar x, \dvar z}})
  && \text{definition of $\delta(\kw{for} \dots)$}
  \\
  &\cup \efor{\etuple{\dvar x, \yone} \in s \cup \ds}
  \delta(\efor{\etuple{\ytwo, \dvar z} \in t}
  \ewhen{\eeq \yone \ytwo} \eset{\etuple{\dvar x, \dvar z}})
  \\
  &= \efor{\etuple{\dvar x, \yone} \in \ds}
  \efor{\etuple{\ytwo, \dvar z} \in t}
  \ewhen{\eeq \yone \ytwo} \eset{\etuple{\dvar x, \dvar z}}
  && \text{applying prior work}
  \\
  &\cup
  \efor{\etuple{\dvar x, \yone} \in s \cup \ds}
  \efor{\etuple{\ytwo, \dvar z} \in \dt}
  \ewhen{\eeq \yone \ytwo} \eset{\etuple{\dvar x, \dvar z}}
  \\
  &= (\ds \relcomp t) \cup ((s \cup \ds) \relcomp \dt)
  && \text{rewriting in terms of ${\relcomp}$}
\end{flail}

\noindent
This, then, is the derivative $\delta(s \relcomp t)$ of relation composition.
With a bit of rewriting, this is equivalent to $(\ds \relcomp t) \cup (s \relcomp
\dt) \cup (\ds \relcomp \dt)$, which is perhaps the derivative a human would
give.

Let's use this to figure out $\phi(\name{trans}\<\eboxvar{e})$. Working inside
out, we start with the derivative of the loop body, $\delta(\dvar e \cup (\dvar
e \relcomp p))$:

\begin{flail}
  \delta({\dvar e \cup (\dvar e \relcomp p)})
  &= \delta\dvar e \cup \delta(\dvar e \relcomp p)\\
  &= \delta\dvar e
  \cup (\delta\dvar e \relcomp p)
  \cup ((\dvar e \cup \delta\dvar e) \relcomp \deep)
  \\
  &= \bot \cup (\bot \relcomp p) \cup ((\dvar e \cup \bot) \relcomp \deep)
  && \delta\dvar e ~\text{is a zero change; insert }\bot
  \\
  &= \dvar e \relcomp \deep
  && \text{propagate}~\bot
\end{flail}

\noindent
The penultimate step requires a new optimization.
%
By definition $\delta\dvar e = \dvar{de}$, but since $\dvar e$ is discrete we
know $\dvar{de}$ is a zero change, so we may safely replace it by $\bot$.

Putting everything together, we have:

\begin{flail}
  \phi(\efixis{p}{\dvar e \cup (\dvar e \relcomp p)}
  &= \phi(\efix \ebox{\fnof{p} \dvar e \cup (\dvar e \relcomp p)})
  && \text{desugaring}
  \\
  &= \semifix\< \phi\ebox{\fnof{p} \dvar e \cup (\dvar e \relcomp p)}
  \\
  &= \semifix\<\bigeboxtuple{\phi({\fnof{p} \dvar e \cup (\dvar e \relcomp p)}),\
  \delta({\fnof{p} \dvar e \cup (\dvar e \relcomp p)})}
  \\
  &= \semifix\<\bigeboxtuple{
      ({\fnof{p} \dvar e \cup (\dvar e \relcomp p)}),\
      ({\fnof{\pboxvar p} \fnof{\deep} \dvar e \relcomp \deep})}
  && \text{previous work}
\end{flail}

\noindent
Examining the recurrence produced by this use of \semifix, we recover the
semi\naive\ transitive closure algorithm from
\cref{section-seminaive-tc-in-datafun}:

\nopagebreak[4]
\begin{align*}
  x_0 &= \bot & x_{i+1} &= x_i \cup \dx_i\\
  \dx_0 &= ({\fnof{p} \dvar e \cup (\dvar e \relcomp p)}) \<\bot
  = \dvar e
  &
  \dx_{i+1} &=
  ({\fnof{\pboxvar p} \fnof{\dx} \dvar e \relcomp \deep})
  \<\ebox{x_i} \<\dx_i
  = \dvar e \relcomp \dx_i
\end{align*}

\setlength\codeoffset{20pt}     %restore.


\section{Implementation and optimization}
\label{section-implementation}

\todolater{expand and edit this section}

We have implemented a compiler from a fragment of Datafun (omitting sum types) to Haskell, available at \url{https://github.com/rntz/datafun/tree/popl20/v4-fastfix}.
%
We use Haskell's \texttt{Data.Set} to represent Datafun sets, and typeclasses to implement Datafun's notions of equality and semilattice types.
%
We do no query planning and implement no join algorithms; relational
joins, written in Datafun as nested \kw{for}-loops, are compiled into nested
loops.
%
Consequently our performance is worse than any real Datalog engine.
%
However, we do implement the $\phi$ translation, along with the following
optimizations:

\begin{enumerate}
\item Propagating $\bot$; for example, rewriting $(e \vee \bot) \rewrites e$ and
  $(\eforin{x}{e} \bot) \rewrites \bot$.

\item Inserting $\bot$ in place of semilattice-valued zero changes (for example,
  changes to discrete variables $\delta \dvar x$). This makes $\bot$-propagation
  more effective.

\item Recognising complex zero change expressions; for example, $\delta e
  \<\ebox{\phi f} \<\delta f$ is a zero change if $\delta e$ and $\delta f$ are.
  This allows more zero changes to be replaced by $\bot$, especially in
  higher-order code such as our regular expression example.
\end{enumerate}

\begin{figure}
  \centering\small\sffamily
  \begin{tikzpicture}[baseline=(current bounding box.center)]
    \begin{axis}[
        title={{\scshape transitive closure on a linear graph}\vphantom{\texttt{/a*/}$\texttt{a}^n$\textsuperscript{N}}},
        xlabel={Number of nodes},
        ylabel={Time (seconds)},
        height=132.88pt, width=213pt, % golden
        legend entries={\naive,semi\naive\ raw,semi\naive\ optimized},
        legend cell align=left,
        legend pos = north west,
        legend style={
          font=\footnotesize,
          draw=none,
          nodes={inner sep=3pt,}
        },
        xtick = {120, 160, ..., 320}, ytick = {0, 40, ..., 150},
      ]
      \addplot [color=red,mark=square*] table [x=n,y=naive] {trans.dat};
      \addplot [color=black,mark=triangle*] table [x=n,y=seminaive_raw] {trans.dat};
      \addplot [color=blue,mark=*] table  [x=n,y=seminaive] {trans.dat};
    \end{axis}
  \end{tikzpicture}
  \hfil
  \begin{tikzpicture}[baseline=(current bounding box.center)]
    \begin{axis}[
        title={{\scshape matching \texttt{/a*/} against} $\texttt{a}^{n}$},
        xlabel={Number of characters},
        height=132.88pt, width=213pt, % golden
        xtick = {120, 160, ..., 320}, ytick = {0, 40, ..., 150},
      ]
      \addplot [color=red,mark=square*] table [x=n,y=naive] {astar.dat};
      \addplot [color=black,mark=triangle*] table [x=n,y=seminaive_raw] {astar.dat};
      \addplot [color=blue,mark=*] table  [x=n,y=seminaive] {astar.dat};
    \end{axis}
  \end{tikzpicture}
  \vspace{\baselineskip}

  \newlength\annoying\setlength\annoying{10pt}
  \hspace{.25\annoying}
  \begin{tikzpicture}[baseline=(current bounding box.center)]
    \begin{semilogyaxis}[
        %% title={{\scshape transitive closure on a linear graph}\vphantom{\texttt{/a*/}$\texttt{a}^n$\textsuperscript{N}}},
        xlabel={Number of nodes},
        ylabel={Speedup factor (logarithmic scale, larger is better)},
        width=213pt, height=345pt,    % golden
        legend entries={
          semi\naive\ optimized,
          semi\naive\ simplified,
          semi\naive\ raw,
          \naive,
        },
        legend cell align=left,
        legend pos=north east,
        legend style={
          font=\footnotesize,
          %% draw=none,
          nodes={inner sep=3pt,},
          at={(0.5, 0.5)}, anchor=center,
        },
        grid=major,ymajorgrids,
        log basis y=2,
        xtick={120, 160, ..., 400},
      ]
      \addplot [color=blue,mark=*] table  [x=n,y=seminaive] {trans-speedup.dat};
      \addplot [color=purple,mark=o] table [x=n,y=seminaive_simple] {trans-speedup.dat};
      \addplot [color=black,mark=triangle*] table [x=n,y=seminaive_raw] {trans-speedup.dat};
      \addplot [color=red,mark=square*] table [x=n,y=naive] {trans-speedup.dat};
    \end{semilogyaxis}
  \end{tikzpicture}
  \hspace{.75\annoying}
  \hfil
  \hspace{\annoying}
  \begin{tikzpicture}[baseline=(current bounding box.center)]
    \begin{semilogyaxis}[
        xlabel={Number of characters},
        width=213pt, height=345pt,    % golden
        grid=major,ymajorgrids,
        log basis y=2,
        xtick={120, 160, ..., 400},
      ]
      \addplot [color=red,mark=square*] table [x=n,y=naive] {astar-speedup.dat};
      \addplot [color=black,mark=triangle*] table [x=n,y=seminaive_raw] {astar-speedup.dat};
      \addplot [color=purple,mark=o] table [x=n,y=seminaive_simple] {astar-speedup.dat};
      \addplot [color=blue,mark=*] table  [x=n,y=seminaive] {astar-speedup.dat};
    \end{semilogyaxis}
  \end{tikzpicture}
  \scriptsize

  \vspace{2\baselineskip}

  %% We color rows according to corresponding lines in the graphs.
  \colorlet{darkred}{red!65!black}
  \colorlet{darkblue}{blue!65!black}

  {\AlegreyaSansTLF\lsstyle
  \setlength\tabcolsep{3.75pt}
  \begin{tabu}{@{}l*{11}{r}@{}}
    & \multicolumn{11}{c}{\scshape graph size / string length}\\
    & 120 & 140 & 160 & 180 & 200 & 220 & 240 & 260 & 280 & 300 & 320
    \\\midrule
    \rowfont{\color{darkred}} \scshape regex search, \naive
    & 1.665 & 2.940 & 4.924 & 7.872 & 12.251 & 18.789 & 27.828 & 48.393 & 69.337 & 96.371 & 131.300
    \\
    \rowfont{\color{darkred}} \scshape transitive closure, \naive
    & 1.568 & 2.843 & 4.797 & 7.756 & 11.944 & 18.521 & 29.415 & 47.446 & 67.845 & 95.142 & 128.403
    \\
    \scshape regex search, semi\naive\ raw
    & 1.317 & 2.410 & 4.047 & 6.568 & 9.909 & 14.840 & 21.636 & 39.629 & 57.017 & 80.622 & 109.707
    \\
    \scshape transitive closure, semi\naive\ raw
    & 1.275 & 2.351 & 4.040 & 6.429 & 9.880 & 14.656 & 22.886 & 39.007 & 56.686 & 79.837 & 109.552
    \\
    \rowfont{\color{Fuchsia}}
    \scshape regex search, semi\naive\ simplified
    & 0.024 & 0.037 & 0.055 & 0.079 & 0.107 & 0.141 & 0.182 & 0.228 & 0.279 & 0.347 & 0.407
    \\
    \rowfont{\color{Fuchsia}}
    \scshape transitive closure, semi\naive\ simplified
    & 0.022 & 0.035 & 0.054 & 0.077 & 0.103 & 0.138 & 0.187 & 0.220 & 0.271 & 0.333 & 0.395
    \\
    \rowfont{\color{darkblue}} \scshape regex search, semi\naive\ optimized
    & 0.023 & 0.036 & 0.057 & 0.079 & 0.113 & 0.142 & 0.181 & 0.227 & 0.283 & 0.355 & 0.416
    \\
    \rowfont{\color{darkblue}} \scshape transitive closure, semi\naive\ optimized
    & 0.022 & 0.035 & 0.054 & 0.081 & 0.102 & 0.137 & 0.174 & 0.216 & 0.272 & 0.336 & 0.407
  \end{tabu}}

  %% %% OLD NUMBERS
  %% {\AlegreyaSansTLF\lsstyle
  %% \setlength\tabcolsep{3.75pt}
  %% \begin{tabu}{@{}l*{11}{r}@{}}
  %%   & \multicolumn{11}{c}{\scshape graph size / string length}\\
  %%   & 120 & 140 & 160 & 180 & 200 & 220 & 240 & 260 & 280 & 300 & 320
  %%   \\\midrule
  %%   \rowfont{\color{darkred}} \scshape regex search, \naive & 1.683 & 2.786 & 4.549 & 7.324 & 11.357 & 17.304 & 25.788 & 45.634 & 65.174 & 90.934 & 123.023
  %%   \\
  %%   \rowfont{\color{darkred}} \scshape transitive closure, \naive & 1.446 & 2.599 & 4.356 & 6.933 & 10.840 & 16.803 & 27.159 & 44.136 & 64.953 & 88.154 & 119.604
  %%   \\
  %%   \scshape regex search, semi\naive\ raw & 1.687 & 2.454 & 4.134 & 6.573 & 9.854 & 14.611 & 21.661 & 39.171 & 56.345 & 79.687 & 108.236
  %%   \\
  %%   \scshape transitive closure, semi\naive\ raw & 1.187 & 2.163 & 4.154 & 5.835 & 8.982 & 13.350 & 21.069 & 36.512 & 53.197 & 75.209 & 101.933
  %%   \\
  %%   \rowfont{\color{darkblue}} \scshape regex search, semi\naive\ optimized& 0.028 & 0.037 & 0.054 & 0.075 & 0.102 & 0.133 & 0.171 & 0.220 & 0.269 & 0.331 & 0.401
  %%   \\
  %%   \rowfont{\color{darkblue}} \scshape transitive closure, semi\naive\ optimized & 0.026 & 0.037 & 0.056 & 0.072 & 0.099 & 0.130 & 0.170 & 0.204 & 0.259 & 0.312 & 0.377
  %% \end{tabu}}

  \caption{\Naive\ vs semi\naive\ evaluation of transitive closure and regex matching in Datafun}
  \label{figure-seminaive-vs-naive-graph}
\end{figure}


\noindent
To test whether the $\phi$ translation can produce the asymptotic performance
gains we claim, we benchmark two example Datafun programs:

\begin{enumerate}
\item Finding the transitive closure of a linear graph using the \name{trans}
  function from \cref{section-datafun-transitive-closure}. We chose this example
  because, as discussed in \cref{section-seminaive-incremental}, it has a well
  understood asymptotic speed-up under semi\naive\ evaluation. This means that
  if we've failed to capture the essence of semi\naive\ evaluation, it should be
  highly visible.

\item Finding all matches of the regular expression \texttt{/a*/} in the string
  $\texttt{a}^n$, using the regex combinators from \cref{section-regex-combinators}.
  Finding all matches for \texttt{/a*/} amounts to finding the reflexive,
  transitive closure of the matches of \texttt{/a/}, and on $\texttt{a}^n$ these
  form a linear graph. Thus it is a close computational analogue of our first
  example, written in a higher-order style. We chose this example to test
  whether our extension of semi\naive\ evaluation properly handles Datafun's
  distinctive feature: higher-order programming.
\end{enumerate}

\noindent
We compiled each program in three distinct ways: \emph{\naive{}}, without the
$\phi$ transform (but with $\bot$-propagation); \emph{semi\naive{} raw}, with
the $\phi$ transform but without further optimization; and \emph{semi\naive{}
  optimized}, with the $\phi$ transform followed by all three optimizations
listed previously. The results are shown in
\cref{figure-seminaive-vs-naive-graph}.
%
The measured times are substantially similar for transitive closure and regex
search across all three optimization levels, suggesting that higher-order code
does not pose a particular problem for our optimizations. However, compared to
\emph{\naive}, the $\phi$ transform alone (\emph{semi\naive\ raw}) provides only
a small speed-up, roughly 10--20\%. Only when followed by other optimizations
(\emph{semi\naive\ optimized}\kern.75pt) does it provide the expected asymptotic
speedup.\footnote{We also tried following the $\phi$ transform with only
  $\bot$-propagation, dropping the other two optimisations. This produced
  essentially the same results as \emph{semi\naive\ optimized}, so we have
  omitted it from \cref{figure-seminaive-vs-naive-graph}. It is unclear whether
  inserting $\bot$ or recognizing complex zero changes are ever necessary to
  achieve an asymptotic speed-up.

  It's also worth addressing the asymptotic performance of
  \emph{semi\naive\ optimized}. On regex search, for example, doubling the string
  length from $160$ to $320$ produces a slowdown factor of $\frac{.401}{.054}
  \approx 7.42$! However, since there are quadratically many matches and we find
  all of them, the best possible runtime is $O(n^2)$. Moreover, our nested-loop
  joins are roughly a factor of $n$ slower than optimal, so we expect $O(n^3)$
  behavior or worse. This back-of-the-envelope estimation predicts a slowdown of
  $2^3 = 8$, reasonably close to $7.42$. Phew!}

We believe this is because both $\phi(\eforin x e ...)$ and $\delta(\eforin x e
...)$ produce loops that iterate over at least every $\dvar x \in \phi e$.
Consulting our logical relation at set type, we see that in this case $e$ and
$\phi e$ will be identical, and so the number of iterations never shrinks.
However, as demonstrated in \cref{section-seminaive-trans}, if the body can be
simplified to $\bot$, then we can eliminate the loop entirely by rewriting
$(\eforin x e \bot)$ to $\bot$, which allows for asymptotic improvement.

As in Datalog, we do not expect semi\naive\ evaluation to be useful on
\emph{all} recursive programs. Under \naive\ evaluation, each iteration towards
a fixed point is more expensive than the last, so as a rule of thumb,
semi\naive\ evaluation is more valuable the more iterations required.

%% TODO: Discuss when \& why inlining might be helpful.
%% We speculate that \emph{function inlining} would also be a helpful optimization.

\section{Change minimization}

\label{section-change-minimization}

Before concluding that we have captured the essence of seminaive evaluation, let's try a small twist on our running example: let's add self-loops to every node in our linear graph, producing the graph $(V,E)$ with $V = \{1, ..., n\}$ and $E = \setfor{(i,j)}{j \in \{i,i+1\}}$.
%
This makes our reachability relation reflexive, changing the transitive closure from the less-than relation $\setfor{(i,j)}{1 \le i < j \le n}$ to the less-than-or-equal-to relation $\setfor{(i,j)}{1 \le i \le j \le n}$.
%
This produces exactly $n$ new paths, namely $\setfor{(i,i)}{1 \le i \le n}$; since we already had quadratically many paths, ideally this won't affect our performance much.

Unfortunately, adding these self-loops produces an asymptotic slowdown, even with our semi\naive\ transformation and all optimizations applied (\`a la \emph{semi\naive\ optimized}):

\nopagebreak[3]
\begin{center}
  \small\sffamily
  \begin{tikzpicture}[baseline=(current bounding box.center)]
    \begin{axis}[
        %% title={\scshape transitive closure without change minimization},
        xlabel={Number of nodes},
        ylabel={Time (seconds)},
        %% height=132.88pt, width=213pt, % golden
        %height=140pt, width=210pt, % 2:3
        height=144pt, width=233pt, % fibonacci/golden
        legend entries={line graph,line graph with self-loops},
        legend cell align=left,
        legend pos = north west,
        legend style={
          font=\footnotesize,
          draw=none,
          nodes={inner sep=3pt,}
        },
        xtick = {120, 160, ..., 400},
        ytick = {0, 240, ..., 790},
      ]
      \addplot [color=Fuchsia,mark=x] table [x=n,y=normal] {trans-loop-extended.dat};
      %% \addplot [color=black,mark=triangle*] table [x=n,y=normal] {trans-loop-extended.dat};
      \addplot [color=red,mark=square*] table [x=n,y=normal_loopy] {trans-loop-extended.dat};
    \end{axis}
  \end{tikzpicture}
\end{center}

\noindent
What's going on here?
%
Recall from \cref{equation-semifix-trans-recurrence}, \cpageref{equation-semifix-trans-recurrence} (and confirmed by \cref{figure-trans-seminaive}) that the semi\naive\ iteration strategy Datafun uses for transitive closure is:

\begin{align*}
  x_0 &= \emptyset
  &
  x_{i+1} &= x_i \cup \dx_i
  \\
  \dx_0 &= \name{edges}
  &
  \dx_{i+1} &= \name{edges} \relcomp \dx_i
\end{align*}

\noindent
The key computation step here is $\dx_{i+1} = \name{edges} \relcomp \dx_i$.
%
In other words, we prepend edges out of each ``frontier'' $\dx_i$ to get the next frontier $\dx_{i+1}$.
%
Ideally, each frontier consists of pairs $(x,y)$ newly discovered to be reachable; by accumulating them into $x_i = \bigcup_{j<i} \dx_j$ we find all such pairs.
%
In a linear graph without self-loops, as we saw in \cref{section-seminaive-incremental}, this strategy discovers each reachable pair exactly once, because $\dx_i$ captures paths of length exactly $i$, and each reachable pair corresponds to a unique path.
%
But if our edge relation is reflexive, any path from $x$ to $y$ can be adjoined to a self-loop to find a longer path from $x$ to $y$; thus $\dx_i \subseteq \dx_{i+1}$.
%
In turn this means that $\dx_i = x_{i+1}$; by adding self-loops we've regressed to \naive\ evaluation!

Taking a logical perspective, at step $i$, \naive\ evaluation finds all derivations of depth $d \le i$, while the ``semi\naive'' strategy we've presented so far finds only derivations of depth $d = i$.
%
This is a clear improvement, but sometimes the same fact may be derived at multiple depths -- as in our loopy linear graph, where derivation depth is path length.
%
We care only about \emph{whether} a fact has a derivation, so anything after the first (shallowest) derivation is redundant.

From an incremental computation perspective, this is a problem of unnecessarily large changes.
%
Our semi\naive\ strategy looks for ``new'' ways to derive a tuple $(x,y)$, based on whatever was ``newly'' derived in the previous step.
%
But our notion of ``new'' is a bit lax, because our derivatives are allowed to be imprecise. Our strategy for finding a fixed point of a function $f : \tset{\fint A} \to \tset{\fint A}$ is:

\begin{align*}
  x_0 &= \emptyset & x_{i+1} &= x_i \cup \dx_i\\
  \dx_0 &= f\<\emptyset &\dx _{i+1} &= f'\<x_i\<\dx_i
\end{align*}

\noindent
For sets, the derivative property guarantees that $f\<x_i \cup f'\<x_i\<\dx_i = f\<x_{i+1}$, but not that $f'\<x_i\<\dx_i = f\<x_{i+1} \setminus f\<x_i$.
%
This is exploited in, among others, the derivative rule $\delta(e_1 \cup e_2) = \delta e_1 \cup \delta e_2$.
%
If $\delta e_1$ and $e_2$ intersect (or $\delta e_2$ and $e_1$ intersect), this generates an overly large change.

A more precise derivative would be $\delta(e_1 \cup e_2) = (\delta e_1 \setminus e_2) \cup (\delta e_2 \setminus e_1)$.
%
However, this does more work, not less: it does not avoid computing ``old'' elements $x \in e_1 \cup e_2$, but rather discards them after-the-fact.
%
Indeed, discovering something twice because it has two different derivations seems in general unavoidable; in graph reachability, for instance, how can we know two different paths lead to the same destination except by following them?

So if computing these overly large changes actually takes \emph{less} work, where does the asymptotic slowdown originate?
%
It happens because rediscovering a reachable pair $(x,y)$ at iteration $i$ causes redundant work in all subsequent iterations, because it is included in $\dx_i$ (treated as ``new'') and used to compute $\dx_{i+1} = f'\<x_i\<\dx_i$.
%
Consequently, $\dx_{i+1}$ will include re-derivations of anything the presence of $(x,y)$ makes ``newly'' deducible; and so on in $\dx_{i+2}, \dx_{i+3},$ \emph{etc.}

While we may not be able to avoid all rediscovery, we can avoid these unnecessary changes accumulating across iterations -- and the resulting asymptotic wastefulness -- by \emph{minimizing our changes.}
%
Let's change our strategy for finding the ``new'' frontier $\dx_{i+1}$ to remove anything that's already in $x_{i+1}$:

\begin{align*}
  \textit{for transitive closure}
  \qquad
  \dx_{i+1} &= (\name{edges} \relcomp \dx_i) \setminus x_{i+1}
  \\
  \textit{or more generally}
  \qquad
  \dx_{i+1} &= (f' \<x_i \<\dx_i) \setminus x_{i+1}
\end{align*}

\noindent
This ensures each $\dx_i$ is minimal, disjoint from $x_i$ and thus all prior $\dx_j$ for $j < i$. (We don't need to do anything to minimize $\dx_0$ since $x_0 = \bot$.) This fixes our asymptotic slowdown:

\nopagebreak[3]
\begin{center}
  \small\sffamily
  \begin{tikzpicture}[baseline=(current bounding box.center)]
    \begin{axis}[
        %% title={\scshape transitive closure without change minimization},
        xlabel={Number of nodes},
        ylabel={Time (seconds)},
        height=144pt, width=233pt, % fibonacci/golden
        legend entries={w/o minimizing changes, minimizing changes},
        legend cell align=left,
        legend pos = north west,
        legend style={
          font=\footnotesize,
          draw=none,
          nodes={inner sep=3pt,}
        },
        xtick = {120, 160, ..., 400},
        ytick = {0, 240, ..., 790},
      ]
      \addplot [color=red,mark=square*] table [x=n,y=normal_loopy] {trans-loop-extended.dat};
      \addplot [color=black,mark=triangle*] table [x=n,y=minimized_loopy] {trans-loop-extended.dat};
    \end{axis}
  \end{tikzpicture}
\end{center}

\begin{figure}
  \centering\small\sffamily

  \begin{tikzpicture}[baseline=(current bounding box.center)]
    \begin{axis}[
        title={\scshape transitive closure without change minimization},
        xlabel={Number of nodes},
        ylabel={Time (seconds)},
        height=132.88pt, width=213pt, % golden
        %height=140pt, width=210pt, % 2:3
        legend entries={line graph,line graph with self-loops},
        legend cell align=left,
        legend pos = north west,
        legend style={
          font=\footnotesize,
          draw=none,
          nodes={inner sep=3pt,}
        },
        xtick = {120, 160, ..., 320},
        ytick = {0, 60, ..., 260},
      ]
      \addplot [color=black,mark=triangle*] table [x=n,y=normal] {trans-loop.dat};
      \addplot [color=red,mark=square*] table [x=n,y=normal_loopy] {trans-loop.dat};
    \end{axis}
  \end{tikzpicture}

  
  \begin{tikzpicture}[baseline=(current bounding box.center)]
    \begin{axis}[
        title={\scshape transitive closure with self-loops},
        xlabel={Number of nodes},
        ylabel={Time (seconds)},
        height=132.88pt, width=213pt, % golden
        %height=140pt, width=210pt, % 2:3
        legend entries={semi\naive,{semi\naive\ minimizing}},
        legend entries={w/o minimizing changes, minimizing changes},
        legend cell align=left,
        legend pos = north west,
        legend style={
          font=\footnotesize,
          draw=none,
          nodes={inner sep=3pt,}
        },
        xtick = {120, 160, ..., 320},
        ytick = {0, 60, ..., 260},
      ]
      \addplot [color=red,mark=square*] table [x=n,y=normal_loopy] {trans-loop.dat};
      \addplot [color=black,mark=triangle*] table [x=n,y=minimized_loopy] {trans-loop.dat};
%      \addplot [color=ForestGreen,mark=star] table [x=n,y=normal] {trans-loop.dat};
    \end{axis}
  \end{tikzpicture}
  \hfil
  \begin{tikzpicture}[baseline=(current bounding box.center)]
    \begin{axis}[
        title={\scshape transitive closure with change minimization},
        xlabel={Number of nodes},
        ylabel={Time (seconds)},
        height=132.88pt, width=213pt, % golden
        %height=140pt, width=210pt, % 2:3
        legend entries={{with loops},{loopless}},
        legend cell align=left,
        legend pos = north west,
        legend style={
          font=\footnotesize,
          draw=none,
          nodes={inner sep=3pt,}
        },
        xtick = {120, 160, ..., 320},
        %ytick = {0, 40, ..., 150},
      ]
      %% \addplot [color=red,mark=square*] table [x=n,y=normal] {trans-loop.dat};
      \addplot [color=black,mark=triangle*] table [x=n,y=minimized_loopy] {trans-loop.dat};
      \addplot [color=blue,mark=*] table [x=n,y=minimized] {trans-loop.dat};
%      \addplot [color=green,mark=x] table [x=n,y=normal] {trans-loop.dat};
    \end{axis}
  \end{tikzpicture}

%  \caption{Semi\naive\ transitive closure on a loopy line graph with and without change minimization}
  \caption{The effect of self-loops and change minimization on semi\naive\ transitive closure on a line graph}
  \label{figure-change-minimization-graph}
\end{figure}


\noindent
If we examine all four options -- with and without self-loops, with and without minimizing $\dx_i$ -- and compare their slowdown factors, taking a loopless graph without change minimization as our baseline (\cref{figure-change-minimization-graph}), we find that (a) loopy graphs without change minimization are asymptotically slow; (b) on loopless graphs, minimizing changes has low constant overhead (versus not minimizing); and (c) moving from a loopless to a loopy graph causes a roughly 2x slowdown when minimizing changes, because we must remove rediscovered paths on every iteration.

Two questions remain: (1) why is change minimization correct? and (2) how can we generalize it from finite sets to all semilattice types?
%
As it happens, the answer to the first question also illuminates the second.
%
Change minimization is correct because it preserves validity of changes: if \(\changesat{\tseteq A}{\dx} x y\), in other words \(x \cup \dx = y\), then we also have \(\changesat{\tseteq A}{\dx \setminus x} x y\), because \(x \cup (\dx \setminus x) = x \cup \dx = y\). Thus minimizing changes preserves the inductive invariant that \(\changes{\dx_i}{x_i}{f \<x_i}\) which guarantees \semifix\ finds a least fixed point.

This condition tells us how to generalize our approach: for each lattice type $L$, we require a change minimization operator \((\setminus)_L : L \to L \to L\) such that if \(\changesat L \dx x y\) then \(\changesat L {\dx \setminus_L x} x y\).\footnotemark\
%
In our runtime library (\cref{figure-runtime-library}), we accomplish this by adding a \lstinline{diff} method to our \lstinline{Semilat} typeclass:

\footnotetext{Given Datafun's monotonicity typing, the reader may wonder whether and with respect to what $(\setminus)_L$ must be monotonic. In practice, on finite sets \(\dx \setminus x\) is monotone in \(\dx\) but not \(x\), and this pattern will hold for our other semilattice types as well. However, monotonicity in either argument is not required for correctness.}

\begin{lstlisting}
  class Semilat a where ...
    diff :: a -> a -> a
    diff dx x = x
\end{lstlisting}

\noindent
This method admits a default implementation, $\dx \setminus_L x = \dx$, which trivially satisfies our correctness condition. Although generic, this is inefficient, for it degenerates to the original non-minimizing implementation of \semifix:

\begin{align*}
  \dx_{i+1} &= (f' \<x_i \<\dx_i) \setminus_L x_{i+1}
  = f' \<x_i \<\dx_i
\end{align*}

\noindent
Instead, we want \(\dx \setminus_L x\) to be as small as possible (whatever that means for our runtime representation of $L$) to reduce the work done by any operators applied after change minimization. So we provide more practical implementations for finite sets and product types:

\begin{align*}
  \dx \setminus_{\tseteq A} x &= \dx \setminus x
  &
  \etuple{\dx, \dy} \setminus_{L \times M} \etuple{x,y}
  &= \etuple{(\dx \setminus_L x),\, (\dy \setminus_M y)}
\end{align*}

\noindent
In our Haskell runtime this becomes:

\nopagebreak[2]
\begin{lstlisting}
  instance Ord a => Semilat (Set a) where ...
    diff = Set.difference
  instance (Semilat a, Semilat b) => Semilat (a,b) where ...
    diff (da,db) (a,b) = (diff da a, diff db b)
\end{lstlisting}

\noindent
Unfortunately, for some semilattices the degenerate default is the \emph{only} valid change minimization operator. This includes any totally-ordered semilattice, such as \(\N^\infty_{\textsf{min}}\), the naturals extended with positive infinity under minimum, which is useful for shortest path computations. This inability to meaningfully minimize changes is concerning for the efficiency of computations which use these semilattices; an important direction for future work would be to characterize the worst-case efficiency of semi\naive\ evaluation over different classes of semilattice.

%% It would be highly desirable to characterize the efficiency 


%% \subsection{Implementing change minimization for Datafun}

%% The preceding graphs were obtained by implementing this revised change-minimizing semi\naive\ evaluation strategy in our Datafun compiler and runtime.
%% %
%% To do this, we have to generalize from the finite set semilattice to all Datafun's semilattice types.
%% %
%% To do this, we must in turn ask why our revised strategy is correct.
%% %
%% \fixme{NOW}{finish section on implementing change minimization}
%% %
%% This is because $\changesat{\tseteq A}{\dx}{x}{y}$ implies $\changesat{\tseteq A}{\dx \setminus x} x y$, so since $\changes{f'\<x_i\<\dx_i}{x_{i+1}}{x_{i+2}}$ the same is true of $(f'\<x_i\<\dx_i) \setminus x_{i+1}$.
%% %
%% Of course, to apply this strategy in Datafun we need to generalize set difference to semilattices other than finite sets, giving a ``change minimization'' operator $(\setminus_L) : \Delta L \to L \to \Delta L$.


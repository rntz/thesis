\begin{figure}[pt]
  \centering
  
  \newlength\descriptionlength
  \setlength\descriptionlength{2\baselineskip}
  %\newcommand\stepnumber[1]{{\bfseries\scshape\osfstyle step #1}}
  \newcommand\updline{\color{purple}}
  \newcommand\reachstep[3]{%
    \hfil
    \begin{minipage}[t]{.48\textwidth}
      \vspace{\baselineskip}
      \ttfamily
      \begin{tabular}{cl@{\qquad}l}
        #3
        \\[2ex]
        \multicolumn{3}{l}{%
          \parbox[t][\descriptionlength][t]{.94\textwidth}{%
            \AlegreyaSansOsF
            {\bfseries\scshape step #1}\hphantom{n}#2}%
        }
      \end{tabular}
    \end{minipage}
    \hfil
  }

  \todo{show control flow of program, and the active edges for each step}
  
  \reachstep{1}{The assignment on line 1 adds itself.}{
    && \parbox[b]{.3\textwidth}{\raggedright\linespread{0}\sffamily\itshape reaching assignments}
    \\[.33ex]
    1 & \tt x := 0          & \updline (x,1)\\
    2 & \tt print x\\
    3 & \tt while true do\\
    4 & \tt\quad print x\\
    5 & \tt\quad x := x + 1}
  %
  \reachstep{2}{Information propagates to line 2.}{
    && \parbox[b]{.3\textwidth}{\raggedright\linespread{0}\sffamily\itshape reaching assignments}
    \\[.33ex]
    1 & \tt x := 0          & (x,1)\\
    2 & \tt print x         & \updline (x,1)\\
    3 & \tt while true do\\
    4 & \tt\quad print x\\
    5 & \tt\quad x := x + 1}
  %
  \reachstep{3}{Information propagates to line 3.}{
    1 & \tt x := 0          & (x,1)\\
    2 & \tt print x         & (x,1)\\
    3 & \tt while true do   & \updline (x,1)\\
    4 & \tt\quad print x\\
    5 & \tt\quad x := x + 1}
  %
  \reachstep{4}{Information propagates to line 4.}{
    1 & \tt x := 0          & (x,1)\\
    2 & \tt print x         & (x,1)\\
    3 & \tt while true do   & (x,1)\\
    4 & \tt\quad print x    & \updline (x,1)\\
    5 & \tt\quad x := x + 1}
  %
  \reachstep{5}{The assignment on line 5 adds itself, discarding the assignment from line 1.}{
    1 & \tt x := 0          & (x,1)\\
    2 & \tt print x         & (x,1)\\
    3 & \tt while true do   & (x,1)\\
    4 & \tt\quad print x    & (x,1)\\
    5 & \tt\quad x := x + 1 & \updline (x,5) }
  %
  \reachstep{6}{Information propagates from line 5 to line 3 because of the loop.}{
    1 & \tt x := 0          & (x,1)\\
    2 & \tt print x         & (x,1)\\
    3 & \tt while true do   & (x,1) \updline (x,5)\\
    4 & \tt\quad print x    & (x,1)\\
    5 & \tt\quad x := x + 1 & (x,5) }

  \vspace{\baselineskip} %HACK to make the spacing look even
  \reachstep{7}{Information propagates to line 4.}{
    1 & \tt x := 0          & (x,1)\\
    2 & \tt print x         & (x,1)\\
    3 & \tt while true do   & (x,1) (x,5)\\
    4 & \tt\quad print x    & (x,1) \updline (x,5)\\
    5 & \tt\quad x := x + 1 & (x,5) }
  %
  \reachstep{8}{No changes -- done!}{
    1 & \tt x := 0          & (x,1)\\
    2 & \tt print x         & (x,1)\\
    3 & \tt while true do   & (x,1) (x,5)\\
    4 & \tt\quad print x    & (x,1) (x,5)\\
    5 & \tt\quad x := x + 1 & (x,5)}
  
  \caption{Reaching definitions, step-by-step}
  \label{figure-reaching-definitions-execution}
\end{figure}

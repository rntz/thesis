\begin{abstract}

\todolater{mine https://flix.dev/paper/oopsla2020a.pdf intro for more examples of uses of Datalog.}

\noindent
The deductive query language Datalog has found a wide array of uses, including
static analysis~\citep{DBLP:conf/datalog/SmaragdakisB10}, business
analytics~\citep{logicblox}, and distributed programming~\citep{dedalus,bloom}.
Datalog is high-level and declarative, but simple and well-studied enough to
admit efficient implementation strategies. For example,
\citeauthor{DBLP:conf/aplas/WhaleyACL05} found they could replace a hand-tuned C
implementation of context-sensitive pointer analysis with a
comparably-performing Datalog program that was 100x
smaller~\citep{whaley-lam,DBLP:conf/aplas/WhaleyACL05}.

However, Datalog's semantics are not stable under extensions. For instance,
adding arithmetic operations breaks Datalog's termination guarantee. Despite
this, nearly all practical implementations extend Datalog beyond its theoretical
core to add niceties such as arithmetic, datatypes, aggregations, and so on.
Moreover, pure Datalog cannot abstract over repeated code: one may express a
static analysis over a \emph{particular} program, but to express the same
analysis over multiple programs, one must duplicate the analysis code for each
program analyzed.

This thesis deconstructs Datalog from a categorical and type theoretic
perspective to determine what makes it tick. Datalog's semantic guarantees are
provided by brute syntactic restrictions, such as stratification and the absence
of function symbols. In place of these, we find compositional semantic
properties such as monotonicity, which we capture using types. We show that
this permits integrating Datalog's features with those of typed functional
languages, such as algebraic data types and higher order functions.

In particular, this thesis makes the following contributions:
\begin{enumerate}
\item We define and expound the semantics and metatheory of Datafun, a pure and
  total higher-order typed functional language capturing the essence of Datalog.
  Where Datalog has predicates defined by a restricted class of Horn clauses,
  Datafun has finite sets and set comprehensions; Datalog's bottom-up recursive
  queries become iterative fixed points; and Datalog's stratification condition
  becomes a matter of tracking monotonicity with types.

\item We show how to generalize semi\naive\ evaluation to handle higher-order
  functions. Semi\naive\ evaluation is a technique from the Datalog literature
  which improves the performance of Datalog's most distinctive feature:
  recursive queries. These are computed iteratively, and under a \naive\
  evaluation strategy, each iteration recomputes all previous values. Semi\naive\
  evaluation computes a safe approximation of the difference between iterations.
  This can asymptotically improve the performance of Datalog queries. Semi\naive\
  evaluation is defined partly as a program transformation and partly as a
  modified iteration strategy, and takes advantage of the first-order nature of
  Datalog. We extend this transformation to handle higher-order programs written
  in Datafun.

\item In the process of generalizing semi\naive\ evaluation, we uncover a theory
  of incremental, monotone, higher-order computation, in which values change
  over time by growing larger, and programs respond incrementally to these
  increases.
\end{enumerate}

\todo{no page numbers here!}

\end{abstract}

\section{Implementation}
\label{section-implementation}

To put the semi\naive\ transformation presented in \cref{chapter-seminaive} to the test, we have implemented it as part of a compiler and runtime for a fragment of Datafun (omitting sum types and pattern-matching), available at \url{https://github.com/rntz/datafun/tree/popl20/v4-fastfix}.
%
In \cref{section-compiler} we sketch the compiler's front-to-back structure, from Datafun source code through several intermediate stages to Haskell output.
%
In \cref{section-runtime} we explore how an example Datafun program gets compiled, exhibit the small auxiliary library necessary to run the compiled outputs, and explain why we chose Haskell as a target language.
%
%% In \cref{section-final-style} we discuss why we chose to implement the compiler in OCaml and in ``final style''~\citep{DBLP:journals/jfp/CaretteKS09}.
%
Finally, in \cref{section-benchmarks} we perform some simple benchmarks to test for the expected efficiency gains.

%% In \cref{section-compiler} we explain the structure of our compiler and why we chose to implement it in OCaml.
%% %
%% Our compiler produces Haskell code which requires a small runtime library to execute; in \cref{section-runtime} we explain this runtime and why we chose Haskell as a target language.
%% %
%% Finally in \cref{section-benchmarks} we perform some simple benchmarks to test for the expected efficiency gains.

%% \begin{itemize}
%% \item the structure of our compiler. why OCaml? final style and why we use it. why do we omit sum types?
%% \item what a compiled result looks like.
%% \item the structure of our runtime implementation. why Haskell? (A: convenient high-level FP language w/ typeclasses.)
%% \item Performance tests. Maybe this becomes its own section?
%% \end{itemize}


\subsection{The compiler structure}\label{section-compiler}

\begin{figure}
  \sffamily
  % https://q.uiver.app/?q=WzAsNixbMCwwLCJcXHRleHR7U291cmNlfSJdLFswLDIsIlxcdGV4dHtTeW50YXh9Il0sWzAsNCwiXFx0ZXh0e1R5cGVkIHN5bnRheH0iXSxbMCw2LCJcXHRleHR7SUx9Il0sWzIsNSwiXFx0ZXh0e0lMIHcvIHplcm99Il0sWzAsOCwiXFx0ZXh0e0hhc2tlbGx9Il0sWzAsMSwiXFx0ZXh0e3BhcnNlfSJdLFsxLDIsIlxcdGV4dHt0eXBlY2hlY2t9Il0sWzIsMywiXFx0ZXh0e2Ryb3AgYm94ZXN9IiwyXSxbMiw0LCJcXHBoaVxcdGV4dHsgdHJhbnNmb3JtfSJdLFs0LDMsIlxcdGV4dHtkcm9wIHplcm9zfSIsMV0sWzQsMywiXFx0ZXh0e3Byb3BhZ2F0ZSB6ZXJvc30iLDEseyJjdXJ2ZSI6LTR9XSxbMyw1LCJcXHRleHR7ZW1pdCBIYXNrZWxsfSJdXQ==
  \[
  \begin{tikzcd}[every label/.append style={font=\footnotesize\itshape}]
	        {\text{Source code}} \\
	        \\
	          {\text{Untyped syntax}} \\
	          \\
	            {\text{Typed syntax}} \\
	            && {\text{IL with zero}} \\
	            {\text{IL}} \\
	            \\
	              {\text{Haskell}}
	              \arrow["{\text{parse}}"', from=1-1, to=3-1]
	              \arrow["{\text{typecheck}}"', from=3-1, to=5-1]
	              \arrow["{\text{drop boxes}}"', from=5-1, to=7-1]
	              \arrow["{\phi\text{ transform}}", from=5-1, to=6-3]
	              \arrow["{\text{do nothing}}"{description}, curve={height=12pt}, from=6-3, to=7-1]
	              \arrow["{\text{insert $\bot$}}"{description}, curve={height=-24pt}, from=6-3, to=7-1]
	              \arrow["{\text{emit}}"', from=7-1, to=9-1]
	              \arrow["{\text{propagate }\bot}", from=7-1, to=7-1, loop left]
  \end{tikzcd}\]
  \caption{Stages of the Datafun compiler}
  \label{figure-compiler-passes}
\end{figure}


The compiler proceeds in several passes, shown as arrows in \cref{figure-compiler-passes}.
%
It begins with fairly standard parsing and typechecking phases. For parsing, we use OCaml's Menhir library.\footnote{\url{http://gallium.inria.fr/~fpottier/menhir/}} The typechecker is bidirectional~\citep{bidirectional} rather than performing full inference\todo{citation}, for simplicity of implementation.
%
After these, we have a choice of paths.

The simplest approach is to translate Datafun fairly directly into Haskell.
%
Many of Datafun's features have direct parallels in Haskell, including sum, product, and function types; Datafun's sets are implemented using Haskell's \texttt{Data.Set} module; Datafun's semilattice types are implemented as a Haskell typeclass; and Datafun's fixed points can be implemented by \naive\ bottom-up iteration.
%
(We discuss these implementation details in \cref{section-runtime}.)

One feature of Datafun that does not translate straightforwardly into Haskell is the discreteness comonad \(\iso\).
%
However, its only purpose is to track (non-)monotonicity; at runtime, \(\iso A\) may as well be \(A\).
%
Thus before emitting Haskell we \emph{``drop boxes'',} eliminating \(\iso\) and its related term syntax by rewriting \(\ebox e \rewrites e\) and \(\eletbox x e f \rewrites \elet{x = e} f\) and dropping the distinction between discrete and monotone variables.
%
This takes us to an intermediate language, unimaginatively dubbed ``IL'', supporting Datafun's computational features but lacking its modal typing, which we translate directly into Haskell.

Thus \naive\ compilation of Datafun code takes a straight path through \cref{figure-compiler-passes}:

\nopagebreak[3]
% https://q.uiver.app/?q=WzAsNSxbMCwwLCJcXHRleHR7U291cmNlfSJdLFsyLDAsIlxcdGV4dHtTeW50YXh9Il0sWzQsMCwiXFx0ZXh0e1R5cGVkIHN5bnRheH0iXSxbNiwwLCJcXHRleHR7SUx9Il0sWzgsMCwiXFx0ZXh0e0hhc2tlbGx9Il0sWzAsMSwiXFx0ZXh0e3BhcnNlfSJdLFsxLDIsIlxcdGV4dHt0eXBlY2hlY2t9Il0sWzIsMywiXFx0ZXh0e2Ryb3AgYm94ZXN9Il0sWzMsNCwiXFx0ZXh0e2VtaXQgSGFza2VsbH0iXV0=
{\small\[\sffamily
\begin{tikzcd}[every label/.append style={font=\footnotesize\itshape}]
	{\text{Source}} && {\text{Syntax}} && {\text{Typed syntax}} && {\text{IL}} && {\text{Haskell}}
	\arrow["{\text{parse}}", from=1-1, to=1-3]
	\arrow["{\text{typecheck}}", from=1-3, to=1-5]
	\arrow["{\text{drop boxes}}", from=1-5, to=1-7]
	\arrow["{\text{emit}}", from=1-7, to=1-9]
\end{tikzcd}\]}

\noindent
For semi\naive\ evaluation, however, we must apply the \emph{$\phi$ transform} from \cref{chapter-seminaive} to the typed syntax before dropping boxes, since the transform works precisely by annotating boxed terms with their zero-changes.
%
As we have no further use for $\iso$, the compiler pass implementing $\phi$ omits boxes from its output.
%
Thus after applying the $\phi$ transform we can directly emit Haskell.

As we saw in \cref{section-seminaive-trans}, however, the $\phi$ transform produces code that can often be simplified by replacing certain expressions by $\bot$. Therefore we implement two optimization passes:

\begin{enumerate}
\item We \emph{propagate $\bot$} by applying the following rewrites to the IL:

  \begin{align*}
    e \vee \bot &\rewrites e
    &
    \bot \vee e &\rewrites e\\
    \eforin{x}{e} \bot &\rewrites \bot
    &
    \eforin x \bot e &\rewrites \bot
    \\
    \elet{x = \bot} e &\rewrites \subone{e}{x}{\bot}
    &
    \elet{x = e} \bot &\rewrites \bot
    \\
    \etuple{\bot, \bot} &\rewrites \bot
    &
    \pi_i\<\bot &\rewrites \bot
  \end{align*}

  \noindent
  The most important rewrite here is $\eforin x e \bot \rewrites \bot$: to evaluate the left hand side directly, we evaluate $e$ and iterate over the resulting set, which takes work proportional to its size; but evaluating the right hand side takes constant work. This is where the asymptotic speedups originate. The other rewrites are useful primarily because they enable this one.

\item To make $\bot$-propagation more effective, we first \emph{insert $\bot$}
  in place of semilattice-valued zero changes.
%
  To this end, the \emph{$\phi$ transform} pass explicitly marks certain terms produced by $\delta$ which are guaranteed to be zero-changes, namely:

  \begin{align*}
    \delta\dvar x &= \dx \quad\text{(for discrete $x$)}
    \\
    \delta \ebox e &= \etuple{}
    \\
    \delta\esetsub{e_i} i &=
    \delta(\eeq{e_1}{e_2}) =
    \delta(\efix e) =
    \delta\bot =
    \bot
  \end{align*}

  The \emph{insert $\bot$} pass replaces these marked expressions with $\bot$ when they have semilattice type, along with some compound terms guaranteed to produce zero-changes:

  \begin{align*}
    & \elet{x = e} f
    && \text{when $f$ is a zero change}\\
    & x
    && \text{when $x$ is let-bound to a zero change}\\
    & e_1\<e_2\<e_3
    && \text{when $e_1, e_3$ are both zero changes}
  \end{align*}

  \noindent
  In case it is not clear, the last case corresponds to the derivative of
  function application, $\delta(e \<f) = \delta e \<\ebox{\phi f} \<\delta f$,
  when neither the function nor the argument are changing.
\end{enumerate}

\noindent
To test whether \emph{insert $\bot$} is actually useful, we also implement a
\emph{do nothing} pass for comparison, which simply ignores the zero-change
annotations produced by the $\phi$ transform.
%
Altogether, we have three new paths through the compiler. First, after parsing and typechecking we we can apply the semi\naive\ transform without further optimizations:
%
{\small\[\sffamily
\begin{tikzcd}[every label/.append style={font=\footnotesize\itshape}]
  {\text{Typed syntax}} && {\text{IL with zero}} && {\text{IL}} && {\text{Haskell}}
  \arrow["{\text{$\phi$ transform}}", from=1-1, to=1-3]
  \arrow["{\text{do nothing}}", from=1-3, to=1-5]
  \arrow["{\text{emit}}", from=1-5, to=1-7]
\end{tikzcd}\]}

\noindent
Second, we can optimize the output of the $\phi$ transform by propagating $\bot$:
%
{\small\[\sffamily
\begin{tikzcd}[every label/.append style={font=\footnotesize\itshape}]
  {\text{Typed syntax}} && {\text{IL with zero}} && {\text{IL}} && {\text{IL}}
  && {\text{Haskell}}
  \arrow["{\text{$\phi$ transform}}", from=1-1, to=1-3]
  \arrow["{\text{do nothing}}", from=1-3, to=1-5]
  \arrow["{\text{propagate $\bot$}}", from=1-5, to=1-7]
  \arrow["{\text{emit}}", from=1-7, to=1-9]
\end{tikzcd}\]}

\noindent
Or finally, we can replace semilattice zero changes with $\bot$ to make $\bot$ propagation more effective:
%
{\small\[\sffamily
\begin{tikzcd}[every label/.append style={font=\footnotesize\itshape}]
  {\text{Typed syntax}} && {\text{IL with zero}} && {\text{IL}} && {\text{IL}}
  && {\text{Haskell}}
  \arrow["{\text{$\phi$ transform}}", from=1-1, to=1-3]
  \arrow["{\text{insert $\bot$}}", from=1-3, to=1-5]
  \arrow["{\text{propagate $\bot$}}", from=1-5, to=1-7]
  \arrow["{\text{emit}}", from=1-7, to=1-9]
\end{tikzcd}\]}


\subsection{Compiling transitive closure}\label{section-compiling-trans}

To understand the compiler's behavior more concretely, let's consider what it does to a Datafun program implementing transitive closure:

\begin{code}
  \name{trans} \isa \iso \tset{\eqt A \x \eqt A} \to \tset{\eqt A \x \eqt A}\\
  \name{trans} \<\pboxvarlong{edge} =
  \efixisraw{\mvar p}{\dvarlong{edge} \vee
    \esetfor{\etuple{\dvar x, \dvar z}}{
    \ptuple{\dvar x, \dvar y} \in \dvarlong{edge},\,
    \ptuple{\peqvar y, \dvar z} \in \mvar p}}
\end{code}

\noindent
This code needs a few changes for the compiler to accept it.
%
First, we must replace $\eqt A$ with a specific, concrete type, as the
compiler does not support polymorphism.
%
Second, the compiler does not support pattern-matching, so we must replace box-patterns with let-unboxing, tuple-patterns with projections, and equality patterns $\peqvar{y}$ with equality tests:

%% We've given the program a concrete type, working on graphs whose vertices are strings, because the compiler does not support polymorphism. We've also avoided use of pattern-matching syntax sugar, which the compiler also does not support. The compiler also lacks support for pattern-matching syntax sugar, and operates on ASCII syntax. Thus desugaring and \XXX\ the actual program text we use is:

\begin{code}
  \name{trans} \isa \iso \tset{\tstring \x \tstring} \to \tset{\tstring \x \tstring}\\
  \name{trans} = \efn{e}
  \\\quad
  \elet{\pboxvarlong{edge} = \mvar e}
  \\\quad
  \efixisraw{\mvar p}
  \\\qquad
  \dvarlong{edge} \vee
    \esetfor{\etuple{\pi_1\< \dvar a,\, \pi_2\< \dvar b}}{
    \dvar a \in \dvarlong{edge},\,
    \dvar b \in \mvar p,\,
    \eeq{\pi_2\<a}{\pi_1\<b}
  }
\end{code}

\noindent
In the ASCII syntax the compiler accepts, this becomes:

%% \begin{verbatim}
%%   @([{str,str}] -> {str,str})       -- type annotation
%%   \e.
%%     let [edge] = e in
%%     fix p is
%%       edge or {(pi1 a, pi2 b) for a in edge for b in p when pi2 a = pi1 b}
%% \end{verbatim}

\begin{lstlisting}[language=datafun]
  @([{str,str}] -> {str,str}) -- type annotation
  \e.
    let [edge] = e in
    fix p is
      edge or {(pi1 a, pi2 b) for a in edge for b in p when pi2 a = pi1 b}
\end{lstlisting}

%% \begin{center}
%%   \setlength\tabcolsep{1em}
%%   \begin{tabular}{>{\ttfamily}ll}
%%     ASCII syntax & Explanation
%%     \\\midrule
%%     @([\{str,str\}] -> \{str,str\}) & (type annotation)
%%     \\
%%     \textbackslash e.
%%     & $\efn{e}$
%%     \\
%%     \ let [edge] = e in
%%     &
%%     $\ \elet{\pboxvarlong{edge} = \mvar E}$
%%     \\
%%     \ fix p is
%%     &
%%     $\ \efixisraw{P}{}$
%%     \\
%%     \ \ edge or
%%     %& ``\texttt{or}'' means semilattice join
%%     & $\ \ \dvarlong{edge} \vee {}$
%%     \\
%%     \ \ \{(pi1 a, pi2 b)
%%     \\
%%     \ \ \ for a in edge for b in p
%%     \\
%%     \ \ \ when pi2 a = pi1 b\}
%%   \end{tabular}
%% \end{center}

\noindent
Passing this through the \naive\ compilation path, {\sffamily\itshape$\text{parse} \to \text{typecheck} \to \text{drop boxes} \to \text{emit}$}, produces closely corresponding Haskell code (also shown in \cref{figure-trans-naive}):

\begin{lstlisting}[emph={fix,union,forIn,guard,set}]
  \e_0 ->
    let edge_1 = e_0 in
    fix (\p_2 ->
      union edge_1
        (forIn edge_1 (\a_3 ->
         forIn p_2 (\b_4 ->
         guard (snd a_3 == fst b_4)
         (set [(fst a_3, snd b_4)])))))
\end{lstlisting}

%% \begin{verbatim}
%%   \e_0 ->
%%     let edge_1 = e_0 in
%%     fix (\p_2 ->
%%       union
%%         edge_1
%%         (forIn edge_1 (\a_3 ->
%%           forIn p_2 (\b_4 ->
%%             guard (snd a_3 == fst b_4)
%%               set [(fst a_3, snd b_4)]))))
%% \end{verbatim}

\noindent
This code has been prettified by removing unnecessary parentheses and adding line breaks and indentation. Besides minor syntactic details, the primary changes are:

\newcommand\runtime[1]{\texttt{\color{runtime}#1}}

\begin{enumerate}
\item Variable names have had unique numeric suffixes added; this is an artifact of the compiler internals.

\item Boxes have been dropped, so \,\verb|let [edge] = e|\, becomes simply \,\lstinline{let edge_1 = e_0}.

\item The set comprehension is translated into nested calls to \runtime{forIn}, \runtime{guard}, and \runtime{set}. This uglifies the code, replacing binding forms with higher-order functions applied to lambdas, but otherwise corresponds to desugaring the comprehension (\cref{figure-sugar}):

  \begin{align*}
    &\phantom{{}\desugars{}}
    \esetfor{\etuple{\pi_1\< \dvar a,\, \pi_2\< \dvar b}}{
    \dvar a \in \dvarlong{edge},\,
    \dvar b \in \mvar p,\,
    \eeq{\pi_2\<a}{\pi_1\<b}}
    \\
    &\desugars
    \eforvar a {\dvarlong{edge}}
    \eforvar b {\mvar p}
    \eforin {\pwild} {\eeq{\pi_2\<\dvar a}{\pi_1\<\dvar b}}
    \eset{\etuple{\pi_1\<\dvar a, \pi_2\<\dvar b}}
  \end{align*}

\item In a similar way, other Datafun features such as fixed points $\efixis x e$ and semilattice join $e_1 \vee e_2$ have been translated into calls to functions like \runtime{fix} and \runtime{union}.
\end{enumerate}

\begin{figure}
  \footnotesize
  \begin{lstlisting}    
module Runtime (Set, Semilat (..), set, guard, forIn,
                  fix, semifix, semifixMinimized) where
import qualified Data.Set as Set
import Data.Set (Set)

class Semilat a where
  (<:) :: a -> a -> Bool
  unions :: [a] -> a
  empty :: a
  empty = unions []
  union :: a -> a -> a
  union x y = unions [x,y]
  diff :: a -> a -> a -- Law: union a (diff da a) = union a da
  diff x y = x -- always lawful but not always efficient

instance Semilat () where
  () <: () = True
  unions _ = ()
instance (Semilat a, Semilat b) => Semilat (a,b) where
  (a,x) <: (b,y) = a <: b && x <: y
  unions ts = (unions lefts, unions rights) where (lefts, rights) = unzip ts
  diff (da,db) (a,b) = (diff da a, diff db b)
instance Semilat Bool where
  x <: y = not x || y
  unions = or
instance Ord a => Semilat (Set a) where
  (<:) = Set.isSubsetOf
  unions = Set.unions
  diff = Set.difference

set :: Ord a => [a] -> Set a
set = Set.fromList

guard :: Semilat a => Bool -> a -> a
guard c x = if c then x else empty

forIn :: Semilat b => Set a -> (a -> b) -> b
forIn set f = unions [f x | x <- Set.toList set]

fix :: Semilat a => (a -> a) -> a
fix f = loop empty
  where loop x = if x' <: x then x else loop x'
          where x' = f x

semifix, semifixMinimized :: Semilat a => ((a -> a), (a -> a -> a)) -> a
semifix (f, df) = loop empty (f empty)
  where loop x dx = if dx <: x then x else loop (union x dx) (df x dx)
semifixMinimized (f, df) = loop empty (f empty)
  where loop x dx = if dx <: x then x else
                      let x' = union x dx in loop x' (df x dx `diff` x')
  \end{lstlisting}
  \caption{Datafun's runtime library}
  \label{figure-runtime-library}
\end{figure}


\noindent
The resulting Haskell code depends on functions ({\color{runtime}highlighted in pink}) supplied by a small auxiliary library~(\cref{figure-runtime-library}).
%
The primary reason we chose Haskell as a target language is because it simplifies this step of translating language features into calls to library functions:
%
Datafun's concept of a ``semilattice type'' $L$, over which  $\vee$, $\bot$, \prim{fix}, and \kw{for}-loops are parameterized, translates directly to a Haskell typeclass \texttt{Semilat} defined in this auxiliary library.
%
This spares the Datafun compiler the work of determining at which types these primitives are actually invoked and generating the library code necessary for that subset of types.

\label{section-runtime}

This \lstinline{Semilat} typeclass has two core methods, \lstinline{(<:)} and \lstinline{unions}:%
%
\footnote{The \lstinline{diff} method on \lstinline{Semilat} and the \lstinline{semifixMinimized} function that uses it will be explained in \cref{section-change-minimization}.\label{footnote-diff}}

\begin{lstlisting}
  class Semilat a where
    (<:) :: a -> a -> Bool
    unions :: [a] -> a
    ... etc ...
\end{lstlisting}

\noindent
The \lstinline{(<:)} operator computes the type's order relation $\le$,\footnote{We did not use Haskell's built-in \lstinline{Ord} typeclass for this because it is intended for total orders, not partial ones.} and is used to determine when fixed point iteration has stabilized: semi\naive\ iteration stabilizes once $\dx_i \le x_i$.
%
(This is an optimization over checking equality $x_i = x_{i+1}$.)
%
The \lstinline{unions} method computes the semilattice join/least upper bound of its list of arguments.
%
Using this we define binary \lstinline{union}/$\vee$ and nullary \lstinline{empty}/$\bot$ wrappers.
%
We provide instances for empty and binary products, booleans, and sets, the Haskell types which our Datafun semilattice types get translated into.

The other functions provided by the runtime library are \lstinline{set}, used to construct literal sets; \lstinline{forIn}, which implements \kw{for}-loops over sets; \lstinline{guard}, which implements \kw{for}-loops over booleans (i.e.\ one-sided conditionals $\efor{e_1} e_2$, also seen as boolean ``guard'' conditions in set comprehensions); and \lstinline{fix}/\lstinline{semifix}, which implement \naive\ and semi\naive\ iteration respectively.\cref{footnote-diff}


%% \subsection{Writing a compiler in final style}\label{section-final-style}

%% \XXX


\subsection{Benchmarking semi\naive\ evaluation}\label{section-benchmarks}

%% We have implemented a compiler from a fragment of Datafun (omitting sum types) to Haskell, available at \url{https://github.com/rntz/datafun/tree/popl20/v4-fastfix}.
%% %
%% We use Haskell's \texttt{Data.Set} to represent Datafun sets, and typeclasses to implement Datafun's notions of equality and semilattice types.
%% %
%% We do no query planning and implement no join algorithms; relational
%% joins, written in Datafun as nested \kw{for}-loops, are compiled into nested
%% loops.
%% %
%% Consequently our performance is worse than any real Datalog engine.
%% %
%% However, we do implement the $\phi$ translation, along with the following
%% optimizations:

%% \begin{enumerate}
%% \item Propagating $\bot$; for example, rewriting $(e \vee \bot) \rewrites e$ and
%%   $(\eforin{x}{e} \bot) \rewrites \bot$.

%% \item Inserting $\bot$ in place of semilattice-valued zero changes (for example,
%%   changes to discrete variables $\delta \dvar x$). This makes $\bot$-propagation
%%   more effective.

%% \item Recognising complex zero change expressions; for example, $\delta e
%%   \<\ebox{\phi f} \<\delta f$ is a zero change if $\delta e$ and $\delta f$ are.
%%   This allows more zero changes to be replaced by $\bot$, especially in
%%   higher-order code such as our regular expression example.
%% \end{enumerate}

To test whether the $\phi$ translation can produce the asymptotic performance
gains we claim, we benchmark two example Datafun programs:

\begin{enumerate}
\item Finding the transitive closure of a linear graph using the \name{trans}
  function from \cref{section-compiling-trans} (first introduced in
  \cref{section-datafun-transitive-closure}). As discussed in
  \cref{section-seminaive-incremental}, transitive closure has a well understood
  asymptotic speed-up under semi\naive\ evaluation. This means that if we've
  failed to capture the essence of semi\naive\ evaluation, it should be highly
  visible.

\item Finding all matches of the regular expression \texttt{/a*/} in the string
  $\texttt{a}^n$, using the regex combinators from
  \cref{section-regex-combinators}. Finding all matches for \texttt{/a*/}
  amounts to finding the reflexive, transitive closure of the matches of
  \texttt{/a/}, and on $\texttt{a}^n$ these form a linear graph.
%
  Thus this is a close analogue of our first example, but written in a
  higher-order style, as we represent regular expressions as functions and
  regex constructors as function combinators.
%
  We chose this example to test whether our extension of semi\naive\ evaluation
  properly handles Datafun's distinctive feature: higher-order programming.
\end{enumerate}

\begin{figure}
  \centering\small\sffamily
  \begin{tikzpicture}[baseline=(current bounding box.center)]
    \begin{axis}[
        title={{\scshape transitive closure on a linear graph}\vphantom{\texttt{/a*/}$\texttt{a}^n$\textsuperscript{N}}},
        xlabel={Number of nodes},
        ylabel={Time (seconds)},
        height=132.88pt, width=213pt, % golden
        legend entries={\naive,semi\naive\ raw,semi\naive\ optimized},
        legend cell align=left,
        legend pos = north west,
        legend style={
          font=\footnotesize,
          draw=none,
          nodes={inner sep=3pt,}
        },
        xtick = {120, 160, ..., 320}, ytick = {0, 40, ..., 150},
      ]
      \addplot [color=red,mark=square*] table [x=n,y=naive] {trans.dat};
      \addplot [color=black,mark=triangle*] table [x=n,y=seminaive_raw] {trans.dat};
      \addplot [color=blue,mark=*] table  [x=n,y=seminaive] {trans.dat};
    \end{axis}
  \end{tikzpicture}
  \hfil
  \begin{tikzpicture}[baseline=(current bounding box.center)]
    \begin{axis}[
        title={{\scshape matching \texttt{/a*/} against} $\texttt{a}^{n}$},
        xlabel={Number of characters},
        height=132.88pt, width=213pt, % golden
        xtick = {120, 160, ..., 320}, ytick = {0, 40, ..., 150},
      ]
      \addplot [color=red,mark=square*] table [x=n,y=naive] {astar.dat};
      \addplot [color=black,mark=triangle*] table [x=n,y=seminaive_raw] {astar.dat};
      \addplot [color=blue,mark=*] table  [x=n,y=seminaive] {astar.dat};
    \end{axis}
  \end{tikzpicture}
  \vspace{\baselineskip}

  \newlength\annoying\setlength\annoying{10pt}
  \hspace{.25\annoying}
  \begin{tikzpicture}[baseline=(current bounding box.center)]
    \begin{semilogyaxis}[
        %% title={{\scshape transitive closure on a linear graph}\vphantom{\texttt{/a*/}$\texttt{a}^n$\textsuperscript{N}}},
        xlabel={Number of nodes},
        ylabel={Speedup factor (logarithmic scale, larger is better)},
        width=213pt, height=345pt,    % golden
        legend entries={
          semi\naive\ optimized,
          semi\naive\ simplified,
          semi\naive\ raw,
          \naive,
        },
        legend cell align=left,
        legend pos=north east,
        legend style={
          font=\footnotesize,
          %% draw=none,
          nodes={inner sep=3pt,},
          at={(0.5, 0.5)}, anchor=center,
        },
        grid=major,ymajorgrids,
        log basis y=2,
        xtick={120, 160, ..., 400},
      ]
      \addplot [color=blue,mark=*] table  [x=n,y=seminaive] {trans-speedup.dat};
      \addplot [color=purple,mark=o] table [x=n,y=seminaive_simple] {trans-speedup.dat};
      \addplot [color=black,mark=triangle*] table [x=n,y=seminaive_raw] {trans-speedup.dat};
      \addplot [color=red,mark=square*] table [x=n,y=naive] {trans-speedup.dat};
    \end{semilogyaxis}
  \end{tikzpicture}
  \hspace{.75\annoying}
  \hfil
  \hspace{\annoying}
  \begin{tikzpicture}[baseline=(current bounding box.center)]
    \begin{semilogyaxis}[
        xlabel={Number of characters},
        width=213pt, height=345pt,    % golden
        grid=major,ymajorgrids,
        log basis y=2,
        xtick={120, 160, ..., 400},
      ]
      \addplot [color=red,mark=square*] table [x=n,y=naive] {astar-speedup.dat};
      \addplot [color=black,mark=triangle*] table [x=n,y=seminaive_raw] {astar-speedup.dat};
      \addplot [color=purple,mark=o] table [x=n,y=seminaive_simple] {astar-speedup.dat};
      \addplot [color=blue,mark=*] table  [x=n,y=seminaive] {astar-speedup.dat};
    \end{semilogyaxis}
  \end{tikzpicture}
  \scriptsize

  \vspace{2\baselineskip}

  %% We color rows according to corresponding lines in the graphs.
  \colorlet{darkred}{red!65!black}
  \colorlet{darkblue}{blue!65!black}

  {\AlegreyaSansTLF\lsstyle
  \setlength\tabcolsep{3.75pt}
  \begin{tabu}{@{}l*{11}{r}@{}}
    & \multicolumn{11}{c}{\scshape graph size / string length}\\
    & 120 & 140 & 160 & 180 & 200 & 220 & 240 & 260 & 280 & 300 & 320
    \\\midrule
    \rowfont{\color{darkred}} \scshape regex search, \naive
    & 1.665 & 2.940 & 4.924 & 7.872 & 12.251 & 18.789 & 27.828 & 48.393 & 69.337 & 96.371 & 131.300
    \\
    \rowfont{\color{darkred}} \scshape transitive closure, \naive
    & 1.568 & 2.843 & 4.797 & 7.756 & 11.944 & 18.521 & 29.415 & 47.446 & 67.845 & 95.142 & 128.403
    \\
    \scshape regex search, semi\naive\ raw
    & 1.317 & 2.410 & 4.047 & 6.568 & 9.909 & 14.840 & 21.636 & 39.629 & 57.017 & 80.622 & 109.707
    \\
    \scshape transitive closure, semi\naive\ raw
    & 1.275 & 2.351 & 4.040 & 6.429 & 9.880 & 14.656 & 22.886 & 39.007 & 56.686 & 79.837 & 109.552
    \\
    \rowfont{\color{Fuchsia}}
    \scshape regex search, semi\naive\ simplified
    & 0.024 & 0.037 & 0.055 & 0.079 & 0.107 & 0.141 & 0.182 & 0.228 & 0.279 & 0.347 & 0.407
    \\
    \rowfont{\color{Fuchsia}}
    \scshape transitive closure, semi\naive\ simplified
    & 0.022 & 0.035 & 0.054 & 0.077 & 0.103 & 0.138 & 0.187 & 0.220 & 0.271 & 0.333 & 0.395
    \\
    \rowfont{\color{darkblue}} \scshape regex search, semi\naive\ optimized
    & 0.023 & 0.036 & 0.057 & 0.079 & 0.113 & 0.142 & 0.181 & 0.227 & 0.283 & 0.355 & 0.416
    \\
    \rowfont{\color{darkblue}} \scshape transitive closure, semi\naive\ optimized
    & 0.022 & 0.035 & 0.054 & 0.081 & 0.102 & 0.137 & 0.174 & 0.216 & 0.272 & 0.336 & 0.407
  \end{tabu}}

  %% %% OLD NUMBERS
  %% {\AlegreyaSansTLF\lsstyle
  %% \setlength\tabcolsep{3.75pt}
  %% \begin{tabu}{@{}l*{11}{r}@{}}
  %%   & \multicolumn{11}{c}{\scshape graph size / string length}\\
  %%   & 120 & 140 & 160 & 180 & 200 & 220 & 240 & 260 & 280 & 300 & 320
  %%   \\\midrule
  %%   \rowfont{\color{darkred}} \scshape regex search, \naive & 1.683 & 2.786 & 4.549 & 7.324 & 11.357 & 17.304 & 25.788 & 45.634 & 65.174 & 90.934 & 123.023
  %%   \\
  %%   \rowfont{\color{darkred}} \scshape transitive closure, \naive & 1.446 & 2.599 & 4.356 & 6.933 & 10.840 & 16.803 & 27.159 & 44.136 & 64.953 & 88.154 & 119.604
  %%   \\
  %%   \scshape regex search, semi\naive\ raw & 1.687 & 2.454 & 4.134 & 6.573 & 9.854 & 14.611 & 21.661 & 39.171 & 56.345 & 79.687 & 108.236
  %%   \\
  %%   \scshape transitive closure, semi\naive\ raw & 1.187 & 2.163 & 4.154 & 5.835 & 8.982 & 13.350 & 21.069 & 36.512 & 53.197 & 75.209 & 101.933
  %%   \\
  %%   \rowfont{\color{darkblue}} \scshape regex search, semi\naive\ optimized& 0.028 & 0.037 & 0.054 & 0.075 & 0.102 & 0.133 & 0.171 & 0.220 & 0.269 & 0.331 & 0.401
  %%   \\
  %%   \rowfont{\color{darkblue}} \scshape transitive closure, semi\naive\ optimized & 0.026 & 0.037 & 0.056 & 0.072 & 0.099 & 0.130 & 0.170 & 0.204 & 0.259 & 0.312 & 0.377
  %% \end{tabu}}

  \caption{\Naive\ vs semi\naive\ evaluation of transitive closure and regex matching in Datafun}
  \label{figure-seminaive-vs-naive-graph}
\end{figure}

\begin{figure}[p]\footnotesize
  \begin{lstlisting}[gobble=2]
  \e_0 ->
    let edge_1 = e_0 in
    fix (\p_2 ->
      union edge_1
        (forIn edge_1 (\a_3 ->
         forIn p_2 (\b_4 ->
         guard (snd a_3 == fst b_4)
         (set [(fst a_3, snd b_4)])))))
  \end{lstlisting}
  \caption{\Naive\ translation of transitive closure}
  \label{figure-trans-naive}
\end{figure}

\begin{figure}[p]\footnotesize
  \begin{lstlisting}[gobble=2]
  \e_0 ->
    let (edge_1, dedge_1) = e_0 in
    semifix
      ((\p_2 -> union edge_1
                   (forIn edge_1 (\a_3 ->
                    forIn p_2 (\b_4 ->
                    guard (snd a_3 == fst b_4)
                    (set [(fst a_3, snd b_4)]))))),
       (\p_2 -> \dp_2 ->
        union dedge_1
          (union
            (forIn dedge_1 (\a_3 ->
             let da_3 = ((), ()) in
             forIn p_2 (\b_4 ->
             guard (snd a_3 == fst b_4)
             (set [(fst a_3, snd b_4)]))))
            (forIn (union edge_1 dedge_1) (\a_3 ->
             let da_3 = ((), ()) in
             union
               (forIn dp_2 (\b_4 ->
                let db_4 = ((), ()) in
                guard (snd a_3 == fst b_4)
                (set [(fst a_3, snd b_4)])))
               (forIn (union p_2 dp_2) (\b_4 ->
                let db_4 = ((), ()) in
                if (snd a_3 == fst b_4) then set [] else
                guard False
                (union (set [(fst a_3, snd b_4)]) (set [])))))))))
  \end{lstlisting}
  \caption{Raw semi\naive\ translation of transitive closure}
  \label{figure-trans-seminaive-raw}
\end{figure}

\begin{figure}[p]\footnotesize
  \begin{lstlisting}
\e_0 ->
  let (edge_1, dedge_1) = e_0 in
  semifix
    ((\p_2 -> union edge_1
                 (forIn edge_1 (\a_3 ->
                  forIn p_2 (\b_4 ->
                  guard (snd a_3 == fst b_4)
                  (set [(fst a_3, snd b_4)]))))),
     (\p_2 -> \dp_2 ->
      union dedge_1
        (union
          (forIn dedge_1 (\a_3 ->
           let da_3 = ((), ()) in
           forIn p_2 (\b_4 ->
           guard (snd a_3 == fst b_4)
           (set [((fst a_3), (snd b_4))]))))
          (forIn (union edge_1 dedge_1) (\a_3 ->
           let da_3 = ((), ()) in
           forIn dp_2 (\b_4 ->
           let db_4 = ((), ()) in
           guard (snd a_3 == fst b_4)
           (set [(fst a_3, snd b_4)])))))))
  \end{lstlisting}
  \caption{Semi\naive\ translation of transitive closure with $\bot$ propagation}
  \label{figure-trans-seminaive-simple}
\end{figure}

\begin{figure}[p]
  \footnotesize
  \begin{lstlisting}
\e_0 ->
  let (edge_1, dedge_1) = e_0 in
  semifix
    ((\p_2 -> union edge_1
                 (forIn edge_1 (\a_3 ->
                  forIn p_2 (\b_4 ->
                  guard (snd a_3 == fst b_4)
                  (set [(fst a_3, snd b_4)]))))),
     (\p_2 -> \dp_2 ->
      forIn edge_1 (\a_3 ->
      let da_3 = ((), ()) in
      forIn dp_2 (\b_4 ->
      let db_4 = ((), ()) in
      guard (snd a_3 == fst b_4)
      (set [(fst a_3, snd b_4)])))))
  \end{lstlisting}
  \caption{Semi\naive\ translation of transitive closure with $\bot$ insertion and propagation}
  \label{figure-trans-seminaive}
\end{figure}


\noindent
We send each program through the four compiler paths described in \cref{section-compiler}: the direct/\naive\ translation (\emph{\naive}); the $\phi$ transform alone (\emph{semi\naive\ raw}); $\phi$ with $\bot$ propagation (\emph{semi\naive\ simplified}); and $\phi$ with $\bot$ insertion and propagation (\emph{semi\naive\ optimized}).
%
We exhibit the four translations of \name{trans} in \cref{figure-trans-naive,figure-trans-seminaive-raw,figure-trans-seminaive-simple,figure-trans-seminaive}. We graph the benchmark results in \cref{figure-seminaive-vs-naive-graph}, separately showing the running times as well as the speedup factor over \naive\ evaluation (on a logarithmic plot).
%
The results support two conclusions:

\begin{enumerate}
\item The $\phi$ transformation combined with optimizations enables asymptotic
  speedups: \emph{semi\naive\ optimized} is dramatically faster than
  \emph{\naive,} and the speedup factor increases with the input size.\footnotemark\
%
  Moreover, the measured times are similar for transitive closure and regex
  search across all optimization levels, suggesting that higher-order code does
  not pose a particular problem for our optimizations.

  \footnotetext{Although faster than \naive\ evaluation,
    \emph{semi\naive\ optimized} is still asymptotically quite slow in these benchmarks. On transitive closure, for example, doubling the graph size from $160$ to $320$ nodes yields a slowdown factor of $\frac{.407}{.054} \approx 7.54$!
%
    However, since there are quadratically many paths and we find all of them, the best possible runtime is $O(n^2)$.
%
    Moreover, our nested-loop relational joins are roughly a factor of $n$ slower than optimal, so we expect $O(n^3)$ behavior, which predicts a slowdown of $2^3 = 8$, reasonably close to $7.54$.}

\item These asymptotic improvements depend on $\bot$ propagation:
  \emph{semi\naive\ raw} yields only a small constant-factor speedup over
  \emph{\naive}, roughly 20\%.
%
  However, the measured times for \emph{semi\naive\ simplified} and
  \emph{semi\naive\ optimized} are effectively identical, so $\bot$ insertion
  appears to be irrelevant, even (somewhat surprisingly) in higher-order code.
\end{enumerate}

\noindent
As alluded to in the previous section, these asymptotic speedups come
from avoiding wasteful loops $(\eforin x {e_1} e_2)$ where $e_1$ grows as our input grows but $e_2$ always produces $\bot$.
%
The $\phi/\delta$ translations alone do not accomplish this: both $\phi(\eforin
x e ...)$ and $\delta(\eforin x e ...)$ produce loops that iterate over at least
every $\dvar x \in \phi e$.
%
Consulting our logical relation (\cref{figure-seminaive-logical-relation}) at
set type, we see that $e$ and $\phi e$ compute identical sets, therefore
the number of iterations never shrinks.
%
For instance, in the \emph{semi\naive\ raw} translation of \name{trans} (\cref{figure-trans-seminaive-raw}) the derivative passed to \lstinline{semifix} contains the following wasteful loop:

\begin{lstlisting}
  forIn (union p_2 dp_2) (\b_4 ->
    let db_4 = ((), ()) in
    if (snd a_3 == fst b_4) then set [] else
    guard False
    (union (set [(fst a_3, snd b_4)]) (set [])))
\end{lstlisting}

\noindent
Each branch of the \lstinline{if}-statement computes an empty set, the first branch explicitly and the else-branch via
%
\lstinline[language={}]{(guard False ...)}.
%
The $\bot$ propagation pass recognizes this and rewrites the entire loop to
$\bot$, removing it from the \emph{simplified} and \emph{optimized} translations
(\cref{figure-trans-seminaive-simple,figure-trans-seminaive}).

\todomaybe{Discuss why bottom insertion is unnecessary and why this is surprising.}{ Looking at the code, seminaive optimized simplifies the programs versus seminaive simplified, and even removes some loops, but the loops it removes are $\eforin x \bot e$ not $\eforin x e \bot$.  We were expecting to see some benefit for higher-order code, because bottom insertion can recognize and simplify zero changes involving use of higher order functions, but we didn't see this. Further investigation needed.

It is unclear whether inserting $\bot$ or recognizing complex zero changes are
ever necessary to achieve an asymptotic speed-up.}

%% Finally, note that, as in Datalog, we do not expect semi\naive\ evaluation to
%% be useful on \emph{all} recursive programs. Under \naive\ evaluation, each
%% iteration towards a fixed point is more expensive than the last, so as a rule of
%% thumb, semi\naive\ evaluation is more valuable the more iterations required.

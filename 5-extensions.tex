\chapter{Extensions and variations}


\section{Monotone ``if'' as first-class construct}
\todo{monotone if as first-class construct}


\section{Finite maps and other semilattices}
\label{section-finite-maps}

\todo{finite maps}


\section{Shrinking differences when computing fixed points seminaively}

\todo{in graphs with cycles in order to get the right asymptotic behavior for graph reachability we need to remove nodes already visited from the frontier/difference set. This requires a difference-shrinking operator, such as set difference for sets; can generalize this to product semilattices and maps but not in general. connects to: finite maps, atomic semilattices.}


\section{Discrete functions}
\label{section-discrete-functions}

\todo{discrete function spaces}


\section{Bounded fixed points}
\label{section-bounded-fixed-points}

\todo{bounded fixed points}


\section{Nested/monotone fixed points}
\label{section-nested-fixed-points}

\todo{nested/monotone fixed points}

\subsubsection{\todo{TAKEN FROM POPL2020}}

The typing rule for $\efix e$ requires $e : \isofixLtoL$.
%
The $\phi$ translation takes advantage of this $\iso$, decorating expressions of
type $\iso A$ with their zero changes.
%
However, it also prevents an otherwise valid idiom: in a nested fixed-point
expression $\efixis{x}{\dots (\efixis{y}{e}) \dots}$, the inner fixed point body $e$
cannot use the monotone variable $x$!
%
This restriction is not present in \citet{datafun}; its addition brings Datafun
closer to Datalog, whose syntax cannot express this sort of nested fixed point.

We suspect it is possible to lift this restriction without losing
semi\naive\ evaluation, by decorating \emph{all} expressions and variables (not
just discrete ones) with zero changes.
%
However, this also invalidates $\delta(\efix f) = \bot$: now that $f$ can
change, so can $\efix f$.
% Argh, using \delta here is technically wrong. because \delta should be
% incrementalizing \phi. this is more like \Derive. But if we use \Derive people
% will be confused, and this is a very technical point.
Luckily, there is a simple and correct solution: $\delta(\efix f) =
\efix \ebox{\delta f \<\ebox{\efix f}}$~\cite{delta-fix}.
%
However, to compute this new fixed point semi\naive{}ly, we need a \emph{second
  derivative}: the zero change to $\delta f \<\ebox{\efix f}$. Indeed, for a
program with fixed points nested $n$ deep, we need $n$\textsuperscript{th}
derivatives. We leave this to future work.

%% Can't we just have \delta produce two expressions: the derivative, and the
%% zero change to the derivative?


\section{``Monotone'' forall and the opposite ordering}

\todo{monotone forall}


\section{What is \texorpdfstring{\boldmath$\delta(\kw{for} ...)$}{\textdelta(for ...)} and why?}

\todo{the many choices of delta(for)}
